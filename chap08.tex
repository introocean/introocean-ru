% -*- coding: utf-8 -*-

\begin{chapter}{Уравнение движения с трением}\label{chap:8}

Обычно большинство внутренних сил трения в океане и атмосфере
относительно малы, и мы вполне можем предположить что поток свободен
от трения. На границах трение, в форме вязкости, становится
важным. Этот тонкий вязкий слой называется пограничным слоем. Внутри
него, скорость потока уменьшается от значений характерных для середины
потока до нуля на твёрдой границе. Если граница не твёрдая, тогда
пограничный слой~--- это тонкий слой быстро изменяющейся скорости где
скоростью на одной стороне границы изменяется скорость на другой её
стороне. Например такой пограничный слой существует на нижней части
атмосферы~--- планетарный пограничный слой описанный нами в Главе~3.
Внутри планетарного пограничного слоя скорости снижаются с многих
метров в секунду в свободной атмосфере, до десятков сантиметров в
секунду над поверхностью моря. Под поверхностью моря, другой
пограничный слой, Экмановский слой описываемый в Главе 9, передаёт
течение с поверхности на глубину. В этой главе мы обсудим роль трения
в потоках жидкости, и устойчивость потоков к небольшим изменениям
скорости или плотности.
%
% Throughout most of the interior of the ocean and atmosphere friction
% is relatively small, and we can safely assume that the flow is
% frictionless. At the boundaries, friction, in the form of viscosity,
% becomes important. This thin, viscous layer is called a
% \textit{boundary layer}\index{boundary layer|textbf}. Within the
% layer, the velocity slows from values typical of the interior to zero
% at a solid boundary. If the boundary is not solid, then the boundary
% layer is a thin layer of rapidly changing velocity whereby velocity on
% one side of the boundary changes to match the velocity on the other
% side of the boundary. For example, there is a boundary layer at the
% bottom of the atmosphere, the planetary boundary layer I described in
% Chapter 3. In the planetary boundary layer, velocity goes from many
% meters per second in the free atmosphere to tens of centimeters per
% second at the sea surface. Below the sea surface, another boundary
% layer, the \index{Ekman layer}Ekman layer described in Chapter 9,
% matches the flow at the sea surface to the deeper flow.

% In this chapter I consider the role of friction in fluid flows, and
% the stability of the flows to small changes in velocity or density.

\begin{section}{Влияние вязкости}
% \section{The Influence of Viscosity}
В последней главе мы записали компоненту уравнения движения жидкости
по оси~$x$ в форме:
\begin{equation}
\frac{\partial{u}}{\partial{t}}+u\,\frac{\partial{u}}{\partial{x}}+v\,
\frac{\partial{u}}{\partial{y}}+w\,\frac{\partial{u}}{\partial{z}}=-\,
\frac{1}{\rho}\frac{\partial{p}}{\partial{x}} + 2\,\Omega\,v\,\sin\vartheta + F_x
\end{equation}
Где $F_x$ означало массовую силу трения. Теперь мы можем рассмотреть как
будет выглядеть этот член если принять во внимание вязкость.
%
% \index{viscosity}Viscosity is the tendency of a fluid to resist
% shear. In the last chapter I wrote the $x$--component of the momentum
% equation for a fluid in the form (7:12a):
% \begin{equation}
% \frac{\partial{u}}{\partial{t}}+u\,\frac{\partial{u}}{\partial{x}}+v\,
% \frac{\partial{u}}{\partial{y}}+w\,\frac{\partial{u}}{\partial{z}}=-\,
% \frac{1}{\rho}\frac{\partial{p}}{\partial{x}} + 2\,\Omega\,v\,\sin\vartheta + F_x
% \end{equation}
% where $F_x$ was a frictional force per unit mass. Now we can consider
% the form of this term if it is due to viscosity.

Молекулы жидкости находящиеся около твёрдой границы могут ударять о
неё и передавать ей импульс (Рисунок 8.1). Молекулы находящиеся далеко
от границы сталкиваясь с молекулами которые уже ударились о неё,
продолжают передачу изменения импульса внутрь потока. Эта передача
импульса~--- молекулярная вязкость. Молекулы, тем не менее, между
столкновениями перемещаются только на микрометры и процесс этот
неэффективен для передачи импульса даже на несколько
сантиметров. Молекулярная вязкость важна только только внутри
нескольких миллиметров границы.
%
% Molecules in a fluid close to a solid boundary sometime strike the
% boundary and transfer momentum to it (figure 8.1). Molecules further
% from the boundary collide with molecules that have struck the
% boundary, further transferring the change in momentum into the
% interior of the fluid. This transfer of momentum is 
% \textit{molecular viscosity}\index{molecular
% viscosity|textbf}\index{viscosity!molecular|textbf}. Molecules,
% however, travel only micrometers between collisions, and the process
% is very inefficient for transferring momentum even a few
% centimeters. Molecular viscosity is important only within a few
% millimeters of a boundary.

Молекулярная вязкость~--- это отношение напряжения~$T_x$,
направленного по касательной к границе потока и сдвига (скорости)
потока на границе. Поэтому напряжение имеет форму:
\begin{equation}
T_{xz} =  \rho \nu \,\frac{\partial{u}}{\partial{z}}
\end{equation}
Где $\nu$~--- это кинематическая молекулярная вязкость. Типичные 
значения~$\nu$ для воды при~$\degCent{20}$ составляют~$10^{-6}\sqmps$.
%
% Molecular viscosity $\rho \nu$ is the ratio of the stress $T$
% tangential to the boundary of a fluid and the velocity shear at the
% boundary. So the stress has the form:
% \begin{equation}
% T_{xz} =  \rho \nu \,\frac{\partial{u}}{\partial{z}}
% \end{equation}
% for flow in the $(x, z)$ plane within a few millimetres of the
% surface, where $\nu$ is the kinematic molecular viscosity. Typically
% $\nu = 10^{-6}$ m$^2$/s for water at 20\degrees{C}.

Распространение (8.2) на три измерения приводит к тензору напряжения
дающему девять компонентов напряжения в точке потока, включающие
давление~--- нормальное напряжение и сдвиговые напряжения. Вывод
тензора напряжения выходит за рамки задач стоящих перед этой книгой,
но вы можете найти детали в Lamb (1945: \S 328) или Kundu (1990: p. 93). 
Для несжимаемой жидкости, массовая сила трения в (8.1) примет
вид:
\begin{equation}
F_x= \frac{\partial }{\partial x} \left[ \nu \frac{\partial u}{\partial x} \right]
   + \frac{\partial }{\partial y} \left[ \nu \frac{\partial u}{\partial y} \right]
   + \frac{\partial }{\partial z} \left[ \nu \frac{\partial u}{\partial z} \right]
= \frac{1}{\rho} \left[ \frac{\partial T_{xx}}{\partial x} +
                        \frac{\partial T_{xy}}{\partial y} +
                        \frac{\partial T_{xz}}{\partial z} \right]
\end{equation}
%
% Generalizing (8.2) to three dimensions leads to a stress tensor giving
% the nine components of stress at a point in the fluid, including
% pressure, which is a normal stress, and shear stresses. A derivation
% of the stress tensor is beyond the scope of this book, but you can
% find the details in Lamb (1945: \S 328) or Kundu (1990: p. 93). For an
% incompressible fluid, the frictional force per unit mass in (8.1)
% takes the form:
% \begin{equation}
% F_x= \frac{\partial }{\partial x} \left[ \nu \frac{\partial u}{\partial x} \right]
%    + \frac{\partial }{\partial y} \left[ \nu \frac{\partial u}{\partial y} \right]
%    + \frac{\partial }{\partial z} \left[ \nu \frac{\partial u}{\partial z} \right]
% = \frac{1}{\rho} \left[ \frac{\partial T_{xx}}{\partial x} +
%                         \frac{\partial T_{xy}}{\partial y} +
%                         \frac{\partial T_{xz}}{\partial z} \right]
% \end{equation}

Эта форма члена трения в уравнении движения была впервые опубликована
Навье (1785--1836) в 1827 году.

\begin{figure}[t!]
\makebox [120mm][c]{\includegraphics{pics/viscositysketch}}
\caption{Молекулы сталкиваясь со стеной и друг с другом передают
количество движения от потока стене, замедляя скорость движения
жидкости.}
\label{fig:viscositysketch}
\end{figure}
%
% \begin{figure}[t!]
% \makebox[120mm] [c]{\includegraphics{viscositysketch}}
% \centering
% \footnotesize
% Figure 8.1 Molecules \rule{0mm}{4ex}colliding with the wall and with
% each other transfer\\momentum from the fluid to the wall, slowing the
% fluid velocity.
%
% \label{fig:viscositysketch}
% \vspace{-3ex}
% \end{figure}

\begin{figure}[t!]
\makebox[120mm] [c]{\includegraphics{pics/reynoldsexp}}
\caption{Аппарат Рейнольдса для исследования турбулентного движения
потока в трубке и фотографии потока близкого к ламинарному (сверху) и
турбулентного (снизу) в прозрачной трубке похожей на ту котороую
использовал Рейнольдс (From Binder 1953).}
\label{fig:reynoldsexp}
\end{figure}
%
% \begin{figure}[t!]
% \makebox[120mm] [c]{\includegraphics{reynoldsexp}}
% \footnotesize
% Figure 8.2 Reynolds \rule{0mm}{4ex}apparatus for investigating the
% transition to turbulence\index{turbulence!transition to} in pipe flow,
% with photographs of near-laminar flow (left) and turbulent flow
% (right) in a clear pipe much like the one used by Reynolds. After
% Binder (1949: 88-89).
% \label{fig:reynoldsexp}
% \vspace{-4ex}
% \end{figure}
\end{section}

\begin{section}{Турбулентность}
% \section{Turbulence}
Если молекулярная вязкость важна только на расстоянии в несколько
миллиметров и не играет роли для большинства потоков в океане, если
конечно вы не зоопланктон, то как влияние границы передаётся внутрь
потока? Ответ: через турбулентность.
%
% \index{viscosity!turbulent}If molecular viscosity is important only
% over distances of a few millimeters, and if it is not important for
% most oceanic flows, unless of course you are a zooplankter trying to
% swim in the ocean, how then is the influence of a boundary transferred
% into the interior of the flow? The answer is: through turbulence.

Турбулентность является результатом нелинейных членов уравнения
движения ($(u\,\partial{u}/\partial{x}$, etc.). Важность (сила
влияния) этих членов устанавливается через безразмерное число, Число
Рейнольдса, которое является отношением нелинейных членов к вязким
членам:
\begin{equation}
Re = \frac{\text{Non-linear Terms}}{\text{Viscous Terms}} 
   = \cfrac{\left(u\,\cfrac{\partial{u}}{\partial{x}}\right) }%
           {\left(\nu\,\cfrac{\partial^2{u}}{\partial{x^2}}\right)} 
   \approx \cfrac{U\,\cfrac{U}{L\vphantom{y}}}{\nu\,\cfrac{U}{L^2}}
   = \frac{UL}{\nu}
\end{equation}
Где $U$~--- типичная (средняя) скорость потока, а $L$~--- типичная
длинна (масштаб) описывающая поток. Вы можете выбрать любое $U$, $L$
должно быть типичным для потока. Например $L$ может быть или средним
поперечным размером потока или его типичным продольным
размером. Обычно в океане $U=0.1\mps$, а $L=1~\text{мегаметр}$, таким
образом $\Reyn=10^11$. Так как нелинейные члены важны при 
$\Reyn > 10$--$1000$, то они естественно важны в океане. 
Океан турбулентен.
%
% Turbulence arises from the non-linear terms in the momentum equation
% $(u\,\partial{u}/\partial{x}$, \textit{etc}.). The importance of these
% terms is given by a non-dimensional number, the Reynolds Number $Re$,
% which is the ratio of the non-linear terms to the viscous terms:
% \begin{equation}
% Re = \frac{\text{Non-linear Terms}}{\text{Viscous
% Terms}} =
% \cfrac{\left( \displaystyle u\,\cfrac{\partial{u}}{\partial{x}}\right) }{\left( \displaystyle \nu\,\cfrac{\partial^2{u}}{\partial{x^2}}\right)
% } \approx \cfrac{\D U\,\cfrac{U}{L\vphantom{y}}}{\D \nu\,\cfrac{U}{L^2}}
% = \frac{UL}{\nu}
% \end{equation}
% where, $U$ is a typical velocity of the flow and $L$ is a typical length
% describing the flow. You are free to pick whatever $U,L$ might be typical of
% the flow. For example $L$ can be either a typical cross-stream distance, or an
% along-stream distance. Typical values in the open ocean are $U = 0.1$ m/s and
% $L = 1$ megameter, so $Re = 10^{11}$. Because non-linear terms are important if
% Re $>$ 10 -- 1000, they are certainly important in the ocean. The ocean is
% turbulent.

Число Рейнольдса названо в честь Осборна Рейнольдса (1842--1912)
проводившего в конце 19го века эксперименты призванные помочь понять
турбулентность. В одном из часто цитируемых экспериментов (Reynolds
1883), он впрыскивал краску в воду текущую с разной скоростью через
трубку (Рисунок 8.2). При малой скорости, течение было спокойным, его
назвали ламинарным течением. При более высоких скоростях течение
становилось нерегулярным и турбулентным. Переход от одного к другому
происходил при $\Reyn = VD/\nu \approx 2000$, где $V$~--- средняя
скорость в трубке, а $D$~--- диаметр трубки.
%
% The Reynolds number is named after Osborne Reynolds (1842--1912) who
% conducted experiments in the late 19th century to understand
% turbulence\index{turbulence!measurement of}.  In one famous experiment
% (Reynolds 1883), he injected dye into water flowing at various speeds
% through a tube (figure 8.2). If the speed was small, the flow was
% smooth. This is called \textit{laminar flow}. At higher speeds, the
% flow became irregular and turbulent. The transition occurred at Re $ =
% VD/\nu \approx 2000$, where $V$ is the average speed in the pipe, and
% $D$ is the diameter of the pipe.

Когда число Рейнольдса становится больше некоторого критического
значения, течение становится всё более и более турбулентным. Заметте
что структура потока является функцией числа Рейнольдса. Все течения с
одинаковой геометрией и одинаковым числом Рейнольдса обладают
одинаковой структурой потока. Таким образом течение вокруг всех
круговых цилиндров, будь они 1мм или 1 м в диаметре, при числе
Рейнольдса равном 20 выглядит также как на рисунке 8.3. Кроме того,
пограничный слой очень близко прилегает к цилиндру, и слишком тонок
для того чтобы быть показанным на рисунке.
%
% As Reynolds number increases above some critical value, the flow
% becomes more and more turbulent. Note that flow pattern is a function
% of Reynold's number. All flows with the same geometry and the same
% Reynolds number have the same flow pattern. Thus flow around all
% circular cylinders, whether 1 mm or 1 m in diameter, look the same as
% the flow at the top of figure 8.3 if the Reynolds number is
% 20. Furthermore, the boundary layer is confined to a very thin layer
% close to the cylinder, in a layer too thin to show in the figure.

\begin{figure}[t!]
\makebox[120mm][c]{\includegraphics{pics/turbsketch}}
\caption{Поток огибающий круглый цилиндр как функция числа
Рейнольдса между единицей и миллионом (From Richardson 1961). 
A~--- зубочистка в потоке 1 мм/с; B~---палец в потоке 2 см/с; 
F~---рука высунутая из окна на скорости 60 миль в час. 
Линии тока течений с одинаковым числом Рейнольдса выглядят одинаково. 
Поток обтекающий цилиндр диаметром 10 см со скоростью 1 см/с выглядит 
так же как поток со скоростью 10 см/с обтекающий цилиндр 
диаметром 1 см так как в обоих случаях $Reyn = 1000$.}
\label{fig:turbsketch}
\vspace{-2ex}
\end{figure}
%
% \begin{figure}[t!]
% \makebox[120mm][c]{\includegraphics{turbsketch}}
% \footnotesize
% Figure 8.3 Flow past a circular \rule{0mm}{3ex}cylinder as a function
% of Reynolds number between one and a million. From Richardson
% (1961). The appropriate flows are: A---a toothpick moving at 1 mm/s;
% B---finger moving at 2 cm/s; F---hand out a car window at 60 mph. All
% flow at the same Reynolds number has the same streamlines. Flow past a
% 10 cm diameter cylinder at 1 cm/s looks the same as 10 cm/s flow past
% a cylinder 1 cm in diameter because in both cases Re $= 1000$.
% \label{fig:turbsketch}
% \vspace{-2ex}
% \end{figure}

\begin{paragraph}{Турбулентное напряжение: Напряжение Рейнольдса}
% \paragraph{Turbulent Stresses: The Reynolds Stress}
Те кто изучал гидромеханику в начале 20 столетия предполагали что
небольшие объёмы воды в турбулентном движении играют такую же роль при
передаче импульса внутрь течения, какую играют молекулы в ламинарном
течении. Исследования в этом направлении привели к идее турбулентного
напряжения.
%
% \index{turbulent!stress}\index{Reynolds Stress}Prandtl, Karman and
% others who studied fluid mechanics in the early 20th century,
% hypothesized that parcels of fluid in a turbulent flow played the same
% role in transferring momentum within the flow that molecules played in
% laminar flow. The work led to the idea of turbulent stresses.

Чтобы увидеть каким образом это напряжение может возникать, рассмотрим
уравнение движения со средними и турбулентными компонентами потока:
\begin{equation}
u=U+u' \,;\quad v = V+v' \,;\quad w=W+w' \, ;\quad p=P+p'
\end{equation}
Где значение U вычислено осреднением по пространству и времени:
\begin{equation}
U = \langle u \rangle =\frac{1}{T}\int^T_0\,u(t)\,dt \quad \text{or}\quad
U = \langle u \rangle =\frac{1}{X}\int^X_0\,u(x)\,dx
\end{equation}
%
% To see how these stresses might arise, consider the momentum equation
% for a flow with mean $(U, V, W)$ and turbulent $(u', v', w')$
% components:
% \begin{equation}
% u=U+u' \,;\quad v = V+v' \,;\quad w=W+w' \, ;\quad p=P+p'
% \end{equation}
% where the mean value $U$ is calculated from a time or space average:
% \begin{equation}
% U = \langle u \rangle =\frac{1}{T}\int^T_0\,u(t)\,dt \quad \text{or}\quad
% U = \langle u \rangle =\frac{1}{X}\int^X_0\,u(x)\,dx
% \end{equation}

Нелинейные члены в уравнении движения могут быть записаны:
\begin{align}
\left< (U+u')\frac{\partial{(U+u')}}{\partial{x}} \right> &= \left<
U\,\frac{\partial{U}}{\partial{x}}\right> +
\left< U\,\frac {\partial{u'}}{\partial{x}}\right> +
\left< u' \,\frac {\partial{U}}{\partial{x}}\right> + \left< u' \,\frac
{\partial{u'}}{\partial{x}}\right> \notag \\
\left< (U+u')\frac{\partial{(U+u')}}{\partial{x}} \right> &= \left<
U\,\frac{\partial{U}}{\partial{x}}\right> +
\left<u' \,\frac{\partial{u'}}{\partial{x}}\right>
\end{align}
Второе уравнение вытекает из первого так как и 
$\langle U\,\partial{u'}/\partial{x}\rangle = 0$ 
и $\langle u'\,\partial{U}/\partial{x}\rangle = 0$, 
вытекающие из определения $U$:
$\langle U \partial{u'}/\partial{x}\rangle 
 = U \partial{\langle u' \rangle }/\partial{x} = 0$.
%
% The non-linear terms in the momentum equation can be written:
% \begin{align}
% \left< (U+u')\frac{\partial{(U+u')}}{\partial{x}} \right> &= \left<
% U\,\frac{\D \partial{U}}{\partial{x}}\right> +
% \left< U\,\frac {\partial{u'}}{\partial{x}}\right> +
% \left< u' \,\frac {\partial{U}}{\partial{x}}\right> + \left< u' \,\frac
% {\partial{u'}}{\partial{x}}\right> \notag \\
% \left< (U+u')\frac{\partial{(U+u')}}{\partial{x}} \right> &= \left<
% U\,\frac{\D \partial{U}}{\partial{x}}\right> +
% \left<u' \,\frac{\partial{u'}}{\partial{x}}\right>
% \end{align}
% The second equation follows from the first since $\langle
% U\,\partial{u'}/\partial{x}\rangle = 0$ and $\langle
% u'\,\partial{U}/\partial{x}\rangle$ $= 0$, which follow from the
% definition of $U$: $\langle U \partial{u'}/\partial{x}\rangle = U
% \partial{\langle u' \rangle }/\partial{x}$ $ = 0$.

Применяя (8.7), уравнение движения можно разделить на два уравнения:
\begin{subequations}
\begin{align}
\frac{\partial{U }}{\partial{x}} 
 + \frac{\partial{V }}{\partial{y}} 
 + \frac{\partial{W }}{\partial{z}} &=0 \\
\frac{\partial{u'}}{\partial{x}} 
 + \frac{\partial{v'}}{\partial{y}} 
 + \frac{\partial{w'}}{\partial{z}} &=0
\end{align}
\end{subequations}
%
% Using (8.5) in (7.19) gives:
% \begin{equation}
% \frac{\partial{U }}{\partial{x}} + \frac{\partial{V }}{\partial{y}}  +\frac{\partial{W }}{\partial{z}} + 
% \frac{\partial{u'}}{\partial{x}} + \frac{\partial{v'}}{\partial{y}}
% +\frac{\partial{w'}}{\partial{z}} =0
% \end{equation}
% Subtracting the mean of (8.8) from (8.8) splits the continuity
% equation into two equations:
% \begin{subequations}
% \begin{align}
% \frac{\partial{U }}{\partial{x}} + \frac{\partial{V }}{\partial{y}}  +\frac{\partial{W }}{\partial{z}} &= 0 \\
% \frac{\partial{u'}}{\partial{x}} + \frac{\partial{v'}}{\partial{y}} +\frac{\partial{w'}}{\partial{z}} &=0
% \end{align}
% \end{subequations}

И $x$ компонента уравнения движения становится:
\begin{equation}
\begin{split}
\frac{DU}{Dt} & = -\frac{1}{\rho}\,\frac{\partial{P}}{\partial{x}}  + 2\Omega V\sin\varphi \\
  & + \frac{\partial }{\partial x} \left[ \nu \frac{\partial U}{\partial x} - \langle u'u'\rangle \right]
    + \frac{\partial }{\partial y} \left[ \nu \frac{\partial U}{\partial y} - \langle u'v'\rangle \right] +
      \frac{\partial }{\partial z} \left[ \nu \frac{\partial U}{\partial z} - \langle u'w'\rangle \right]
\end{split}
\end{equation}
где 2Wv'sinj отброшено из за малости. Таким образом дополнительная
сила на единицу массы будет составлять:
\begin{equation}
F_x=-\frac{\partial}{\partial{x}}\langle u'u' \rangle
-\frac{\partial}{\partial{y}}\langle u'v' \rangle
-\frac{\partial}{\partial{z}}\langle u'w'\rangle
\end{equation}
%
% Using (8.5) in (8.1) taking the mean value of the resulting equation,
% then simplifying using (8.7), the x-component of the momentum equation
% for the mean flow becomes:
% \begin{equation}
% \begin{split}
% \frac{DU}{Dt} & = -\frac{1}{\rho}\,\frac{\partial{P}}{\partial{x}}  + 2\Omega V\sin\varphi \\
%   & + \frac{\partial }{\partial x} \left[ \nu \frac{\partial U}{\partial x} - \langle u'u'\rangle \right]
%     + \frac{\partial }{\partial y} \left[ \nu \frac{\partial U}{\partial y} - \langle u'v'\rangle \right] +
%       \frac{\partial }{\partial z} \left[ \nu \frac{\partial U}{\partial z} - \langle u'w'\rangle \right]
% \end{split}
% \end{equation}
% The derivation is not as simple as it seems. See Hinze (1975: 22) for
% details. Thus the additional force per unit mass due to the
% turbulence\index{turbulent!stress} is:
% \begin{equation}
% F_x=-\frac{\partial}{\partial{x}}\langle u'u' \rangle
% -\frac{\partial}{\partial{y}}\langle u'v' \rangle
% -\frac{\partial}{\partial{z}}\langle u'w'\rangle
% \end{equation}

Члены $\rho{\langle u' u' \rangle}$, $\rho{\langle u' v'\rangle}$, и
$\rho{\langle u' w' \rangle}$ передают (количество движения) импульс
по направлениям $x$, $y$, и $z$ . 
Например, член $\rho{\langle u' w'\rangle}$ описывает направленный
книзу транспорт восточнонаправленного импульса (количества движения)
через горизонтальную плоскость. Так как они передают количество
движения (импульс) и впервые были описаны Рейнольдсом, называют их
\emph{Напряжениями Рейнольдса}.
%
% The terms $\rho{\langle u' u' \rangle}$, $\rho{\langle u' v'
% \rangle}$, and $\rho{\langle u' w' \rangle}$ transfer eastward
% momentum ($\rho u' $) in the $x$, $y$, and $z$ directions. For
% example, the term $\rho{\langle u' w' \rangle}$ gives the downward
% transport \index{transport!momentum}of eastward momentum across a
% horizontal plane. Because they transfer momentum, and because they
% were first derived by Osborne Reynolds, they are called
% \textit{Reynolds Stresses}\index{Reynolds Stress|textbf}.
\end{paragraph}
\end{section}

\begin{section}{Расчёт напряжений Рейнольдса}
% \section{Calculation of Reynolds Stress:}
Члены трения, такие как $\partial{\langle u'w'\rangle}/\partial{z}$
являются виртуальным напряжением (сравни Goldstein, 1965: 69 \& 80). 
Теперь мы предположим что они играют такую же роль как члены
вязкости в уравнении движения. В связи с этим встаёт проблема
получения значений или функциональной формы уравнения для напряжений
Рейнольдса. Используется несколько приближений.
%
% \index{Reynolds Stress!calculation of}The Reynolds stresses such as
% $\partial{\langle u'w'\rangle}/\partial{z}$ are called virtual
% stresses (cf. Goldstein, 1965: \S 69 \& \S 81) because we assume that
% they play the same role as the viscous terms in the equation of
% motion. To proceed further, we need values or functional form for the
% Reynolds stress. Several approaches are used.

\begin{paragraph}{From Experiments}
\paragraph{From Experiments}
We can calculate Reynolds stresses from direct measurements of ($u',
v', w'$) made in the laboratory or ocean. This is accurate, but hard
to generalize to other flows. So we seek more general approaches.
%
% We can calculate Reynolds stresses from direct measurements of ($u',
% v', w'$) made in the laboratory or ocean. This is accurate, but hard
% to generalize to other flows. So we seek more general approaches.
\end{paragraph}

\begin{paragraph}{По аналогии с молекулярной вязкостью}
% \paragraph{By Analogy with Molecular Viscosity}
Давайте вернёмся к примеру изображённому на Рисунке 8.1, который
показывает пограничный слой над плоской поверхностью в $x$, $y$
проекции. Предположим что течение над поверхностью турбулентное. Это
очень распространённый тип течения в пограничном слое и мы будем
описывать его неоднократно в последующих главах. Это может быть поток
ветра над поверхностью моря или течение в придонном пограничном слое
океана или течение в перемешанном слое на поверхности.
%
% Let's return to the example in figure 8.1, which shows flow above a
% surface in the $x$, $y$ plane. Prandtl, in a revolutionary paper
% published in 1904, stated that turbulent viscous effects are only
% important in a very thin layer close to the surface, the boundary
% layer. Prandtl's invention of the boundary layer allows us to describe
% very accurately turbulent flow of wind above the sea surface, or flow
% at the bottom boundary layer in the ocean, or flow in the mixed
% layer\index{mixed layer} at the sea surface. See the box
% \textit{Turbulent Boundary Layer Over a Flat Plate}.

Предположим что течение над границей постоянно в направлении $x$, $y$, что
статистические свойства потока изменяются только по направлению $z$, и
что течение установившееся. Следовательно члены 
напряжения~$\partial /\partial t = \partial /\partial x = \partial /\partial y = 0$
будут:
\begin{equation}
2 \Omega V \sin \varphi + \frac{\partial }{\partial z} 
 \left[\nu \frac{\partial U}{\partial z} - \langle u'w'\rangle \right] = 0
\end{equation}
%
% To calculate flow in a boundary layer, we assume that flow is constant
% in the $x$, $y$ direction, that the statistical properties of the flow
% vary only in the $z$ direction, and that the mean flow is
% steady. Therefore $\partial /\partial t = \partial /\partial x =
% \partial /\partial y = 0$, and (8.10) can be written:
% \begin{equation}
% 2 \Omega V \sin \varphi + \frac{\partial }{\partial z} \left[\nu \frac{\partial
% U}{\partial z} - \langle u'w'\rangle \right] = 0
% \end{equation}

Также по анологии с (8.2)
\begin{equation}
- \rho \langle u'w'\rangle = T_{xz} = \rho A_z \frac{\partial U}{\partial z}
\end{equation}
где $A_z$ это вихревая (турбулентная) вязкость, которая заменяет
молекулярную вязкость~$\nu$ в уравнении (8.2). Тогда
\begin{equation} 
\frac{\partial{T_{xz}}}{\partial{z}} =
\frac{\partial}{\partial{z}}\left(A_z\frac{\partial{U}}{\partial{z}}\right)
\approx A_z \frac{\partial^2 U}{\partial z^2}
\end{equation}
предполагая $A_z$ или постоянной или изменяющейся гораздо медленнее в
направлении $z$ чем $\partial U / \partial z$. Таким образом мы
предположим позже что $A_z \approx z$.  Уравнения движения для
компонентов $x$ и $y$ для однородного, устойчивого турбудентного
пограничного слоя над или под горизонтальной поверхностью будут:
\begin{subequations}
\begin{align}
\rho fV + \frac{\partial {T_{xz}}}{\partial z} & = 0 \\
\rho fU - \frac{\partial {T_{yz}}}{\partial z} & = 0
\end{align}
\end{subequations}
где $f=2\omega \sin \varphi$~--- – параметр Кориолиса и мы
отбросили член молекулярной вязкости так как он гораздо меньше чем
турбулентная вязкость. Заметьте что (8.14b) получайется из такой же
производной по y компоненте уравнения движения. Нам понадобится 8.14
когда мы будем описывать движение у поверхности.
%
% We now assume, in analogy with (8.2)
% \begin{equation}
% - \rho \langle u'w'\rangle = T_{xz} = \rho A_z \frac{\partial U}{\partial z}
% \end{equation}
% where $A_z$ is an \textit{eddy viscosity}\index{eddy
% viscosity|textbf}\index{viscosity!eddy|textbf} or \textit{eddy
% diffusivity}\index{eddy diffusivity|textbf} which replaces the
% molecular viscosity $\nu$ in (8.2). With this assumption,
% \begin{equation} 
% \frac{\partial{T_{xz}}}{\partial{z}} =
% \frac{\partial}{\partial{z}}\left(A_z\frac{\partial{U}}{\partial{z}}\right)
% \approx A_z \frac{\partial^2 U}{\partial z^2}
% \end{equation}
% where I have assumed that $A_z$ is either constant or that it varies
% more slowly in the $z$ direction than $\partial U / \partial
% z$. Later, I will assume that $A_z \approx z$.
% 
% Because eddies can mix heat, salt, or other properties as well as
% momentum, I will use the term eddy diffusivity. It is more general
% than eddy viscosity, which is the mixing\index{mixing!of momentum} of
% momentum.
%
% The $x$ and $y$ momentum equations for a homogeneous, steady-state,
% turbulent boundary layer above or below a horizontal surface are:
% \begin{subequations}
% \begin{align}
% \rho fV + \frac{\partial {T_{xz}}}{\partial z} & = 0 \\
% \rho fU - \frac{\partial {T_{yz}}}{\partial z} & = 0
% \end{align}
% \end{subequations}
% where $f=2\omega \sin \varphi$ is the Coriolis
% parameter\index{Coriolis parameter}, and I have dropped the molecular
% viscosity term because it is much smaller than the turbulent eddy
% viscosity. Note, (8.15b) follows from a similar derivation from the
% $y$-component of the momentum equation. We will need (8.15) when we
% describe flow near the surface.

\begin{center}
\textbf{Турбулентный пограничный слой над плоской поверхностью.}
\end{center}
%%\begin{table}[h!]
%%\fbox{\parbox{119mm}{
%%\centering \small
%%\begin{minipage}{11.5cm}
%%\begin{center}\textbf{The Turbulent Boundary Layer \rule{0mm}{3ex}Over a Flat
%%Plate}\\
%%\end{center}
Теория распределения средней скорости в турбулентном пограничном слое
над плоской поверхностью разрабатывалась независимо Тейлором
G.I.Taylor (1886--1975), Прандтлем L. Prandtl (1875--1953), и Карманом
T. von Karman (1818--1963) с 1915 по 1935. Эта империческая теория,
иногда называемая теорией длинны перемешивания, хорошо предсказывает
профиль средней скорости у границы. Нам будет интересно то что она
может предсказывать средний поток воздуха над морем. Далее
представлена упрощённая версия этой теории, применённая к ровной
поверхности. Начнём с предположения что средний поток в пограничном
слое установившийся и изменяется только по направлению z. Внутри
нескольких миллиметров границы трение важно и (8.2) будет иметь
решение.
\begin{equation}
U = \frac{T_x}{\rho \nu} \,z
\end{equation}
и средняя скорость линейно изменяется с изменением расстояния над
границей. Обычно (8.15) записывают в безразмерной форме:
\begin{equation}
\frac{U}{u^*} = \frac{u^* z}{\nu}
\end{equation}
где $u^{*2} \equiv T_x/\rho$ это Скорость Трения. Дальше от границы
течение турбулентно и молекулярное трение не важно. В этих условиях мы
можем использовать (8.12), и
\begin{equation}
A_z \frac{\partial U}{\partial z} = u^{*2}
\end{equation}
%
% \vspace{-1.5 ex} \hspace*{1 em}\index{turbulent!boundary layer}The
% revolutionary concept of a boundary layer was invented by Prandtl in
% 1904 (Anderson, 2005). Later, the concept was applied to flow over a
% flat plate by G.I. Taylor (1886--1975), L. Prandtl (1875--1953), and
% T. von Karman (1881--1963) who worked independently on the theory from
% 1915 to 1935. Their empirical theory, sometimes called the
% \textit{mixing-length theory}\index{mixing-length theory|textbf}
% predicts well the mean velocity profile close to the boundary. Of
% interest to us, it predicts the mean flow of air above the sea. Here's
% a simplified version of the theory applied to a smooth surface.
%
% \hspace*{1 em} We begin by assuming that the mean flow in the boundary
% layer is steady and that it varies only in the $z$ direction. Within a
% few millimeters of the boundary, friction is important and (8.2) has
% the solution
% \begin{equation}
% U = \frac{T_x}{\rho \nu} \,z
% \end{equation}
% and the mean velocity varies linearly with distance above the
% boundary. Usually (8.16) is written in dimensionless form:
% \begin{equation}
% \frac{U}{u^*} = \frac{u^* z}{\nu}
% \end{equation}
% where $u^{*2} \equiv T_x/\rho$ is the \textit{friction
% velocity}\index{friction velocity|textbf}.
%
% \hspace*{1 em}Further from the boundary, the flow is turbulent, and
% molecular friction is not important. In this regime, we can use
% (8.13), and
% \begin{equation}
% A_z \frac{\partial U}{\partial z} = u^{*2}
% \end{equation}

Прандтль и Тэйлор предположили что большие вихри более эффективны в
передаче импульса чем маленькие, и поэтому $A_z$ должно изменяться с
расстоянием от стенки. Карман предположил что оно имеет форму 
$A_z = \kappa z u^*$, где $\kappa$~--- безразмерная константа. С этим
приближением уравнение дла профиля средней скорости принимает вид
\begin{equation}
\kappa z u^* \frac{\partial U}{\partial z} = u^{*2}
\end{equation}
%
% \hspace*{1 em}Prandtl and Taylor assumed that large eddies are more
% effective in mixing\index{mixing!of momentum} momentum than small
% eddies, and therefore $A_z$ ought to vary with distance from the
% wall. Karman assumed that it had the particular functional form $A_z =
% \kappa z u^*$, where $\kappa$ is a dimensionless constant. With this
% assumption, the equation for the mean velocity profile becomes
% \begin{equation}
% \kappa z u^* \frac{\partial U}{\partial z} = u^{*2}
% \end{equation}

Так как $U$~--- функция только одной переменной~$z$, мы можем 
записать $dU = u^*/(\kappa z) \, dz$, что имеет решение
\begin{equation}
U = \frac{u^*}{\kappa} \ln \left(\frac{z}{z_0}\right)
\end{equation}
где $z_0$ это расстояние от границы на которой скорость стремится к
нулю. Для воздушного потока над морем, $\kappa = 0.4$ и выражение для~$z_0$ по
Charnock's (1955) $z_0 = 0.0156 \, u^{*2}/g$. Профиль средней скорости в
атмосферном пограничном слое описанном в \S4.3 хорошо соответствует
логарифмическому (8.20), также как средяя скорость в первых метрах под
поверхностью моря. К тому же использование (4.1) в определении
скорости трения и использование (8.20), даёт Чарноковскую форму
коэффициента сопротивления как функцию скорости ветра на Рисунке 4.6.
%
% \hspace*{1 em}Because $U$ is a function only of $z$, we can write 
% $dU = u^*/(\kappa z) \, dz$, which has the solution
% \begin{equation}
% U = \frac{u^*}{\kappa} \ln \left(\frac{z}{z_0}\right)
% \end{equation}
% where $z_0$ is distance from the boundary at which velocity goes to
% zero.
%
% \hspace*{1 em}For airflow over the sea, $\kappa = 0.4$ and $z_o$ is
% given by Charnock's (1955) relation $z_0 = 0.0156 \, u^{*2}/g$. The
% mean velocity in the boundary layer just above the sea surface
% described in \S 4.3 fits well the logarithmic profile of (8.20), as
% does the mean velocity in the upper few meters of the sea just below
% the sea surface. Furthermore, using (4.2) in the definition of the
% friction velocity\index{friction velocity!and wind stress}, then using
% (8.20) gives Charnock's form of the drag
% coefficient\index{drag!coefficient} as a function of wind
% speed.\rule[-1ex]{0mm}{1ex}
%%\vspace{0.5ex}
%%\end{minipage}
%%}}
%%\vspace{-3ex}
%%\end{table}

Приближение по которому $A_z$ изменяется с расстоянием от границы хорошо
работает при описании потока над плоской поверхностью, где $U$~---
функция расстояния от поверхности~$z$, и W, средняя скорость
перпендикулярная плоскости равняется нулю. (Смотри блок Турбулентный
пограничный слой над плоской поверхностью). Это классическое
приближение, впервые описанное в 1925 году Прандтлем, ввёдшим
концепцию пограничного слоя, и другими. Здесь $A_z$ определяется с
помощью эмпирического подбора на основе данных собранных в
аэродинамических трубах или измерений в поверхностном пограничном слое
в море. Смотри Hinze (1975, \S5--2 and \S7--5) и Goldstein (1965:\S80) 
для более подробного теоретического описания турбулентного потока
у плоской поверхности.
%
% The assumption that an eddy viscosity $A_z$ can be used to relate the
% Reynolds stress to the mean flow works well in turbulent boundary
% layers. However $A_z$ cannot be obtained from theory. It must be
% calculated from data collected in wind tunnels or measured in the
% surface boundary layer at sea. See Hinze (1975, \S5--2 and\S7--5) and
% Goldstein (1965: \S80) for more on the theory of
% turbulence\index{turbulence!theory of} flow near a flat plate.

Приближение (8.13) и классическая теория хорошо работают только когда
трение гораздо больше силы Кориолиса. Это верно для потоков воздуха в
пределах нескольких десятков метров над поверхностью моря и для
потоков воды в пределах нескольких метров под его поверхностью. Не
ясна возможность применения этой техники к другим потокам в
океане. Например поток в перемешанном слое на глубинах ниже десяти
метров классической теорией турбулентности описывается хуже. Tennekes
and Lumley (1970: 57) писали:
%
% Prandtl's theory based on assumption (8.13) works well only where
% friction is much larger than the Coriolis force. This is true for air
% flow within tens of meters of the sea surface and for water flow
% within a few meters of the surface. The application of the technique
% to other flows in the ocean is less clear. For example, the flow in
% the mixed layer\index{mixed layer!theory} at depths below about ten
% meters is less well described by the classical turbulent
% theory. Tennekes and Lumley (1990: 57) write:
\begin{quotation}
Модели с длинной перемешивания и вихревой вязкостью должны
использоваться только для получения аналитических выражений для
напряжений Рейнольдаса и профилей средней скорости если они желетельны
для целей подгонки кривой в турбулентных потоках характеризующихся
единым масштабом длинны и единым масштабом скорости. Применения теории
длинны перемешивания в турбулентных потоках масштабы которых
неизвестны, необходимо избегать.
%
% Mixing-length and eddy viscosity models should be used only to
% generate analytical expressions for the Reynolds stress and
% mean-velocity profile if those are desired for curve fitting purposes
% in turbulent flows characterized by a single length scale and a single
% velocity scale. The use of mixing-length theory\index{mixing-length
% theory} in turbulent flows whose scaling laws are not known beforehand
% should be avoided.
\end{quotation}

Проблемы с приближением вихревой вязкости:
%
% Problems with the eddy-viscosity approach:
\begin{enumerate}
\item 
Кроме как в пограничных слоях толщиной в несколько метров,
геофизические потоки могут находится под влиянием нескольких
характерных масштабов. Например в пограничном слое атмосферы над морем
по крайней мере три масштаба могут быть важны:I) высота над уровнем
моря~$z$, II) масштаб Монина~--- Обухова~$L$, обсуждавшийся в \S4.3, и
III) средняя скорость~$U$ разделённая на параметр Кориолиса~$U/f$.
%
% \vitem Except in boundary layers a few meters thick, geophysical flows
% may be influenced by several characteristic scales. For example, in
% the atmospheric boundary layer above the sea, at least three scales
% may be important: i) the height above the sea $z$, ii) the
% Monin-Obukhov scale $L$ discussed in \S4.3, and iii) the typical
% velocity $U$ divided by the Coriolis parameter\index{Coriolis
% parameter} $U/f$.

\item
Скорости $u'$, $v’$, $w'$~--- являются свойствами жидкости, в то время
как $A_z$~--- свойство потока.
%
% \vitem The velocities $u',\,w'$ are a property of the \textit{fluid},
% while $A_z$ is a property of the \textit{flow};

\item
 Члены вихревой вязкости несимметричны:
\begin{align*}
\langle u'v' \rangle &= \langle v'u' \rangle\,;\quad \text{но} \\
A_x \frac{\partial{V}}{\partial{x}} &\neq A_y \frac{\partial{U}}{\partial{y}}
\end{align*}
%
% \vitem 
% Eddy viscosity terms are not symmetric:
% \begin{align}
% \langle u'v' \rangle &= \langle v'u' \rangle\,;\quad \text{but} \notag \\
% A_x \frac{\partial{V}}{\partial{x}} &\neq A_y \frac{\partial{U}}{\partial{y}}
% \notag
% \end{align}
\end{enumerate}
\end{paragraph}

\begin{paragraph}{Из статистической теории турбулентности}
% \paragraph{From a Statistical Theory of Turbulence}
Напряжение Рейнольдса может быть расчитано с помощью различных теорий
которые относят $\langle u'u' \rangle$ к корреляциям более высокого
порядка в форме $\langle u'u'u' \rangle$. Тогда встаёт проблема: как
посчитать члены более высокого порядка? Это \emph{closure problem} в
турбулентности. Не существует общего решения, но приближения ведут к
пониманию некоторых форм турбулентности, таких как изотропная
турбулентность на решётке в аэродинамической трубе (Batchelor
1967). \emph{Изотропная турбулентность}~--- это турбулентность чьи
статистические характеристики не зависят от нарпавления.
%
% The Reynolds stresses can be calculated from various theories which
% relate $\langle u'u' \rangle$ to higher order correlations of the form
% $\langle u'u'u' \rangle$. The problem then becomes: How to calculate
% the higher order terms? This is the \textit{closure
% problem}\index{closure problem|textbf}\index{turbulence!closure
% problem|textbf} in turbulence.  There is no general solution, but the
% approach leads to useful understanding of some forms of turbulence
% such as isotropic turbulence downstream of a grid in a wind tunnel
% (Batchelor 1967). \textit{Isotropic turbulence}\index{isotropic
% turbulence|textbf}\index{turbulence!isotropic|textbf} is turbulence
% with statistical properties that are independent of direction.

Приближение может быть изменено специальным образом для потока в
океане. В идеализированном случае сильного Рейнольдсовского течения мы
можем посчитать статистические характеристики потока в
термодинамическом равновесии. Так как реальный поток в океане далёк от
равновесия, мы предположим что он будет стремиться к
равновесию. Holloway(1986) пересмотрел это приближение, показав каким
образом его можно использовать для описания влияния турбулентности на
перемешивание и транспорт тепла.
%
% The approach can be modified somewhat for flow in the ocean. In the
% idealized case of high Reynolds flow, we can calculate the statistical
% properties of a flow in thermodynamic equilibrium. Because the actual
% flow in the ocean is far from equilibrium, we assume it will evolve
% towards equilibrium. Holloway (1986) provides a good review of this
% approach, showing how it can be used to derive the influence of
% turbulence\index{turbulent!mixing} on mixing and heat
% transports\index{transport!heat}. One interesting result of the work
% is that zonal mixing\index{mixing!zonal} ought to be larger than
% meridional mixing\index{mixing!meridional}.
\end{paragraph}

\begin{paragraph}{Резюме:}
% \paragraph{Summary}
Турбулентные вихревые вязкости $A_x$, $A_y$, и $A_z$ не могут быть точно
вычислены для большинства потоков в океане.
%
% The turbulent eddy viscosities $A_x$, $A_y$, and $A_z$ cannot be
% calculated accurately for most oceanic flows.

\begin{enumerate}
\item
Они могут быть оценены исходя из измерений турбулентных потоков. Тем
не менее измерения в океане сложны; а измерения в лабораториях
несмотря на всю их точность, не могут достигнуть чисел Рейнольдса в
$10^{11}$, типичных для океана.
%
% \vitem They can be estimated from measurements of turbulent
% flows. Measurements in the ocean, however, are difficult. Measurements
% in the lab, although accurate, cannot reach Reynolds numbers of
% $10^{11}$ typical of the ocean.

\item
Статистическая теория турбулентности даёт хорошее понимание роли
турбулентности в океане и в данный момент это область активных
исследований.
%
% \vitem 
% The statistical theory of turbulence\index{turbulence!theory of} gives
% useful insight into the role of turbulence in the ocean, and this is
% an area of active research.
\end{enumerate}

\begin{table}
\caption{Некоторые значения вязкости}
\begin{tabular}{rcl}
\hline
$\nu_{\text{water}}$               &$=$& $10^{-6}\sqmps$ \\
$\nu_{\text{tar at }\degCent{15}}$ &$=$& $10^6\sqmps$    \\
$\nu_{\text{glacier ice}}$         &$=$& $10^{10}\sqmps$ \\
$A_{y\text{ ocean}}$               &$=$& $10^4\sqmps$    \\ 
$A_{z\text{ ocean}}$               &$=$& $(10^{-5} - 10^{-3})\sqmps$  \\ 
\hline
\end{tabular}
\end{table}
%
%
% \begin{table}[h!]\centering \small
% \begin{tabular*}{62mm}{@{}rcl@{}}
% \multicolumn{3}{@{}l@{}}{\bfseries Table 8.1 Some \rule[-1ex]{0mm}{1ex}Values for Viscosity}
% \\
% \hline
% 
% $\nu_{water}$                       &$=$&\rule{0mm}{3ex}$10^{-6}$     m$^2$/s    \\
% $\nu_{tar\,at\,15^\circ{C}}$ &$=$&\rule{0mm}{3ex}$10^6$         m$^2$/s    \\
% $\nu_{glacier\,ice}$              &$=$&\rule{0mm}{3ex}$10^{10}$     m$^2$/s    \\
% $A_{y ocean}$                                     &$=$&\rule{0mm}{3ex}$10^4$         m$^2$/s    \\ 
% $A_{z ocean}$                      &$=$&\rule{0mm}{3ex}$(10^{-5} - 10^{-3})$    m$^2$/s    \\ 
% [0.5ex]
% \hline
% \end{tabular*} \\[0.5ex]
% \vspace{-3ex}
% \end{table}
\end{paragraph}
\end{section}

% \begin{section}{Mixing in the Ocean}
% % \section{Mixing in the Ocean}
% \index{mixing!oceanic} \index{mixing} Turbulence in the ocean leads to
% mixing. Because the ocean has stable stratification, vertical
% displacement must work against the buoyancy\index{buoyancy}
% force. Vertical mixing requires more energy than horizontal mixing. As
% a result, horizontal mixing along surfaces of constant density is much
% larger than vertical mixing across surfaces of constant density. The
% latter, however, usually called \textit{diapycnal
% mixing}\index{mixing!diapycnal|textbf}\index{diapycnal mixing|textbf},
% is very important because it changes the vertical structure of the
% ocean, and it controls to a large extent the rate at which deep water
% eventually reaches the surface in mid and low latitudes.
%
% The equations describing mixing depend on many processes. See Garrett
% (2006) for a good overview of the subject. Here I consider some simple
% flows. A simple equation for vertical mixing\index{mixing!vertical} by
% eddies of a tracer $\Theta$ such as salt or temperature is:
% \begin{equation}
% \frac{\partial \Theta}{\partial t} + W\,\frac{\partial \Theta}{\partial z} =
% \frac{\partial }{\partial z} \left( A_z \frac{\partial \Theta}{\partial z}
% \right) + S
% \end{equation}
% where $A_z$ is the vertical eddy diffusivity, $W$ is a mean vertical
% velocity, and $S$ is a source term.
%
% \begin{paragraph}{Average Vertical Mixing}
% % \paragraph{Average Vertical Mixing}
% \index{mixing!average vertical} Walter Munk (1966) used a very simple
% observation to calculate vertical mixing in the ocean. He observed
% that the ocean has a thermocline\index{thermocline} almost everywhere,
% and the deeper part of the thermocline\index{thermocline} does not
% change even over decades (figure 8.4). This was a remarkable
% observation because we expect downward mixing would continuously
% deepen the thermocline. But it doesn't. Therefore, a steady-state
% thermocline requires that the downward mixing of heat by
% turbulence\index{turbulent!mixing} must be balanced by an upward
% transport of heat\index{transport!heat upward} by a mean vertical
% current $W$. This follows from (8.21) for steady state with no sources
% or sinks of heat:
% \begin{equation}
% W \frac{\partial T}{\partial z} = A_z \frac{\partial^2 T}{\partial z^2}
% \end{equation}
% where $T$ is temperature as a function of depth in the thermocline.
%
% The equation has the solution:
% \begin{equation}
% T \approx T_0 \exp (z/H)
% \end{equation}
% where $H=A_z/W$ is the scale depth of the
% thermocline\index{thermocline}, and $T_0$ is the temperature near the
% top of the thermocline. Observations of the shape of the deep
% thermocline are indeed very close to a exponential function.  Munk
% used an exponential function fit through the observations of $T(z)$ to
% get $H$.
%
% \begin{figure}[t!]
% \makebox[121mm] [c]{\includegraphics{mixing}}
% \footnotesize
% Figure 8.4 Potential \rule{0mm}{3ex}temperature measured as a function
% of depth (pressure) near 24.7\degrees N, 161.4\degrees W in the
% central North Pacific by the \textit{Yaquina} in 1966 ($\bullet$), and
% by the \textit{Thompson} in 1985 $\left( _\square \;\right)$. Data
% from \textit{Atlas of Ocean Sections} produced by Swift, Rhines, and
% Schlitzer.
% \label{fig:mixing}
% \vspace{-3ex}
% \end{figure}
%
% Munk calculated $W$ from the observed vertical distribution of
% $^{14}$C, a radioactive isotope of carbon, to obtain a vertical time
% scale. In this case, $S=-1.24 \times 10^{-4}$ years$^{-1}$. The length
% and time scales gave $W=1.2$ cm/day and
% \begin{equation}
% \left< A_z \right> = 1.3 \times 10^{-4} \text{ m$^2$/s} \qquad \text{Average
% Vertical Eddy Diffusivity}
% \end{equation}
% where the brackets denote average eddy diffusivity in the
% thermocline\index{thermocline!eddy diffusivity in}.
%
% Munk also used $W$ to calculate the average vertical flux of water
% through the thermocline in the Pacific, and the flux agreed well with
% the rate of formation of bottom water assuming that bottom water
% upwells almost everywhere at a constant rate in the Pacific. Globally,
% his theory requires upward mixing of 25 to 30 Sverdrups of water,
% where one Sverdrup is $10^6$ cubic meters per second.
% \end{paragraph}
%
% \begin{paragraph}{Measured Vertical Mixing}
% % \paragraph{Measured Vertical Mixing}
% \index{mixing!measured vertical|(} 
% \index{mixing!vertical!measured}Direct observations of vertical mixing
% required the development of techniques for measuring: i) the fine
% structure of turbulence\index{turbulent!fine structure}, including
% probes able to measure temperature and salinity with a spatial
% resolution of a few centimeters (Gregg 1991), and ii) the distribution
% of tracers such as sulphur hexafluoride (SF$_6$) which can be easily
% detected at concentrations as small as one gram in a cubic kilometer
% of seawater.
%
% Direct measurements of open-ocean
% turbulence\index{turbulence!measurement of} and the diffusion of
% SF$_6$ yield an eddy diffusivity:
% \begin{equation}
% A_z \approx 1 \times 10^{-5} \text{ m$^2$/s} \qquad \text{Open-Ocean Vertical
% Eddy Diffusivity}
% \end{equation}
% For example, Ledwell, Watson, and Law (1998) injected 139 kg of SF$_6$
% in the Atlantic near 26\degrees N, 29\degrees W 1200 km west of the
% Canary Islands at a depth of 310 m. They then measured the
% concentration for five months as it mixed over hundreds of kilometers
% to obtain a diapycnal eddy
% diffusivity\index{mixing!diapycnal}\index{diapycnal mixing} of $A_z =
% 1.2 \pm 0.2 \times 10^{-5}$ m$^2$/s.
%
% The large discrepancy between Munk's calculation of the average eddy
% diffusivity for vertical mixing and the small values observed in the
% open ocean has been resolved by recent studies that show:
% \begin{equation}
% A_z \approx 10^{-3} \to 10^{-1} \text{ m$^2$/s} \qquad \text{Local Vertical Eddy Diffusivity}
% \end{equation}
%
% Polzin et al. (1997) measured the vertical structure of temperature in
% the Brazil Basin in the south Atlantic. They found $A_z > 10^{-3}$
% m$^2$/s close to the bottom when the water flowed over the western
% flank of the mid-Atlantic ridge at the eastern edge of the
% basin. Kunze and Toole (1997) calculated enhanced eddy diffusivity as
% large as $A= 10^{-3}$ m$^2$/s above Fieberling Guyot in the Northwest
% Pacific and smaller diffusivity along the flank of the seamount. And,
% Garbato et al (2004) calculated even stronger mixing in the Scotia Sea
% where the Antarctic Circumpolar Current flows between Antarctica and
% South America.
%
% The results of these and other experiments show that mixing occurs
% mostly by breaking internal waves and by shear at oceanic boundaries:
% along continental slopes, above seamounts and mid-ocean ridges, at
% fronts, and in the mixed layer\index{mixed layer!mixing in} at the sea
% surface. To a large extent, the mixing is driven by deep-ocean tidal
% currents\index{mixing!tidal}\index{mixing!of deep waters}, which
% become turbulent when they flow past obstacles on the sea floor,
% including seamounts and mid-ocean ridges (Jayne et al, 2004).
%
% Because water is mixed along boundaries or in other regions
% (Gnadadesikan, 1999), we must take care in interpreting temperature
% profiles such as that in figure 8.4. For example, water at 1200 m in
% the central north Atlantic could move horizontally to the Gulf
% Stream\index{Gulf Stream!and mixing}, where it mixes with water from
% 1000 m. The mixed water may then move horizontally back into the
% central north Atlantic at a depth of 1100 m. Thus parcels of water at
% 1200 m and at 1100 m at some location may reach their position along
% entirely different paths.
% \end{paragraph}
%
% \begin{paragraph}{Measured Horizontal Mixing}
% % \paragraph{Measured Horizontal Mixing}
% \index{mixing!average horizontal|(} Eddies mix fluid in the
% horizontal, and large eddies mix more fluid than small eddies. Eddies
% range in size from a few meters due to
% turbulence\index{turbulent!mixing} in the
% thermocline\index{thermocline} up to several hundred kilometers for
% geostrophic\index{geostrophic currents!eddies} eddies discussed in
% Chapter 10.
%
% In general, mixing depends on Reynolds number $R$ (Tennekes 1990:
% p. 11)
% \begin{equation}
% \frac{A}{\gamma} \approx \frac{A}{\nu} \sim \frac{UL}{\nu} = R
% \end{equation}
% where $\gamma$ is the molecular diffusivity of heat, $U$ is a typical
% velocity in an eddy, and $L$ is the typical size of an
% eddy. Furthermore, horizontal eddy diffusivity are ten thousand to ten
% million times larger than the average vertical eddy diffusivity.
%
% Equation (8.27) implies $A_x\sim UL$. This functional form agrees well
% with Joseph and Sender's (1958) analysis, as reported in (Bowden 1962)
% of spreading of radioactive tracers, optical turbidity, and
% Mediterranean Sea water in the north Atlantic. They report
% \begin{gather}
% A_x = P L \\
% 10 \text{ km} < L < 1500 \text{ km} \notag \\
% P = 0.01 \pm 0.005 \text{ m/s} \notag
% \end{gather}
% where $L$ is the distance from the source, and $P$ is a constant.
%
% The horizontal eddy diffusivity (8.28) also agrees well with more
% recent reports of horizontal diffusivity. Work by Holloway (1986) who
% used satellite altimeter observations of geostrophic
% currents\index{geostrophic currents!altimeter observations of},
% Freeland who tracked \textsc{sofar} underwater floats, and Ledwell
% Watson, and Law (1998) who used observations of currents and tracers
% to find
% \begin{equation}
% A_x \approx 8 \times 10^2 \text{ m$^2$/s} \qquad \text{Geostrophic Horizontal
% Eddy Diffusivity}
% \end{equation}
% Using (8.28) and the measured $A_x$ implies eddies with typical scales
% of 80 km, a value near the size of geostrophic
% eddies\index{geostrophic currents!eddies} responsible for the mixing.
%
% Ledwell, Watson, and Law (1998) also measured a horizontal eddy
% diffusivity. They found
% \begin{equation}
% A_x \approx 1 \text{ -- } 3 \text{ m$^2$/s} \qquad \text{Open-Ocean Horizontal
% Eddy Diffusivity}
% \end{equation}
% over scales of meters due to turbulence\index{turbulent!mixing} in the
% thermocline\index{thermocline!mixing in} probably driven by breaking
% internal waves. This value, when used in (8.28) implies typical
% lengths of 100 m for the small eddies responsible for mixing in this
% experiment.\index{mixing!average horizontal|)}
% \end{paragraph}
%
% \begin{paragraph}{Comments on horizontal mixing}
% % \paragraph{Comments on horizontal mixing}
%
% \begin{enumerate}
% \vitem Horizontal eddy diffusivity is $10^5 - 10^8$ times larger than
% vertical eddy diffusivity.
%
% \vitem \index{mixing!horizontal}Water in the interior of the ocean
% moves along sloping surfaces of constant density with little local
% mixing until it reaches a boundary where it is mixed vertically. The
% mixed water then moves back into the open ocean again along surfaces
% of constant density (Gregg 1987).
%
% One particular case is particularly noteworthy. When water, mixing
% downward through the base of the mixed layer,\index{mixed layer!mixing
% through base of} flows out into the thermocline along surfaces of
% constant density, the mixing leads to the \textit{ventilated
% thermocline}\index{thermocline!ventilated|textbf} model of oceanic
% density distributions.
%
% \vitem Observations of mixing in the ocean imply that numerical models
% of the oceanic circulation should use mixing schemes that have
% different eddy diffusivity parallel and perpendicular to surfaces of
% constant density, not parallel and perpendicular to level
% surfaces\index{level surface} of constant $z$ as I used
% above. Horizontal mixing along surfaces of constant $z$ leads to
% mixing across layers of constant density because layers of constant
% density are inclined to the horizontal by about $10^{-3}$ radians (see
% \S10.7 and figure 10.13).
%
% Studies by Danabasoglu, McWilliams, and Gent (1994) show that
% numerical models using isopycnal and diapycnal
% mixing\index{mixing!diapycnal}\index{diapycnal mixing} leads to much
% more realistic simulations of the oceanic circulation.
%
% \vitem Mixing is horizontal and two dimensional for horizontal scales
% greater than $NH/(2f)$ where $H$ is the water depth, $N$ is the
% stability frequency\index{stability!frequency} (8.36), and $f$ is the
% Coriolis parameter (Dritschel, Juarez, and Ambaum (1999).
% \end{enumerate}
% \end{paragraph}
% \vspace{-2ex}
% \end{section}

\begin{section}{Устойчивость}
% \section{Stability}
В предидущем параграфе мы говорили о том что поток жидкости с большими
числами Рейнольдса турбулентен. Это одна из форм неустойчивости. В
океане существует много других форм неустойчивости. Здесь мы обсудим
три самые важные I)статическую устойчивость – связанную с
изменением плотности с глубиной,II)динамическую устойчивость связанную
со сдвигом скорости, и III) двойную диффузию, связанную с градиентами
солёности и температуры.
%
% We saw in section 8.2 that fluid flow with a sufficiently large
% Reynolds number is turbulent. This is one form of instability. Many
% other types of instability occur in the in the ocean. Here, let's
% consider three of the more important ones: i) \textit{static
% stability}\index{stability!static|textbf} associated with change of
% density with depth, ii) \textit{dynamic
% stability}\index{stability!dynamic|textbf} associated with velocity
% shear, and iii) \textit{double-diffusion}\index{double diffusion}
% associated with salinity and temperature gradients in the ocean.

\begin{paragraph}{Статическая устойчивость и частота устойчивости}
% \paragraph{Static Stability and the Stability Frequency} 
Сначала рассмотрим статическую устойчивость. Если более плотная вода
находится над менее плотной, то жидкость неустойчива. Более плотная
вода будет опускаться под менее плотную. И наоборот, если менее
плотная вода находится над более плотной, граница раздела между ними
устойчива. Но насколько устойчива? Можно предположить что чем больше
контраст плотности вдоль поверхности раздела, тем она устойчивей. Это
пример статической устойчивости. Статическая устойчивость важна в
любом стратифицированном потоке, где плотность увеличивается с
глубиной, и нам необходим критерий для оценки важности (величины)
устойчивости.
%
% Consider first static stability. If more dense water lies above less
% dense water, the fluid is unstable. The more dense water will sink
% beneath the less dense. Conversely, if less dense water lies above
% more dense water, the interface between the two is stable.  But how
% stable? We might guess that the larger the density contrast across the
% interface, the more stable the interface. This is an example of static
% stability.  Static stability is important in any \textit{stratified}
% flow where density increases with depth, and we need some criterion
% for determining the importance of the stability.

\begin{figure}[b!]
\makebox [120mm][c]{\includegraphics{pics/stabilitysketch}}
\caption{Рисунок для расчёта статической устойчивости и частоты
стратификации.}
\label{fig:stabilitysketch}
%\vspace{-2ex}
\end{figure}
%
% \begin{figure}[b!]
% \vspace{1ex}
% \makebox[120mm] [c]{\includegraphics{stabilitysketch}}
% \centering
% \footnotesize
% Figure 8.5 Sketch for \rule{0mm}{4ex}calculating static stability and
% stability frequency\index{stability!frequency!sketch of}.
%
% \label{fig:stabilitysketch}
% %\vspace{-2ex}
% \end{figure}

\textbf{Статическая Устойчивость и Частота Стратификации}
Рассмотрим частицу воды вертикально перемещённую в стратифицированной
жидкости (Рисунок 8.4). Сила плавучести F действующая на перемещённую
частицу равна разности между её массой $V g \rho '$ и массой окружающей воды
$V g \rho_2$, где $V$~--- объём частицы:
\begin{displaymath}
F=V\,g\,(\rho_2-\rho{'})
\end{displaymath}
Ускорение перемещённой частицы составит:
\begin{equation}
a=\frac{F}{m}=\frac{g\,(\rho_2-\rho{'})}{\rho{'}}
\end{equation}
но
\begin{align}
\rho_2  &= \rho + \left( \frac{d {\rho}}{d {z}}\right)_{water}
\delta z \\
\rho{'} &= \rho + \left( \frac{d {\rho}}{d {z}}\right)_{parcel}
\delta z
\end{align}
Подставляя (8.21) и (8.22) в (8.20), и игнорируя члены
пропорциональные $\delta{z^2}$, получим:
\begin{equation}
E = -\frac{1}{\rho}\,\Biggl[\left(\frac{d \rho}{d {z}}\right)_{water}
- \,\left(\frac{d \rho}{d {z}}\right)_{parcel}\Biggr]
\end{equation}
где $E \equiv -a/(g \, \delta z)$ устойчивость столба воды. Это может
быть записанов в терминах измеренной температуры и солёности t(z),
S(z) (Mc Dougall?, 1987; Sverdrup, Johnson, and Fleming, 1942: 416; or
Gill, 1982: 50):
\begin{equation}
E = \alpha\left(\frac{dt}{dz} - g\rho\Gamma\right) - \beta\frac{dS}{dz},
\end{equation}
где
\begin{equation}
\alpha = -\frac{1}{\rho}\left.\frac{\partial \rho}{\partial t}\right|_{S,p},\qquad
\beta  = -\frac{1}{\rho}\left.\frac{\partial \rho}{\partial S}\right|_{t,p},\qquad
\Gamma = \left. \frac{\partial t}{\partial p}\right|_{\text{adiabatic}}
\end{equation}
and where $\alpha$ is the thermal expansion coefficient, $\beta$ is
the saline contraction coefficient, and $\Gamma$ is the adiabatic
lapse rate, the change of temperature with pressure as the water
parcel moves without exchanging heat with it's surroundings. $p$ is
pressure, $t$ is temperature in celsius, $\rho$ is density, and $S$ is
salinity.
%
% Consider a parcel of water that is displaced vertically and
% adiabatically in a stratified fluid (figure 8.5). The
% buoyancy\index{buoyancy} force $F$ acting on the displaced parcel is
% the difference between its weight $V g \rho '$ and the weight of the
% surrounding water $V g \rho_2$, where $V$ is the volume of the parcel:
% \begin{displaymath}
% F=V\,g\,(\rho_2-\rho{'})
% \end{displaymath}
% The acceleration of the displaced parcel is:
% \begin{equation}
% a=\frac{F}{m}=\frac{g\,(\rho_2-\rho{'})}{\rho{'}}
% \end{equation}
% but
% \begin{align}
% \rho_2  &= \rho + \left( \frac{d {\rho}}{d {z}}\right)_{water}
% \delta z \\
% \rho{'} &= \rho + \left( \frac{d {\rho}}{d {z}}\right)_{parcel}
% \delta z
% \end{align}
% Using (8.32) and (8.33) in (8.31), ignoring terms proportional to
% $\delta{z^2}$, we obtain:
% \begin{equation}
% E = -\frac{1}{\rho}\,\Biggl[\left(\frac{d \rho}{d {z}}\right)_{water}
% - \,\left(\frac{d \rho}{d {z}}\right)_{parcel}\Biggr]
% \end{equation}
% where $E \equiv -a/(g \, \delta z)$ is the
% \textit{stability}\index{stability|textbf} of the water column
% (McDougall, 1987; Sverdrup, Johnson, and Fleming, 1942: 416; or Gill,
% 1982: 50).

В верхнем километре океана устойчивость большая, и первый член в
(8.23) гораздо больше чем второй. Первый член пропорционален изменению
плотности в столбе жидкости; второй пропорционален сжимаемости морской
воды, которая очень мала. Пренебрегая вторым членом, мы можем записать
уравнение устойчивости (равновесия).
\begin{equation}
\boxed{E \approx -\frac{1}{\rho}\,\frac{d{\rho}}{d{z}} }
\end{equation}
Приближение использованное нами для того чтобы вывести уравение (8.26)
правомерно для $E> 50\times 10^{-8}$/м.
%
% In the upper kilometer of the ocean stability is large, and the first
% term in (8.34) is much larger than the second. The first term is
% proportional to the rate of change of density of the water column. The
% second term is proportional to the compressibility of sea water, which
% is very small. Neglecting the second term, we can write the
% \textit{stability equation}\index{stability!equation|textbf}:
% \begin{equation}
% \boxed{E \approx -\frac{1}{\rho}\,\frac{d{\rho}}{d{z}} }
% \end{equation}
% The approximation used to derive (8.35) is valid for $E > 50 \times
% 10^{-8}$/m.


Глубже в океане изменение плотности настолько малы что мы должны
рассматривать небольшие изменения плотности частицы воды вызванные
изменеиями в давлении при её вертикальном перемещении и использовать
для расчётов формулу (8.24).
%
% Below about a kilometer in the ocean, the change in density with depth
% is so small that we must consider the small change in density of the
% parcel due to changes in pressure as it is moved vertically.

Устойчивость определяется исходя из условия:
\begin{align*}
E>0 & \quad \text{Stable} \\
E=0 & \quad \text{Neutral Stability} \\
E<0 & \quad \text{Unstable}
\end{align*}
В верхнем километре океана, $z<1\,000\m$, $E=(100\text{--}1000)*10^{-8}$/м, а в
глубоководных желобах $z > 7\,000\m$, $E=1*10^{-8}$/м.
%% !!! числа разные
%
% Stability is defined such that
% \begin{align}
% E>0 & \quad \text{Stable} \notag \\
% E=0 & \quad \text{Neutral Stability} \notag \\
% E<0 & \quad \text{Unstable} \notag
% \end{align}
% In the upper kilometer of the ocean, $z < 1,000$ m, $E = (50$---$1000)
% \times 10^{-8}$/m, and in deep trenches where $z > 7,000$ m, $E = 1
% \times 10^{-8}$/m.

\begin{figure}[t!]
\makebox [120mm][c]{\includegraphics{pics/stabilityfreq}} 
\caption{Частота стратификации измеренная в Тихом океане Слева:
Устойчивость глубокого термоклина на востоке Куросио. Справа:
Устойчивость неглубокого термоклина характерного для тропиков. Имейте
в виду разницу масштабов.}  
\label{fig:stabilityfreq}
\end{figure}
%
% \begin{figure}[t!]
% %\vspace{-1ex}
% \makebox[120mm] [c]{\includegraphics{stabilityfreq}} 
% \footnotesize
% Figure 8.6. Observed \rule{0pt}{3ex} stability
% frequency\index{stability!frequency!in Pacific} in the Pacific.
% \textbf{Left:} Stability of the deep
% thermocline\index{thermocline!stability of} east of the
% Kuroshio\index{Kuroshio!thermocline}.  \textbf{Right:} Stability of a
% shallow thermocline typical of the tropics. Note the change of scales.
% \label{fig:stabilityfreq}
% \vspace{-3ex}
% \end{figure}

Влияние устойчивости обычно выражается через частоту устойчивости $N$:
\begin{equation}
N^2 \equiv -g E
\end{equation}
Частоту устойчивости часто называют Частота Вяйселя Брантля~--- или
частотой стратификации. Частота выражает величину устойчивости, и
является фундоментальной переменной в динамике стратифицированной
жидкости. В простейшем виде, она может быть интерпретирована как
вертикальная частота (колебаний) вызванная вертикальным перемещением
частицы жидкости. Таким образом это максимальная частота внутренних
волн в океане. Обычные значения N составляют несколько периодов в час
(рис 8.5).
%
% The influence of stability is usually expressed by a
% \textit{stability frequency}\index{stability!frequency|textbf}
% $N$:
% \begin{equation}
% N^2 \equiv -g E
% \end{equation}
% The stability frequency\index{stability!frequency} is often called the
% \textit{buoyancy frequency}\index{buoyancy!frequency|textbf} or the
% \textit{Brunt-Vaisala frequency}\index{Brunt-Vaisala
% frequency|textbf}. The frequency quantifies the importance of
% stability, and it is a fundamental variable in the dynamics of
% stratified flow. In simplest terms, the frequency can be interpreted
% as the vertical frequency excited by a vertical displacement of a
% fluid parcel. Thus, it is the maximum frequency of internal waves in
% the ocean. Typical values of $N$ are a few cycles per hour (figure 8.6).
\end{paragraph}

\begin{paragraph}{Динамическая Устойчивость и Число Ричардсона}
% \paragraph{Dynamic Stability and Richardson's Number}
Если скорость изменяется с глубиной в устойчивом стратифицированном
потоке, тогда поток может стать неустойчивым если изменеие скорости с
глубиной, сдвиг скорости, достаточно большой. Самый простой пример это
ветер дующий над океаном. В этом случае устойчивость вдоль поверхности
моря очень большая. We might say it is infinite because there is a
step discontinuity in r, and (8.27) is infinite. Кроме того, ветер
дующий над океаном вызывает волны, и если он достаточно сильный,
поверхность становится нестабильной и волны обрушаются.
%
% If velocity changes with depth in a stable, stratified flow, then the
% flow may become unstable if the change in velocity with depth, the
% \textit{current shear}\index{current shear|textbf}, is large
% enough. The simplest example is wind blowing over the ocean. In this
% case, stability is very large across the sea surface. We might say it
% is infinite because there is a step discontinuity in $\rho$, and
% (8.36) is infinite. Yet, wind blowing on the ocean creates waves, and
% if the wind is strong enough, the surface becomes unstable and the
% waves break.

Это пример \emph{динамической неустойчивости} при которой устойчивая
жидкость становится неустойчивой благодаря сдвигу скорости. Другой
пример динамической неустойчивости это это неустойчивость
Кельвина-Гельмгольца, наблюдающаяся когда контраст в жидкости со
сдвигом скорости гораздо меньше чем на поверхности моря, как например
в термоклине или на вершине стабильного атмосферного пограничного слоя
(Рисунок 8.6)
%
% This is an example of \textit{dynamic
% instability}\index{instability!dynamic|textbf}\index{dynamic
% instability|textbf} in which a stable fluid is made unstable by
% velocity shear.  Another example of dynamic instability, the
% Kelvin-Helmholtz instability, occurs when the density contrast in a
% sheared flow is much less than at the sea surface, such as in the
% thermocline\index{thermocline} or at the top of a stable, atmospheric
% boundary layer (figure 8.7).

\begin{figure}[t!]
\makebox [120mm][c]{\includegraphics{pics/helmholtz}} 
\caption{Волнистые облака демонстрирующие неустойчивость
Кельвина-Гельмгольца на вершине устойчивого пограничного атмосферного
слоя.(взято из NOAA Forecast Systems Laboratory.). Note that the
billows become large enough that more dense air overlies less dense
air, and the billows collapse into turbulence.}
\label{fig:helmholtz}
\end{figure}
%
% \begin{figure}[t!]
% \makebox[120mm] [c]{\includegraphics{helmholtz}} 
% \footnotesize Figure 8.7 Billow clouds showing a Kelvin-Helmholtz
% \rule{0mm}{4ex}instability at the top of a stable atmospheric
% layer. Some billows can become large enough that more dense air
% overlies less dense air, and then the billows collapse into
% turbulence\index{turbulent!mixing}. Photography copyright Brooks
% Martner, \textsc{noaa} Environmental Technology Laboratory.
% \label{fig:helmholtz}
% \vspace{-3ex}
% \end{figure}

Относительная важность статической устойчивости и динамической
неустойчивости выражается Число Ричардсона.
\begin{equation}
\boxed{R_i\equiv\frac{g\,E}{(\partial{U}/\partial{z})^2} }
\end{equation}
где в числителе стоит величина статической устойчивости, а в
знаменателе величина сдвига скорости.
\begin{align*}
R_i &>0.25 \quad \text{Stable} \\
R_i &<0.25 \quad \text{Velocity Shear Enhances Turbulence} 
\end{align*}
%
% The relative importance of static stability and dynamic instability is
% expressed by the \textit{Richardson Number}\index{Richardson
% Number|textbf}:
% \begin{equation}
% \boxed{R_i\equiv\frac{g\,E}{(\partial{U}/\partial{z})^2} }
% \end{equation}
% where the numerator is the strength of the static stability, and the
% denominator is the strength of the velocity shear.
% \begin{align}
% R_i &>0.25 \quad \text{Stable} \notag \\
% R_i &<0.25 \quad \text{Velocity Shear Enhances Turbulence} \notag
% \end{align}

Заметте что число Ричардсона не единственный критерий
неустойчивости. Для появления турбулентности число Рейнольдса должно
быть большим, а число Ричардсона меньше 0,25. эти признаки встречаются
в некоторых океанских течениях. Турбулентность перемешивает жидкость
по вертикали приводя к вихревой вязкости и вихревой диффузии. Так как
океан в основном сильно стратифицирован а течения в нём слабые,
турбулентное перемешивание прерывистое и нечастое событие. Измерения
плотности как функции глубины редко демонстрирует наличие более
плотной среды над над менее плотной как происходит в обрушающихся
волнах (Рисунок 8,6) (Moum and Caldwell 1985).
%
% Note that a small Richardson number is not the only criterion for
% instability.  The Reynolds number must be large and the Richardson
% number must be less than 0.25 for turbulence. These criteria are met
% in some oceanic flows. The turbulence mixes fluid in the vertical,
% leading to a vertical eddy viscosity and eddy diffusivity. Because the
% ocean tends to be strongly stratified and currents tend to be weak,
% turbulent mixing is intermittent and rare. Measurements of density as
% a function of depth rarely show more dense fluid over less dense fluid
% as seen in the breaking waves in figure 8.7 (Moum and Caldwell 1985).
\end{paragraph}

\begin{paragraph}{Двойная Диффузия и Солёностные Пальцы}
% \paragraph{Double Diffusion and Salt Fingers}
В некоторых районах океана менее плотная вода находится над более
плотной, однако столб воды неустойчив даже если течения
отсутствуют. Неустойчивость вызывается из за того что молекулярная
диффузия тепла происходит в 100 раз быстрее молекулярной диффузии
соли. Эта неустойчивость была впервые открыта Мелвином Штерном (Meivin
Stern) в 1960, который сразу понял её значение для океанографии.
%
% \index{double diffusion!salt fingers}In some regions of the ocean,
% less dense water overlies more dense water, yet the water column is
% unstable even if there are no currents. The instability occurs because
% the molecular diffusion of heat is about 100 times faster than the
% molecular diffusion of salt. The instability was first discovered by
% Melvin Stern in 1960 who quickly realized its importance in
% oceanography.

\begin{figure}[h!]
\makebox [120mm][c]{\includegraphics{pics/saltfingers}} 
\caption{Слева: Начальное распределение плотности по
вертикали. Справа: Через некоторое время, диффузия тепла приводит к
образованию тонкого неустойчивого слоя между двумя
изначальноустойчивыми слоями. Тонкий неустойчивый слой погружается в
нижний слой в виде солёностных пальцев. Вертикальный масштаб пальцев
составляет несколько сантиметров.}
\label{fig:saltfingers}
\vspace{-2ex}
\end{figure}
%
% \begin{figure}[h!]
% \makebox[120mm] [c]{\includegraphics{saltfingers}} \footnotesize
% Figure 8.8 \textbf{Left:} Initial \rule{0mm}{4ex}distribution of
% density in the vertical. \textbf{Right:} After some time, the
% diffusion of heat leads to a thin unstable layer between the two
% initially stable layers. The thin unstable layer sinks into the lower
% layer as salty fingers. The vertical scale in the figures is a few
% centimeters.
% \label{fig:saltfingers}
% \vspace{-2ex}
% \end{figure}

Рассмотрим два слоя толщиной в несколько метров каждый разделённых
чёткой границей (Рисунок 8.7). Если верхний слой более тёплый и
солёный, а нижний холоднее и менее солёный чем верхний, поверхность
раздела между ними становится неустойчивой даже если верхний слой
менее плотный чем нижний.
%
% Consider two thin layers a few meters thick separated by a sharp
% interface (figure 8.8). If the upper layer is warm and salty, and if
% the lower is colder and less salty than the upper layer, the interface
% becomes unstable even if the upper layer is less dense than the lower.

Что здесь происходит. Тепло диффундирует через границу быстрее чем
соль, приводя к образованию тонкого холодного и солёного слоя между
двумя первоначальными слоями. Холодный солёный слой более плотный чем
холодный менее солёный слой под ним и более плотная вода начинает
погружаться в менее плотную. Так как слой тонкий, жидкость погружается
в виде пальцев диаметром в сантиметр или около того, не слишком
отличающихся по размерам и форме от наших. Это Солёностные
Пальцы. Так как через границу раздела диффундируют два компонента,
процесс называется Двойная Диффузия.
%
% Here's what happens. Heat diffuses across the interface faster than
% salt, leading to a thin, cold, salty layer between the two initial
% layers. The cold salty layer is more dense than the cold, less-salty
% layer below, and the water in the layer sinks. Because the layer is
% thin, the fluid sinks in fingers 1--5 cm in diameter and 10s of
% centimeters long, not much different in size and shape from our
% fingers. This is \textit{salt fingering}\index{salt
% fingering|textbf}. Because two constituents diffuse across the
% interface, the process is called \textit{double
% diffusion}\index{double diffusion|textbf}.

Существует три другие вариации на эту тему. Две переменных каждая из
которых взята по времени приводят к четырём возможным комбинациям.
%
% There are four variations on this theme. Two variables taken two at a
% time leads to four possible combinations:
\begin{enumerate}
\item
Тёплая солёная над холодной менее солёной: Ведёт к солёностным
пальцам. Встречается в центральных областях субтропических кругов
течений, на востоке тропической Северной Атлантики и Северо-Восточной
Атлантики ниже течения выходящего из Средиземного моря.
%
% \vitem \textit{Warm salty over colder less salty}. This process is
% called \textit{salt fingering}. It occurs in the thermocline below the
% surface waters of sub-tropical gyres and the western tropical north
% Atlantic, and in the North-east Atlantic beneath the outflow from the
% Mediterranean Sea. Salt fingering eventually leads to density
% increasing with depth in a series of steps. Layers of constant-density
% are separated by thin layers with large changes in density, and the
% profile of density as a function of depth looks like stair
% steps. Schmitt et al (1987) observed 5--30 m thick steps in the
% western, tropical north Atlantic that were coherent over 200--400 km
% and that lasted for at least eight months. Kerr (2002) reports a
% recent experiment by Raymond Schmitt, James Leswell, John Toole, and
% Kurt Polzin showed salt fingering off Barbados mixed water 10 times
% faster than turbulence\index{turbulent!mixing}.

\item
Холодная менее солёная над тёплой более солёной: Двойная диффузия
делает более чёткой границу раздела между двумя слоями приводя к
ступенчатому увеличению плотности с глубиной. Вот что происходит в
этом случае. Двойная диффузия приводит к образованию тонкого тёплого
менее солёного слоя под верхним холодным менее солёным слоем. Тонкий
слой воды увеличивается и перемешивается с водой из верхнего
слоя. Такой же процесс происходит и в нижнем слое где на границе
раздела формируется более холодный и солёный слой. В результате
конвекции в верхнем и нижнем слоях, граница раздела становится очень
чёткой и любые небольшие градиенты плотности в слоях
уменьшаются. Eventually step discontinuities of density are created
that are separated by layers of constant density. Этот процесс
называется Дифузивная Конвекция.
%
% \vitem \textit{Colder less salty over warm salty}.  This process is
% called \textit{diffusive convection}\index{diffusive
% convection|textbf}. It is much less common than salt fingering, and it
% us mostly found at high latitudes. Diffusive convection also leads to
% a stair step of density as a function of depth. Here's what happens in
% this case. Double diffusion leads to a thin, warm, less-salty layer at
% the base of the upper, colder, less-salty layer. The thin layer of
% water rises and mixes with water in the upper layer. A similar
% processes occurs in the lower layer where a colder, salty layer forms
% at the interface. As a result of the convection in the upper and lower
% layers, the interface is sharpened. Any small gradients of density in
% either layer are reduced. Neal et al (1969) observed 2--10 m thick
% layers in the sea beneath the Arctic ice.

\item
Холодная солёная над более тёплой менее солёной. Всегда статически
неустойчива.
%
% \vitem \textit{Cold salty over warmer less salty}. Always statically
% unstable.

\item
Более тёплая менее солёная, над холодной солёной. Всегда устойчиво и
двойная диффузия размывает границу раздела между двумя слоями.
%
% \vitem \textit{Warmer less salty over cold salty}. Always stable and
% double diffusion diffuses the interface between the two layers.
\end{enumerate}

Примеры двойной диффузии часто встречаются в океане. В главе 7 мы
посмотрели что тёплая солёная вода втекает в в Атлантику через
Гибралтар. Это вода более плотная чем боолее холодная но распреснённая
вода на поверхности Северной Атлантики. Воды средиземного моря
погружаются на глубину примерно километра, где они имеют такую же
плотность что и окружающие воды. Затем они распространяются в Северную
Атлантику как тёплая солёная интрузия между холодными менее солёными
водами. Двойная диффузия приводит к скачкам плотности вверху интрузии
и солёностным пальцам под интрузией.

Может показаться что образование солёностных пальцев не важно из за
того что они сами по себе слишком малы. Но это не так. в отличие от
турбулентного перемешивания, это процесс происходящий постоянно, и
суммарный эффекты за много лет велики. Образование солёностных пальцев
и Двойная Диффузия? возможно наиболее важные процессы вызывающие
вертикальное перемешивания на больших пространствах океана.

% Double diffusion mixes ocean water, and it cannot be
% ignored. Merryfield et al (1999), using a numerical model of the ocean
% circulation that included double diffusion, found that
% double-diffusive mixing changed the regional distributions of
% temperature and salinity although it had little influence on
% large-scale circulation of the ocean.
\end{paragraph}
\end{section}

\begin{section}{Перемешивание в океане}
Неустойчивость океана ведёт к перемешиванию. Так как океан обладает
устойчивой стратификацией и любое вертикальное перемещение должно
работать против сил плавучести, вертикальное перемешивание требует
гораздо больше энергии чем горизонтальное перемешивание. В результате
горизонтальное перемешивание вдоль поверхностей с постоянной
плотностью, гораздо больше чем вертикальное перемешивание вдоль
поверхностей с постоянной плотностью. Последнее, тем не менее, обычно
называемое диапикнальным перемешиванием, очень важно, так как оно
изменяет вертикальную структуру океана и в большой мере контролирует
уровень при котором глубинные воды в конце концов достигнут
поверхности в средних и низких широтах.

В океане турбулентное вихревое перемешивание гораздо важнее чем
молекулярная диффузия (Munk 1966). Уравнение для вертикального
вихревого перемешивания для параметра~$\Theta$ такого как солёность или
температура имеет вид:
\begin{equation}
\frac{\partial \Theta}{\partial t} + W \frac{\partial \Theta}{\partial z}
 = \frac{\partial}{\partial z} \left(K_z\frac{\partial \Theta}{\partial z}\right) + S
\end{equation}
Где $K_z$~--- это коэффициент вертикальной вихревой диффузии, $W$~---
средняя вертикальная скорость и $S$~--- source term.

\begin{figure}[t!]
\makebox [121mm][c]{\includegraphics{pics/mixing}}
\caption{Potential temperature measured as a function of depth
(pressure) near \latlon{24.7}{N}, \latlon{161.4}{W} in the central
North Pacific by the \textit{Yaquina} in 1966 ($\bullet$), and by the
\textit{Thompson} in 1985 $\left( _\square \;\right)$. Data from
\textit{Atlas of Ocean Sections} produced by Swift, Rhines, and
Schlitzer.}
\label{fig:mixing}
\end{figure}

\begin{paragraph}{Average Vertical Mixing}
Walter Munk (1966) used a very simple observation to calculate
vertical mixing in the ocean. He observed that the ocean has a
thermocline almost everywhere, and the deeper part of the thermocline
does not change even over decades (Figure 8.8). This was a remarkable
observation because we expect downward mixing would continuously
deepen the thermocline. But it doesn't. Therefore, a steady-state
thermocline requires that the downward mixing of heat by turbulence
must be balanced by an upward transport of heat by a mean vertical
current W. This follows from, (8.29) for steady state with no sources
or sinks:
\begin{equation}
W \frac{\partial T}{\partial z} = A_z \frac{\partial^2 T}{\partial z^2}
\end{equation}
where $T$ is temperature as a function of depth in the thermocline.

The equation has the solution:
\begin{equation}
T \approx T_0 \exp (z/H)
\end{equation}
where $H=A_z/W$ is the scale depth of the
thermocline\index{thermocline}, and $T_0$ is the temperature near the
top of the thermocline. Observations of the shape of the deep
thermocline are indeed very close to a exponential function.  Munk
used an exponential function fit through the observations of $T(z)$ to
get $H$.

Munk calculated $W$ from the observed vertical distribution of
$^{14}$C, a radioactive isotope of carbon, to obtain a vertical time
scale. In this case, $S=-1.24 \times 10^{-4}$ years$^{-1}$. The length
and time scales gave $W=1.2$ cm/day and
\begin{equation}
\left< A_z \right> = 1.3 \times 10^{-4} \text{ m$^2$/s} \qquad \text{Average
Vertical Eddy Diffusivity}
\end{equation}
where the brackets denote average eddy diffusivity in the
thermocline\index{thermocline!eddy diffusivity in}.

Munk also used $W$ to calculate the average vertical flux of water
through the thermocline in the Pacific, and the flux agreed well with
the rate of formation of bottom water assuming that bottom water
upwells almost everywhere at a constant rate in the Pacific. Globally,
his theory requires upward mixing of 25 to 30 Sverdrups of water,
where one Sverdrup is $10^6$ cubic meters per second.
\end{paragraph}

\begin{paragraph}{Измеренное вертикальное перемешивание.}
Прямые наблюдения за перемешиванием требуют разработки специальной
техники для измерения: i) тонкой структуры турбулентности, включая
зонды способные измерять температуру и солёность с пространственным
разрешением в несколько сантиметров (Greegg 1991) и i) распределение
трассеров таких как серный ??гексафторид (SF 6 ) в концентрациях
10--15 моль (Ledwell, Watson, and Law 1993).

Прямые измерения турбулентности в открытом океане и диффузии SF6
приведут к коэффициенту вихревой диффузии:
\begin{equation}
K_z \approx 1 \times 10^{-5} \text{ m$^2$/s} \qquad \text{Open-Ocean Vertical
Eddy Diffusivity}
\end{equation}

Например Ledwell, Watson, and Law (1991) выпустили 139 кг SF$_6$ в
Атлантику около \latlon{26}{N}, \latlon{29}, это 1200 км западнее
канарских островов, на глубине 310 метров. Затем в течении пяти
месяцев, пока трассер распространялся на сотни километров, они
проводили измерения его концентрации и получили диапикнальный
коэффициент вихревого перемешивания 
равный $K_z = 1.1 \pm 0.2 \times 10^{-5}\sqmps$.

These and other open-ocean experiments indicate that turbulent mixing
is driven by breaking internal waves and shear instability at
boundaries. Furthermore, mixing by turbulence seems to be more
important than double diffusion (Gregg 1987).

The large discrepancy between Munk's calculation of the the mean eddy
diffusivity for vertical mixing and the observed values in the open
ocean led to further experiments to resolve the difference. Two recent
experiments are especially interesting
\begin{equation}
K_z \approx 10^{-3}\sqmps \text{Rough Bottom Vertical Eddy Diffusivity}
\end{equation}
because they indicate that over seamounts and ridges

Polzin et al., (1997) measured the vertical structure of temperature
in the Brazil Basin in the South Atlantic. They found $K_z > 10^{-3}\sqmps$
close to the bottom when the water flowed over the western flank of
the mid-Atlantic ridge at the eastern edge of the basin. Kunze and
Toole (1997) calculated enhanced eddy diffusivity as large as 
$K = 10^{-3}\sqmps$ above Fieberling Guyot in the Northwest Pacific and
smaller diffusivity along the flank of the seamount.

The results of these and other experiments show that mixing occurs
mostly at oceanic boundaries: along continental slopes, above
seamounts and mid-ocean ridges, at fronts, and in the mixed layer at
the sea surface.

Still, the observed mixing in the open ocean away from boundaries is
too small to account for the mixing calculated by Munk. Recent work
reported at the World Ocean Circulation Experiment Conference on
Circulation and Climate 1998 and work by Munk and Wunsch (1998) and
Webb and Suginohara (2001) indicate that the dilemma may be resolved
several ways:
\begin{enumerate}
\item
First, separate studies by Gargett, Salmon, and Marotzke show that we
must separate the concept of deep convection from that of the
meridional overturning circulation (see chapter 13). Deep convection
may mix properties not mass. The mass of upwelled water required by
Munk may be overestimated, and the vertical mixing needed to balance
the upwelling may be smaller than he calculated. Webb and Suginohara
(2001) note that the upwelled water may be as small as 8Sv.

\item 
Second, mixing probably takes place along boundaries or in the source
regions for thermocline waters (Gnadadesikan, 1999). For example,
water at 1200m in the central North Atlantic could move horizontally
to the Gulf Stream, where it mixes with water from 1000m. The mixed
water may then move horizontally back into the central North Atlantic
at a depth of 1100m. Thus water at 1200m and at 1100m may reach their
position along entirely different paths.
\end{enumerate}
\end{paragraph}


\begin{paragraph}{Измерение горизонтального перемешивания.}
Вихри перемешивают жидкость по горизонтали и большие вихри
перемешивают больше воды чем маленькие. Вихри варьируются по размеру
от нескольких метров (вызванные турбулентностью в термоклине) до
нескольких сотен километров (геострофические вихри описываемые в главе
10).

В основном перемешивание зависит от числа рейнольдса R (Tennekes 1990:
p. 11)
\begin{equation}
\frac{K}{\gamma} \approx \frac{K}{\nu} \sim \frac{UL}{\nu} = R
\end{equation}
Где $\gamma$~--- это коэффициент молекулярной диффузии тепла. Кроме
того, горизонтальный коэффициент вихревой диффузии больше среднего
вертикального коэффициента вихревой диффузии в тысячи а иногда и в
миллионы раз.

Уравнение (8.35) предпологает $K_x \sim UL$. Эта функциональная форма
хорошо соотносится с анализом спрединга радиоактивных трассеров,
оптической плотности и вод Средиземного моря в Северной Атлантике
проведённым Джозефом и Сендером (Joseph and Sender's (1958)). Они
сообщают
\begin{gather}
K_x = P L \\
10 \text{ km} < L < 1500 \text{ km} \notag \\
P = 0.01 \pm 0.005 \text{ m/s} \notag
\end{gather}
Где $L$~--- расстояние от источника, а $U$~--- константа.

Горизонтальный коэффициент вихревой диффузии (8.35) также хорошо
согласуется с более современными данными по горизонтальной
диффузии. Работы Холловэя (Holloway (1986)), который использовал
наблюдения спутниковых альтиметров за геострофическими течениями,
Фриланда и др. (Freeland et al. (1975)) котрый проследил подводные
течения SOFAR, и Маквильямса (Mc Williams? (1976)) и Ledwell et al.,
(1998) которые использовали наблюдения за течениями и трассерами,
показывают что
\begin{equation}
K_x \approx 8 \times 10^2 \sqmps \text{ Geostrophic Horizontal Eddy Diffusivity}
\end{equation}

Используя (8.36) и измеренный $K_x$ получаем вихри с характерным
масштабом 80 км, приблизительным размером геострофических вихрей
ответственных за перемешивание.

Ledwell, Watson, and Law (1991) также измеряли горизонтальный
коэффициент турбулентной вихревой диффузии. Они получили:
\begin{equation}
K_x \approx 1 \text{--}  3\sqmps \text{Open-Ocean Horizontal Eddy Diffusivity}
\end{equation}
over scales of meters due to turbulence in the thermocline probably
driven by breaking internal waves. Это значение, при использовании
(8.36) подразумевает типичный размер в 100 метров для маленьких вихрей
ответственных за перемешивание в этом эксперименте.
\end{paragraph}

\begin{paragraph}{Комментарии к горизонтальном перемешиванию.}
\begin{enumerate}
\item
Эксперименты в открытом океане упомянутые выше показывают что
турбулентное перемешивание вызывается обрушающимися внутренними
волнами и сдвиговой неустойчивостью на границах. К тому же
турбулентное перемешивание представляется более важным чем двойная
диффузия (Gregg 1987). Над выступающими батиметрическими
элементами.Результаты позволяют предположить что перемешивание
происходит в основном на океанских пограничных поверхностях:вдоль
континентальных склонов, над подводными горами и срединно
океаническими хребтами, на фронтах и в перемешанном слое на
поверхности моря. Вода в глубинах океана похоже движется вдоль
наклонных поверхностей постоянной плотности с небольшим локальным
перемешиванием пока не достигает какой ни будь границы где
перемешивается в вертикальном направлении. Затем перемешанная вода
возвращается назад в открытый океан снова вдоль поверхностей
постоянной плотности (Gregg 1985). Один частный случай заслуживает
особого упоминания. Когда вода перемешиваемая вниз через границу
перемешанного слоя втекает в термоклин вдоль поверхностей постоянной
плотности, перемешивание приводит к распределению плотности по модели
вентилируемого термоклина.

\item
Наблюдения за перемешиванием в океане показывают что численные модели
океанической циркуляции должны использовать такие схемы перемешивания
в которых используются различные коэффициенты вихревой диффузии
параллельные и перпендикулярные поверхностям постоянной плотности, а
не уровенным поверхностям постоянного значения~$z$, как мы это делали
выше. Горизонтальное перемешивание вдоль поверхностей постоянного
значения z приводит к перемешиванию поперёк поверхностей с постоянной
плотностью, так как поверхности постоянной плотности наклонены по
отношению к горизонтали приблизительно на 10-3 радиан (смотри главу
10.7, рис. 10.13). Работы Danabasoglu, Mc Williams, and Gent (1994)
показали что численные модели использующие изопикнальное и
диапикнальное перемешивание дают гораздо более реалистичные картины
океанической циркуляции.

\item
Наблюдаемое перемешивание в открытом океане, вдалеке от границ,
слишком мало для того чтобы объяснить значения перемешивания
полученные Мунком. Современные работы доложенные на конференции
посвящённой Эксперименту по Циркуляции Мирового Океана на отделении
Циркуляция и Климат в 1998 году Мунком и Вуншем (Munk and Wunsch
(1998)) показывают что эта диллема может быть решена несколькими
способами.
\begin{enumerate}
\item
Первое. Независимые исследования Гаргетта, Салмона и Мароцке (Gargett,
Salmon, and Marotzke) показывают что мы должны разделять концепцию
глубинной конвекции от концепции циркуляции меридионального
опрокидывания (смотри главу 13). Глубинная циркуляция может
перемешивать характеристики, но не массы и масса глубинных вод
поднятых на поверхность указанная Мунком может быть завышена, а
вертикальное перемешивание необходимое для компенсации этого
апвеллинга может быть меньше чем им было посчитано.

\item
Второе. Перемешивание вероятно имеет место вдоль границ или в областях
являющихся источниками вод термоклина (Gnadadesikan, 1999). Например,
вода на глубине 1200 метров в центре Северной Атлантики может
перемещаться горизонтально по направлению к Гольфстриму где
перемешивается с водой с глубины 1000 метров. Перемешанные воды могут
затем горизонтально переместиться назад в центр Северной Атлантики на
глубину 1100 метров. Таким образом воды на глубине 1200 м и 1100 м
могут занять положение вдоль совершенно разных траекторий
\end{enumerate}
\end{enumerate}
\end{paragraph}
\end{section}

\begin{section}{Основные концепции}
% \section{Important Concepts}
\begin{enumerate}
\item
Трение в океане важно только на расстояниях в несколько
миллиметров. Для большинства течений трением можно пренебречь.
%
% \item Friction in the ocean is important only over distances of a few
% millimeters. For most flows, friction can be ignored.

\item
Океан турбулентен для всех потоков чьи характерные размеры превышают
несколько сантиметров, но теория турбулентного потока в океане плохо
разработана (сложна для понимания)
%
% \vitem The ocean is turbulent for all flows whose typical dimension
% exceeds a few centimeters, yet the theory for turbulent flow in the
% ocean is poorly understood.

\item
Влияние турбулентности это функция числа Рейнольдса потока. Потоки с
одинаковой геометрией и одинаковым числом Рейнольдса обладают
одинаковыми линиями тока. 
%
% \vitem The influence of turbulence\index{turbulence!Reynolds number}
% is a function of the Reynolds number of the flow. Flows with the same
% geometry and Reynolds number have the same streamlines.

\item
Океанографы предполагают что турбулентность на расстояниях больших чем
несколько сантиметров влияет на поток также как молекулярная вязкость
влияет на поток на гораздо меньших расстояниях.
% \vitem Oceanographers assume that turbulence influences flows over
% distances greater than a few centimeters in the same way that
% molecular viscosity influences flow over much smaller distances.

\item
Учитывание влияния турбулентности приводит к появлению членов
Рейнольдовского стресса в уравнении движения.
% \vitem The influence of turbulence leads to Reynolds stress terms in
% the momentum equation.

\item
Влияние статической устойчивости в океане выражается через частоту,
частоту устойчивости. Чем больше частота, тем боле устойчив столб
жидкости.
% \vitem The influence of static stability in the ocean is expressed as
% a frequency, the stability frequency\index{stability!frequency}. The
% larger the frequency, the more stable the water column.

\item
Влияние устойчивости связанной со сдвигом скорости (сдвиговой
устойчивости) выражается через число Ричардсона. Чем больше сдвиг
скорости и меньше статическая устойчивость, тем больше вероятность
того что течение станет турбулентным.
% \vitem The influence of shear stability is expressed through the
% Richardson number. The greater the velocity shear, and the weaker the
% static stability, the more likely the flow will become turbulent.

\item
Молекулярная диффузия тепла гораздо больше чем диффузия соли. Это
приводит к неустойчивости вызываемой двойной диффузией, которая
изменяет распределение плотности в столбе воды во многих регионах
океана.
% \vitem Molecular diffusion of heat is much faster than the diffusion
% of salt.  This leads to a double-diffusion instability which modifies
% the density distribution in the water column in many regions of the
% ocean.

\item
Неустойчивость в океане приводит к перемешиванию. Перемешивание
поперёк поверхностей с постоянной плотностью гораздо меньше чем
перемешивание вдоль этих поверхностей.
% \vitem Instability in the ocean leads to mixing. Mixing across
% surfaces of constant density is much smaller than mixing along such
% surfaces.

\item
Расчитанные средние коэффициенты вихревой диффузии в глубинах океана
гораздо меньше чем измеренные.

% \vitem Horizontal eddy diffusivity in the ocean is much greater than
% vertical eddy diffusivity.

\item
Измерения коэффициента вихревой диффузии показывают что вода
перемешивается по вертикали около океанических границ (например над
подводными горами и срединно океаничесими хребтами). Это может
объяснять малые измеренные значения коэффициента диффузии в открытом
океане.
%
% \vitem Measurements of eddy diffusivity indicate water is mixed 
% vertically near oceanic boundaries such as above seamounts and
% mid-ocean ridges.  
\end{enumerate} 
\end{section}
\end{chapter}
