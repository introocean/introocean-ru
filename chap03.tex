% -*- coding: utf-8 -*-

\begin{chapter}{Физические параметры океана}
%\chapter{The Physical Setting}
Земля имеет форму сжатого у полюсов эллипсоида вращения с экваториальным
%% в оригинале еще "an ellipse rotated about its minor axis". Если это
%% относится к определению эллипсоида вращения, то это не принципиально, 
%% т.к. эллипс симметричен по двум осям, а если это о том, что Земля вращается
%% вокруг оси... Тоже вроде известный и без того факт.
радиусом
$$
R_e=6\,378.1349\mbox{~км (West,1982)},
$$ 
который немного больше полярного радиуса 
$$
 R_p=6\,356.7497\mbox{~км}.
$$
Эта разница образуется за
%% в оригинале This small equatorial bulge, "пучность", "горб"???
счёт вращения Земли. %% ... вокруг малой оси эллипсоида (см. выше)
%
% Earth \index{earth!radii of}is an oblate ellipsoid, an ellipse rotated about 
% its minor axis, with an equatorial radius of $R_e = 6,378.1349$ km 
% (West, 1982) slightly greater than the polar radius of $R_p = 6,356.7497$ km.
% The small equatorial bulge is due to earth's rotation.

Расстояния на земной поверхности измеряются в различных
единицах; наиболее распространёнными являются градусы широты и
долготы, метры, мили и морские мили. \emph{Широта}~--- это угол между
вертикалью на местности и экваториальной плоскостью. Меридиан~--- это
линия пересечения земной поверхности с плоскостью, перпендикулярной 
экваториальной плоскости и проходящей через ось вращения
Земли. \emph{Долгота}~--- это угол между нулевым меридианом и любым другим,
где нулевым является меридиан, проходящий через Королевскую Гринвичскую
обсерваторию в Англии. Таким образом, долгота измеряется на восток и
запад от Гринвича.
%
% Distances on earth are measured in many different units, the most common 
% are degrees of latitude or longitude, meters, miles, and nautical miles.
% \textit{Latitude}\index{latitude|textbf} is the angle between the local 
% vertical and the equatorial plane. A meridian is the intersection at earth's 
% surface of a plane  perpendicular to the equatorial plane and passing 
% through earth's axis of rotation. \textit{Longitude}\index{longitude|textbf}
% is the angle between the standard meridian and any other meridian, where 
% the standard meridian is the one that passes through a point at the
% Royal Observatory at Greenwich, England.  Thus longitude is measured 
% east or west of Greenwich.

За исключением экватора, градус широты на земной поверхности по длине
отличается от градуса долготы. Широта измеряется вдоль 
большого круга с радиусом~$R$, где $R$~--- средний радиус Земли. Долгота
измеряется на окружностях с радиусом~$R \cos(\varphi)$, где $\varphi$~--- широта. 
Таким образом, $1^\circ\mbox{~широты} = 111\mbox{~км}$, 
а~$1^\circ\mbox{~долготы} = 111\cos(\varphi)~\mbox{км}$. 
%%Когда требуется особая точность, следует учитывать и тот факт, 
%%что Земля~--- не сфера, так что широта тоже немного изменяется с удалением 
%%от экватора, но значения, приведённые здесь, вполне достаточны для наших целей.
%
% A degree of latitude is not the same length as a degree of longitude except
% at the equator. Latitude is measured along great circles with radius $R$,
% where $R$ is the mean radius of earth. Longitude is measured along circles 
% with radius $R \cos \varphi$, where $\varphi$ is latitude. Thus 
% $1^{\circ}$ latitude $ = 111$ km, and $1^{\circ}$ longitude 
% $= 111 \cos \varphi$ km.

Так как расстояние в градусах долготы не постоянно, океанографы
измеряют расстояние на картах, используя градусы широты.
%
% Because distance in degrees of longitude is not constant, oceanographers 
% measure distance on maps using degrees of latitude.

И морские мили, и метры исторически связаны с размерами Земли. В 1670~г.\
%% викарий церкви Святого Павла в Лионе 
Габриэль Мутон предложил
десятичную систему измерений, основанную на одной минуте дуги большого
круга Земли. Длина этой дуги позднее вошла в определение морской мили,
а предложение Мутона привело к созданию метрической
системы, основанной на другой единице длины~--- метре, который
первоначально предполагался равным одной десятимиллионной расстояния от
экватора до полюса вдоль Парижского меридиана. Хотя от взаимосвязи
морских миль и метров с размерами Земли вскоре отказались, ввиду её
непрактичности, погрешность приближённых значений, вычисленных таким образом,
достаточно мала. В самом деле, пусть длина меридиана%
\remark{Найденная как периметр эллипса с большой и малой полуосями, 
равными $R_e$ и~$R_p$ соответственно.} 
приближенно равна~$40\,008\mbox{~км}$. Отсюда одна десятимиллионная длины 
квадранта (дуги, составляющей четверть окружности) равна~$1.0002\mbox{~м}$.
%% на самом деле, это четверть длины периметра эллипса, НО определение квадранта
%% http://slovari.yandex.ru/dict/bse/article/00033/57200.htm
%% дается для окружности.
В случае морской мили поступаем аналогично: поделив длину меридиана 
на~$360 * 60 = 21600$~угловых минут, получаем~$1.8522\mbox{~км}$. Данное
значение очень близко к официальному определению \emph{международной морской
мили}: $1\mbox{~миля} \equiv 1.852\mbox{~км}$.
%
% Nautical miles and meters are connected historically to the size of earth. 
% Gabriel Mouton proposed in 1670 a decimal system of measurement based on 
% the length of an arc that is one minute of a great circle of earth.  
% This eventually became the nautical mile. Mouton's decimal system 
% eventually became the metric system based on a different unit of length,
% the meter, which was originally intended to be one ten-millionth 
% the distance from the Equator to the pole along the Paris meridian. 
% Although the tie between nautical miles, meters, and earth's radius was 
% soon abandoned because it was not practical, the approximations are very 
% good. For example, earth's polar circumference is approximately 40,008 km. 
% Therefore one ten-millionth of a quadrant is 1.0002 m. Similarly,
% a nautical mile should be 1.8522 km, which is very close to the official 
% definition of 
% the\index{nautical mile|textbf}\index{international nautical mile|textbf} 
% \textit{international nautical mile}: 1 nm $\equiv$ 1.8520 km.


\begin{section}{Океаны и моря}
% \section{Ocean and Seas}
Будем полагать, что существует единый мировой океан, условно поделенный
на три именованные части, также называемые <<океанами>>: Атлантический, 
Тихий и Индийский. Границы океанов задаются соглашениями, принятыми 
Международной гидрографической организацией. Моря, которые считаются 
частью океанов, определяются различными способами; мы рассмотрим два из них. 
%
% There is only one ocean. It is divided into three named parts by 
% international agreement: the Atlantic, Pacific, and Indian 
% ocean\index{ocean!defined} (International Hydrographic Bureau, 1953)%
%\index{International Hydrographic Bureau}. Seas, which are part of the ocean, 
% are defined in several ways. I consider two.


\begin{description}
\item[Атлантический Океан] (рис.~3.1) расположен к северу от
Антарктиды и включает Арктическое море%
\remark{\label{remark:threeoceans}
Существуют различные мнения о том, следует ли считать Северный Ледовитый
океан морем в составе Атлантического океана (как это делает автор), либо 
отдельным океаном (согласно действующей в настоящий момент 3-й редакции 
стандарта Международной гидрографической организации 
\href{http://www.iho.shom.fr/publicat/free/files/S23_1953.pdf}%
{\textsl{Limits of oceans and seas}}, 
\texttt{http://www.iho.shom.fr/publicat/free/files/S23\_1953.pdf}).
}, европейское
Средиземноморье и американское Средиземноморье (Карибское
море). Границей между Атлантическим и Индийским океанами является
меридиан мыса Игольный (\latlon{20}{E}). Граница между Атлантическим и Тихим
океанами на юге~--- линия между мысом Горн и Южными Шетландскими
островами, а на севере~--- Берингов пролив, отделяющий Тихий океан от
Арктического моря, входящего в состав Атлантического океана.
%
% \textbf{The Atlantic Ocean} \index{ocean!Atlantic Ocean}extends
% northward from Antarctica and includes all of the Arctic Sea, the
% European Mediterranean, and the American Mediterranean more
% commonly known as the Caribbean sea (figure 3.1). The boundary
% between the Atlantic and Indian Ocean is the meridian of Cape
% Agulhas (20\degrees E). The boundary between the Atlantic and
% Pacific is the line forming the shortest distance from Cape
% Horn to the South Shetland Islands. In the north, the Arctic Sea
% is part of the Atlantic Ocean, and the Bering Strait is the
% boundary between the Atlantic and Pacific.


%% Рисунок 3.1 Атлантический Океан в равноплощадной проэкции. Глубины в
%% метрах, взяты из ETOPO 30. Изобата 200 метров показывает границу
%% континетального щельфа.

\item[Тихий Oкеан] (рис.~3.2) простирается к северу от Антарктиды до Берингова
пролива. Граница между Тихим и Индийским океаном лежит на линии, проходящей
от Малайского полуострова через Суматру, Яву, Тимор, австралийский мыс
Лондондерри и Тасманию, а от Тасмании до Антарктиды~--- на меридиане мыса 
Северо-Восточный (\latlon{147}{E}).
%
% \textbf{The Pacific Ocean} \index{ocean!Pacific Ocean}extends
% northward from Antarctica to the Bering Strait (figure 3.2). The
% boundary between the Pacific and Indian Ocean follows the line
% from the Malay Peninsula through Sumatra, Java, Timor, Australia
% at Cape Londonderry, and Tasmania. From Tasmania to Antarctica it
% is the meridian of South East Cape on Tasmania 147\degrees E.


%% Рисунок 3.2 Тихий Океан в равноплощадной проэкции. Глубины в метрах,
%% взяты из ETOPO 30. Изобата 200 метров показывает границу
%% континетального щельфа.

\item[Индийский Океан] (рис.~3.3) простирается от Антарктиды до Евразийского
континента, включая в себя Красное море и Персидский залив. 
%
% \textbf{The Indian Ocean} \index{ocean!Indian Ocean}extends from
% Antarctica to the continent of Asia including the Red Sea and
% Persian Gulf (figure 3.3). Some authors use the name Southern
% Ocean to describe the ocean surrounding Antarctica.

%% Рисунок 3.2 Индийский Океан в равноплощадной проэкции. Глубины в
%% метрах, взяты из ETOPO 30. Изобата 200 метров показывает границу
%% континетального щельфа.
\end{description}

Некоторые авторы используют название Южный океан для вод вокруг Антарктиды.%
\remark{
Южный океан включен в проект очередной, 4-й редакции стандарта 
(\href{http://www.iho-ohi.net/mtg_docs/com_wg/S-23WG/S-23WG_Misc/Draft_2002/Draft_2002.htm}%
{\texttt{http://www.iho-ohi.net/mtg\_docs/com\_wg/S-23WG/S-23WG\_Misc/Draft\_2002/Draft\_2002.htm}}).
} 

Существуют различные типы морей. Мы ограничимся двумя:
\begin{description}
\item[Средиземные моря] большей частью окружены сушей. Согласно этому
определению, Арктическое и Карибское моря~--- средиземные, Арктическое
cредиземное и Карибское cредиземное.
%
% \textbf{Mediterranean Seas} \index{seas!Mediterranean}are mostly
% surrounded by land. By this definition, the Arctic and Caribbean
% Seas are both Mediterranean Seas, the Arctic Mediterranean and the
% Caribbean Mediterranean.

\item[Окраинные моря] определяются только изрезанностью побережья.
Примерами окраинных морей являются Аравийское и Южно-Китайское моря.
%
% \textbf{Marginal Seas} \index{seas!marginal}are defined by only an
% indentation in the coast. The Arabian Sea and South China Sea are
% marginal seas.
\end{description}
\end{section}


\begin{section}{Размеры океанов}
% \section{Dimensions of the ocean}
Океаны и моря покрывают $70.8\%$~земной поверхности, что 
составляет~$361\,254\,000~\mbox{км}^2$. Площади океанов значительно 
различаются (табл.~3.1):
%
% \index{ocean!dimensions of}The ocean and seas cover 70.8\% 
% of the surface of earth, which amounts to 361,254,000 km$^2$. 
% The areas of the named parts vary considerably (table 3.1).

\begin{tabular}{lr}
%% Таблица 3.1. Площадь океанов $^{\dag }$
Тихий Океан         & $181.34 \times 10^6 \mbox{км}^2$ \\
Атлантический Океан & $106.57 \times 10^6 \mbox{км}^2$ \\
Индийский Океан     & $ 74.12 \times 10^6 \mbox{км}^2$ \\
%% $^{\dag }$ From Menard and Smith (1966)
\end{tabular}
%
% \begin{table} [b!]\centering \small
% \vspace{-3ex}
% \begin{tabular*}{65mm}{@{}l @{\extracolsep{\fill}} r@{}}
% \multicolumn{2}{@{}l@{}}{\bfseries Table 3.1 Surface Area of the ocean} $^{\dag }$ \\
% \hline
% \rule{0ex}{2.5ex}Pacific Ocean  & $181.34 \times 10^6 \hbox{ km}^2$        \\
%                  Atlantic Ocean   & $ 106.57 \times 10^6 \hbox{ km}^2$        \\
%                 Indian Ocean  & $74.12 \times 10^6 \hbox{ km}^2$        \\[0.5ex]
% \hline
% \multicolumn{2}{@{}l@{}}  {\rule{0ex}{2.5ex}$^{\dag }$ From Menard and Smith (1966)}
% \end{tabular*} \\[0.5ex]
% \vspace{-3ex}
% \end{table}


Горизонтальные размеры океанов изменяются от $1500\mbox{~км}$~--- минимальной
ширины Атлантического океана, до $13000\mbox{~км}$~--- его протяженности
с севера на юг либо ширины Тихого океана.
При этом типичные глубины составляют~$3$--$4\mbox{~км}$.
Таким образом, горизонтальные размеры океанских бассейнов в $1000\mbox{~раз}$
больше, чем вертикальные.  Масштабы Тихого океана можно представить
себе с помощью обычного листа бумаги $8.5 \times 11~\mbox{дюймов}$:
задав коэффициент масштабирования~$10\mbox{~дюймов} = 10\,000\mbox{~км}$,
получим, что ширина океана сравнима с размерами листа, а глубина 
в~$3\mbox{~км}$, которая в выбранном масштабе равна~$0.003\mbox{~дюйма}$,
соответствует типичной толщине листа.
%
% Oceanic dimensions range from around 1500 km for the minimum width of the
% Atlantic to more than 13,000 km for the north-south extent of the Atlantic 
% and the width of the Pacific. Typical depths are only 3--4 km. So horizontal
% dimensions of ocean basins are 1,000 times greater than the vertical
% dimension. A scale model of the Pacific, the size of an $8.5 \times 11$ in 
% sheet of paper, would have dimensions similar to the paper: a width 
% of 10,000 km scales to 10 in, and a depth of 3 km scales to 0.003 in, 
% the typical thickness of a piece of paper.


Таким образом, графики поперечного сечения океана для
удобства использования должны иметь сильно преувеличенный вертикальный
масштаб. Как правило, его выбирают в 200 раз большим, 
чем горизонтальный (рис.~3.4). Это преувеличение искажает наши
представления об океане. Края океанических бассейнов (континентальные
склоны), которые на рис.~3.4 выглядят крутыми обрывами 
(\latlon{41}{W}, \latlon{12}{E}), на самом деле представляют собой 
пологие склоны, понижающиеся на~$1\mbox{~м}$ по вертикали 
на каждые~$20\mbox{~м}$ по горизонтали.
%
% Because the ocean is so thin, cross-sectional plots of ocean basins must 
% have a greatly exaggerated vertical scale to be useful. Typical plots have
% a vertical scale that is 200 times the horizontal scale (figure 3.4). This 
% exaggeration distorts our view of the ocean. The edges of the ocean basins, 
% the continental slopes, are not steep cliffs as shown in the figure 
% at 41\degrees W and 12\degrees E. Rather, they are gentle slopes dropping 
% down 1 meter for every 20 meters in the horizontal.

Малое отношение глубин океанических бассейнов к их ширине также играет 
важную роль в теории океанских течений. Так, вертикальные скорости 
должны быть гораздо меньше, чем горизонтальные. Даже на расстояниях
порядка нескольких сотен километров вертикальные скорости должны составлять 
менее $1\%$~горизонтальных. Мы используем эту информацию позже для того, 
чтобы упростить уравнение движения.
%
% The small ratio of depth to width of the ocean basins is very important 
% for understanding ocean currents. Vertical velocities must be much smaller
% than horizontal velocities. Even over distances of a few hundred kilometers,
% the vertical velocity must be less than 1\% of the horizontal velocity.
% I will use this information later to simplify the equations of motion.

В то же время, относительно малые вертикальные скорости существенно влияют
на турбулентность. Трёхмерная турбулентность по своей природе сильно
отличается от двумерной. В двумерной турбулентности вихревые линии
всегда должны быть вертикальны, так что растяжение вихря невелико.
С другой стороны, в трёхмерном случае растяжение вихря
играет фундаментальную роль.
%
% The relatively small vertical velocities have great influence on
% turbulence\index{turbulence}. Three dimensional turbulence is fundamentally
% different than two-dimensional turbulence\index{turbulence!two dimensional}.
% In two dimensions, vortex lines must always be vertical, and there can be
% little vortex stretching. In three dimensions, vortex stretching plays
% a fundamental role in turbulence.

%% Рисунок 3.4 Профиль Северной Атлантики вдоль 25°S демонстрирующий
%% континетальный шельф Южной Америки, подводную гору около 35°W,
%% срединный Атлантический Хребет около 14°W, Хребет Вальвис около 6°E и
%% узкий континетальный шельф Южной Африки. Верхний: Вертикальное
%% увеличение масштаба 180:1. Нижний: Вертикальное увеличение масштаба
%% 30:1. Если нарисовать график в действительной пропорции, то он будет
%% тоньше чем линия обозначающая поверхность моря на нижнем графике
%% рисунка.
\end{section}

\begin{section}{Элементы рельефа}
% \section{Sea-Floor Features}
Земная кора делится на два типа: сравнительно тонкая (около~$10\mbox{~км}$), 
но более плотная океаническая и более толстая (около~$40\mbox{~км}$), но
менее плотная континентальная. Участки коры континентального типа погружаются
в более плотное вещество мантии не так глубоко, как участки океанического типа,
так что средняя высота их поверхности относительно уровня моря имеет два
различных значения: континенты в среднем возвышаются на~$1100\mbox{~м}$, 
а дно океанов погружено на~$-3400\mbox{~м}$ (рис.~3.5).
%
% Earth's rocky surface is divided into two types: oceanic, with a thin
% dense crust about 10 km thick, and continental, with a thick light crust 
% about 40 km thick. The deep, lighter continental crust floats higher on 
% the denser mantle than does the oceanic crust, and the mean height of
% the crust relative to sea level has two distinct values: continents
% have a mean elevation of 1100 m, the ocean has a mean depth of -3400 m 
% (figure 3.5).

%% Рисунок 3.5 Слева: Гистограмма превышений суши и глубины дна океана в
%% процентном отношении к площади Земли. Видно явное различие между
%% континентами и морским дном. Справа: Гипсографическая кривая. Кривые
%% посчитаны по данным ETOPO 30.

Объём воды в океанах превышает объём океанических бассейнов, так что её часть
покрывает низменные окраины континентов. Образующиеся при этом мелководные 
моря называются континентальным шельфом. 
Ширина некоторых из них (например, Южно-Китайского моря) превосходит~$1100\km$,
а типичная глубина большинства сравнительно невелика: $50$--$100\m$.
Наиболее важными участками шельфа считаются Восточно-Китайское море, 
Берингово море, Северное море, Большая Ньюфаундлендская банка, Патагонский
шельф, Арафурское море и залив Карпентария, а также Сибирский
шельф. Мелководные моря помогают рассеиванию (диссипации) приливов,
они часто являются зонами высокой биологической продуктивности и, как правило,
входят в исключительные экономические зоны близлежащих стран.
%% ??? Grand Banks --- что имелось в виду: Большая Багамская банка 
%% (Grand Banks of Bahamas не гуглится) или
%% Grand Banks of Newfoundland 
%% http://www.britannica.com/EBchecked/topic/241118/Grand-Banks -- про нее
%% http://slovari.yandex.ru/dict/bse/article/00009/46500.htm -- она же
%
% The volume of the water in the ocean exceeds the volume of the ocean basins, 
% and some water spills over on to the low lying areas of the continents. These
% shallow seas are the continental shelves. Some, such as the South China Sea,
% are more than 1100 km wide. Most are relatively shallow, with typical depths 
% of 50--100 m. A few of the more important shelves are: the East China Sea,
% the Bering Sea, the North Sea, the Grand Banks, the Patagonian Shelf,
% the Arafura Sea and Gulf of Carpentaria, and the Siberian Shelf. The shallow
% seas help dissipate tides, they are often areas of high biological
% productivity, and they are usually included in the exclusive economic zone
% of adjacent countries.

Земная кора разделена на большие плиты, которые движутся относительно
друг друга. Новая кора создаётся в срединно-океанических хребтах, а
старая исчезает в глубоководных желобах. Относительное движение
литосферных плит порождает большое количество элементов морского
%% ??? due to plate tectonics "тектоника плит" --- это название геол. теории?
%% или еще и "процесса движения и деформации"?
дна. Эти элементы, изображённые на рис.~3.6, включают в себя
срединно-океанические хребты, глубоководные желоба, бассейны и островные дуги.
Названия элементов рельефа морского дна утверждены Международной
гидрографической организацией, а определения, приведенные ниже, даются 
согласно работам Sverdrup, Johnson, and Fleming (1942), Shepard (1963)
и~Dietrich et al. (1980).
%
% The crust is broken into large plates that move relative to each other. New
% crust is created at the mid-ocean ridges, and old crust is lost at trenches.
% The relative motion of crust, due to plate tectonics, produces the
% distinctive features of the sea floor sketched in figure 3.6, including
% mid-ocean ridges, trenches, island arcs, and basins. 
% \index{ocean!features of|(}The names of the sub-sea features have been
% defined by the International Hydrographic
% Organization\index{International Hydrographic Bureau} (1953), and
% the following definitions are taken from Sverdrup, Johnson, 
% and Fleming (1942), Shepard (1963), and Dietrich et al. (1980).

%% Рисунок 3.6 Схематический профиль океана демонстрирующий основные
%% элементы морского дна. Обратите внимание на то что уклоны сильно
%% преувеличены.

\begin{description}
\item[Бассейн] 
Понижение морского дна, напоминающее по своей форме круг или овал.
%
% \textit{Basins} \index{basins|textbf}are deep depressions of the sea floor
% of more or less circular or oval form.

\item[Каньон]
Относительно узкая глубокая долина с крутыми склонами, проходящая по
континентальному шельфу и континентальному склону, глубина
которой постоянно увеличивается.
%% "долина" отсюда: http://slovari.yandex.ru/dict/gl_natural/article/3004/300_4104.HTM
%% из формулировки неясно, вдоль или поперек шельфа идет
%% еще определение: http://slovari.yandex.ru/dict/bse/article/00032/20400.htm
%
% \textit{Canyons} \index{canyon|textbf}are relatively narrow, deep furrows 
% with steep slopes, cutting across the continental shelf and slope, with
% bottoms sloping continuously downward.

\item[Континентальный шельф]
Зона, смежная с континентом (или окружающая остров), простирающаяся от
линии малой воды до глубины (как правило, порядка~$120\m$), на которой
%% ??? линии наибольшего отлива, уреза малой воды, и т.п.
%% http://www.multitran.ru/c/m.exe?l1=1&l2=2&s=low-water+line
%% первоначальный вариант перевода содержал термин "горизонт меженных вод"
%% но это вроде (http://slovari.yandex.ru/dict/bse/article/00046/79300.htm)
%% относится к рекам и сезонным изменениям, а не к приливам/отливам
обнаруживается резкое или хотя бы достаточно ярко выраженное увеличение 
крутизны склона в направлении больших глубин (рис.~3.7).
%
% \textit{Continental shelves} \index{continental shelves|textbf}are zones 
% adjacent to a continent (or around an island) and extending from
% the low-water line to the depth, usually about 120 m, where there is a marked
% or rather steep descent toward great depths. (figure 3.7)
 
\item[Континентальный склон]
Уклон в сторону моря от границы шельфа к большим глубинам.
%
% \textit{Continental slopes} \index{continental slopes|textbf}are
% the declivities seaward from the shelf edge into greater depth.

\item[Равнина]
Плоская поверхность океанского дна, обнаруженная во многих глубоких бассейнах.
%
% \textit{Plains} \index{plains|textbf}are very flat surfaces found in many
% deep ocean basins.

\item[Хребет]
Вытянутое узкое поднятие морского дна с крутыми склонами и
неравномерной (нерегулярной) топографией.
%
% \textit{Ridges} \index{ridges|textbf}are long, narrow elevations of the sea
% floor with steep sides and rough topography.

\item[Подводная гора]
Изолированное или относительно изолированное поднятие, возвышающееся
на~$1000\m$ и более над дном океана, со сравнительно небольшой площадью 
вершины (рис.~3.8).
%
% \textit{Seamounts} \index{seamounts|textbf}are isolated or comparatively
% isolated elevations rising 1000 m or more from the sea floor and with small
% summit area (figure 3.8).

\item[Разлом]
Наиболее глубокий участок хребта, отделяющего океанические бассейны друг от 
друга или от близлежащего морского дна.
%% В БСЭ есть термин "силл": 
%% http://slovari.yandex.ru/dict/bse/article/00070/97600.htm
%% но он вроде совсем не о том
%
% \textit{Sills} \index{sills|textbf}are the low parts of the ridges separating
% ocean basins from one another or from the adjacent sea floor.

\item[Глубоководный желоб]
Протяжённое, узкое и глубокое понижение морского дна с относительно
крутыми склонами (рис.~3.9).
%
% \textit{Trenches} \index{trenches|textbf}are long, narrow, and deep
% depressions of the sea floor, with relatively
% steep sides (figure 3.9).\index{ocean!features of|)}
\end{description}

Элементы подводного рельефа оказывают важное влияние на циркуляцию
океанов. Хребты разделяют глубинные воды океанов на отдельные бассейны. 
Вода, находящаяся глубже разлома, не
может перемещаться из одного бассейна в другой. Десятки тысяч
изолированных пиков, подводных гор, разбросаны по дну океана. Они
преграждают путь течениям и вызывают турбулентность, которая приводит
к вертикальному перемешиванию вод.
%
% Sub-sea features strongly influences the ocean circulation.
% Ridges separate deep waters of the ocean into distinct basins.
% Water deeper than the sill\index{sills} between two basins cannot move
% from one to the other. Tens of thousands of seamounts are scattered
% throughout the ocean basins. They interrupt ocean currents, and produce
% turbulence\index{turbulence!in deep ocean} leading to vertical
% mixing\index{mixing!vertical} in the ocean.

%%Рисунок 3.7 Пример континентального шельфа, шельф у побережья Монтерея
%%в Калифорнии, здесь можно видеть каньон Монтерей и другие. Каньоны
%%часто встречаются на шельфе и обычно простираются через весь шельф и
%%континентальный склон. Права на рисунок принадлежат Monterey Bay
%%Aquarium Research Institute (MBARI).

%%Рисунок 3.8 Пример подводной горы~--- гайот Вилд. Гайот~--- это
%%морская гора с плоской вершиной, а плоская она из за волнового
%%воздействия происходившего пока гора находилась над уровнем моря. Так
%%как морская гора зависит от тектоники плит, то она понемногу
%%погружается. Глубины были посчитаны на основе данных эхолокации вдоль
%%маршрутов судна (тонкие прямые линии), дополненными данными
%%гидролокатора бокового обзора. Глубина в сотнях метров.

%%Рисунок 3.9 Пример глубоководного жёлоба~--- Алеутский Желоб;
%%островная дуга, Алеутские Острова и континентальный шельф, Берингово
%%море. Островная дуга состоит из вулканов образовавшихся когда
%%океаническая кора погружаясь в желоб, плавилась, и поднималась к
%%поверхности. Наверху Карта Алеутского региона на севере Тихого
%%Океана. Внизу профиль через регион.
\end{section}

\begin{section}{Измерение глубины океана}
% \section{Measuring the Depth of the Ocean}
Глубина океана может быть измерена двумя способами: 1) эхолотом,
установленным на судне, или 2) спутниковым альтиметром.
% The depth of the ocean is usually measured two ways: 1) using acoustic
% echo-sounders on ships, or 2) using data from satellite altimeters.

\begin{paragraph}{Эхолоты.}
Большинство карт океана созданы на основе измерений, сделанных при помощи
эхолотов. Этот прибор посылает звуковой импульс частотой
$10$--$30\kHz$ и принимает сигнал, отражённый от морского дна. Временной
интервал между посылкой импульса и приходом эха, умноженный на скорость
звука, даёт удвоенную глубину океана (рис.~3.10).
%
% \paragraph{Echo Sounders} \index{echo sounders|(}Most maps of the ocean
% are based on measurements made by echo sounders. The instrument transmits
% a burst of 10--30 kHz sound\index{sound!used to measure depth} and listens
% for the echo from the sea floor. The time interval between transmission
% of the pulse and reception of the echo, when multiplied by the velocity
% of sound, gives twice the depth of the ocean (figure 3.10).

Впервые трансатлантическое эхолотирование было выполнено в 1922~г.\ %
американским эсминцем <<Стюарт>>, 
а первые систематические промеры производились немецким
исследовательским судном <<Метеор>> в ходе экспедиции в южную
Атлантику в~1925--1927~гг. В настоящее время океанографические и военные суда
во время плавания ведут эхолотирование практически непрерывно.
Миллионы миль профилей глубины, записанных на бумагу,
%% ship-track = "морской профиль"
%% (http://www.multitran.ru/c/m.exe?CL=1&l1=1&s=ship-track),
%% но "профиль глубины" точнее в данном контексте?
были оцифрованы и занесены в базы данных, на основе которых и
составляются батиметрические карты. Распределение судовых маршрутов по
поверхности океана неравномерно. В южном полушарии они пролегают
довольно далеко друг от друга даже возле Австралии (рис.~3.11), 
а в уже хорошо картированных районах, таких как Северная Атлантика, 
достаточно близко.
%
% The first transatlantic echo soundings were made by the U.S. Navy
% Destroyer \textit{Stewart} in 1922. This was quickly followed by
% the first systematic survey of an ocean basin, made by the German
% research and survey ship \textit{Meteor} during its expedition to
% the south Atlantic from 1925 to 1927. Since then, oceanographic
% and naval ships have operated echo sounders almost continuously
% while at sea. Millions of miles of ship-track data recorded on
% paper have been digitized to produce data bases used to make maps.
% The tracks are not well distributed. Tracks tend to be far apart
% in the southern hemisphere, even near Australia (figure 3.11) and
% closer together in well mapped areas such as the North
% Atlantic.\index{echo sounders|)}

%% Этот фрагмент отсутствует в оригинале, но содержит весьма полезную
%% информацию. Даже жаль выкидывать.
%%Измерения глубин эхолотированием широко используются, но у этого
%%метода есть свои ошибки.
%%\begin{enumerate}
%%\item
%%Скорость звука изменяется на $\pm 4\mbox{\%}$ в разных районах
%%океана. Используя таблицы средних скоростей звука можно уменьшить
%%ошибку измерений до $\pm 1\mbox{\%}$. Смотри параграф 3,6 для большей
%%информации о звуке в океане.
%%
%%\item
%%От малых глубин эхо может прийти не точно под корабль, а на его
%%борт. Это может вызвать небольшие ошибки в холмистых районах.
%%
%%\item
%%Местоположение корабля плохо определялось до появления в шестидесятых
%%спутниковой навигации. Ошибки могли составлять десятки километров
%%особенно в облачных регионах где невозможны астрономические
%%наблюдения.
%% 
%%\item
%%Иногда скопления зоопланктона и косяки рыбы в неглубоких райнонах
%%вызывали ошибки, приводившие к появлению на некоторых батиметрических
%%картах ложных подводных гор. Эта ошибка устраняется путём повторного
%%исследования спорных мест.
%%
%%\item
%%Некоторые районы океана (размером до 500 километров) ни разу не были
%%исследованы эхолокаторами. Это создаёт значительные пробелы в наших
%%знаниях об океанских глубинах
%%\end{enumerate}

%% Рисунок 3.10 Слева: Эхолокаторы измеряют глубину океана посылая
%% звуковой импульс и измеряя время затраченное им чтобы отразится от
%% поверхности и вернутся обратно. Справа: Время записывается с помощью
%% иглы оставляющей след на медленно движещемся рулоне бумаги. (From
%% Dietrich, et al. 1980)


%% Рисунок 3.11 Расположение данных эхолотирования использованных для
%% картирования океана около Австралии. Заметте что имеются большие
%%пространства где нет данных.

Использование эхолотов дает наиболее точные данные о глубине океана:
их погрешность составляет~$\pm1$\%.
% Echo sounders \index{echo sounders!errors in measurement}make the most 
% accurate measurements of ocean depth.
% Their accuracy\index{accuracy!echo sounders} is $\pm$1\%.
\end{paragraph}

\begin{paragraph}{Спутниковая альтиметрия.}
Пробелы в наших знаниях о глубинах океана между маршрутами судов
теперь заполнены данными спутниковой альтиметрии. Альтиметры измеряют
(профилируют) форму морской поверхности, которая некоторым образом 
связана с рельефом дна. Чтобы понять, почему это
происходит, мы вначале должны обсудить то, как гравитация влияет на
уровень моря.
%
% \paragraph{Satellite Altimetry}
% \index{satellite altimetry!use in measuring depth} Gaps in our knowledge
% of ocean depths between ship tracks have now been filled
% by satellite-altimeter data. Altimeters profile the shape of the sea surface,
% and its shape is very similar to the shape of the sea floor
% (Tapley and Kim, 2001; Cazenave and Royer, 2001; Sandwell and Smith, 2001).
% To see this, we must first consider how gravity influences sea level.

\begin{subparagraph}{Взаимосвязь уровня моря и рельефа дна.}
% \textit{The Relationship Between Sea Level and the Ocean's Depth}
Избыток массы на дне океана, например подводная гора, увеличивает местную
гравитацию. Плотность скальных пород, образующих гору, в три раза превышает
плотность воды, поэтому масса горы соответственно больше массы воды, 
которую она замещает. В свою очередь, увеличение силы тяжести притягивает 
к горе воду, изменяя форму морской поверхности (рис.~3.12).
%
% Excess mass at the sea floor, for example the mass of a seamount, increases
% local gravity because the mass of the seamount is larger than the mass
% of water it displaces. Rocks are more than three times denser than water.
% The excess mass increases local gravity, which attracts water toward
% the seamount. This changes the shape of the sea surface (figure 3.12).

Рассмотрим это явление более подробно. С достаточной точностью можно считать, 
что поверхность моря~--- частный случай уровенной поверхности, называемой
%% ??? "поверхность уровня" или "эквипотенциальная поверхность"
%% (http://www.multitran.ru/c/m.exe?CL=1&l1=1&s=level+surface)
%% звучит лучше, но похоже, что "Уровенная поверхность"
%% как раз и есть геодезический термин: 
%% http://slovari.yandex.ru/dict/bse/article/00082/56900.htm
%% но звучит некрасиво :-(
\emph{геоидом} (см. врезку). По определению, уровенная поверхность 
представляет собой множество точек с одинаковым гравитационным потенциалом
и в каждой своей точке перпендикулярна силе тяжести. 
В частности, она должна быть перепендикулярна локальной вертикали, определяемой
при помощи \emph{отвеса}, то есть <<небольшого груза, свободно подвешенного 
на нити, по которой определяют вертикальное направление>> (Толковый словарь 
русского языка Ушакова%
\remark{
\href{http://slovari.yandex.ru/dict/ushakov/article/ushakov/15-2/us290207.htm}%
{\texttt{http://slovari.yandex.ru/dict/ushakov/article/ushakov/15-2/us290207.htm}}%
}).
%
% Let's make the concept more exact. To a very good approximation, 
% the sea surface is a particular \textit{level surface}
% \index{level surface|textbf}called the \textit{geoid} (see box).
% By definition a level surface is a surface of constant gravitational
% potential, and it is everywhere perpendicular to gravity. In particular,
% it must be perpendicular to the local vertical determined by a plumb line,
% which is ``a line or cord having at one end a metal weight for determining
% vertical direction'' (Oxford English Dictionary).

Избыток массы подводной горы притягивает грузик отвеса, тем самым немного 
отклоняя его нить от направления к центру масс Земли в сторону
горы. Так как поверхность моря должна быть перепендикулярна вектору силы
тяжести, над подводной горой образуется небольшая вспученность,
как показано на рис.~3.12. Обычные подводные горы вызывают
%% Предложение "If there were no bulge, the sea surface would not be
%% perpendicular to gravity." выкинуто, поскольку фактически повторяет 
%% предыдущее.
вспученности высотой~$1$--$20\m$ на расстоянии~$100$--$200\km$. Конечно, такое
изменение высоты слишком мало, чтобы быть обнаруженным с корабля, однако
спутниковым альтиметром это сделать довольно просто. Глубоководные желоба 
вызывают дефицит масс и создают понижения морской поверхности.
%
% The excess mass of the seamount attracts the plumb line's weight, causing
% the plumb line to point a little toward the seamount instead of toward
% earth's center of mass. Because the sea surface must be perpendicular
% to gravity, it must have a slight bulge above a seamount as shown
% in figure 3.12. If there were no bulge, the sea surface would not be
% perpendicular to gravity. Typical seamounts produce a bulge that is 1--20 m
% high over distances of 100--200 kilometers. This bulge is far too small
% to be seen from a ship, but it is easily measured by satellite altimeters.
% Oceanic trenches have a deficit of mass, and they produce a depression
% of the sea surface.

Взаимосвязь между формой морской поверхности и глубиной не очень
строга. Она зависит от прочности дна, возраста элементов рельефа, толщины
слоя осадочных пород. 
Если подводная гора <<плавает>> на поверхности дна,
словно лёд на воде, то гравитационный сигнал будет слабее, чем если бы
она покоилась на дне, как лёд, лежащий на столе. В результате
взаимосвязь силы тяжести и рельефа дна изменяется от места к месту.
%
% The correspondence between the shape of the sea surface and the depth
% of the water is not exact. It depends on the strength of the sea floor,
% the age of the sea-floor feature, and the thickness of sediments.
% If a seamount floats on the sea floor like ice on water, the gravitational
% signal is much weaker than it would be if the seamount rested on the sea
% floor like ice resting on a table top. As a result, the relationship
% between gravity and sea-floor topography varies from region to region.

Глубина, измеряемая эхолотами, используется для того, чтобы определить
эту взаимосвязь. Затем с помощью альтиметрии проводится интерполяция
между измерениями эхолотов (Smith and Sandwell, 1994). 
%% Используя этот способ, можно расчитать
%% глубины океана с точностью до $\pm 100$~метров.
%
% Depths measured by acoustic echo sounders are used to determine the regional
% relationships. Hence, altimetry is used to interpolate between acoustic echo
% sounder measurements (Smith and Sandwell, 1994).
\end{subparagraph}

\begin{subparagraph}{Системы спутниковой альтиметрии.}
Рассмотрим, каким образом альтиметры измеряют форму земной
поверхности. Системы спутниковой альтиметрии включают в себя радар для
измерения высоты спутника над земной поверхностью и систему слежения
для определения высоты спутника в геоцентрической системе
координат. Система измеряет превышение уровня моря относительно центра
масс Земли (рис.~3.13) и, тем самым, определяет форму морской
поверхности.
%
% \textit{Satellite-altimeter systems}
% \index{satellite altimetry!systems|textbf}Now let's see how altimeters
% measure the shape of the sea surface. Satellite altimeter systems include
% a radar to measure the height of the satellite above the sea surface and
% a tracking system to determine the height of the satellite in geocentric
% coordinates. The system measures the height of the sea surface relative
% to the center of mass of earth (figure 3.13). This gives the shape of the sea
% surface.

В околоземное космическое пространство выведено большое количество 
альтиметрических спутников, предназначенных для изучения морского геоида
и влияния на него элементов подводного рельефа. Наиболее важные 
альтиметрические данные были получены спутниками 
Seasat~(1978), GEOSAT~(1985--1988), 
ERS-1~(1991--1996), ERS-2~(1995--), Topex/Poseidon~(1992--2006),
Jason~(2002--) и~Envisat~(2002). Спутники~Topex/Poseidon и~Jason 
специально предназначены для измерения высоты морской поверхности с высокой
точностью, достигающей~$\pm 0.05\m$.
%
% Many altimetric satellites have flown in space. All observed the marine
% geoid\index{geoid} and the influence of sea-floor features on
% the geoid\index{geoid}. The altimeters that produced the most useful
% data include Seasat (1978)\index{Seasat}, \textsc{geosat} (1985--1988),
% \textsc{ers}--1\index{ERS satellites} (1991--1996),
% \textsc{ers}--2 (1995-- ), Topex/Poseidon\index{Topex/Poseidon} (1992--2006),
% Jason\index{Jason} (2002--), and Envisat (2002)\index{Envisat}.
% Topex/Poseidon and Jason were specially designed to make extremely accurate
% measurements of sea-surface height. They measure sea-surface height with
% an accuracy
% of $\pm 0.05$ m\index{Jason!accuracy of}\index{Topex/Poseidon!accuracy of}.
\end{subparagraph}


%% Рисунок 3.13 Спутниковый альтиметр измеряет высоту спутника над
%% уровнем моря. При вычитании этого значения из высоты орбиты спутника
%% получим уровень моря относительно центра Земли. Форма поверхности
%% изменяется под воздействием вариаций силы тяжести, которые вызывают
%% индуляции геоида, и под воздействием океанаских течений, которые
%% приводят к образованию океанической топографии, отклонениям
%% поверхности моря от геоида. Референц элипсоид~--- наиболее близкая
%% сглаженная апроксимация геоида.

\begin{subparagraph}{Спутниковые альтиметрические карты дна.}
Орбиты спутников Seasat, Geosat, ERS-1 и~ERS-2 располагались таким образом,
что расстояние между маршрутами измерений на поверхности, равное~$3$--$10\km$,
оказалось достаточным для картирования геоида. На основе показаний альтиметров 
спутников GEOSAT и~ERS-1, объединенных с данными эхолотирования, были 
построены карты морского дна с пространственным разрешением~$5$--$10\km$
и средней погрешностью по глубине, равной~$\pm 100\m$ Smith and Sandwell (1997).
\end{subparagraph}
% 
% \textit{Satellite Altimeter Maps of the Sea-floor Topography}
% Seasat\index{Seasat}, \textsc{geosat}\index{Geosat}, \textsc{ers}--1,
% and \textsc{ers}--2\index{ERS satellites} were
% \index{satellite altimetry!maps of the sea-floor topography}operated
% in orbits with ground tracks spaced 3--10 km apart, which was sufficient
% to map the geoid\index{geoid}. By combining data from echo-sounders
% with data from \textsc{geosat} and \textsc{ers}--1 altimeter systems,
% Smith and Sandwell (1997) produced maps of the sea floor with horizontal
% resolution of 5--10 km and a global average depth accuracy of $\pm 100$ m.
\end{paragraph}

%% Box: caption{Геоид}
\begin{center}\textbf{Геоид}\end{center}
Уровенная поверхность, соответствующая невозмущённому уровню моря,
называется \emph{геоидом}. В первом приближении, геоид~--- это эллипсоид,
соответствующий поверхности однородной (не имеющей внутренних течений)
жидкости, совершающей твердотельное вращение. Во втором приближении, геоид
%% жидкости на твёрдом вращающемся теле
%% ??? homogeneous fluid in solid-body rotation --- уточнить перевод,
%% возможно, что здесь речь идет не о поверхности тв. тела, а о каком-то
%% названии такого вращения
%% http://en.wikipedia.org/wiki/Liquid_mirror:
%% In fluid mechanics, the state when no part of the fluid has motion 
%% relative to any other part of the fluid is called 'solid body rotation'.
%% твердотельное вращение? но это ridid-body rotation 
%% (http://vo.astronet.ru/dict/?lang=ru&word=%23rigidBodyRotation)
отличается от элипсоида из-за локальных неоднородностей силы
тяжести. Эти отклонения называются \emph{ундуляциями геоида}. Максимальная их
амплитуда ориентировочно равна~$\pm 60\m$. В третьем приближении, геоид
отличается от поверхности моря, поскольку океаны далеко не
спокойны. Отклонения уровня моря от геоида называют
\emph{топографией}. Обозначают её так же, как наземную топографию, например,
высотой, нанесённой на топографическую карту.
%
% The \index{geoid|textbf}level surface\index{level surface} that corresponds
% to the surface of an ocean at rest is a special surface, the \textit{geoid}.
% To a first approximation, the geoid\index{geoid} is an ellipsoid that
% corresponds to the surface of a rotating, homogeneous fluid in solid-body
% rotation, which means that the fluid has no internal flow. To a second
% approximation, the geoid differs from the ellipsoid because of local
% variations in gravity. The deviations are called \textit{geoid undulations}.
% \index{geoid!undulations|textbf}The maximum amplitude of the undulations is
% roughly $\pm 60$ m. To a third approximation, the geoid deviates from the sea
% surface because the ocean is not at rest. The deviation of sea level from
% the geoid\index{geoid} is defined to be the \textit{topography}.
% \index{topography|textbf}The definition is identical to the definition
% for land topography, for example the heights given on a topographic map.

Топография океана определяется приливами и океанскими поверхностными
течениями, которые будут рассмотрены подробнее в гл.~10 и~17.
Максимальная амплитуда топографии составляет приблизительно~$\pm 1\m$, 
таким образом, она мало сравнима с ундуляциями геоида.
%
% The ocean's topography is caused by tides, heat content of the water,
% and ocean surface currents. I will return to their influence in chapters 10
% and 17. The maximum amplitude of the topography is roughly $\pm 1$ m, so it
% is small compared to the geoid\index{geoid} undulations.

Ундуляции геоида вызываются локальными вариациями силы тяжести
вследствие неравномерного распределения массы на дне океана. В местах 
расположения подводных гор наблюдается избыток массы благодаря их плотности,
что ведет к образованию на геоиде выпуклости (см.~ниже). 
В районах глубоководных желобов наблюдается дефицит масс и, соответственно, 
прогиб геоида. Таким образом, геоид взаимосвязан с рельефом дна, 
и карты морского геоида имеют заметное сходство с батиметрическими.
%
% Geoid  undulations are caused by local variations in gravity due to
% the uneven distribution of mass at the sea floor. Seamounts have an excess
% of mass because they are more dense than water. They produce an upward bulge
% in the geoid (see below). Trenches have a deficiency of mass. They produce
% a downward deflection of the geoid\index{geoid}. Thus the geoid\index{geoid}
% is closely related to sea-floor topography. Maps of the oceanic
% geoid\index{geoid} have a remarkable resemblance to the sea-floor topography.

%%Рисунок 3.12 Подводная гора гораздо плотнее чем морская вода, поэтому
%%она увеличивает локальную силу тяжести и заставляет отвесную линию
%%(стрелочки) отклоняться в сторону горы. Так как поверхность океана в
%%спокойном состоянии должна быть перпендикулярна силе тяжести,
%%поверхность моря и геоид в этом месте должны иметь небольшую
%%выпуклость как показано на рисунке. Эта выпуклость легко измеряется
%%спутниковыми альтиметрами. Следовательно данные альтиметров могут
%%использоваться для картирования морского дна. Следует понимать что
%%выпуклость поверхности моря на рисунке сильно преувеличена, подводная
%%гора высотой в 2 километра вызывает выпуклость высотой приблизительно
%%10 метров.

%% End of box
\end{section}

\begin{section}{Батиметрические карты и базы данных}
% \section{Sea Floor Charts and Data Sets}
Почти все доступные результаты эхолотирования были оцифрованы и собраны вместе,
чтобы на их основе построить батиметрические карты. В результате дальнейшей
обработки эти данных были созданы цифровые базы данных, которые получили
широкое распространение на CD-ROM. Эти данные были дополнены данными
альтиметрических спутников для того, чтобы создать карты морского дна с
пространственным разрешением около~$3\km$.
%
% \index{ocean!maps of}Almost all echo-sounder data have been digitized
% and combined to make sea-floor charts. Data have been further processed
% and edited to produce digital data sets which are widely distributed
% in \textsc{cd-rom} format. These data have been supplemented with data
% from altimetric satellites to produce maps of the sea floor with
% horizontal resolution around 3 km.

Британский центр океанографических данных, 
%% (The British Oceanographic Data Centre), 
%% перевод названия: http://data.oceaninfo.ru/codes/codeview?id=161&s=2&l=ru
действуя по поручению Межправительственной океанографической комиссии ЮНЕСКО
%% Intergovernmental Oceanographic Commission of UNESCO
и Международной гидрографической организации,
%% International Hydrographic Organization
публикует электронный атлас <<Общая батиметрическая карта океанов>> 
%% http://dic.academic.ru/dic.nsf/eng_rus_technic/68943/GEBCO
%% НО в http://slovari.yandex.ru/dict/bse/article/00006/94300.htm она же 
%% называется "Генеральная батиметрическая карта океанов"
(также известный как GEBCO, то есть, General Bathymetric Chart of the
Oceans). Этот атлас содержит, в основном, изобаты, линию берега 
и путевые линии, построенные на основе 5-й редакции Общей батиметрической 
%% tracklines ???
карты океанов, изданной в масштабе~$1:10\,000\,000$.
Исходные изолинии были нарисованы от руки согласно оцифрованным данным 
эхолотирования.
%
% The British Oceanographic Data Centre publishes the General Bathymetric
% Chart of the ocean (\textsc{gebco})\index{bathymetric charts!GEBCO} Digital
% Atlas on behalf of the Intergovernmental Oceanographic Commission
% of \textsc{unesco} and the International Hydrographic
% Organization\index{International Hydrographic Organization}.
% The atlas consists primarily of the location of depth
% contours, coastlines, and tracklines from the \textsc{gebco} 5th Edition
% published at a scale of 1:10 million. The original contours were drawn
% by hand based on digitized echo-sounder data plotted on base maps.

Национальный центр геофизических данных США выпустил CD-ROM
%% NGDC: http://www.spsl.nsc.ru/win/nelbib/ecolos/geo_gis.htm
ETOPO-2, содержащий значения как глубин океана, измеренных при помощи эхолотов
и спутниковых альтиметров, так и высот на суше.
Интерполяция данных осуществлялась на сетке с шагом~$2'$ (2 морские мили).
Данные по океану в области от~\latlon{64}{N} до~\latlon{72}{S} взяты из
работы Smith and Sandwell (1997), в которой результаты эхолотирования
объединены с показаниями альтиметров, установленных на спутниках~GEOSAT
и~ERS-1; в области к северу от~\latlon{64}{N}~--- согласно Международной
батиметрической карте Северного Ледовитого океана, а в области 
%% International Bathymetric Chart of the Arctic Ocean
южнее~\latlon{72}{S}~--- в соответствии с Цифровой базой батиметрических
данных переменного разрешения US Naval Oceanographic Office.
%% Digital Bathymetric Data Base Variable Resolution
%% US Naval Oceanographic Office
Данные по рельефу суши основаны на результатах проекта GLOBE, в ходе которого 
по сведениям, предоставленным многими государствами, были построены
цифровые модели рельефа суши с шагом сетки~$0.5'$ ($0.5$~морской мили).
%
% The U.S. National Geophysical Data Center\index{bathymetric charts!ETOPO-2}
% publishes the \textsc{etopo-2 cd-rom} containing digital values of oceanic
% depths from echo sounders and altimetry and land heights from surveys.
% Data are interpolated to a 2-minute (2 nautical mile) grid. Ocean data
% between 64\degrees N and 72\degrees S are from the work of Smith
% and Sandwell (1997), who combined echo-sounder data with altimeter data
% from \textsc{geosat} and \textsc{ers--1}. Seafloor data northward
% of 64\degrees N are from the International Bathymetric Chart
% of the Arctic Ocean.  Seafloor data southward of 72\degrees S are from
% are from the US Naval Oceanographic Office's Digital Bathymetric Data Base
% Variable Resolution. Land data are from the \textsc{globe} Project, 
% that produced a digital elevation model with 0.5-minute (0.5 nautical mile)
% grid spacing using data from many nations. 

%% Рисунок 3.14 Карта глубин океана с разрешением 3 км созданная по
%% данным спутниковых альтиметрических наблюдений поверхности моря. (По
%% Smith and Sandwell).

%% ??? в последней редакции PDF ссылок на этот рисунок нет.
%% кроме того, в новой редакции отсутствует разъяснение, что
%% > ... Несмотря на то, что значения на этой карте нанесены на пятимильную 
%% > сетку, данные использованные при её изготовлении часто имеют гораздо 
%% > большее пространственное разрешение, особенно в Южном Океане, 
%% > где расстояния между маршрутами кораблей в некоторых регионах может 
%% > достигать 500 км.
%% Без него становится непонятной оговорка в разделе 2.5 об осторожности с
%% выбором массивов данных.

Правительства разных стран публикуют карты побережья и гаваней. В США за эту
деятельность отвечает NOAA National Ocean Service, которая выпускает 
навигационные карты гаваней и вод материковой отмели.
%% offshore waters = "вод материковой отмели"
%% http://www.multitran.ru/c/m.exe?CL=1&l1=1&s=offshore+waters
%% но еще и "открытого моря" (там же).
%% кроме того, материковая отмель --- уж не континентальный ли это шельф?
%
% National governments publish coastal and harbor maps. In the USA, the
% \textsc{noaa} National Ocean Service publishes nautical charts useful for
% navigation of ships in harbors and offshore waters.
\end{section}

\begin{section}{Звук в океане}
% \section{Sound in the Ocean}
Звук обеспечивает единственный приемлемый способ передачи информации
на большие расстояния в океане. При помощи звука измеряются характеристики
дна океана и его глубина, а также температура и параметры течений. Киты и
другие животные, обитающие в океане, используют звук для навигации, общения
друг с другом на больших расстояниях и поиска пищи.
%% уместно ли говорить "навигация" по отношению к китам? Они штурманских 
%% классов не заканчивали.
%
% Sound\index{sound!in ocean} provides the only convenient means for 
% transmitting information over great distances in the ocean.
% Sound\index{sound!use of} is used to measure the properties of the sea 
% floor, the depth of the ocean, temperature, and currents. Whales and other
% ocean animals use sound to navigate, communicate over great distances,
% and find food.

%\paragraph{Sound Speed}
Скорость звука в воде зависит от температуры, солёности и 
давления (MacKenzie, 1981; Munk et al. 1995: 33):
\begin{equation}
\begin{split}\label{MacKenzieFormula}
  C & = 1448.96 + 4.591\,t - 0.05304\,t^2 + 0.0002374\,t^3+ 0.01630\,Z \\
    & + (1.340 - 0.01025\,t) (S - 35) + 1.675 \times 10^{-7}\,Z^2 
      - 7.139 \times 10^{-13}\,t\,Z^3
\end{split}
\end{equation}
где $C$~--- скорость в м/с, $t$~--- температура в градусах Цельсия, 
$S$~--- солёность (см.\ определение в гл.~6) в промилле
%% соленость в промилле (http://en.wikipedia.org/wiki/Sound_speed#Seawater)
и~$Z$~--- глубина в метрах. Точность этой формулы примерно~$0.1\mps$
(Dushaw, et al. 1993). Существуют и другие популярные формулы скорости звука,
например, формула Вильсона Wilson (1960), которую широко использовал
военно-морской флот США.
%
% The sound speed \index{sound!speed}in the ocean varies with
% temperature, salinity, and pressure (MacKenzie, 1981; Munk
% et al. 1995: 33):
% \begin{align}
%   C & = 1448.96 + 4.591\,t - 0.05304\,t^2 + 0.0002374\,t^3+ 0.0160\,Z \\
%   &+ (1.340 - 0.01025\,t) (S - 35) + 1.675 \times 10^{-7}\,Z^2 - 7.139 \times
% 10^{-13}\,t\,Z^3 \notag
% \end{align}
% where $C$ is speed in m/s, $t$ is temperature in Celsius, $S$ is salinity 
% (see Chapter 6 for a definition of salinity), and $Z$ is depth in meters.
% The equation has an accuracy\index{accuracy!equation!sound speed} of
% about 0.1 m/s (Dushaw et al. 1993). Other sound-speed equations have been
% widely used, especially an equation proposed by Wilson (1960) which has been
% widely used by the U.S. Navy.

В обычных условиях скорость звука~$C$ составляет от~$1450$ до~$1550\mps$ 
(рис.~3.15). 
Используя формулу~\ref{MacKenzieFormula}, мы можем оценить влияние на скорость 
звука небольших изменений температуры, глубины и солёности, часто
происходящих в океане. Так, скорость звука
изменяется на~$40\mps$ при увеличении температуры на~$10^\circ$~Цельсия, 
на~$16\mps$ при увеличении глубины на~$1000\m$ и на~$1.5\mps$ при
увеличении солёности на~1~промилле. Таким образом, основные причины
изменения скорости звука~--- это температура и глубина (давление). 
Изменения солёности слишком малы, чтобы оказывать существенное влияние.
%
% For typical oceanic conditions, $C$ is usually between 1450 m/s and 1550 m/s 
% (figure 3.15). Using (3.1), we can calculate the sensitivity of $C$ to 
% changes of temperature, depth, and salinity typical of the ocean. 
% The approximate values are: 40 m/s per 10\degrees C rise of temperature, 
% 16 m/s per 1000 m increase in depth, and 1.5 m/s per 1 increase in salinity. 
% Thus the primary causes of variability of sound 
% speed\index{sound!speed!variation of} is temperature and depth (pressure).
% Variations of salinity are too small to have much influence.

Если изобразить на графике скорость звука как функцию глубины, то мы
увидим, что её минимум приходится примерно на~$1000\m$. Водный слой, 
расположенный на этой глубине, получил за свои особые свойства название 
\emph{подводного звукового канала}.  Он присутствует во всех океанах, 
а в высоких широтах обычно выходит на поверхность.
%% в индекс добавить два различных сокращения: DSC/SOFAR
%
% If we plot sound speed\index{sound!speed!as function of depth} as a function 
% of depth, we find that the speed usually has a minimum at a depth around
% 1000 m (figure 3.16). The depth of minimum speed is called the
% \textit{sound channel}\index{sound!channel|textbf}. It occurs in
% all ocean, and it usually reaches the surface at very high latitudes.

Важность подводного звукового канала в том, что звук в нем может 
распространяться очень далеко, иногда проходя половину пути вокруг Земли. 
Кратко, принцип действия подводного звукового канала состоит в следующем: 
звуковые лучи, которые начинают выходить из канала, отражаются обратно 
к его центру. Лучи, распространяющиеся вверх под небольшими углами к 
горизонтали, отражаются книзу, а лучи, распространяющиеся вниз, отклоняются 
кверху соответственно (рис.~3.16). Глубина канала изменяется
от~$10$ до~$1200\m$ в зависимости от местоположения.
%
% The sound channel\index{sound!channel} is important because sound in the
% channel can travel very far, sometimes half way around the earth. Here is 
% how the channel works: Sound\index{sound!rays} rays that begin to travel
% out of the channel are refracted back toward the center of the channel.
% Rays propagating upward at small angles to the horizontal are bent downward,
% and rays propagating downward at small angles to the horizontal are bent
% upward (figure 3.16). Typical depths of the chan\-nel vary from 10 m to
% 1200 m depending on geographical area.

%%Рисунок 3,15 Процессы создающие звуковой канал в океане. Слева
%%Температура Т и солёность S измеренные во время рейса судна Hakuho
%%Maru № KH-87–1, станция JT, 28 января 1987 на широте 33°52.90'N и
%%долготе 141°55.80'E в северной части Тихого Океана. В центре:
%%Изменение скорости звука в зависимости от изменений температуры
%%солёности и глубины. Справа: Звуковой канал на глубине около 1 км,
%%определяемяй как область минимальной скорости звука на графике
%%зависимости скорости звука от глубины. (Данные из JPOTS Editorial
%%Panel, 1991).

\begin{paragraph}{Поглощение звука водной средой.}
Поглощение звука (абсорбция) на единицу расстояния зависит от интенсивности 
звука~$I$:
\begin{equation}
dI = -k I_0 \, dx,
\end{equation}
где $I_0$~--- интенсивность до поглощения, а $k$~--- коэффициент
поглощения, зависящий от частоты звука. Решение данного уравнения:
\begin{equation}
I = I_0 \exp(-kx)
\end{equation}
Типичные значения~$k$ (в децибелах на километр) составляют: $0.08\dBpkm$
при~$1000\Hz$ и~$50\dBpkm$ при~$100\,000\Hz$. Децибелы считаются таким
образом: $\mbox{дБ} = 10 \log(I / I_0)$, где $I_0$~--- первоначальная мощность
звука, $I$~--- мощность звука после поглощения.
%
% Absorption of sound \index{sound!absorption of}per unit distance
% depends on the intensity $I$ of the sound:
% \begin{equation}
% dI = -k I_0 \, dx
% \end{equation}
% where $I_0$ is the intensity before absorption and $k$ is an absorption
% coefficient which depends on frequency of the sound. The equation has the
% solution:
% \begin{equation}
% I = I_0 \exp(-kx)
% \end{equation}
% Typical values of $k$ (in decibels dB per kilometer) are: 0.08 dB/km
% at 1000 Hz, and 50 dB/km at 100,000 Hz. Decibels are calculated from:
% $dB = 10 \log(I/I_0)$, where $I_0$ is the original acoustic power,
% $I$ is the acoustic power after absorption.

Например, пройдя расстояние~$1\km$, сигнал с частотой~$1000\Hz$ ослабнет всего
на~$1.8\%$: $I = 0.982 I_0$. На том же расстоянии сигнал с 
частотой~$100\,000\Hz$ уменьшится на~$I = 10^{-5} I_0$. Таким образом, сигнал 
частотой~$30\,000\Hz$, обычно используемый при эхолотировании морского дна,
совсем немного ослабевает, проходя от поверхности до дна и обратно.
%
% For example, at a range of 1 km a 1000 Hz signal is attenuated by only 
% 1.8\%: $I = 0.982 I_0$. At a range of 1 km a 100,000 Hz signal is reduced
% to $I = 10^{-5} I_0$. The 30,000 Hz signal used by typical echo sounders to
% map the ocean's depths are little attenuated going from the surface to 
% the bottom and back.

Сигналы очень низкой, менее $500\Hz$, частоты были зафиксированы в подводном
звуковом канале на расстоянии мегаметров. В 1960~г.\ звук частотой~$15\Hz$ от
взрывов в подводном звуковом канале у австралийского города Перт был слышен
около Бермудских островов; он прошёл почти полмира. 
Дальнейшие эксперименты показали, что сигнал частотой~$57\Hz$, посланный 
в подводный звуковой канал около острова Херд (\latlon{75}{E}, \latlon{53}{S}), 
может быть зафиксирован на Бермудах в Атлантике и в Монтерее (Калифорния) на
побережье Тихого океана (Munk et al. 1994).
%
% Very low frequency sounds in the sound channel\index{sound!channel}, 
% those with frequencies below 500 Hz have been detected at distances
% of megameters. In 1960 15-Hz sounds from explosions set off in the sound
% channel\index{sound!channel} off Perth Australia were heard in the sound
% channel near Bermuda, nearly halfway around the world. Later experiment
% showed that 57-Hz signals transmitted in the sound channel near Heard
% Island (75\degrees E, 53\degrees S) could be heard at Bermuda in the Atlantic
% and at Monterey, California in the Pacific (Munk et al. 1994).
\end{paragraph}

%% Рисунок 3.16 Звуковые лучи в океане для источника вблизи оси звукового
%% канала. (Взято из Munk et al. 1995).

\begin{paragraph}{Использование звука}
Поскольку низкочастотные звуки распространяются на большие расстояния, 
военно-морской флот США в 1950-х разместил на дне океана систему микрофонов
как в глубоких, так и в мелких водах, подключив их к наземным станциям. Эта
система акустической разведки SOSUS (Sound Surveillance System), первоначально
предназначенная для слежения за подводными лодками, нашла немало и других
применений. Так, она использовалась для поиска и слежения за китами на
расстоянии до~$1\,700\km$, а также для обнаружения подводных вулканических
извержений.
\end{paragraph}
%
% \paragraph{Use of Sound}
% Because low frequency sound \index{sound!use of}can be heard at
% great distances, the US Navy, in the 1950s, placed arrays of
% microphones on the sea floor in deep and shallow water and
% connected them to shore stations. The Sound Surveillance System
% \textsc{sosus}, although designed to track submarines, has found
% many other uses. It has been used to listen to and track whales up
% to 1,700 km away, and to find the location of sub-sea volcanic
% eruptions.
%\end{paragraph}
\end{section}

\begin{section}{Основные концепции}
% \section{Important Concepts}
\begin{enumerate}
\item
Если уменьшить ширину океана до 8 дюймов, то его глубина в том же масштабе
будет соответствовать толщине листа бумаги. Благодаря этому, поля скорости 
в океане близки к двумерным, а вертикальные скорости гораздо меньше
горизонтальных.
%
% \item If the ocean were scaled down to a width of 8 inches it  would have
% depths about the same as the thickness of a piece of paper. As a result, the
% velocity field in the ocean is nearly 2-dimensional. Vertical velocities 
% are much smaller than horizontal velocities.


\item
Количество океанов равно трём.%
\remark{О различных точках зрения на этот вопрос см. примечание 
на стр.~\pageref{remark:threeoceans}.}
%
% \vitem There are only three official ocean.

\item
Объём воды превышает вместительность океанических бассейнов, так что океаны
затапливают побережье континентов, образуя континентальный шельф.
%
% \vitem The volume of ocean water exceeds the capacity of the ocean basins,
% and the ocean overflows on to the continents creating continental shelves.

\item
Измерение глубины океанов и составление карт морского дна производится на 
основе информации, полученной при помощи установленных на судах эхолотов.
Принцип действия эхолота состоит в измерении времени, требуемого звуковому
импульсу для прохождения от поверхности до дна и в обратном направлении.
Карты глубин имеют в некоторых регионах малое пространственное разрешение,
поскольку эти регионы редко посещатся кораблями, так что маршруты измерений
расположены далеко друг от друга.
%
% \vitem The depths of the ocean are mapped by echo sounders which measure
% the time required for a sound\index{sound!used to measure depth} pulse to
% travel from the surface to the bottom and back. Depths measured by ship-based
% echo sounders have been used to produce maps of the sea floor. The maps have
% poor horizontal resolution in some regions because the regions were seldom
% visited by ships and ship tracks are far apart.

\item
Еще одним способом измерения глубин служат спутниковые альтиметрические 
системы, которые профилируют форму поверхности моря. Элементы подводного
рельефа вызывают изменение гравитации в месте своего расположения, что,
в свою очередь, влияет на форму морской поверхности в этом районе.
У современных карт, основанных на спутниковых альтиметрических измерениях 
и данных эхолотирования, ошибка по глубине составляет~$\pm 100\m$, 
а пространственное разрешение~---~$\pm 3\km$.
%
% \vitem The depths of the ocean are also measured by satellite altimeter
% systems which profile the shape of the sea surface. The local shape of the
% surface is influenced by changes in gravity due to sub-sea features. Recent
% maps based on satellite altimeter measurements of the shape of the sea
% surface combined with ship data have depth accuracy of $ \pm $100 m and
% horizontal resolutions of $ \pm $3 km.

\item
Скорость звука в океане составляет обычно~$1480\mps$ и определяется, в
основном, температурой, меньше~--- давлением, и совсем мало~--- 
солёностью. Зависимость скорости звука от температуры и глубины
создаёт в океане подводный звуковой канал, в котором звук может
путешествовать на огромные расстояния. Так, сигнал частотой менее~$500\Hz$
может обойти полмира при условии, что на его пути не встретится суша.
%
% \vitem Typical sound\index{sound!speed!typical} speed in the ocean is 
% 1480 m/s. Speed depends primarily on temperature, less on pressure, and very
% little on salinity. The variability of sound speed as a function of pressure
% and temperature produces a horizontal sound channel in the ocean. Sound in
% the channel can travel great distances. Low-frequency sounds below 500 Hz
% can travel halfway around the world provided the path is not interrupted
% by land.
\end{enumerate}
\end{section}

\end{chapter}
