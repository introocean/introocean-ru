% -*- coding: utf-8 -*-

\begin{chapter}{Физические параметры океана}
%\chapter{The Physical Setting}
Земля~--- это сплюснутый элипсоид, элипсоид вращения с экваториальным
радиусом равным 
$$
R_e=6\,378.1349\mbox{~км (West,1982)},
$$ 
который немного больше полярного радиуса, равного 
$$
 R_p=6\,356.7497\mbox{~км}.
$$
Эта разница образуется за счёт вращения Земли.
%
% Earth \index{earth!radii of}is an oblate ellipsoid, an ellipse rotated about 
% its minor axis, with an equatorial radius of $R_e = 6,378.1349$ km 
% (West, 1982) slightly greater than the polar radius of $R_p = 6,356.7497$ km.
% The small equatorial bulge is due to earth's rotation.
Расстояние на земле измеряется в различных
единицах, наиболее распространёнными являются градусы широты и
долготы, метры, мили и морские мили. Широта~--- это угол между
вертикалью на местности и экваториальной плоскостью. Меридиан~--- это
линия пересечения плоскости перпендикулярной экваториальной плоскости
на земной поверхности и проходящая через ось вращения
Земли. Долгота~--- это угол между нулевым меридианом и любым другим,
где нулевым является меридиан проходящий через Королевскую Гринвичскую
Обсерваторию в Англии. Таким образом долгота измеряется на восток и
запад от Гринвича.
%
% Distances on earth are measured in many different units, the most common 
% are degrees of latitude or longitude, meters, miles, and nautical miles.
% \textit{Latitude}\index{latitude|textbf} is the angle between the local 
% vertical and the equatorial plane. A meridian is the intersection at earth's 
% surface of a plane  perpendicular to the equatorial plane and passing 
% through earth's axis of rotation. \textit{Longitude}\index{longitude|textbf}
% is the angle between the standard meridian and any other meridian, where 
% the standard meridian is the one that passes through a point at the
% Royal Observatory at Greenwich, England.  Thus longitude is measured 
% east or west of Greenwich.

За исключением экватора градус широты на земной поверхности по длинне
отличается от градуса долготы. Широта измеряется вдоль окружности
большого круга с радиусом R, где R~--- средний радиус Земли. Долгота
измеряется вдоль кругов с радиусом Rcos(j), где j~--- широта. Таким
образом 1 градус широты равен 111 км, а один градус долготы =
111cos(j) км. При работе всегда нужно помнить что Земля не сфера и
широта тоже немного изменяется с удалением от экватора, но значения
приведённые здесь вполне достаточны для наших целей.
%
% A degree of latitude is not the same length as a degree of longitude except
% at the equator. Latitude is measured along great circles with radius $R$,
% where $R$ is the mean radius of earth. Longitude is measured along circles 
% with radius $R \cos \varphi$, where $\varphi$ is latitude. Thus 
% $1^{\circ}$ latitude $ = 111$ km, and $1^{\circ}$ longitude 
% $= 111 \cos \varphi$ km.

Так как расстояние в градусах долготы не постоянно, океанографы
измеряют расстояние на картах используя градусы широты.
%
% Because distance in degrees of longitude is not constant, oceanographers 
% measure distance on maps using degrees of latitude.

И морские мили и метры исторически связаны с размерами Земли. В 1670
году викарий церкви Святого Павла в Лионе Габриэль Моутон предложил
десятичную систему измерений, основанную на одной минуте дуги большого
круга Земли. Это привело в конечном счёте к созданию метрической
системы, основанной на другой единице длинны – метре, который
первоначально предполагался равным одной миллионной расстояния от
экватора до полюса вдоль Парижского меридиана. Хотя от взаимосвязи
морских миль и метров с размерами Земли вскоре отказались, ввиду её
непрактичности, приближённые значения до сих пор используют. Например
длинна окружности проходящей через полюс Земли приблизительно равна
2pRe=40,075 км. Следовательно одна десятитысячная четверти этого круга
(квадранта) равна 1,0019 м. Рассуждая подобным образом в случае с
морской милей, получим что она должна быть равна 2pRe/(360*60) = 1.855
км. Это очень близко к официальному определении МЕЖДУНАРОДНОЙ МОРСКОЙ
МИЛИ: 1 мм = 1,852 км.
%
% Nautical miles and meters are connected historically to the size of earth. 
% Gabriel Mouton proposed in 1670 a decimal system of measurement based on 
% the length of an arc that is one minute of a great circle of earth.  
% This eventually became the nautical mile. Mouton's decimal system 
% eventually became the metric system based on a different unit of length,
% the meter, which was originally intended to be one ten-millionth 
% the distance from the Equator to the pole along the Paris meridian. 
% Although the tie between nautical miles, meters, and earth's radius was 
% soon abandoned because it was not practical, the approximations are very 
% good. For example, earth's polar circumference is approximately 40,008 km. 
% Therefore one ten-millionth of a quadrant is 1.0002 m. Similarly,
% a nautical mile should be 1.8522 km, which is very close to the official 
% definition of 
% the\index{nautical mile|textbf}\index{international nautical mile|textbf} 
% \textit{international nautical mile}: 1 nm $\equiv$ 1.8520 km.


\begin{section}{Океаны и моря}
% \section{Ocean and Seas}
Согласно Международному Гидрографическому Стандарту на нашей планете
только три океана.
%
% There is only one ocean. It is divided into three named parts by 
% international agreement: the Atlantic, Pacific, and Indian 
% ocean\index{ocean!defined} (International Hydrographic Bureau, 1953)%
%\index{International Hydrographic Bureau}. Seas, which are part of the ocean, 
% are defined in several ways. I consider two.


\begin{description}
\item[Атлантический Океан] (рис 3,1) простирается на север от
Антарктики и включает всё Арктическое Море (СЛО), Европейское
Средиземноморье и Американское Средиземноморье (Карибское
море). Границей между Атлантическим и Индийским океанами является
меридиан Мыса Игольный (20° E). Граница между Атлантическим и Тихим
Океанами~--- линия между Мысом Горн и Северо Шетландскими
островами. На севере Арктические моря являются частью Атлантического
океана, а Беренгов пролив~--- границей между Атлантикой и Тихим
Океаном.
%
% \textbf{The Atlantic Ocean} \index{ocean!Atlantic Ocean}extends
% northward from Antarctica and includes all of the Arctic Sea, the
% European Mediterranean, and the American Mediterranean more
% commonly known as the Caribbean sea (figure 3.1). The boundary
% between the Atlantic and Indian Ocean is the meridian of Cape
% Agulhas (20\degrees E). The boundary between the Atlantic and
% Pacific is the line forming the shortest distance from Cape
% Horn to the South Shetland Islands. In the north, the Arctic Sea
% is part of the Atlantic Ocean, and the Bering Strait is the
% boundary between the Atlantic and Pacific.


%% Рисунок 3.1 Атлантический Океан в равноплощадной проэкции. Глубины в
%% метрах, взяты из ETOPO 30. Изобата 200 метров показывает границу
%% континетального щельфа.

\item[Тихий Oкеан] простирается на север от Антарктики до Беренгова
пролива. Граница между Тихим и Индийским океаном лежит на линии идущей
от Малайского Полуострова через Суматру, Яву, Тимор, до австралийского
мыса Лондондерри, а от Тасмании до Антарктики на меридиане мыса Северо
Восточный (147 ° E).
%
% \textbf{The Pacific Ocean} \index{ocean!Pacific Ocean}extends
% northward from Antarctica to the Bering Strait (figure 3.2). The
% boundary between the Pacific and Indian Ocean follows the line
% from the Malay Peninsula through Sumatra, Java, Timor, Australia
% at Cape Londonderry, and Tasmania. From Tasmania to Antarctica it
% is the meridian of South East Cape on Tasmania 147\degrees E.


%% Рисунок 3.2 Тихий Океан в равноплощадной проэкции. Глубины в метрах,
%% взяты из ETOPO 30. Изобата 200 метров показывает границу
%% континетального щельфа.

\item[Индийский Океан] простирается от Антарктики до Евразийского
континента, включая в себя Красное Море и Персидский Залив. Некоторые
авторы используют название Южный Океан для вод вокруг Антарктиды.
%
% \textbf{The Indian Ocean} \index{ocean!Indian Ocean}extends from
% Antarctica to the continent of Asia including the Red Sea and
% Persian Gulf (figure 3.3). Some authors use the name Southern
% Ocean to describe the ocean surrounding Antarctica.



%% Рисунок 3.2 Индийский Океан в равноплощадной проэкции. Глубины в
%% метрах, взяты из ETOPO 30. Изобата 200 метров показывает границу
%% континетального щельфа.
\end{description}

Существует много типов морей. Мы упомянем о двух:
\begin{itemize}
\item 
Средиземные Mоря большей частью окружены сушей. Согласно этому
определению Арктическое и Карибское моря~--- средиземные, Арктическое
Средиземное и Карибское Средиземное.
%
% \textbf{Mediterranean Seas} \index{seas!Mediterranean}are mostly
% surrounded by land. By this definition, the Arctic and Caribbean
% Seas are both Mediterranean Seas, the Arctic Mediterranean and the
% Caribbean Mediterranean.

\item
Окраинные Mоря определяются только изрезанностью побережья.
%
% \textbf{Marginal Seas} \index{seas!marginal}are defined by only an
% indentation in the coast. The Arabian Sea and South China Sea are
% marginal seas.
\end{itemize}
\end{section}


\begin{section}{Размеры океанов}
% \section{Dimensions of the ocean}
Океаны и моря покрывают 70,8\% земной поверхности, что составляет 361
254 000 квадратных километров. Площади океанов значительно различаются
(таблица 3,1) и Тихий из них самый большой.
%
% \index{ocean!dimensions of}The ocean and seas cover 70.8\% 
% of the surface of earth, which amounts to 361,254,000 km$^2$. 
% The areas of the named parts vary considerably (table 3.1).

\begin{tabular}{lr}
%% Таблица 3.1. Площадь Океанов (Dietrich, et al. (1980: стр3)
Тихий Океан         & $181.34 \times 10^6 \mbox{Км}^2$ \\
Индийский Океан     & $ 74.12 \times 10^6 \mbox{Км}^2$ \\
Атлантический Океан & $106.57 \times 10^6 \mbox{Км}^2$ \\
\end{tabular}
%
% \begin{table} [b!]\centering \small
% \vspace{-3ex}
% \begin{tabular*}{65mm}{@{}l @{\extracolsep{\fill}} r@{}}
% \multicolumn{2}{@{}l@{}}{\bfseries Table 3.1 Surface Area of the ocean} $^{\dag }$ \\
% \hline
% \rule{0ex}{2.5ex}Pacific Ocean  & $181.34 \times 10^6 \hbox{ km}^2$        \\
%                  Atlantic Ocean   & $ 106.57 \times 10^6 \hbox{ km}^2$        \\
%                 Indian Ocean  & $74.12 \times 10^6 \hbox{ km}^2$        \\[0.5ex]
% \hline
% \multicolumn{2}{@{}l@{}}  {\rule{0ex}{2.5ex}$^{\dag }$ From Menard and Smith (1966)}
% \end{tabular*} \\[0.5ex]
% \vspace{-3ex}
% \end{table}


Горизонтальные размеры океанов изменяются от 1500~км~--- минимальной
ширины Атлантики, до 13000~км~--- простирания Атлантики с севера на юг
и ширины Тихого Океана.  При этом типичные глубины составляют 3--4~км.
Таким образом горизонтальные размеры океанских бассейнов в 1000~раз
больше чем вертикальные.  Масштабы Тихого океана можно представить
себе с помощью обычного листа бумаги 8,5*11 дюймов (А4), при этом
ширина океана в 10~000~км будет соответствовать 10~дюймам~--- ширине
листа, а 3х километровая глубина 0,003~дюймам~--- типичной толщине
листа бумаги.
%
% Oceanic dimensions range from around 1500 km for the minimum width of the
% Atlantic to more than 13,000 km for the north-south extent of the Atlantic 
% and the width of the Pacific. Typical depths are only 3--4 km. So horizontal
% dimensions of ocean basins are 1,000 times greater than the vertical
% dimension. A scale model of the Pacific, the size of an $8.5 \times 11$ in 
% sheet of paper, would have dimensions similar to the paper: a width 
% of 10,000 km scales to 10 in, and a depth of 3 km scales to 0.003 in, 
% the typical thickness of a piece of paper.


Из за того что океаны такие тонкие, графики их поперечного сечения для
удобства использования должны иметь сильно преувеличенный вертикальный
масштаб. Обычно у таких графиков вертикальный масштаб в 200 раз больше
чем горизонтальный (Рис. 3,4). Это преувеличение искажает наши
представления об океане. Края океанических бассейнов (континентальные
склоны) не крутые обрывы как показано на рисунке 3.4 (41W 12 E). Они
скорее являются пологими склонами, понижающимися на 1 метр по
вертикали на каждые 20 метров по горизонтали.
%
% Because the ocean is so thin, cross-sectional plots of ocean basins must 
% have a greatly exaggerated vertical scale to be useful. Typical plots have
% a vertical scale that is 200 times the horizontal scale (figure 3.4). This 
% exaggeration distorts our view of the ocean. The edges of the ocean basins, 
% the continental slopes, are not steep cliffs as shown in the figure 
% at 41\degrees W and 12\degrees E. Rather, they are gentle slopes dropping 
% down 1 meter for every 20 meters in the horizontal.

Малое отношение глубин океанических бассейнов к их ширине имеет свои
последствия для динамики. Вертикальные скорости должны быть гораздо
слабее чем горизонтальные. Даже при масштабах движения в несколько
сотен километров вертикальные скорости должны составлять порядка 1\%
от горизонтальных. Мы используем эту информацию позже для того чтобы
упростить уравнение движения.
%
% The small ratio of depth to width of the ocean basins is very important 
% for understanding ocean currents. Vertical velocities must be much smaller
% than horizontal velocities. Even over distances of a few hundred kilometers,
% the vertical velocity must be less than 1\% of the horizontal velocity.
% I will use this information later to simplify the equations of motion.

На первый взгляд относительно малые значения вертикальных скоростей
должны мало влиять на динамику, но всё меняется когда мы начинаем
принимать во внимание турбулентность. Трёхмерная турбулентность сильно
отличается от двухмерной. В двухмерной турбулентности вихревые линии
всегда должны быть вертикальны и здесь может быть только небольшое
растяжение вихря. В трёхмерной же турбулентности растяжение вихря
играет фундаментальную роль.
%
% The relatively small vertical velocities have great influence on
% turbulence\index{turbulence}. Three dimensional turbulence is fundamentally
% different than two-dimensional turbulence\index{turbulence!two dimensional}.
% In two dimensions, vortex lines must always be vertical, and there can be
% little vortex stretching. In three dimensions, vortex stretching plays
% a fundamental role in turbulence.

%% Рисунок 3.4 Профиль Северной Атлантики вдоль 25°S демонстрирующий
%% континетальный шельф Южной Америки, подводную гору около 35°W,
%% срединный Атлантический Хребет около 14°W, Хребет Вальвис около 6°E и
%% узкий континетальный шельф Южной Африки. Верхний: Вертикальное
%% увеличение масштаба 180:1. Нижний: Вертикальное увеличение масштаба
%% 30:1. Если нарисовать график в действительной пропорции, то он будет
%% тоньше чем линия обозначающая поверхность моря на нижнем графике
%% рисунка.
\end{section}

\begin{section}{Элементы рельефа}
% \section{Sea-Floor Features}
Земная кора делится на два типа: регионы с тонкой, около 10 км
корой~--- океаны, и регионы с толстой около 40 км корой~--- 
континенты. Блоки земной коры плавают в более плотном материале мантии
и средняя высота их поверхности относительно уровня моря имет два
различных значения: континенты в среднем возвышаются на 840 м, а дно
океанов погружено на 3 432 м (Рис 3,5)
%
% Earth's rocky surface is divided into two types: oceanic, with a thin
% dense crust about 10 km thick, and continental, with a thick light crust 
% about 40 km thick. The deep, lighter continental crust floats higher on 
% the denser mantle than does the oceanic crust, and the mean height of
% the crust relative to sea level has two distinct values: continents
% have a mean elevation of 1100 m, the ocean has a mean depth of -3400 m 
% (figure 3.5).

%% Рисунок 3.5 Слева: Гистограмма превышений суши и глубины дна океана в
%% процентном отношении к площади Земли. Видно явное различие между
%% континентами и морским дном. Справа: Гипсографическая кривая. Кривые
%% посчитаны по данным ETOPO 30.

Объём воды в океанах превышает объём океанических бассейнов и часть
воды распространяется (разливается) над опущенными частями
континентов. Эти мелководные моря называются континентальными
шельфами. Некоторые, такие как Северо Китайское море, больше 1000~км
шириной. Большинство из них относительно мелководны, с типичными
глубинами 50--100~м. Наиболее важными являются Восточно Китайское
Море, Берингово Море, Северное Море, Большая Багамская Банка, Шельф
Патагонии, Арафурское Море и залив Карпентария а также Сибирский
Шельф. Мелководные моря помогают рассеиванию (дессипации) приливов,
они часто являются зонами высокой продуктивности и входят в особую
экономическую зону близлежащих стран.
%
% The volume of the water in the ocean exceeds the volume of the ocean basins, 
% and some water spills over on to the low lying areas of the continents. These
% shallow seas are the continental shelves. Some, such as the South China Sea,
% are more than 1100 km wide. Most are relatively shallow, with typical depths 
% of 50--100 m. A few of the more important shelves are: the East China Sea,
% the Bering Sea, the North Sea, the Grand Banks, the Patagonian Shelf,
% the Arafura Sea and Gulf of Carpentaria, and the Siberian Shelf. The shallow
% seas help dissipate tides, they are often areas of high biological
% productivity, and they are usually included in the exclusive economic zone
% of adjacent countries.

Земная кора разделена на большие плиты которые движутся относительно
друг друга. Новая кора создаётся в срединно океанических хребтах, а
старая исчезает в глубоководных желобах. Относительное движение
литосферных плит создаёт большое количество элементов морского
дна. Эти элементы, изображённые на рисунке 3,6, включают в себя
срединно океанические хребты, глубоководные желоба, островные дуги,
бассейны и подводные горы.
%
% The crust is broken into large plates that move relative to each other. New
% crust is created at the mid-ocean ridges, and old crust is lost at trenches.
% The relative motion of crust, due to plate tectonics, produces the
% distinctive features of the sea floor sketched in figure 3.6, including
% mid-ocean ridges, trenches, island arcs, and basins. 
% \index{ocean!features of|(}The names of the sub-sea features have been
% defined by the International Hydrographic
% Organization\index{International Hydrographic Bureau} (1953), and
% the following definitions are taken from Sverdrup, Johnson, 
% and Fleming (1942), Shepard (1963), and Dietrich et al. (1980).

%% Рисунок 3.6 Схематический профиль океана демонстрирующий основные
%% элементы морского дна. Обратите внимание на то что уклоны сильно
%% преувеличены.

Названия элементов рельефа морского дна были оговорены Международной
Гидрографической комиссией и следующие определения взяты из Dietrich
et. al.(1980).

\begin{description}
\item[Бассейн] 
Понижение морского дна более менее выравненной формы и различной
протяжённости.
%
% \textit{Basins} \index{basins|textbf}are deep depressions of the sea floor
% of more or less circular or oval form.

\item[Каньон]
Относительно узкое глубокое понижение с крутыми склонами, глубина
которого постоянно увеличивается книзу.
%
% \textit{Canyons} \index{canyon|textbf}are relatively narrow, deep furrows 
% with steep slopes, cutting across the continental shelf and slope, with
% bottoms sloping continuously downward.

\item[Континентальный Шельф]
Зона смежная с континентом (или вокруг острова), простирающаяся от
горизонта меженных вод (минимального наблюдённого или расчитанного
уровня моря) до глубины которая обычно отмечена увеличением уклона в
сторону больших глубин.
%
% \textit{Continental shelves} \index{continental shelves|textbf}are zones 
% adjacent to a continent (or around an island) and extending from
% the low-water line to the depth, usually about 120 m, where there is a marked
% or rather steep descent toward great depths. (figure 3.7)
 
\item[Континентальный Склон]
Уклон в сторону моря от границы шельфа к большим глубинам.
%
% \textit{Continental slopes} \index{continental slopes|textbf}are
% the declivities seaward from the shelf edge into greater depth.

\item[Равнина]
Плоская, слабо покатая или близкая к равнинному часть морского
дна. (Например абиссальная равнина).
%
% \textit{Plains} \index{plains|textbf}are very flat surfaces found in many
% deep ocean basins.

\item[Хребет]
Вытянутое узкое поднятие морского дна с крутыми склонами и
неравномерной (нерегулярной) топографией.
%
% \textit{Ridges} \index{ridges|textbf}are long, narrow elevations of the sea
% floor with steep sides and rough topography.

\item[Подводная Гора]
Изолированное или относительно изолированное поднятие, возвышающееся
на 1000 метров и более над дном океана, ограниченное вершиной.
%
% \textit{Seamounts} \index{seamounts|textbf}are isolated or comparatively
% isolated elevations rising 1000 m or more from the sea floor and with small
% summit area (figure 3.8).

\item[Разлом]
Нижняя часть хребта, отделяющая океанические бассейны друг от друга
или от близлежащего морского дна.
%
% \textit{Sills} \index{sills|textbf}are the low parts of the ridges separating
% ocean basins from one another or from the adjacent sea floor.

\item[Глубоководный Желоб]
Протяжённое, узкое и глубокое понижение морского дна с относительно
крутыми склонами.
%
% \textit{Trenches} \index{trenches|textbf}are long, narrow, and deep
% depressions of the sea floor, with relatively
% steep sides (figure 3.9).\index{ocean!features of|)}
\end{description}

Подводные элементы оказывают важное влияние на циркуляцию
океанов. Хребты в районе разломов (рифтовых долин) разделяют глубинные
воды океанов на отдельные бассейны. Вода находящаяся глубже разлома не
может перемещаться из одного бассейна в другой. Десятки тысяч
изолированных пиков~--- подводных гор, разбросаны по дну океана. Они
преграждают путь течениям и вызывают турбулентность, которая приводит
к вертикальному перемешиванию вод.
%
% Sub-sea features strongly influences the ocean circulation.
% Ridges separate deep waters of the ocean into distinct basins.
% Water deeper than the sill\index{sills} between two basins cannot move
% from one to the other. Tens of thousands of seamounts are scattered
% throughout the ocean basins. They interrupt ocean currents, and produce
% turbulence\index{turbulence!in deep ocean} leading to vertical
% mixing\index{mixing!vertical} in the ocean.

%%Рисунок 3.7 Пример континентального шельфа, шельф у побережья Монтерея
%%в Калифорнии, здесь можно видеть каньон Монтерей и другие. Каньоны
%%часто встречаются на шельфе и обычно простираются через весь шельф и
%%континентальный склон. Права на рисунок принадлежат Monterey Bay
%%Aquarium Research Institute (MBARI).

%%Рисунок 3.8 Пример подводной горы~--- гайот Вилд. Гайот~--- это
%%морская гора с плоской вершиной, а плоская она из за волнового
%%воздействия происходившего пока гора находилась над уровнем моря. Так
%%как морская гора зависит от тектоники плит, то она понемногу
%%погружается. Глубины были посчитаны на основе данных эхолокации вдоль
%%маршрутов судна (тонкие прямые линии), дополненными данными
%%гидролокатора бокового обзора. Глубина в сотнях метров.

%%Рисунок 3.9 Пример глубоководного жёлоба~--- Алеутский Желоб;
%%островная дуга, Алеутские Острова и континентальный шельф, Берингово
%%море. Островная дуга состоит из вулканов образовавшихся когда
%%океаническая кора погружаясь в желоб, плавилась, и поднималась к
%%поверхности. Наверху Карта Алеутского региона на севере Тихого
%%Океана. Внизу профиль через регион.
\end{section}

\begin{section}{Измерения глубин океана}
% \section{Measuring the Depth of the Ocean}
Глубина океана может быть измерена двумя способами 1) эхолокатором
установленном на судне, или 2) спутниковыми альтиметрами.
% The depth of the ocean is usually measured two ways: 1) using acoustic
% echo-sounders on ships, or 2) using data from satellite altimeters.

\begin{subsection}{Эхолокаторы.}
Большинство карт океана созданы на основе измерений сделанных
эхолокаторами. Этот прибор посылает звуковой импульс частотой
10--30~кГц и принимает сигнал отражённый от морского дна. Временной
интервал между посылом импульса и приходом эха, умноженный на скорость
звука, даёт удвоенную глубину океана. Первое трансатлантическое
эхолотирование было выполнено американским эсминцем «Стюарт» в 1922
году. Первые систематические промеры были выполнены германским
исследовательским судном «Метеор» во время его экспедиции в северную
Атлантику 1925--1927 годов. Теперь океанографические и военные суда
во время плавания практически непрерывно производят
эхолотирование. Милионы миль вдольпутевых данных записанных на бумагу
были оцифрованы для того чтобы создать базы данных на основе которых и
делаются батиметрические карты. Распределение судовых маршрутов по
поверхности океана неравномерно. В южном полушарии они пролегают
довольно далеко друг от друга, даже возле Австралии, а в уже хорошо
картированных районах, таких как Северная Атлантика, довольно близко.
%
% \paragraph{Echo Sounders} \index{echo sounders|(}Most maps of the ocean
% are based on measurements made by echo sounders. The instrument transmits
% a burst of 10--30 kHz sound\index{sound!used to measure depth} and listens
% for the echo from the sea floor. The time interval between transmission
% of the pulse and reception of the echo, when multiplied by the velocity
% of sound, gives twice the depth of the ocean (figure 3.10).
%
% The first transatlantic echo soundings were made by the U.S. Navy
% Destroyer \textit{Stewart} in 1922. This was quickly followed by
% the first systematic survey of an ocean basin, made by the German
% research and survey ship \textit{Meteor} during its expedition to
% the south Atlantic from 1925 to 1927. Since then, oceanographic
% and naval ships have operated echo sounders almost continuously
% while at sea. Millions of miles of ship-track data recorded on
% paper have been digitized to produce data bases used to make maps.
% The tracks are not well distributed. Tracks tend to be far apart
% in the southern hemisphere, even near Australia (figure 3.11) and
% closer together in well mapped areas such as the North
% Atlantic.\index{echo sounders|)}

Измерения глубин эхолотированием широко используются, но у этого
метода есть свои ошибки.
\begin{enumerate}
\item
Скорость звука изменяется на $\pm 4\mbox{\%}$ в разных районах
океана. Используя таблицы средних скоростей звука можно уменьшить
ошибку измерений до $\pm 1\mbox{\%}$. Смотри параграф 3,6 для большей
информации о звуке в океане.

\item
От малых глубин эхо может прийти не точно под корабль, а на его
борт. Это может вызвать небольшие ошибки в холмистых районах.

\item
Местоположение корабля плохо определялось до появления в шестидесятых
спутниковой навигации. Ошибки могли составлять десятки километров
особенно в облачных регионах где невозможны астрономические
наблюдения.
 
\item
Иногда скопления зоопланктона и косяки рыбы в неглубоких райнонах
вызывали ошибки, приводившие к появлению на некоторых батиметрических
картах ложных подводных гор. Эта ошибка устраняется путём повторного
исследования спорных мест.

\item
Некоторые районы океана (размером до 500 километров) ни разу не были
исследованы эхолокаторами. Это создаёт значительные пробелы в наших
знаниях об океанских глубинах
\end{enumerate}

%% Рисунок 3.10 Слева: Эхолокаторы измеряют глубину океана посылая
%% звуковой импульс и измеряя время затраченное им чтобы отразится от
%% поверхности и вернутся обратно. Справа: Время записывается с помощью
%% иглы оставляющей след на медленно движещемся рулоне бумаги. (From
%% Dietrich, et al. 1980)


%% Рисунок 3.11 Расположение данных эхолотирования использованных для
%% картирования океана около Австралии. Заметте что имеются большие
%%пространства где нет данных.

% Echo sounders \index{echo sounders!errors in measurement}make the most 
% accurate measurements of ocean depth.
% Their accuracy\index{accuracy!echo sounders} is $\pm$1\%.
\end{subsection}

\begin{subsection}{Спутниковая Альтиметрия}

Пробелы в наших знаниях о глубинах океана между маршрутами судов
теперь заполнены данными спутниковой альтиметрии. Альтиметры измеряют
(профилируют) форму морской поверхности, а форма морской поверхности
очень похожа на форму морского дна. Чтобы понять почему это
происходит, мы вначале должны обсудить то как гравитация влияет на
уровень моря.
%
% \paragraph{Satellite Altimetry}
% \index{satellite altimetry!use in measuring depth} Gaps in our knowledge
% of ocean depths between ship tracks have now been filled
% by satellite-altimeter data. Altimeters profile the shape of the sea surface,
% and its shape is very similar to the shape of the sea floor
% (Tapley and Kim, 2001; Cazenave and Royer, 2001; Sandwell and Smith, 2001).
% To see this, we must first consider how gravity influences sea level.

Взаимоотношение между уровнем моря и рельефом дна.
% \textit{The Relationship Between Sea Level and the Ocean's Depth}
Избыток мыссы на дне океана, например масса горы, увеличивает местную
гравитацию, поскольку масса горы больше массы воды которую она
замещает, ведь камень в три раза плотнее воды. Избыток массы
увеличивает местную гравитацию, которая притягивает воду к подводной
горе. Это изменяет форму морской поверхности (Рис 3,12).
%
% Excess mass at the sea floor, for example the mass of a seamount, increases
% local gravity because the mass of the seamount is larger than the mass
% of water it displaces. Rocks are more than three times denser than water.
% The excess mass increases local gravity, which attracts water toward
% the seamount. This changes the shape of the sea surface (figure 3.12).

Давайте рассмотрим это более подробно. В первом приближении
поверхность моря~--- частный случай уровенной поверхности называемой
геоидом (смотри блок ниже). По определению уровенная поверхность везде
перпендикулярна силе тяжести. В частности она должна быть
перепендикулярна отвесной линии в данном конкретном месте (которая
определяется путём подвешивания какой нибудь массы (грузика) на
верёвке). Таким образом отвесная линия перпендикулярна локальной
уровенной поверхности и используется (в особенности геодезистами и
топографами) для того чтобы определить её положение.
%
% Let's make the concept more exact. To a very good approximation, 
% the sea surface is a particular \textit{level surface}
% \index{level surface|textbf}called the \textit{geoid} (see box).
% By definition a level surface is a surface of constant gravitational
% potential, and it is everywhere perpendicular to gravity. In particular,
% it must be perpendicular to the local vertical determined by a plumb line,
% which is ``a line or cord having at one end a metal weight for determining
% vertical direction'' (Oxford English Dictionary).

Избыток массы подводной горы притягивает грузик отвеса, заставляя
линию отвеса немного отклонятся от центра масс Земли в сторону
горы. Так как поверхность моря должна быть перепендикулярна силе
тяжести, над подводной горой будет находиться небольшая вспученность
как показано на рисунке. Если бы её не было, поверхность моря не была
бы перпендикулярна силе тяжести. Обычные подводные горы вызывают
вспученности высотой 1--20~м на расстоянии 100--200~км. Конечно это
очень мало и засечь с корабля такие изменения невозможно, однако
альтиметром это сделать довольно просто. Глубоководные желоба вызывают
дефицит масс и создают понижения морской поверхности.
%
% The excess mass of the seamount attracts the plumb line's weight, causing
% the plumb line to point a little toward the seamount instead of toward
% earth's center of mass. Because the sea surface must be perpendicular
% to gravity, it must have a slight bulge above a seamount as shown
% in figure 3.12. If there were no bulge, the sea surface would not be
% perpendicular to gravity. Typical seamounts produce a bulge that is 1--20 m
% high over distances of 100--200 kilometers. This bulge is far too small
% to be seen from a ship, but it is easily measured by satellite altimeters.
% Oceanic trenches have a deficit of mass, and they produce a depression
% of the sea surface.

Взаимосвязь между формой морской поверхности и глубиной не очень
строгая. Она зависит от расчленённости дна и возраста его
элементов. Если подводная гора как бы колышится на поверхности дна,
словно лёд на воде, то гравитационный сигнал будет слабее чем если бы
она покоилась на дне, как лёд лежащий на столе. В результате
взаимосвязь силы тяжести и рельефа дна изменяется от места к месту.
%
% The correspondence between the shape of the sea surface and the depth
% of the water is not exact. It depends on the strength of the sea floor,
% the age of the sea-floor feature, and the thickness of sediments.
% If a seamount floats on the sea floor like ice on water, the gravitational
% signal is much weaker than it would be if the seamount rested on the sea
% floor like ice resting on a table top. As a result, the relationship
% between gravity and sea-floor topography varies from region to region.

Глубина измеряемая эхолотами используется для того чтобы определить
эту взаимосвязь. Затем с помощью альтиметрии проводится интерполяция
между измерениями эхолотов. Используя этот способ можно расчитать
глубины океана с точностью до $\pm 100$~метров.
%
% Depths measured by acoustic echo sounders are used to determine the regional
% relationships. Hence, altimetry is used to interpolate between acoustic echo
% sounder measurements (Smith and Sandwell, 1994).


\begin{subsubsection}{Системы спутниковой альтиметрии}
Теперь посмотрим каким образом альтиметры измеряют форму земной
поверхности. Системы спутниковой альтиметрии включают в себя радар для
измерения высоты спутника над земной поверхностью и систему слежения
для определения высоты спутника в геоцентрической системе
координат. Система измеряет превышение уровня моря относительно центра
масс Земли (Рис 3,13). Таким образом получается форма морской
поверхности.
%
% \textit{Satellite-altimeter systems}
% \index{satellite altimetry!systems|textbf}Now let's see how altimeters
% measure the shape of the sea surface. Satellite altimeter systems include
% a radar to measure the height of the satellite above the sea surface and
% a tracking system to determine the height of the satellite in geocentric
% coordinates. The system measures the height of the sea surface relative
% to the center of mass of earth (figure 3.13). This gives the shape of the sea
% surface.

В космосе находится много альтиметрических спутников. Все они обладают
достаточной точностью для того чтобы изучать морской геоид и влияние
на него элементов подводного рельефа. Обычно она варьирует от
нескольких метров для прибора спутника GEOSAT до $\pm 0,05$~м для прибора
спутника TOPEX/POSEIDON. Наиболее используемые спутники это Seasat
(1978), GEOSAT (1985–1988), ERS-1 (1991--1996), ERS-2(1995-), и
TOPEX/POSEIDON (1992-). У спутников Seasat, ERS-1 и ERS-2, также есть
инструменты для измерения ветра, волнения и других
параметров. TOPEX/POSEIDON и GEOSAT~--- преимущественно
альтиметрические спутники.
%
% Many altimetric satellites have flown in space. All observed the marine
% geoid\index{geoid} and the influence of sea-floor features on
% the geoid\index{geoid}. The altimeters that produced the most useful
% data include Seasat (1978)\index{Seasat}, \textsc{geosat} (1985--1988),
% \textsc{ers}--1\index{ERS satellites} (1991--1996),
% \textsc{ers}--2 (1995-- ), Topex/Poseidon\index{Topex/Poseidon} (1992--2006),
% Jason\index{Jason} (2002--), and Envisat (2002)\index{Envisat}.
% Topex/Poseidon and Jason were specially designed to make extremely accurate
% measurements of sea-surface height. They measure sea-surface height with
% an accuracy
% of $\pm 0.05$ m\index{Jason!accuracy of}\index{Topex/Poseidon!accuracy of}.

\end{subsubsection}


%% Рисунок 3.13 Спутниковый альтиметр измеряет высоту спутника над
%% уровнем моря. При вычитании этого значения из высоты орбиты спутника
%% получим уровень моря относительно центра Земли. Форма поверхности
%% изменяется под воздействием вариаций силы тяжести, которые вызывают
%% индуляции геоида, и под воздействием океанаских течений, которые
%% приводят к образованию океанической топографии, отклонениям
%% поверхности моря от геоида. Референц элипсоид~--- наиболее близкая
%% сглаженная апроксимация геоида.

\begin{subsubsection}{Спутниковые альтиметрические карты дна.}
Seasat, Geosat, ERS-1 и ERS-2 были запущены на орбиту с целью
картирования морского геоида. Их орбиты располагались таким образом
что расстояние между маршрутами измерений на повоерхности составляло
3--10~км, что достаточно для картирования геоида. Первые измерения
сделанные спутником GEOSAT были засекречены американскими военными. Но
к 1996 году геоид был картирован европейцами и американцы открыли
данные GEOSATа. В результате сравнения данных со всех альтиметрических
спутников были уменьшены ошибки связанные с приливами и течениями и
создана карта геоида с пространственным разрешением 3км.
\end{subsubsection}
% 
% \textit{Satellite Altimeter Maps of the Sea-floor Topography}
% Seasat\index{Seasat}, \textsc{geosat}\index{Geosat}, \textsc{ers}--1,
% and \textsc{ers}--2\index{ERS satellites} were
% \index{satellite altimetry!maps of the sea-floor topography}operated
% in orbits with ground tracks spaced 3--10 km apart, which was sufficient
% to map the geoid\index{geoid}. By combining data from echo-sounders
% with data from \textsc{geosat} and \textsc{ers}--1 altimeter systems,
% Smith and Sandwell (1997) produced maps of the sea floor with horizontal
% resolution of 5--10 km and a global average depth accuracy of $\pm 100$ m.

\end{subsection}

%% Box: caption{Геоид}
Уровенная поверхность соответствующая невозмущённому уровню моря
называется геоидом. В первом приближении геоид это элипсоид
соответствующий поверхности однородной (не имеющей внутренних течений)
жидкости на твёрдом вращающемся теле. Во втором приближении геоид
отличается от элипсоида из за локальных неоднородностей силы
тяжести. Эти отклонения называются индуляциями геоида. Максимальная их
амплитуда ориентировочно равна~$\pm 60$~м. В третьем приближении геоид
отличается от поверхности моря, поскольку океаны далеко не
спокойны. Отклонения уровня моря от геоида называют
топографией. Обозначают её так же как наземную топографию, например
высотой нанесённой на топографическую карту.
%
% The \index{geoid|textbf}level surface\index{level surface} that corresponds
% to the surface of an ocean at rest is a special surface, the \textit{geoid}.
% To a first approximation, the geoid\index{geoid} is an ellipsoid that
% corresponds to the surface of a rotating, homogeneous fluid in solid-body
% rotation, which means that the fluid has no internal flow. To a second
% approximation, the geoid differs from the ellipsoid because of local
% variations in gravity. The deviations are called \textit{geoid undulations}.
% \index{geoid!undulations|textbf}The maximum amplitude of the undulations is
% roughly $\pm 60$ m. To a third approximation, the geoid deviates from the sea
% surface because the ocean is not at rest. The deviation of sea level from
% the geoid\index{geoid} is defined to be the \textit{topography}.
% \index{topography|textbf}The definition is identical to the definition
% for land topography, for example the heights given on a topographic map.

Топография океана вызывается приливами и океанскими поверхностными
течениями, в этой связи мы вернёмся к ней в главах 10 и
18. Максимальная амплитуда топографии составляет приблизительно 
$\pm 1$~м, таким образом она мало сравнима с индуляциями геоида.
%
% The ocean's topography is caused by tides, heat content of the water,
% and ocean surface currents. I will return to their influence in chapters 10
% and 17. The maximum amplitude of the topography is roughly $\pm 1$ m, so it
% is small compared to the geoid\index{geoid} undulations.

Индуляции геоида вызваны локальными вариациями силы тяжести,
вследствии необычнного распределения массы на дне океана. У подводных
гор наблюдается избыток масс благодаря их плотности и они образуют
выпуклости на геоиде (смотри ниже). У глубоководных желобов
наблюдается дефицит масс и они вызывают прогибы геоида. Таким образом
геоид взаимосвязан с рельефом дна и карты морского геоида имеют
заметное сходство с батиметрическими.
%
% Geoid  undulations are caused by local variations in gravity due to
% the uneven distribution of mass at the sea floor. Seamounts have an excess
% of mass because they are more dense than water. They produce an upward bulge
% in the geoid (see below). Trenches have a deficiency of mass. They produce
% a downward deflection of the geoid\index{geoid}. Thus the geoid\index{geoid}
% is closely related to sea-floor topography. Maps of the oceanic
% geoid\index{geoid} have a remarkable resemblance to the sea-floor topography.



%%Рисунок 3.12 Подводная гора гораздо плотнее чем морская вода, поэтому
%%она увеличивает локальную силу тяжести и заставляет отвесную линию
%%(стрелочки) отклоняться в сторону горы. Так как поверхность океана в
%%спокойном состоянии должна быть перпендикулярна силе тяжести,
%%поверхность моря и геоид в этом месте должны иметь небольшую
%%выпуклость как показано на рисунке. Эта выпуклость легко измеряется
%%спутниковыми альтиметрами. Следовательно данные альтиметров могут
%%использоваться для картирования морского дна. Следует понимать что
%%выпуклость поверхности моря на рисунке сильно преувеличена, подводная
%%гора высотой в 2 километра вызывает выпуклость высотой приблизительно
%%10 метров.

%% End of box
\end{section}

\begin{section}{Батиметрические карты и базы данных}
% \section{Sea Floor Charts and Data Sets}
Данные эхолотирования были оцифрованы, нанесены на карты, по ним
построили изолинии и создали батиметрические карты. В дальнейшем их
обработали для создания цифровых баз данных, которые теперь широко
распространены на CD-ROM. Эти данные были дополнены данными
альтиметрических спутников для того чтобы создать карты морского дна с
пространственным разрешением 3км.
%
% \index{ocean!maps of}Almost all echo-sounder data have been digitized
% and combined to make sea-floor charts. Data have been further processed
% and edited to produce digital data sets which are widely distributed
% in \textsc{cd-rom} format. These data have been supplemented with data
% from altimetric satellites to produce maps of the sea floor with
% horizontal resolution around 3 km.

Британский Океанографический Центр Данных (The British Oceanographic
Data Centre), опубликовал Электронный Атлас «Генеральная
Батиметрическая Карта Океанов» (General Bathymetric Chart of the
Oceans (GEBCO)) по поручению Межправительственной Океанографической
Комиссии ЮНЕСКО (Intergovernmental Oceanographic Commission of UNESCO)
и Международной Гидрографической Организации (International
Hydrographic Organization). Атлас содержит в основном изобаты, линию
берега и путевые линии взятые из пятого выпуска Генеральной
Батиметрической Карты Океанов изданной в масштобе 1:10 000
000. Исходные изолинии были нарисованы от руки основываясь на
оцифрованных данных эхолотирования.
%
% The British Oceanographic Data Centre publishes the General Bathymetric
% Chart of the ocean (\textsc{gebco})\index{bathymetric charts!GEBCO} Digital
% Atlas on behalf of the Intergovernmental Oceanographic Commission
% of \textsc{unesco} and the International Hydrographic
% Organization\index{International Hydrographic Organization}.
% The atlas consists primarily of the location of depth
% contours, coastlines, and tracklines from the \textsc{gebco} 5th Edition
% published at a scale of 1:10 million. The original contours were drawn
% by hand based on digitized echo-sounder data plotted on base maps.

Американский Национальный Центр Геофизических Данных выпустил CD-ROM
Топографическая Основа (Terrain Base) содержащий значения (глубин)
интерполированных на пятимильной сетке. Большинство материалов были
изначально собраны U.S. Defense Mapping Agency, U.S. Navy
Oceanographic Office, и U.S. National Ocean Service. Несмотря на то
что значения на этой карте нанесены на пятимильную сетку, данные
использованные при её изготовлении часто имеют гораздо большее
пространственное разрешение, особенно в Южном Океане, где расстояния
между маршрутами кораблей в некоторых регионах может достигать 500
км. На этом же CD находятся сглаженные значения полученные на основе
тех же данных, но интерполированные на 30 мильной сетке.

% The U.S. National Geophysical Data Center\index{bathymetric charts!ETOPO-2}
% publishes the \textsc{etopo-2 cd-rom} containing digital values of oceanic
% depths from echo sounders and altimetry and land heights from surveys.
% Data are interpolated to a 2-minute (2 nautical mile) grid. Ocean data
% between 64\degrees N and 72\degrees S are from the work of Smith
% and Sandwell (1997), who combined echo-sounder data with altimeter data
% from \textsc{geosat} and \textsc{ers--1}. Seafloor data northward
% of 64\degrees N are from the International Bathymetric Chart
% of the Arctic Ocean.  Seafloor data southward of 72\degrees S are from
% are from the US Naval Oceanographic Office's Digital Bathymetric Data Base
% Variable Resolution. Land data are from the \textsc{globe} Project, 
% that produced a digital elevation model with 0.5-minute (0.5 nautical mile)
% grid spacing using data from many nations. 

Американский Национальный Центр Геофизических Данных также выпустил
батиметрический атлас океанов основанный на измерениях высоты
поверхности моря сделанных спутником GEOSAT (смотри главу 10 где
обсуждается спутниковая альтиметрия). Пространственное разрешение этой
карты 3--4км а точность глубины $\pm 100$~м (Smith and Sandwell,
1994). Эта карта более детальна чем ETOPO-5 так как спутниковые данные
перекрывают регионы между маршрутами судов (рис 3,14). Данные со
спутников ERS-1 и ERS-2 также использовались чтобы получить похожие
карты, особенно в широтах недоступных для GEOSATа.

%% Рисунок 3.14 Карта глубин океана с разрешением 3 км созданная по
%% данным спутниковых альтиметрических наблюдений поверхности моря. (По
%% Smith and Sandwell).

%% ??? в последней редакции PDF ссылок на этот рисунок нет.
%% кроме того, в новой редакции отсутствует разъяснение, что
%% > ... Несмотря на то, что значения на этой карте нанесены на пятимильную 
%% > сетку, данные использованные при её изготовлении часто имеют гораздо 
%% > большее пространственное разрешение, особенно в Южном Океане, 
%% > где расстояния между маршрутами кораблей в некоторых регионах может 
%% > достигать 500 км.
%% Без него становится непонятной оговорка в разделе 2.5 об осторожности с
%% выбором массивов данных.

% National governments publish coastal and harbor maps. In the USA, the
% \textsc{noaa} National Ocean Service publishes nautical charts useful for
% navigation of ships in harbors and offshore waters.
\end{section}

\begin{section}{Звук в океане}
% \section{Sound in the Ocean}
Звук обеспечивает единственный приемлемый способ передачи информации
на большие расстояния в океане, и это единственный сигнал который
можно использовать для того чтобы узнать что находится на нескольких
десятках метров под дном океана. Звук используется для измерения
параметров морского дна, глубины океана, температуры и течений.
%
% Sound\index{sound!in ocean} provides the only convenient means for 
% transmitting information over great distances in the ocean.
% Sound\index{sound!use of} is used to measure the properties of the sea 
% floor, the depth of the ocean, temperature, and currents. Whales and other
% ocean animals use sound to navigate, communicate over great distances,
% and find food.


%\paragraph{Sound Speed}
Скорость звука в воде зависит от температуры, солёности и давления
(Рис 3,10). Одна из формул для скорости звука, хорошо работающая до
глубины 1000 м, имеет вид: 
\begin{equation}
C = 1449.2 + 4.6 T - 0.055 T^2 + 0.00029
T^3 + (1.34 - 0.01 T) (S - 35) + 0.016 Z
\end{equation}
где C~--- скорость в м/с, T --- температура в градусах цельсия, S ---
солёность в практических единицах солёности (practical salinity units)
и Z --- глубина в метрах. Точность этой формулы примерно 0,1 м/с
(Dushaw, et al. 1993). Существуют и другие формулы для скорости звука,
например формула Вильсона Wilson (1960) котроую широко использовал
военный флот США.
%
% The sound speed \index{sound!speed}in the ocean varies with
% temperature, salinity, and pressure (MacKenzie, 1981; Munk
% et al. 1995: 33):
% \begin{align}
%   C & = 1448.96 + 4.591\,t - 0.05304\,t^2 + 0.0002374\,t^3+ 0.0160\,Z \\
%   &+ (1.340 - 0.01025\,t) (S - 35) + 1.675 \times 10^{-7}\,Z^2 - 7.139 \times
% 10^{-13}\,t\,Z^3 \notag
% \end{align}
% where $C$ is speed in m/s, $t$ is temperature in Celsius, $S$ is salinity 
% (see Chapter 6 for a definition of salinity), and $Z$ is depth in meters.
% The equation has an accuracy\index{accuracy!equation!sound speed} of
% about 0.1 m/s (Dushaw et al. 1993). Other sound-speed equations have been
% widely used, especially an equation proposed by Wilson (1960) which has been
% widely used by the U.S. Navy.

В обычных условиях скорость звука мало меняется, от 1450 до
1550~м/с. Исполльзуя формулу 3,1 мы можем расчитать скорость звука при
небольших изменениях температуры глубины и солёности часто
встречающихся в океане. В обычной океанской воде скорость звука
изменяется на 40~м/с при увеличении температуры на 10 градусов
цельсия, на 16~м/с при увеличении глубины на 1000~м и на 1,5~м/с при
увеличении солёности на 1\%????. Таким образом основные причины
изменения скорости звука это температура и глубина
(давление). Изменения солёности слишком малы и не оказывают
существенного влияния.
%
% For typical oceanic conditions, $C$ is usually between 1450 m/s and 1550 m/s 
% (figure 3.15). Using (3.1), we can calculate the sensitivity of $C$ to 
% changes of temperature, depth, and salinity typical of the ocean. 
% The approximate values are: 40 m/s per 10\degrees C rise of temperature, 
% 16 m/s per 1000 m increase in depth, and 1.5 m/s per 1 increase in salinity. 
% Thus the primary causes of variability of sound 
% speed\index{sound!speed!variation of} is temperature and depth (pressure).
% Variations of salinity are too small to have much influence.

Если изобразить на графике скорость звука как функцию глубины, то мы
увидим что её минимум приходится примерно на 1000 метров. Глубина
минимальной скорости звука существует почти во всех морях кроме очень
высокоширотных. Эта глубина называется звуковым каналом. Он есть во
всех океанах, а в высоких широтах выходит на поверхность.
%
% If we plot sound speed\index{sound!speed!as function of depth} as a function 
% of depth, we find that the speed usually has a minimum at a depth around
% 1000 m (figure 3.16). The depth of minimum speed is called the
% \textit{sound channel}\index{sound!channel|textbf}. It occurs in
% all ocean, and it usually reaches the surface at very high latitudes.

Звуковой канал очень важен. Рефракция в нём позволяет звуку
распространяться на огромные расстояния. Звуковые лучи которые
начинают выходить из канала отражаются обратно к его центру. Лучи
распространяющиеся вверх под небольшими углами к горизонтали
отражаются книзу, а лучи распространяющиеся вниз под небольшими углами
к горизонтали отклоняются кверху (рис 3,16). Глубина канала изменяется
от 10 до 1200~м в зависимости от географического района.
%
% The sound channel\index{sound!channel} is important because sound in the
% channel can travel very far, sometimes half way around the earth. Here is 
% how the channel works: Sound\index{sound!rays} rays that begin to travel
% out of the channel are refracted back toward the center of the channel.
% Rays propagating upward at small angles to the horizontal are bent downward,
% and rays propagating downward at small angles to the horizontal are bent
% upward (figure 3.16). Typical depths of the chan\-nel vary from 10 m to
% 1200 m depending on geographical area.

%%Рисунок 3,15 Процессы создающие звуковой канал в океане. Слева
%%Температура Т и солёность S измеренные во время рейса судна Hakuho
%%Maru № KH-87–1, станция JT, 28 января 1987 на широте 33°52.90'N и
%%долготе 141°55.80'E в северной части Тихого Океана. В центре:
%%Изменение скорости звука в зависимости от изменений температуры
%%солёности и глубины. Справа: Звуковой канал на глубине около 1 км,
%%определяемяй как область минимальной скорости звука на графике
%%зависимости скорости звука от глубины. (Данные из JPOTS Editorial
%%Panel, 1991).

\begin{paragraph}{Поглощение (абсорбция) звука}
Поглащение звука на единицу глубины зависит от интенсивности звука~---$I$
\begin{equation}
dI = k I_o dx
\end{equation}
где $I_o$~--- интенсивность до поглощения а $k$~--- коэффициент
поглощенгия зависящий от частоты звука. У этого уравнения есть
решение:
\begin{equation}
I = I_o \exp(k x)
\end{equation}
Типичные значения~$k$ (в децибелах на километр) составляют: 0,08~дБ/км
при 1000~Гц; и 50~дБ/км при 100~000~Гц. Децибелы считаются таким
образом; $\mbox{дБ} = 10 \log(I / I_o)$. Где $I_o$~--- первоначальная мощьность
звука, $I$~--- мощность звука после поглощения.
%
% Absorption of sound \index{sound!absorption of}per unit distance
% depends on the intensity $I$ of the sound:
% \begin{equation}
% dI = -k I_0 \, dx
% \end{equation}
% where $I_0$ is the intensity before absorption and $k$ is an absorption
% coefficient which depends on frequency of the sound. The equation has the
% solution:
% \begin{equation}
% I = I_0 \exp(-kx)
% \end{equation}
% Typical values of $k$ (in decibels dB per kilometer) are: 0.08 dB/km
% at 1000 Hz, and 50 dB/km at 100,000 Hz. Decibels are calculated from:
% $dB = 10 \log(I/I_0)$, where $I_0$ is the original acoustic power,
% $I$ is the acoustic power after absorption.

Нарпимер на расстоячнии 1~км сигнал с частотой 1000~Гц ослабнет всего
на 1,8\%:I = 0.982 Io. На том же расстоянии сигнал с частотой
100~000~Гц уменьшится на I = 10-5 Io. Таким образом сигнал частотой
30~000~Гц, обычно используемый при эхолотировании морского дна совсем
немного ослабевает проходя от поверхности до дна и обратно.
%
% For example, at a range of 1 km a 1000 Hz signal is attenuated by only 
% 1.8\%: $I = 0.982 I_0$. At a range of 1 km a 100,000 Hz signal is reduced
% to $I = 10^{-5} I_0$. The 30,000 Hz signal used by typical echo sounders to
% map the ocean's depths are little attenuated going from the surface to 
% the bottom and back.

Очень низкочастотные сигналы (менее 500~Гц) в звуковом канале, были
зафиксированы на расстоянии мегаметров. В 1960 звук частотой 15~Гц от
взрывов в звуковом канале у Австралийского города Перт был слышен в
звуковом канале около Бермудских островов, он прошёл почти пол
мира. Дальнейшие эксперименты показали что сигнал частотой 57~Гц
посланный в звуковом канале около острова Херд (75°E, 53°S) может быть
зафиксирован на Бермудах в Атлантике и в Калифорнии, находящейся в
Тихом океане (Munk et al. 1994).
%
% Very low frequency sounds in the sound channel\index{sound!channel}, 
% those with frequencies below 500 Hz have been detected at distances
% of megameters. In 1960 15-Hz sounds from explosions set off in the sound
% channel\index{sound!channel} off Perth Australia were heard in the sound
% channel near Bermuda, nearly halfway around the world. Later experiment
% showed that 57-Hz signals transmitted in the sound channel near Heard
% Island (75\degrees E, 53\degrees S) could be heard at Bermuda in the Atlantic
% and at Monterey, California in the Pacific (Munk et al. 1994).
\end{paragraph}

%% Рисунок 3.16 Звуковые лучи в океане для источника вблизи оси звукового
%% канала. (Взято из Munk et al. 1995).

% \paragraph{Use of Sound}
% Because low frequency sound \index{sound!use of}can be heard at
% great distances, the US Navy, in the 1950s, placed arrays of
% microphones on the sea floor in deep and shallow water and
% connected them to shore stations. The Sound Surveillance System
% \textsc{sosus}, although designed to track submarines, has found
% many other uses. It has been used to listen to and track whales up
% to 1,700 km away, and to find the location of sub-sea volcanic
% eruptions.
%\end{paragraph}
\end{section}

\begin{section}{Основные Концепции}
% \section{Important Concepts}
\begin{enumerate}
\item
Если уменьшить ширину океана до 8 дюймов, его глубина будет
соответствовать толщине листа бумаги. Поэтому поля скорости в океане
близки к двухмерным. Вертикальные скорости гораздо меньше
горизонтальных.
%
% \item If the ocean were scaled down to a width of 8 inches it  would have
% depths about the same as the thickness of a piece of paper. As a result, the
% velocity field in the ocean is nearly 2-dimensional. Vertical velocities 
% are much smaller than horizontal velocities.


\item
Официально на планете только 3 океана.
%
% \vitem There are only three official ocean.

\item
Объём воды превышает вместительность океанических бассейнов, и океаны
переливаются на континенты создавая континентальный шельф.
%
% \vitem The volume of ocean water exceeds the capacity of the ocean basins,
% and the ocean overflows on to the continents creating continental shelves.

\item
Глубина океанов картируется с помощью эхолокаторов, которые измеряют
время затрачиваемое звуком на то чтобы пройти от поверхности до дна и
обратно. Глубины измеряемые эхолокаторами закреплёнными на судах
используются для создания карт морского дна. Для некоторых регионов
эти карты имеют маленькое пространственное разрешение, так как там
редко появляются корабли и лежат они в стороне от основных
транспортных марщрутов.
%
% \vitem The depths of the ocean are mapped by echo sounders which measure
% the time required for a sound\index{sound!used to measure depth} pulse to
% travel from the surface to the bottom and back. Depths measured by ship-based
% echo sounders have been used to produce maps of the sea floor. The maps have
% poor horizontal resolution in some regions because the regions were seldom
% visited by ships and ship tracks are far apart.

\item
Глубина океана также измеряется спутниковыми альтиметрическими
системами, которые профилируют форму поверхности моря. На поверхность
моря оказывают влияния изменения силы тяжести, вызванные элементами
подводного рельефа. У современных карт, основанных на спутниковых
альтиметрических измерениях и данных эхолотирования, ошибка по глубине
составляет $\pm 100$~м а пространственное разрешение $\pm 3$~км.
%
% \vitem The depths of the ocean are also measured by satellite altimeter
% systems which profile the shape of the sea surface. The local shape of the
% surface is influenced by changes in gravity due to sub-sea features. Recent
% maps based on satellite altimeter measurements of the shape of the sea
% surface combined with ship data have depth accuracy of $ \pm $100 m and
% horizontal resolutions of $ \pm $3 km.

\item
Обычно скорость звука в океане составляет 1480 м/с. Скорость зависит в
основном от температуры, меньше от давления и совсем мало от
солёности. Изменение скорости звука как функции температуры и глубины
создают горизонтальный звуковой канал в океане. Звук в канале может
путешествовать на огромные расстояния; и сигнал частотой менее 500 Гц
может обойти пол мира.
%
% \vitem Typical sound\index{sound!speed!typical} speed in the ocean is 
% 1480 m/s. Speed depends primarily on temperature, less on pressure, and very
% little on salinity. The variability of sound speed as a function of pressure
% and temperature produces a horizontal sound channel in the ocean. Sound in
% the channel can travel great distances. Low-frequency sounds below 500 Hz
% can travel halfway around the world provided the path is not interrupted
% by land.
\end{enumerate}
\end{section}

\end{chapter}
