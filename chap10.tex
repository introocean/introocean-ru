% -*- coding: utf-8 -*-

\begin{chapter}{Геострофические течения}\label{chap:10}
% \chapter{Geostrophic Currents}
В открытом океане, на масштабах превышающих несколько десятков
километров и периодах более нескольких дней и вне приповерхностного и
придонного Экмановских пограничных слоев, имеет место
т.н. геострофический баланс~-- почти точное равенство
горизонтального градиента давления и Кориолисовой силы, связанной с
горизонтальными течениями.
%
% Within the ocean's interior away from the top and bottom Ekman
% layers\index{Ekman layer}, for horizontal distances exceeding a few
% tens of kilometers, and for times exceeding a few days, horizontal
% pressure gradients in the ocean almost exactly balance the Coriolis
% force resulting from horizontal currents. This balance is known as the
% \textit{geostrophic balance}\index{geostrophic balance|textbf}.

В вертикальном направлении главные действующие силы~--- вертикальный
градиент давления и сила тяжести, также сбалансированы с относительной
точностью в несколько миллионных долей, поэтому давление в данном
месте почти полностью определяется весом находящегося выше водяного
столба. Доминирующие по горизонтали силы – градиент давления и сила
Кориолиса, на достаточно больших масштабах длины и времени,
уравновешивают друг друга до тысячных долей (См. «Рамку» ).
%
% The dominant forces acting in the vertical are the vertical pressure
% gradient and the weight of the water. The two balance within a few
% parts per million. Thus pressure at any point in the water column is
% due almost entirely to the weight of the water in the column above the
% point. The dominant forces in the horizontal are the pressure gradient
% and the Coriolis force. They balance within a few parts per thousand
% over large distances and times (See Box).

И тот и другой балансы указывают на то, что вязкие и нелинейные члены
в уравнениях движения пренебрежимо малы. Насколько допустимо это
приближение ? Рассмотрим вначале эффект вязкости. Хорошо известно, что
обычная гребная лодка массой сто килограммов проходит около десяти
метров, прежде чем остановиться после прекращения гребли, а
супертанкер, движущийся со скоростью лодки, остановится через
несколько километров. Для кубического километра воды
массой~$10^{15}\kg$ разумной оценкой времени движения до остановки
является величина порядка суток. Мезомасштабные вихри в океане
содержат примерно 1000 кубических километров воды, поэтому наша
интуитивная оценка малости сил вязкости представляется
правдоподобной. Конечно, интуиция может и подвести, поэтому нам нужно
вновь вернуться к численным оценкам.
%
% Both balances require that viscosity and nonlinear terms in the
% equations of motion be negligible. Is this reasonable? Consider
% viscosity. We know that a rowboat weighing a hundred kilograms will
% coast for maybe ten meters after the rower stops. A super tanker
% moving at the speed of a rowboat may coast for kilometers. It seems
% reasonable, therefore that a cubic kilometer of water weighing
% $10^{15}$ kg would coast for perhaps a day before slowing to a
% stop. And oceanic mesoscale eddies\index{mesoscale eddies} contain
% perhaps 1000 cubic kilometers of water. Hence, our intuition may lead
% us to conclude that neglect of viscosity is reasonable.  Of course,
% intuition can be wrong, and we need to refer back to scaling
% arguments.

% \begin{figure} [t!]
% \fbox{\parbox{12cm}{ 
% \centering \small
% \begin{minipage}{11.5cm}
\begin{center}
\textbf{Масштабирование уравнений: геострофическое приближение.}
% \textbf{Scaling the Equations: The \rule{0mm}{3ex} Geostrophic Approximation}\\
\end{center}
% \vspace{-1em} 
Попытаемся упростить уравнения движения в глубине океана ниже
поверхностного Экмановского пограничного слоя, для чего оценим
характерный вклад каждого члена, отбрасывая малые слагаемые,
существенно не влияющие на решение. Для толщи океана характерные
величины горизонтального размера~$L$, горизонтальной скорости~$U$, 
глубины~$H$, параметр Кориолиса~$f$, ускорения силы тяжести~$g$ 
и плотности~$\rho$:
\begin{align*}
 L & \approx 10^6\m     & H_1 & \approx 10^3\m  
   & f & \approx 10^{-4}\radps  & \rho &\approx 10^3\kgpcm \\
 U &\approx 10^{-1}\mps  & H_2 &\approx 1\m   & \rho &\approx 10^3\kgpcm 
   & g &\approx 10\mpsqs
\end{align*}
где~$H_1$ и~$H_2$~--- характерные масштабы изменения давления в
вертикальном и горизонтальном направлениях.
%
% \hspace*{1em}\index{geostrophic approximation}We wish to simplify the
% \rule{0mm}{3ex}equations of motion to obtain solutions that describe
% the deep-sea conditions well away from coasts and below the Ekman
% boundary layer at the surface. To begin, let's examine the typical
% size of each term in the equations in the expectation that some will
% be so small that they can be dropped without changing the dominant
% characteristics of the solutions. For interior, deep-sea conditions,
% typical values for distance $L$, horizontal velocity $U$, depth $H$,
% Coriolis parameter\index{Coriolis parameter} $f$, gravity $g$, and
% density $\rho$ are:
% \vspace{-2ex}
% \begin{align}
% L\, &\approx\, 10^6 \text{ m}     & H_1\, &\approx\,10^3 \text{ m}  & f\,    &\approx\,10^{-4}\,\text{ s}^{-1}  & \rho\, &\approx\,10^3 \text{ kg/m}^3 \notag \\
% U\, &\approx\,10^{-1} \text{ m/s} & H_2\, &\approx\, 1 \text{ m}    & \rho\, &\approx\,10^3 \text{ kg/m}^3      & g\,    &\approx\,10 \text{ m/s}^2  \notag 
% \end{align}
% where $H_1$ and $H_2$ are typical depths for pressure in the vertical
% and horizontal.

Из этих величин можно получить оценки вертикальной скорости~$W$,
давления~$P$ и характерного времени~$T$:
\begin{align*}
\frac{\partial W}{\partial z} 
  & = -\left(\frac{\partial U}{\partial x} + \frac{\partial v}{\partial y}\right) \\
\frac{W}{H_1} 
  & =\frac{U}{L}; \quad W =\frac{UH_1}{L} = \frac{10^{-1}\,10^3}{10^6}\mps = 10^{-4}\mps \\
P & =\rho g H_1 =10^3\,10^1\,10^3 = 10^7\Pa; 
   \quad \partial{p} / \partial{x} = \rho g H_2 / L = 10^{-2}\Papm \\ 
T & = L/U =10^7\seconds
\end{align*}
Уравнение для вертикальной скорости:
\begin{align*}
\frac{\partial w}{\partial t} + u \frac{\partial w}{\partial x} 
  + v \frac{\partial w}{\partial y} + w \frac{\partial w}{\partial z}
  & = -\frac{1}{\rho}\frac{\partial p}{\partial z} + 2\Omega u \cos\varphi - g\\
\frac{W}{T} + \frac{UW}{L} + \frac{UW}{L} + \frac{W^2}{H}
  &=\frac{P}{\rho\,H_1} + f U - g\\
 10^{-11} + 10^{-11} + 10^{-11} + 10^{-11} & = 10\quad + 10^{-5} - 10 
\end{align*}
Отсюда следует, что единственными значимыми членами являются:
\begin{displaymath}
 \frac{\partial p}{\partial z}=-\rho g \quad \text{Correct to}\quad 1:10^6.
\end{displaymath}
Для составляющей горизонтальной скорости по оси х имеем:
\begin{align*}
\frac{\partial u}{\partial t} + u \frac{\partial u}{\partial x} 
   + v \frac{\partial u}{\partial y} + w \frac{\partial u} {\partial z} 
 &= -\frac{1}{\rho}\frac{\partial p}{\partial x} + fv \\ 
10^{-8} + 10^{-8} + 10^{-8} + 10^{-8} & =\quad 10^{-5} + 10^{-5}
\end{align*}
Таким образом, сила Кориолиса компенсирует градиент давления с
точностью до одной тысячной, что и называется 
\emph{геострофическим балансом}, а \emph{геострофическими уравнениями} являются:
\begin{displaymath}
 \frac{1}{\rho} \frac{\partial p}{\partial x}  =f v; \quad
 \frac{1}{\rho} \frac{\partial p}{\partial y}  =-f u; \quad
 \frac{1}{\rho} \frac{\partial p}{\partial z}  = -g
\end{displaymath}
Этот баланс применим к потокам с горизонтальным масштабом больше,
примерно, $50\km$ и на временах превышающих несколько дней.

% \hspace*{1em}From these variables we can calculate typical values for
% vertical velocity $W$, pressure $P$, and time $T$:
% \begin{align} \notag
% \frac{\partial W}{\partial z} & =-\,\left(\frac{\partial U}{\partial x}\,
% +\,\frac{\partial v}{\partial y}\right) \notag \\
% \frac{W}{H_1} & =\frac{U}{L}; \quad W =\frac{UH_1}{L}\,=\,
% \frac{10^{-1}\,10^3}{10^6}\text{ m/s} =10^{-4} \text{m/s} \notag \\
% P & =\rho g H_1 =10^3\,10^1\,10^3=10^7 \text{ Pa;} \quad \partial{p} / \partial{x} = \rho g H_2 / L = 10^{-2} \text{Pa/m} \notag
% \\ T & = L/U =10^7 \text{ s} \notag
% \end{align}
% The momentum equation for vertical velocity is therefore:
% \begin{align}
% \frac{\partial w}{\partial t}\,+\,u\,\frac{\partial w}{\partial x}\,+\,v\,
% \frac{\partial w}{\partial y}\,+\,w\,\frac{\partial w}{\partial z}&
% =-\frac{1}{\rho}\frac{\partial p}{\partial z}\,+\,2\Omega \, u \cos\varphi -g
% \notag\\ \notag
% \frac{W}{T}\,+\,\frac{UW}{L}\;+\;\frac{UW}{L}\,+\;\;\;\frac{W^2}{H}&=\frac{P}{\rho\,H_1}\,+\,f\,U\,-\,g
% \notag\\ \notag 10^{-11} + 10^{-11} + 10^{-11} + 10^{-11} & =10\quad + 10^{-5} - 10
% \notag
% \end{align}
% and the only important balance in the vertical is hydrostatic:
% \begin{displaymath}
% \frac{\partial p}{\partial z}=-\rho g \quad \text{Correct to}\quad 1:10^6.
% \end{displaymath}
% The momentum equation for horizontal velocity in the $x$ direction is:
% \begin{align}
% \frac{\partial u}{\partial t}\,+\,u\,\frac{\partial u}{\partial
% x}\,+\,v\,\frac{\partial u}{\partial y}\,+\,w\,\frac{\partial u} {\partial z} &
% =-\frac{1}{\rho}\frac{\partial p}{\partial x}\,+fv \notag\\ \notag 10^{-8} +\;\;
% 10^{-8} + \:\;10^{-8} + \;10^{-8} & =\quad 10^{-5}\; + 10^{-5}
% \notag
% \end{align}
% Thus the Coriolis force balances the pressure gradient within one part
% per thousand. This is called the \textit{geostrophic
% balance}\index{geostrophic balance|textbf}, and the
% \textit{geostrophic equations}\index{geostrophic equations|textbf}
% are:
% \begin{displaymath}
% \frac{1}{\rho}\,\frac{\partial p}{\partial x}  =f v; \quad
% \frac{1}{\rho}\,\frac{\partial p}{\partial y}  =-f u; \quad
% \frac{1}{\rho}\,\frac{\partial p}{\partial z}  = -g
% \end{displaymath} \notag
% This balance applies to oceanic flows with horizontal dimensions
% larger than roughly 50 km and times greater than a few days.
% \vspace{0.7ex}
% \end{minipage}}}
% \vspace{-5ex}
% \end{figure}

\begin{section}{Гидростатическое равновесие.}
% \section{Hydrostatic Equilibrium}
Прежде чем детально рассматривать геострофический баланс, обратим
внимание на простейшее решение уравнений движения~--- уравнений для
импульса~--- для океана в состоянии покоя, что вводит гидростатическое
давление в его толще. Для получения этого решения положим жидкость
неподвижной в начальный момент:
\begin{equation}
  u = v = w = 0;
\end{equation}
и стационарной в последующие:
\begin{equation}
 \frac{du}{dt}=\frac{dv}{dt}=\frac{dw}{dt} = 0;
\end{equation}
а также примем условие отсутствия трения:
\begin{equation}
 f_x = f_y =f_z = 0.
\end{equation}
%
%  \index{hydrostatic equilibrium}Before describing the geostrophic
% balance, let's first consider the simplest solution of the momentum
% equation, the solution for an ocean at rest. It gives the hydrostatic
% pressure within the ocean. To obtain the solution, we assume the fluid
% is stationary:
% \begin{equation}
%  u=v=w=0;
% \end{equation}
% the fluid remains stationary:
% \begin{equation}
% \frac{du}{dt}=\frac{dv}{dt}=\frac{dw}{dt} = 0;
% \end{equation}
% and, there is no friction:
% \begin{equation}
% f_x=f_y=f_z=0.
% \end{equation}

При этих предположениях уравнения движения запишутся как:
\begin{equation}
\frac{1}{\rho}\frac{\partial p}{\partial x}=0; \qquad \qquad
\frac{1}{\rho}\frac{\partial p}{\partial y}=0; \qquad \qquad
\frac{1}{\rho}\frac{\partial p}{\partial z}=-\,g(\varphi,z)
\end{equation}
где явно указана зависимость, по причинам которые будут указаны ниже,
ускорения силы тяжести от широты и вертикальной координаты.
%
% With these assumptions the momentum equation (7.12) becomes:
% \begin{equation}
% \frac{1}{\rho}\frac{\partial p}{\partial x}=0; \qquad \qquad
% \frac{1}{\rho}\frac{\partial p}{\partial y}=0; \qquad \qquad
% \frac{1}{\rho}\frac{\partial p}{\partial z}=-\,g(\varphi,z)
% \end{equation}
% where I have explicitly noted that gravity $g$ is a function of
% latitude $\varphi$ and height $z$. I will show later why I have kept
% this explicit.

Из уравнения (10.4) следует, что поверхности постоянного давления,
называемые \emph{изобарическими поверхностями}, являются, в тоже время,
поверхностями постоянного уровня. Последнее уравнение в (10.4) может
быть проинтегрировано, что дает выражение для давления на любой
глубине:
\begin{equation}
 p=\int_{-h}^0 g(\varphi,z)\,\rho(z)\,dz
\end{equation}
где плотность в состоянии покоя~$\rho$ есть функция глубины.

Во многих случаях, $g$ и~$\rho$ являются константами, 
и~$p = \rho \,g\,h$. Позднее будет показано, (10.5) выполняется 
с точностью до одной миллионной, даже если океан не находится в 
состоянии покоя.
%
% Equations (10.4) require surfaces of constant pressure to be level
% surface\index{level surface} (see page 30). A surface of constant
% pressure is an \textit{isobaric surface}\index{isobaric
% surface|textbf}. The last equation can be integrated to obtain the
% pressure at any depth $h$. Recalling that $\rho$ is a function of
% depth for an ocean at rest.
% \begin{equation}
% p=\int_{-h}^0\,g(\varphi,z)\,\rho(z)\,dz
% \end{equation}
% For many purposes, $g$ and $\rho$ are constant, and $p = \rho \,g\,h$.
% Later, I will show that (10.5) applies with an
% accuracy\index{accuracy!equation!momentum} of about one part per
% million even if the ocean is not at rest.

В системе СИ единицей давления является паскаль Па. Другой
распространенной единицей является бар, причем $1\oneBar = 10^5\Pa$
(Таблица 10.1). Поскольку глубина моря в метрах почти точно численно
совпадает с давлением в децибарах, океанографы предпочитают выражать
давление именно в децибарах.
%
% The\index{pressure!units of} SI unit for pressure is the pascal
% (Pa). A bar is another unit of pressure. One bar is exactly $10^5$ Pa
% (table 10.1). Because the depth in meters and pressure in decibars are
% almost the same numerically, oceanographers prefer to state pressure
% in decibars.

\begin{table}[h!]
\caption{Единицы давления}
\small \centering
\begin{tabular}{lcl}
\hline
1 паскаль (Па)        &=& $1\mbox{~Н/$\mbox{м}^2$}$ 
                       =  $1\mbox{~кг} \cdot \mbox{с}^{-2} \cdot \mbox{м}^{-1}$\\ 
1 бар                 &=& $10^5\Pa$ \\
1 децибар             &=& $10^4\Pa$ \\
1 миллибар            &=& $100\Pa$ \\
\hline
\end{tabular}
\end{table}
%
% \begin{table}[h!]\small \centering
% \vspace{-1ex}
% \begin{tabular*}{70mm}{lcl}
% \multicolumn{3}{@{}l@{}}{\bfseries Table\rule[-1ex]{0mm}{1ex} 10.1 Units of
% Pressure} \\
% \hline
% 1 \rule{0mm}{2.5ex}pascal (Pa) &=& 1 N/m$^2$ = 1 kg$\cdot$s$^{-2}\cdot $m$^{-1}$
% \\ 1 bar                 &=& 10$^5$ Pa \\
% 1 decibar             &=& 10$^4$ Pa \\
% 1 millibar            &=& 100 Pa \\
% \hline
% \end{tabular*} \\[0.5ex]
% \vspace{-2ex}
% \end{table}
\end{section}

\begin{section}{Геострофические уравнения}
% \section{Geostrophic Equations}
Из геострофического баланса следует равенство силы Кориолиса
горизонтальному градиенту давления. Уравнения геострофического баланса
получаются из уравнений движения при следующих допущениях: движение
происходит без ускорения, т.е. $du/dt = dv/dt = dw/dt = 0$; горизонтальные
составляющие скорости намного больше вертикальной $w \ll u,v$;
единственной внешней силой является сила тяжести; трение пренебрежимо
мало. При этом система уравнений (7.12) запишется как:
\begin{equation}
 \frac{\partial p}{\partial x}= \rho fv; \quad
 \frac{\partial p}{\partial y}= - \rho f u; \quad
 \frac{\partial p}{\partial z}= - \rho g
\end{equation}
где $f = 2 \Omega \sin \varphi$~--- параметр Кориолиса. 
Уравнения (10.6) называются \emph{геострофическими уравнениями}.
%
% The geostrophic \index{geostrophic currents!equations for|(}balance
% requires that the Coriolis force \index{Coriolis force}balance the
% horizontal pressure gradient. The equations for geostrophic balance
% are derived from the equations of motion assuming the flow has no
% acceleration, $du/dt = dv/dt = dw/dt = 0$; that horizontal velocities
% are much larger than vertical, $w \ll u,v$; that the only external
% force is gravity; and that friction is small. With these assumptions
% (7.12) become
% %\begin{subequations}
% \begin{equation}
% \frac{\partial p}{\partial x}= \rho fv; \quad
% \frac{\partial p}{\partial y}= - \rho f u; \quad
% \frac{\partial p}{\partial z}= - \rho g
% \end{equation}
% where $f = 2 \Omega \sin \varphi$ is the Coriolis
% parameter\index{Coriolis parameter}. These are the \textit{geostrophic
% equations}\index{geostrophic equations|textbf}.

Их можно переписать в виде:
\begin{subequations}
 \begin{equation}
  u= -\frac{1}{f\rho}\frac{\partial p}{\partial y}; \qquad
  v= \frac{1}{f\rho}\frac{\partial p}{\partial x}
 \end{equation}
 \begin{equation}
  p=p_0+\int_{-h}^{\zeta} g(\varphi,z)\rho(z)\,dz
 \end{equation}
\end{subequations}
где $p_0$~--- атмосферное давление при~$z = 0$, а $\zeta$~---
возвышение уровня морской поверхности. Заметим, что возвышение уровня
может быть как выше, так и ниже поверхности~$z = 0$, а поверхностный
градиент давления компенсируется поверхностным течением~$u_s$.
%
% The equations can be written:
% \begin{subequations}
% \begin{equation}
% u= -\frac{1}{f\rho}\frac{\partial p}{\partial y}; \qquad
% v= \frac{1}{f\rho}\frac{\partial p}{\partial x}
% \end{equation}
% \begin{equation}
% p=p_0+\int_{-h}^{\,\zeta}\,g(\varphi,z)\rho(z)dz
% \end{equation}
% \end{subequations}
% where $p_0$ is atmospheric pressure at $z = 0$, and $\zeta$ is the
% height of the sea surface. Note that I have allowed for the sea
% surface to be above or below the surface $z = 0$; and the pressure
% gradient at the sea surface is balanced by a surface current $u_s$.

Подстановка (10.7b) в (10.7a) дает 
\begin{subequations}
\begin{align*}
  u&= -\frac{1}{f\rho}
       \,\frac{\partial}{\partial y}\int_{-h}^{0} g(\varphi,z)\,\rho(z)\,dz 
      -\frac{g}{f}\,\frac{\partial \zeta}{\partial y} \\
 u &= -\frac{1}{f\rho}
       \,\frac{\partial}{\partial y}\int_{-h}^0 g(\varphi,z)\,\rho(z)\,dz - u_s
\end{align*}
где использовано приближение Буссинеска, т.е. сохранение точного
значения для плотности~$\rho$ только при вычислении давления.
%
% Substituting (10.7b) into (10.7a) gives:
% \begin{subequations}
% \begin{align}
% u\,&= -\frac{1}{f\rho}\,\frac{\partial}{\partial
% y}\int_{-h}^{0}\,g(\varphi,z)\,\rho(z)\,dz -
% \frac{g}{f}\,\frac{\partial \zeta}{\partial y} \notag \\
% u &= -\frac{1}{f\rho}\,\frac{\partial}{\partial
% y}\int_{-h}^0\,g(\varphi,z)\,\rho(z)\,dz - u_s
% \end{align}
% where I have used the Boussinesq approximation\index{Boussinesq
% approximation}, retaining full accuracy\index{accuracy!Boussinesq
% approximation} for $\rho$ only when calculating pressure.

Аналогично, можно получить уравнение для компоненты~$v$
\begin{align*}
 v &= \frac{1}{f\rho}
      \,\frac{\partial}{\partial x}\int_{-h}^0 g(\varphi,z)\,\rho(z)\,dz 
     + \frac{g}{f}\,\frac{\partial \zeta}{\partial x} \\
 v &= \frac{1}{f\rho}
      \,\frac{\partial}{\partial x}\int_{-h}^0 g(\varphi,z)\,\rho(z)\,dz + v_s
\end{align*}
\end{subequations}
первый член в (10.8b) обращается в нуль, а горизонтальные градиенты
давления в толще океана равны градиенту при~$z = 0$. Это случай
т.н. баротропного течения, описываемого в \S~10.4.
%
% In a similar way, we can derive the equation for $v$.
% \begin{align}
% v&= \frac{1}{f\rho}\,\frac{\partial}{\partial
% x}\int_{-h}^0\,g(\varphi,z)\,\rho(z)\,dz + \frac{g}{f}\,\frac{\partial
% \zeta}{\partial x} \notag \\
%  v&= \frac{1}{f\rho}\,\frac{\partial}{\partial x}\int_{-h}^0
% \,g(\varphi,z)\,\rho(z)\,dz + v_s
% \end{align}
% \end{subequations}
%
% If the ocean is homogeneous and density and gravity are constant, the
% first term on the right-hand side of (10.8) is equal to zero; and the
% horizontal pressure gradients within the ocean are the same as the
% gradient at $z = 0$. This is barotropic flow described in \S10.4.

В стратифицированном океане, горизонтальный градиент давления состоит
их двух вкладов~--- один связан с наклоном уровня морской поверхности,
другой с горизонтальными изменениями плотности. В этом случае эти
уравнения включают бароклинное течение, также рассматриваемое
в~\S~10.4. Первый член в (10.8b) связан с изменениями
плотности~$\rho(z)$ и называется относительной скоростью. Таким
образом, расчет геострофических течений из возмущений плотности
требует задания скорости~$\left(u_0, v_0\right)$ на поверхности моря
или на некоторой глубине.
%
% If the ocean is stratified, the horizontal pressure gradient has two
% terms, one due to the slope at the sea surface, and an additional term
% due to horizontal density differences. These equations include
% baroclinic flow also discussed in \S10.4. The first term on the
% right-hand side of (10.8) is due to variations in density $\rho (z)$,
% and it is called the relative velocity. Thus calculation of
% geostrophic currents from the density distribution requires the
% velocity $\left(u_0, v_0\right)$ at the sea surface or at some other
% depth.

\begin{figure}[h!]
\makebox[120mm][c]{\includegraphics{pics/surfacesketch}}
\caption{Figure 10.1 Sketch defining $\zeta$ and $r$, used for
calculating pressure just below the sea surface.}
\label{fig:surfacesketch}
\end{figure}
%
% \begin{figure}[h!]
% \makebox[120mm][c]{\includegraphics{surfacesketch}}
% \centering
% \footnotesize
% Figure 10.1 Sketch \rule{0mm}{3ex}defining $\zeta$ and $r$, used for
% calculating pressure just below the sea surface.
%
% \label{fig:surfacesketch}
% \vspace{-3ex}
% \end{figure}
\end{section}

\begin{section}{Расчет геострофических течений по альтиметрическим данным.}
% \section{Surface Geostrophic Currents From Altimetry}
Если применить геострофическое приближение к поверхности~$z = 0$ это
ускорение, например, в 2 метрах ниже поверхности моря~$z = -r$
(Рис. 10.1). Давление на этой поверхности дается:
\begin{equation}
 p = \rho\,g\,\left(\zeta + r\right)
\end{equation}
в предположении практического постоянства~$g$ и~$\rho$ в слое в несколько
метров под поверхностью. Подставляя это в (10.7а) получаем для
компонент поверхностного течения $(u_s, v_s)$:
\begin{equation}
 u_s =-\frac{g}{f}\frac{\partial\zeta}{\partial y}; \qquad \qquad
 v_s = \frac{g}{f}\frac{\partial\zeta}{\partial x}
\end{equation}
где $g$ это ускорение силы тяжести, $f$~--- параметр Кориолиса, 
$\zeta$~--- возвышение уровня над уровенной поверхностью.
%
% \index{geostrophic currents!surface}\index{geostrophic currents!from
% altimetry|(}The geostrophic approximation applied at $z = 0$ leads to
% a very simple relation: surface geostrophic currents are proportional
% to surface slope. Consider a level surface\index{level surface}
% slightly below the sea surface, say two meters below the sea surface,
% at $z = -r$ (figure 10.1).
%
% The pressure on the level surface\index{level surface} is:
% \begin{equation}
% p = \rho\,g\,\left(\zeta + r\right)
% \end{equation}
% assuming $\rho$ and $g$ are essentially constant in the upper few
% meters of the ocean.
%
% Substituting this into (10.7a), gives the two components ($u_s, v_s$)
% of the surface geostrophic current:
% \begin{equation}
% u_s =-\frac{g}{f}\frac{\partial\zeta}{\partial y}; \qquad \qquad
% v_s =\frac{g}{f}\frac{\partial\zeta}{\partial x}
% \end{equation}
% where $g$ is gravity, $f$ is the Coriolis parameter\index{Coriolis
% parameter}, and $\zeta$ is the height of the sea surface above a level
% surface\index{geostrophic currents!from altimetry|)}\index{geostrophic
% currents!equations for|)}.

\begin{paragraph}{Топография океанской поверхности.}
% \paragraph{The Oceanic Topography}
В \S~3.4 мы определили топографию морской поверхности~$\zeta$ как
высоту этой поверхности над некоторой определенной поверхностью,
геоидом, определяемом как поверхность океана в состоянии
покоя. Поэтому, согласно (10.10), составляющие скорости поверхностного
геострофического течения пропорциональны наклону топографии
(Рис. 10.2) , величина которого может быть измерена с помощью методов
спутниковой альтиметрии, при условии, что нам известна форма геоида.
%
% \index{topography!oceanic|textbf}In \S 3.4 we define the topography of
% the sea surface $\zeta$ to be the height of the sea surface relative
% to a particular level surface\index{level surface}, the
% geoid\index{geoid}; and we defined the geoid\index{geoid} to be the
% level surface\index{level surface} that coincided with the surface of
% the ocean at rest.  Thus, according to (10.10) the surface geostrophic
% currents are proportional to the slope of the topography (figure
% 10.2), a quantity that can be measured by satellite altimeters if the
% geoid\index{geoid} is known.

\begin{figure}[h!]
\makebox[120mm][c]{\includegraphics{pics/geostrophicsketch}}
\caption{Наклон морской поверхности относительно геоида 
$(\partial\zeta/\partial x)$ прямо связан со скоростью геострофического 
течения~$v_s$, направленной в северном полушарии (как на рисунке) от нас. 
Наклон в~$1\m$ на~$100\km$, что соответствует углу в 10 микрорадиан, 
вызывает сильные поверхностные течения.}
\label{fig:geostrophicsketch}
\end{figure}
%
% \begin{figure}[h!]
% \vspace{-2ex}
% \makebox[120mm][c]{\includegraphics{geostrophicsketch}}
% \footnotesize
% Figure 10.2 The \rule{0mm}{3ex}slope of the sea surface relative to
% the geoid\index{geoid} $(\partial\zeta/\partial x)$ is directly
% related to surface geostrophic currents $v_s$.  The slope of 1 meter
% per 100 kilometers (10 $\mu$rad) is typical of strong currents.  $V_s$
% is into the paper in the northern hemisphere.
% \label{fig:geostrophicsketch}
% \vspace{-2ex}
% \end{figure}

Поскольку геоид является равновесной поверхностью, то это поверхность
постоянного геопотенциала. Чтобы убедиться в этом, рассмотрим работу
необходимую для перемещения массы~$m$ на высоту~$h$ перпендикулярно
равновесной поверхности. Её величина равна~$W = mgh$ и изменение
потенциальной энергии на единицу массы есть~$gh$. Поэтому равновесные
поверхности являются поверхностями постоянного \emph{геопотенциала}~$\Phi = gh$.
%
% Because the geoid\index{geoid} is a level surface\index{level
% surface}, it is a surface of constant geopotential. To see this,
% consider the work done in moving a mass $m$ by a distance $h$
% perpendicular to a level surface\index{level surface}. The work is
% $W=mgh$, and the change of potential energy per unit mass is
% $gh$. Thus level surfaces\index{level surface} are surfaces of
% constant geopotential, where the
% \textit{geopotential}\index{geopotential} $\Phi = gh$.

Форма топографии океанской поверхности формируется под действием
различных факторов: приливов, течений, изменений барометрического
давления, приводящих к эффекту обратного барометра. Поскольку
топография формируется в ходе этих динамических процессов, она часто
обозначается как \emph{динамическая топография}. Характерные значения
топографии составляют примерно одну сотую долю от величины неровностей
геоида, поэтому форма морской поверхности определяется локальными
вариациями силы тяжести, а течения дают значительно меньший вклад.
Типичные значения амплитуды вариаций морской топографии лежат в
пределах~$\pm 1\m$. (Рис. 10.3). Характерные наклоны имеют 
порядок~$\partial\zeta/\partial x \approx \mbox{1 -- 10 микрорадиан}$ 
для соответствующих скоростей в средних широтах
порядка $0.1$--$1.0\mps$.
%
% Topography is due to processes that cause the ocean to move: tides,
% ocean currents, and the changes in barometric pressure that produce
% the inverted barometer effect. Because the ocean's topography is due
% to dynamical processes, it is usually called \textit{dynamic
% topography}\index{dynamic
% topography|textbf}\index{topography!dynamic|textbf}. The topography is
% approximately one hundredth of the geoid
% undulations\index{geoid!undulations}. Thus the shape of the sea
% surface is dominated by local variations of gravity. The influence of
% currents is much smaller.  Typically, sea-surface topography has
% amplitude of $\pm$1m (figure 10.3). Typical slopes are
% $\partial\zeta/\partial x \approx $ 1--10 microradians for $v = $
% 0.1--1.0 m/s at mid latitude.

Форма геоида, сглаженная по горизонтали на масштабах приблизительно
превышающих $400\km$, известна с точностью~$\pm 1\mm$ из спутниковых данных,
собранных во время проведения Гравитационного и Климатического
эксперимента GRACE.
%
% The height of the geoid\index{geoid}, smoothed over horizontal
% distances greater than roughly 400 km, is known with an
% accuracy\index{accuracy!geoid} of $\pm$1mm from data collected by the
% Gravity Recovery and Climate Experiment
% \textsc{grace}\index{GRACE|textbf} satellite mission.

\begin{figure}[t!]
\makebox[120mm][c]{\includegraphics{pics/sshprofile}}
\caption{Альтиметрические наблюдения спутника Topex/Poseidon в районе
Гольфстрима. Топография океанской поверхности, которая в данном случае
формируется, в основном, течениями, получается после вычитания
высотных измерений из параметров местного геоида. Эти параметры были
получены группой исследователей из Огайского университета из судовых
гравиметрических измерений.}
\label{sshprofile}
\end{figure}
%
% \begin{figure}[t!]
% %\vspace{-2ex}
% \makebox[120mm][c]{\includegraphics{sshprofile}}
% \footnotesize
% Figure 10.3 Topex/Poseidon\index{Topex/Poseidon!observations of Gulf
% Stream} \rule{0mm}{3ex}altimeter observations of the Gulf
% Stream\index{Gulf Stream!mapped by Topex/Poseidon}\index{geostrophic
% currents!from altimetry}. When the altimeter observations are
% subtracted from the local geoid\index{geoid}, they yield the oceanic
% topography, which is due primarily to ocean currents in this
% example. The gravimetric geoid\index{geoid} was determined by the Ohio
% State University from ship and other surveys of gravity in the
% region. From Center for Space Research, University of Texas.
% \label{sshprofile}
% \vspace{-5ex}
% \end{figure}
\end{paragraph}

\begin{paragraph}{Спутниковая альтиметрия.}
% \paragraph{Satellite Altimetry}
Для измерений океанской топографии требуется альтиметрия особой
точности. Первые системы такого рода, установленные на спутниках
Seasat, Geosat,

ERS-1, ERS-2, были разработаны для измерений недельной
изменчивости течений. Спутник TROPEX/POSEIDON, запущенный в 1992 году
был первым аппаратом, предназначенным для существенно более точных
наблюдений постоянной (усредненной по времени) океанической
циркуляции, приливов и изменчивости гиромасштабных течений. За ним в
2001 году последовал спутник Jason в 2008 году~--- Jason-2.
%
% \index{satellite altimetry}Very accurate, satellite-altimeter systems
% are needed for measuring the oceanic topography. The first systems,
% carried on Seasat, Geosat\index{Geosat}, \textsc{ers}--1, and
% \textsc{ers}--2\index{ERS satellites} were designed to measure
% week-to-week variability of
% currents. Topex/Poseidon\index{Topex/Poseidon}, launched in 1992, was
% the first satellite designed to make the much more accurate
% measurements necessary for observing the permanent (time-averaged)
% surface circulation of the ocean, tides, and the variability of
% gyre-scale currents. It was followed in 2001 by Jason\index{Jason} and
% in 2008 by Jason-2\index{Jason-2}.

Ввиду того, что локальные характеристики геоида до 2004 года не были
известны с достаточной точностью, орбиты альтиметрических спутников
строились таким образом, чтобы они проходили над данным пунктом строго
через определенный временной интервал, что обеспечивало возможность
относительных измерений. Так, орбиты спутников TROPEX/POSEIDON и Jason
повторяют один и тот же трек (проекцию на земную поверхность) через
каждые 9.9156 суток. Вычитая результаты высотных измерений для двух
последовательных треков, имеющих одинаковое расположение, получают
данные об изменениях океанской поверхности, т.к. параметры геоида,
остающиеся постоянными, вычитаются из данных и их точное знание не
требуется. Таким образом, можно выделить изменения в течениях, такие
как мезомасштабные вихри, при условии исключения приливов из данных
(Рис. 10.4). Мезомастабная изменчивость включает вихри диаметром,
приблизительно, от~$20$ до~$500\km$.
%
% Because the geoid\index{geoid} was not well known locally before about
% 2004, altimeters were usually flown in orbits that have an exactly
% repeating ground track. Thus Topex/Poseidon\index{Topex/Poseidon} and
% Jason\index{Jason} fly over the same ground track every 9.9156
% days. By subtracting sea-surface height from one traverse of the
% ground track from height measured on a later traverse, changes in
% topography can be observed without knowing the geoid\index{geoid}. The
% geoid\index{geoid} is constant in time, and the subtraction removes
% the geoid, revealing changes due to changing currents, such as
% mesoscale eddies\index{mesoscale eddies}, assuming tides have been
% removed from the data (figure 10.4). Mesoscale variability includes
% eddies with diameters between roughly 20 and 500 km.

\begin{figure}[t!]
\makebox[120mm][c]{\includegraphics{pics/sshvariability}}
\caption{Глобальное распределение стандартного отклонения топографии
океанской поверхности в см по данным альтиметрических измерений
спутникаов TROPEX/POSEIDON и ERS за период с 12/92 г. по 11/98
г. Изменения высоты поверхности отражают изменчивость поверхностных
геострофических течений. По данным AVISO.}
\label{fig:sshvariability}
\end{figure}
%
% \begin{figure}[t!]
% \makebox[120mm][c]{\includegraphics{sshvariability}}
% \footnotesize
% Figure 10.4 Global distribution of \rule{0mm}{3ex}standard deviation
% of topography from Topex/Poseidon\index{Topex/Poseidon!observations of
% topography} altimeter data from 10/3/92 to 10/6/94. The height
% variance is an indicator of variability of surface geostrophic
% currents\index{geostrophic currents!from altimetry}. From Center for
% Space Research, University of Texas.
% \label{fig:sshvariability}
% \vspace{-4ex}
% \end{figure}

Высокая точность и низкая погрешность измерений альтиметрической
системы TROPEX/POSEIDON позволяет получать данные о топографии
океанской поверхности с точностью~$\pm (2\mbox{--}5)\cm$ (Chelton et al, 2001). 
Это, в свою очередь, позволяет измерять:
%
% The great accuracy\index{accuracy!altimeter} and precision of the
% Topex/Poseidon\index{Topex/Poseidon!accuracy of} and
% Jason\index{Jason!accuracy of} altimeter systems allow them to measure
% the oceanic topography over ocean basins with an accuracy of
% $\pm$(2--5) cm (Chelton et al, 2001). This allows them to measure:
\begin{enumerate}
\item
Изменения объема океана и скорость подъема его уровня с 
точностью~$\pm 0.4\mmperyr$, начиная с 1993 года (Nerem et al, 2006);
%
% \vitem Changes in the mean volume of the ocean and sea-level rise with
% an accuracy of $\pm 0.4$ mm/yr since 1993 (Nerem et al, 2006);

\item
Сезонный ход нагрева и охлаждения океана (Chambers et al 1998);
%
% \vitem Seasonal heating and cooling of the ocean (Chambers et al 1998);

\item
Высоты приливов в открытом океане с точностью~$\pm(1\mbox{--}2)\cm$ (Shum et al., 1997);
%
% \vitem Open ocean tides with an accuracy of $\pm$(1--2) cm (Shum et al, 1997);

\item
Диссипацию приливной энергии (Egbert and Ray, 1999; Rudnick et al, 2003);
%
% \vitem Tidal dissipation (Egbert and Ray, 1999; Rudnick et al, 2003);

\item
Постоянные поверхностные геострофические течения (Рис. 10.5);
%
% \vitem The permanent surface geostrophic current system (figure 10.5);

\item
Изменчивость поверхностных геострофических течений на всех масштабах
(Рис. 10.4);
%
% \vitem Changes in surface geostrophic currents on all scales 
% (figure 10.4); and

\item
Изменчивость топографии экваториальной системы течений, таких как
течения связанные с явлением Эль-Ниньо (Рис. 10.6)
%
% \vitem Variations in topography of equatorial current systems such as
% those associated with El Ni\~{n}o (figure 10.6).
\end{enumerate}

\begin{figure}[t!]
\makebox[120mm][c]{\includegraphics{pics/sshmean}}
\caption{Глобальное распределение усредненной по времени за период с
1992 - 2002 топографии океана по спутниковым и дрифтерным данным,
данным по ветру и модели GRACE Gravity Model-01 построенное Nikolai
Maximenko (IPRC) и Peter Niiler (SIO).  Геострофические течения
параллельны изолиниям. Сравните с Рис 2.8, построенному по
гидрографическим данным. Для большей информании смотрите статью
Maximenko (2005) . Данные Asia-Pacific Data-Research Center.}
\label{fig:sshmean}
\vspace{-4ex}
\end{figure}
%
% \begin{figure}[t!]
% \makebox[120mm][c]{\includegraphics{sshmean}}
% \footnotesize
% Figure 10.5 Global distribution of \rule{0pt}{3ex} time-averaged
% topography of the ocean from Topex/Pos\-eidon altimeter data from
% 10/3/92 to 10/6/99 relative to the \textsc{jgm}--3
% geoid\index{geoid}. Geostrophic cur\-rents at the ocean
% surface\index{geostrophic currents!from altimetry} are parallel to the
% contours. Compare with figure 2.8 calculated from hydrographic
% data\index{hydrographic data}. From Center for Space Research,
% University of Texas.
% \label{fig:sshmean}
% \vspace{-4ex}
% \end{figure}

\begin{figure}[t!]
\makebox[120mm][c]{\includegraphics{pics/texas-may01}}
\caption{Долготно временная развертка аномалий уровня в экваториальной
зоне Тихого океана по наблюдениям Topex/Poseidon в период исследований
явления Эль-Ниньйо 1997~--- 1998 гг. Теплые аномалии обозначены
светло-серым цветом, холодные – темно-серым. Аномалии возвышения
уровня получены из осредненных за 10 дней отклонений от трех летней
средней за период с 3-го октября 1992 г. по 8 октября 1995
г. поверхности. Данные сглажены гауссовой весовой функцией с полуосями
5о по долготе и 2о по широте. Слева приведены номера циклов
спутниковых данных. По данным Центра космических исследований
Техаского университета.}
\label{texas-may01}
\end{figure}
%
% \begin{figure}[t!]
% \makebox[120mm][c]{\includegraphics{texas-may01}}
% \footnotesize
% Figure 10.6 Time-longitude \rule{0pt}{3ex}plot of sea-level
% anomalies\index{anomalies!sealevel} in the Equatorial Pacific observed
% by Topex/Poseidon during the 1997--1998 El
% Ni\~{n}o\index{Topex/Poseidon!observations of El Ni\~{n}o}. Warm
% anomalies\index{anomalies!sea-surface temperature} are light gray,
% cold anomalies are dark gray. The anomalies are computed from 10-day
% deviations from a three-year mean surface from 3 Oct 1992 to 8 Oct
% 1995. The data are smoothed with a Gaussian weighted filter with a
% longitudinal span of 5\degrees and a latitudinal span of
% 2\degrees. The annotations on the left are cycles of satellite
% data. From Center for Space Research, University of Texas.
% \label{texas-may01}
% \vspace{-3ex}
% \end{figure}
%
% \vspace{-1ex}
\end{paragraph}

\begin{paragraph}{Ошибки альтиметрических измерений. (TROPEX/POSEIDON и Jason).}
% \paragraph{Altimeter Errors (Topex/Poseidon and Jason)}
Наиболее точными альтиметрическими измерениями являются данные
спутников TROPEX/POSEIDON и Jason. Основными источниками ошибок этих
данных являются (Chelton et al, 2001):
%
%  \index{satellite altimetry!errors in}\index{Topex/Poseidon}The most
% accurate observations of the sea-surface topography are from
% Topex/Poseidon and Jason\index{Jason}. Errors for these satellite
% altimeter system are due to (Chelton et al, 2001):
\begin{enumerate}
\item
Инструментальный шум, волнение на поверхности, водяной пар, свободные
электроны в ионосфере, масса атмосферного столба. Оба спутника
оборудованы высокоточной альтиметрической системой, способной измерять
высоту спутника над поверхностью моря в пределах~$\pm\degrees{66}$ по
широте с погрешностью~$\acc{1}{2}{\cm}$ и точностью~$\acc{2}{5}{\cm}$.
Эта система состоит из двухчастотного радарного альтиметра,
измеряющего высоту над поверхностью моря, влияние ионосферы и высоту
волнения, и трех частотного микроволнового радиометра для измерения
содержания водяного пара в тропосфере.
%
% \vitem Instrument noise, ocean waves, water vapor, free electrons in
% the ionosphere, and mass of the atmosphere. Both satellites carried a
% precise altimeter system able to observe the height of the satellite
% above the sea surface between $\pm$66\degrees\ latitude with a
% precision of $\pm$(1--2) cm and an accuracy\index{accuracy!altimeter}
% of $\pm$(2--5) cm. The systems consist of a two-frequency radar
% altimeter to measure height above the sea, the influence of the
% ionosphere, and wave height, and a three-frequency microwave
% radiometer able to measure water vapor in the troposphere.

\item
Ошибки положения трека. На борту спутника установлены три системы,
обеспечивающие его ориентацию в пространстве и эфемеридное
сопровождение с точностью~$\acc{1}{3.5}{\cm}$.
%
% \vitem Tracking errors. The satellites carried three tracking systems
% that enable their position in space, the ephemeris, to be determined
% with an accuracy\index{accuracy!satellite tracking systems} of
% $\pm$(1--3.5) cm.

\item
Ошибки измерений. Спутник измеряет высоты вдоль наземных треков,
циклически повторяющихся с точностью~$\pm 1\km$ через каждые
$9.9156$~дней. Поскольку течения измеряются только вдоль
подспутниковых треков и спутник не может регистрировать топографию
между треками, то возникают ошибки измерений. Также спутниковые данные
не позволяют регистрировать изменчивость топографии с периодами
меньшими~$2\times 9.9156$ дней (см.~\S~16.3).
%
% \vitem Sampling error. The satellites measure height along a ground
% track that repeats within $\pm$1 km every 9.9156 days. Each repeat is
% a cycle. Because currents are measured only along the sub-satellite
% track, there is a sampling error\index{sampling error}. The satellite
% cannot map the topography between ground tracks, nor can they observe
% changes with periods less than $2 \times 9.9156$ d (see \S 16.3).

\item
Ошибки формы геоида. Топография геоида плохо известна на масштабах
меньше ста километров, где доминируют локальные эффекты. Для
исследований главных особенностей постоянных поверхностных
геострофических течений используются топографические карты, сглаженные
на больших масштабах (Рис 10.5). Новые спутниковые системы GRACE и
CHAMP производят гравиметрические измерения с точностью достаточной
для того, чтобы пренебречь ошибками в форме геоида на масштабах
больших~$100\km$.
%
% \vitem Geoid error\index{geoid!errors}. The permanent topography is
% not well known over distances shorter than a hundred kilometers
% because geoid errors dominate for short distances. Maps of topography
% smoothed over greater distances are used to study the dominant
% features of the permanent geostroph\-ic currents at the sea surface
% (figure 10.5). New satellite systems \textsc{grace}\index{GRACE} and
% \textsc{champ} are measuring earth's gravity accurately enough
% \index{accuracy!geoid} that the geoid error is now small enough to
% ignore over distances greater than 100 km.
\end{enumerate}
Совокупный эффект всех вышеперечисленных ошибок приводит к точности
измерения высоты спутника над морской поверхностью~$\acc{2}{5}{\cm}$
вгеоцентрическойсистемекоординат.
%
% Taken together, the measurements of height above the sea and the
% satellite position give sea-surface height in geocentric coordinates
% within $\pm$(2--5) cm. Geoid error adds further errors that depend on
% the size of the area being measured.
\end{paragraph}
\end{section}

\begin{section}{Расчет геострофических течений по гидрографическим данным}
% \section{Geostrophic Currents From Hydrography}
Геострофические уравнения широко используются в океанографии для
расчетов глубинных течений. Основная идея состоит в использовании
гидрографических данных о температуре, солености или теплопроводности
и давлении для расчета поля плотности из уравнения состояния морской
воды, которая используется в (10.7b) для вычисления интегрального поля
давления, из которого, в свою очередь, рассчитываются геострофические
течения по уравнениям (10.8a,b). Однако, как правило, постоянная
интегрирования в (10.8) неизвестна, поэтому таким методом можно
получить лишь поле относительных скоростей.
%
% \index{geostrophic currents!from hydrographic data|(}The geostrophic
% equations are widely used in oceanography to calculate currents at
% depth. The basic idea is to use hydrographic\index{hydrographic data}
% measurements of temperature, salinity or conductivity, and pressure to
% calculate the density field of the ocean using the equation of state
% of sea water. Density is used in (10.7b) to calculate the internal
% pressure field, from which the geostrophic currents are calculated
% using (10.8a, b). Usually, however, the constant of integration in
% (10.8) is not known, and only the relative velocity field can be
% calculated.

Может возникнуть вопрос, почему бы не измерять непосредственно
давление, как это делается в метеорологии для расчета скорости ветра?
И нужны ли измерения давления для расчета плотности из уравнения
состояния? Ответ заключается в том, что даже очень небольшие изменения
глубины приводят к большим перепадам давления из-за большой плотности
воды. Ошибки в величине давления, вызванные ошибками определения
глубины избыточного давления, значительно больше ошибок при оценки
давления по скоростям течений. Например, используя (10.7a) для
градиента давления, соответствующего скорости течения $10\cmps$ на
широте~$\degrees{30}$, получаем величину~$7.5\times 10^{-3}\Papm$
или~$750\Pa$ на~$100\km$. Из уравнения гидростатики (10.5) перепад
давления в~$750\Pa$ соответствует изменению глубины на~$7.4\cm$, т.е. в
этом примере нам необходимо знать глубину избыточного давления с
точностью существенно лучшей, чем~$7.4\cm$, что не представляется
возможным.
%
% At this point, you may ask, why not just measure pressure directly as
% is done in meteorology, where direct measurements of pressure are used
% to calculate winds.  And, aren't pressure measurements needed to
% calculate density from the equation of state? The answer is that very
% small changes in depth make large changes in pressure because water is
% so heavy. Errors in pressure caused by errors in determining the depth
% of a pressure gauge are much larger than the pressure due to
% currents. For example, using (10.7a), we calculate that the pressure
% gradient due to a 10 cm/s current at 30\degrees latitude is $7.5
% \times 10^{-3}$ Pa/m, which is 750 Pa in 100 km. From the hydrostatic
% equation (10.5), 750 Pa is equivalent to a change of depth of 7.4
% cm. Therefore, for this example, we must know the depth of a pressure
% gauge with an accuracy\index{accuracy!pressure} of much better than
% 7.4 cm. This is not possible.

\begin{paragraph}{Геопотенциальные поверхности в толще океана.}
% \paragraph{Geopotential Surfaces Within the Ocean}
Градиенты давления на произвольной глубине должны рассчитываться на
поверхностях постоянного геопотенциала, также как это делалось для
поверхностных градиентов относительно геоида при расчете поверхностных
геострофических течений.

Уже в 1910 году, VilhelmBjerknes (BjerknesandSandstrom, 1910) пришел к
выводу, что эти поверхности не соответствуют фиксированным высотам в
атмосфере из-за того, ускорение силы тяжести g не является постоянной
величиной. Поэтому при расчете давления уравнение (10.4) должно
учитывать изменения g как по вертикали, так и по горизонтали
(Saunders, Fofonoff, 1976).
%
% \index{geopotential!surface}Calculation of pressure gradients within
% the ocean must be done along surfaces of constant geopotential just as
% we calculated surface pressure gradients relative to the
% geoid\index{geoid} when we calculated surface geostrophic currents. As
% long ago as 1910, Vilhelm Bjerknes (Bjerknes and Sandstrom, 1910)
% realized that such surfaces are not at fixed heights in the atmosphere
% because $g$ is not constant, and (10.4) must include the variability
% of gravity in both the horizontal and vertical directions (Saunders
% and Fofonoff, 1976) when calculating pressure in the ocean.

\emph{Геопотенциал $\Phi$} определяется как:
\begin{equation}
 \Phi =\int_0^z\,g dz
\end{equation}
В системе единиц СИ величина~$\Phi/9.8$ почти точно совпадает с
высотой в метрах, поэтому в метеорологическом сообществе общепринятой
традицией, основанной на предложении Бьеркнеса, является использование
динамических метров~$D = \Phi/10$ 0 для построения естественной
вертикальной координаты. Позднее они были заменены на геопотенциальные
метры, определяемые как~$Z = \Phi/9.80$. Геопотенциальный метр
является мерой работы, необходимой для подъема единичной массы от
уровня моря до высоты~$z$. Харальд Сврдруп, ученик Бьеркнеса, перенес
это понятие в океанографию, и глубины в океане часто указывают в
геопотенциальных метрах. Разница между высотами разделенными
постоянными шагом и соответствующими эвидистантными уровнями
потенциала может быть значительной. Так, глубина в 1000 динамических
метров соответствует $1017.40\m$ на полюсе и~$1022.78\m$ на экваторе,
т.е. разница составляет~$5.38\m$. Заметим, что глубина в
геопотенциальных метрах, глубина в метрах и давление в децибарах
численно практически совпадают. На глубине~$1\m$ давление приблизительно
равно~$1.007\dBar$ и эта глубина соответствует $1.00$~геопотенциальному
метру.

% The \textit{geopotential $\Phi$}\index{geopotential|textbf} is:
% \begin{equation}
% \Phi =\int_0^z\,g dz
% \end{equation}
% Because $\Phi/9.8$ in SI units has almost the same numerical value as
% height in meters, the meteorological community accepted Bjerknes'
% proposal that height be replaced by \textit{dynamic
% meters}\index{dynamic meter|textbf} $D = \Phi/10$ to obtain a natural
% vertical coordinate. Later, this was replaced by the
% \textit{geopotential meter}\index{geopotential!meter|textbf} (gpm) 
% $Z = \Phi/9.80$. The geopotential meter is a measure of the work required
% to lift a unit mass from sea level to a height $z$ against the force
% of gravity. Harald Sverdrup, Bjerknes' student, carried the concept to
% oceanography, and depths in the ocean are often quoted in geopotential
% meters. The difference between depths of constant vertical distance
% and constant potential can be relatively large. For example, the
% geometric depth of the 1000 dynamic meter surface is 1017.40 m at the
% north pole and 1022.78 m at the equator, a difference of 5.38 m.
%
% Note that depth in geopotential meters, depth in meters, and pressure
% in decibars are almost the same numerically. At a depth of 1 meter the
% pressure is approximately 1.007 decibars and the depth is 1.00
% geopotential meters.
\end{paragraph}

\begin{paragraph}{Уравнения геострофических течений в толще океана.}
% \paragraph{Equations for Geostrophic Currents Within the Ocean}
Для расчета скоростей геострофических течений в толще океана
необходимо знать горизонтальный градиент давления на данной
глубине. Для этого используются два подхода:
%
% \index{geostrophic currents!equations for}To calculate geo\-strophic
% currents, we need to calculate the horizontal pressure gradient
% with\-in the ocean. This can be done using either of two approaches:
\begin{enumerate}
\item
Рассчитывается наклон поверхности постоянного давления к поверхности
постоянного геопотенциала. Мы использовали этот метод и данные по
наклонам морской поверхности, полученные из спутниковых
альтиметрических измерений, для расчетов поверхностных геострофических
течений. При этом морская поверхность является поверхностью
постоянного давления, а геопотенциальной поверхностью является геоид.
%
% \vitem Calculate the slope of a constant pressure surface relative to
% a surface of constant geopotential. We used this approach when we used
% sea-surface slope from altimetry to calculate surface geostrophic
% currents. The sea surface is a constant-pressure surface. The constant
% geopotential surface was the geoid\index{geoid}.

\item
Рассчитываются изменения давления на поверхности постоянного
геопотенциала, т.н. \emph{геопотенциальной поверхности}.
%
% \vitem Calculate the change in pressure on a surface of constant
% geopotential. Such a surface is called a \textit{geopotential
% surface}\index{geopotential!surface|textbf}.
\end{enumerate}

\begin{figure}[h!]
\makebox[120mm][c]{\includegraphics{pics/hydrosketch}}
\caption{Схематичный рисунок, поясняющий процедуру расчета
геострофических течений по гидрографическим данным.} 
\label{fig:hydrosketch}
\end{figure}
%
% \begin{figure}[h!]
% \vspace{-2ex}
% \makebox[120mm][c]{\includegraphics{hydrosketch}}
% \centering
% \footnotesize
% Figure 10.7. Sketch of \rule{0mm}{3ex}geometry used for calculating
% geostrophic current from hydrography.
%
% \label{fig:hydrosketch}
% \vspace{-2ex}
% \end{figure}

В океанографии обычно рассчитывают наклоны поверхностей постоянного
давления. Основные этапы такого расчета следующие:
%
% Oceanographers usually calculate the slope of constant-pressure
% surfaces.  The important steps are:
\begin{enumerate}
\item
Вычисляются разности геопотенциала~$\left( \Phi_A - \Phi_B \right)$ 
между поверхностями постоянного давления $\left( P_1 , P_2 \right)$ 
на гидрографических станциях~A и~B (Рис. 10.7), что то же самое, 
что и определение~$\zeta$ на поверхности.
%
% \vitem Calculate differences in geopotential $\left( \Phi_A - \Phi_B
% \right)$ between two constant-pressure surfaces $\left( P_1 , P_2
% \right)$ at hydrographic stations\index{hydrographic stations} A and B
% (figure 10.7). This is similar to the calculation of $\zeta$ of the
% surface layer.

\item
Рассчитывается наклон верхней поверхности давления к нижней.
%
% \vitem Calculate the slope of the upper pressure surface relative to
% the lower.

\item
Рассчитывается геострофический поток на верхней поверхности
относительно потока на нижней, т.е. потоковый сдвиг.
%
% \vitem Calculate the geostrophic current at the upper surface relative
% to the current at the lower. This is the current shear.

\item
Потоковый сдвиг интегрируется от некоторой глубины, на которой течение
известно, и получается зависимость течения от глубины. Например, можно
интегрировать от поверхности вглубь, используя спутниковые
альтиметрические данные о поверхностных геострофических течениях или
интегрировать вверх от слоя, на котором принимается условие отсутствия
течений.
%
% \vitem Integrate the current shear from some depth where currents are
% known to obtain currents as a function of depth. For example, from the
% surface downward, using surface geostrophic currents observed by
% satellite altimetry, or upward from an assumed level of no motion.
\end{enumerate}
Для расчета геострофических течений в океанографии используется
несколько модифицированная форма уравнения гидростатики. Вертикальный
градиент давления (10.6) записывается в виде:
\begin{subequations}
\begin{align}
 \frac{\delta p}{\rho}=\alpha\,\delta p &=-g\,\delta z \\
 \alpha\,\delta p&=\delta\Phi
\end{align}
\end{subequations}
где~$\alpha = \alpha(S,t,p)$~--- \emph{удельныйобъем}, а (10.12b)
следует из (10.11). Дифференцируя (10.12a) по горизонтальной
координате~$x$, приходим, с использованием (10.6) и с 
обозначением~$f = 2 \Omega\sin \phi $, к следующей форме уравнений 
геострофического баланса:
\begin{subequations}
 \begin{align}
  \alpha\frac{\partial p}{\partial x} 
    =\frac{1}{\rho}\,\frac{\partial p}{\partial x} 
  & =-2\,\Omega \,v\sin \varphi \\
  \frac{\partial \Phi \left( p=p_0 \right)} {\partial x}
  & =-2\,\Omega \,v \, \sin{\varphi}
\end{align}
\end{subequations}
где~$\Phi$ это геопотенциал на поверхности постоянного давления.
%
% To calculate geostrophic currents oceanographers use a modified form
% of the hydrostatic equation. The vertical pressure gradient (10.6) is
% written
% \begin{subequations}
% \begin{align}
% \frac{\delta p}{\rho}=\alpha\,\delta p &=-g\,\delta z \\
% \alpha\,\delta p&=\delta\Phi
% \end{align}
% \end{subequations}
% where $\alpha = \alpha(S,t,p)$ is the \textit{specific
% volume}\index{specific volume|textbf}, and (10.12b) follows from
% (10.11). Differentiating (10.12b) with respect to horizontal distance
% $x$ allows the geo\-stro\-phic balance to be written in terms of the
% slope of the constant-pressure surface using (10.6) with $f = 2 \Omega
% \sin \phi $:
% \begin{subequations}
% \begin{align}
% \alpha\,\frac{\partial p}{\partial x} =\frac{1}{\rho}\,\frac{\partial
% p}{\partial x} &=-2\,\Omega \,v\sin \varphi \\
% \frac{\partial \Phi \left( p=p_0 \right)} {\partial x}
%  &= - 2 \, \Omega \,v \, \sin{\varphi}
% \end{align}
% \end{subequations}
% where $\Phi$ is the geopotential at the constant-pressure surface.

Теперь рассмотрим, как гидрографические данные используются для
определения величины~$\partial \Phi/\partial x$ на поверхности
постоянного давления. Интегрирование (10.12а) между двумя
поверхностями постоянного давления $\left( P_1 , P_2 \right)$, как
показано на Рис. 10.7, дает разность геопотенциала между ними. На
гидрографической станции~A имеем:
\begin{equation}
 \Phi\left(P_{1A}\right)-\Phi\left(P_{2A}\right)
   =\int_{P_{1A}}^{P_{2A}} \alpha\left(S,t,p\right)\,dp
\end{equation}
Удельный объем записывается как сумма некоторого базового значения и
отклонения от него:
\begin{equation}
 \alpha(S,t,p)=\alpha(35,0,p)+\delta
\end{equation}
где~$\alpha (35,0,p)$ это удельный объем морской воды с соленостью~$35$ при
температуре~$\degCent{0}$ и давлении~$p$, ?$\delta$~представляет собой его
аномалию. Используя (10.15) и (10.14) получаем:
\begin{align}
 \Phi(P_{1A})-\Phi(P_{2A})
   & = \int_{P_{1A}}^{P_{2A}} \alpha(35,0,p)\, dp
      +\int_{P_{1A}}^{P_{2A}} \delta \,dp \\
 \Phi(P_{1A})-\Phi(P_{2A})
   & = \left(\Phi_1-\Phi_2 \right)_{std} + \Delta\Phi_A
\end{align}
где $(\Phi_1-\Phi_2 )_{std}$ \emph{стандартное геопотенциальное
расстояние} между поверхностями~$P_1$, и~$P_2$, а
\begin{equation}
 \Delta\Phi_A =\int_{P_{1A}}^{P_{2A}} \,\delta\, dp
\end{equation}
есть аномалия этого расстояния, обозначаемая как геопотенциальная
аномалия. Геометрическое расстояние между $\Phi_1$ и $\Phi_2$
приблизительно равно $(\Phi_2 - \Phi_1) /g$, $g= 9.8\mpsqs$~--- приближенное
значение ускорения силы тяжести. Геопотенциальная аномалия составляет
примерно~$0.1\%$ от значения стандартного геопотенциального расстояния,
т.е. является малой величиной.
%
% Now let's see how hydrographic data\index{hydrographic data} are used
% for evaluating $\partial \Phi/\partial x$ on a constant-pressure
% surface. Integrating (10.12b) between two constant-pressure surfaces
% $\left( P_1 , P_2 \right)$ in the ocean as shown in figure 10.7 gives
% the geopotential difference between two constant-pressure surfaces. At
% station A the integration gives:
% \begin{equation}
% \Phi\left(P_{1A}\right)-\Phi\left(P_{2A}\right)=\int_{P_{1A}}^{P_{2A}}
% \alpha\left(S,t,p\right)dp
% \end{equation}
% The specific volume anomaly is written as the sum of two parts:
% \begin{equation}
% \alpha(S,t,p)=\alpha(35,0,p)+\delta
% \end{equation}
% where $\alpha (35,0,p)$ is the specific volume of sea water with
% salinity of 35, temperature of 0\degrees C, and pressure $p$. The
% second term $\delta$ is the \textit{specific volume
% anomaly}\index{specific volume!anomaly|textbf}. Using (10.15) in
% (10.14) gives:
% \begin{align}
% \Phi(P_{1A})-\Phi(P_{2A})&=\int_{P_{1A}}^{P_{2A}}\,\alpha(35,0,p)\, dp +\int_{P_{1A}}^{P_{2A}}
% \delta \,dp \notag \\
% \Phi(P_{1A})-\Phi(P_{2A})&=\left(\Phi_1-\Phi_2 \right)_{std}
% +\Delta\Phi_A \notag
% \end{align}
% where ($\Phi_1-\Phi_2 )_{std}$ is the \textit{standard geopotential
% distance}\index{standard geopotential distance|textbf} between two
% constant-pressure surfaces $P_1$ and $P_2$, and
% \begin{equation}
% \Delta\Phi_A =\int_{P_{1A}}^{P_{2A}} \,\delta\, dp
% \end{equation}
% is the anomaly of the geopotential distance between the surfaces. It
% is called the \textit{geopotential
% anomaly}\index{geopotential!anomaly|textbf}. The geometric distance
% between $\Phi_2$ and $\Phi_1$ is numerically approximately $(\Phi_2 -
% \Phi_1) /g$ where $g= 9.8$m/s$^2$ is the approximate value of
% gravity. The geopotential anomaly is much smaller, being approximately
% 0.1\% of the standard geopotential distance.

Теперь рассмотрим геопотенциальную аномалию между двумя
поверхностями~$P_1$ и~$P_2$ рассчитываемую на двух гидрографических
станциях~А и~В, находящихся на расстоянии~$L$ друг от друга
(Рис. 10.7). Для простоты примем нижнюю поверхность постоянного
давления в качестве отсчетной. Тогда поверхности постоянного давления
и геопотенциальные поверхности совпадают, и на этой глубине нет
геострофического течения. Наклон верхней поверхности дается:
\begin{displaymath}
 \frac{\Delta\Phi_B - \Delta\Phi_A}{L} 
  =\text{наклон поверхности постоянного давления $P_2$}
\end{displaymath}
т.к. стандартное геопотенциальное расстояние на станциях А и В одно и тоже.

Скорость геострофического течения на поверхности определяется 
из (10.13b) как:
\begin{equation}
  V =\frac{\left(\Delta\Phi_B - \Delta\Phi_A\right)}{2\Omega\,L\, \sin\varphi }
\end{equation}
где~$V$ это скорость на верней геопотенциальной поверхности. Скорость~$V$
перпендикулярна плоскости, содержащей гидрографические станции и
направлена, в северном полушарии, от читателя (Рис. 10.7). \emph{Полезное
мнемоническое правило состоит в том, что более теплая, менее плотная
вода движется вниз по правоориентированному в северном полушарии
течению.}
%
% Consider now the geopotential anomaly between two pressure surfaces
% $P_1$ and $P_2$ calculated at two hydrographic
% stations\index{hydrographic stations} A and B a distance $L$ meters
% apart (figure 10.7). For simplicity we assume the lower
% constant-pressure surface is a level surface\index{level
% surface}. Hence the constant-pressure and geopotential surfaces
% coincide, and there is no geostrophic velocity at this depth. The
% slope of the upper surface is
% \begin{displaymath}
% \frac{\Delta\Phi_B - \Delta\Phi_A}{L} =\text{slope of constant-pressure
% surface $P_2$}
% \end{displaymath}
% because the standard geopotential distance is the same at stations A
% and B. The geostrophic velocity\index{geostrophic currents} at the
% upper surface calculated from (10.13b) is:
% \begin{equation}
% V =\frac{\left(\Delta\Phi_B - \Delta\Phi_A\right)}{2\Omega\,L\, \sin\varphi }
% \end{equation}
% where $V$ is the velocity at the upper geopotential surface. The
% velocity $V$ is perpendicular to the plane of the two hydrographic
% stations\index{hydrographic stations} and directed into the plane of
% figure 10.7 if the flow is in the northern hemisphere. \textit{A
% useful rule of thumb is that the flow is such that warmer, lighter
% water is to the right looking downstream in the northern hemisphere.}

В принципе, мы могли бы рассчитать наклон поверхности постоянного
давления используя плотность~$\rho$ вместо удельного
объёма~$\alpha$. Я использовал удельный объём, т.к. это является
общепринятой практикой в океанографии для чего существуют таблицы и
компьютерные программы для расчетов аномалий. Эта практика сложилась
на основе расчетных методов, разработанных задолго до появления
калькуляторов и компьютеров, когда все вычисления производились
вручную или с помощью механических калькуляторов, таблиц и номограмм.
%
% Note that I could have calculated the slope of the constant-pressure
% surfaces using density $\rho$ instead of specific volume $\alpha$. I
% used $\alpha$ because it is the common practice in oceanography, and
% tables of specific volume anomalies\index{anomalies!specific volume}
% and computer code to calculate the anomalies are widely available. The
% common practice follows from numerical methods developed before
% calculators and computers were available, when all calculations were
% done by hand or by mechanical calculators with the help of tables and
% nomograms. Because the computation must be done with an
% accuracy\index{accuracy!density} of a few parts per million, and
% because all scientific fields tend to be conservative, the common
% practice has continued to use specific volume anomalies rather than
% density anomalies\index{anomalies!density}.
\end{paragraph}

\begin{paragraph}{Баротропные и бароклинные течения:}
% \paragraph{Barotropic and Baroclinic Flow:}
Если бы океан представлял собой однородную среду с постоянной
плотностью, то поверхности постоянного давления всегда были бы
параллельны морской поверхности и скорости геострофических течений
были бы независимы от глубины. В этом случае, относительная скорость
равна нулю и \emph{гидрографические} данные не дают информации о
геострофических течениях. Если плотность варьируется по глубине, но не
зависит от горизонтальных координат, то поверхности постоянного
давления всегда параллельны морской поверхности и уровням постоянной
плотности~--- \emph{изопикническим поверхностям}. В этом случая
относительный поток также равен нулю. Оба случая являются примерами
\emph{баротропного течения}.
%
% If the ocean were homogeneous with constant density, then
% constant-pressure surfaces would always be parallel to the sea
% surface, and the geostrophic velocity would be independent of
% depth. In this case the relative velocity is zero, and hydrographic
% data\index{hydrographic data!and geostrophic currents} cannot be used
% to measure the geostrophic current. If density varies with depth, but
% not with horizontal distance, the constant-pressure surfaces are
% always parallel to the sea surface and the levels of constant density,
% the \textit{isopycnal surfaces}\index{isopycnal surfaces|textbf}. In
% this case, the relative flow is also zero. Both cases are examples of
% \textit{barotropic flow}.

\emph{Баротропные течения} возникают тогда, когда уровни постоянного
давления в океане всегда остаются параллельными поверхностям
постоянной плотности. Заметим, что некоторые исследователи называют
усредненный по вертикали полный поток баротропной компонентой
течения. Wunsh (1996:74) даже предлагает отказаться от использования
термина баротропный ввиду его многократного употребления в самых
различных смыслах.
%
% \textit{Barotropic flow}\index{barotropic flow|textbf} occurs when
% levels of constant pressure in the ocean are always parallel to the
% surfaces of constant density. Note, some authors call the vertically
% averaged flow the barotropic component of the flow.  Wunsch (1996: 74)
% points out that barotropic is used in so many different ways that the
% term is meaningless and should not be used.

\emph{Бароклииный поток} возникает при отличном от нуля наклоне
поверхностей постоянной постоянного давления к поверхностям постоянной
плотности. Рис. 10.8 хорошо демонстрирует, как поверхности постоянной
плотности меняют глубину своего залегания более чем на $1\km$ на
горизонтальной протяженности в $100\km$ в районе
Гольфстрима. Бароклинный поток меняется с глубиной, поэтому сдвиговое
течение может быть рассчитано по гидрографическим данным. При этом
обратите внимание, что для жидкости в состоянии покоя поверхности
постоянной плотности имею нулевой наклон к поверхностям постоянного
давления.
%
% \textit{Baroclinic flow}\index{baroclinic flow|textbf} occurs when
% levels of constant pressure are inclined to surfaces of constant
% density. In this case, density varies with depth and horizontal
% position. A good example is seen in figure 10.8 which shows levels of
% constant density changing depth by more than 1 km over horizontal
% distances of 100 km at the Gulf Stream\index{Gulf Stream!is
% baroclinic}. Baroclinic flow varies with depth, and the relative
% current can be calculated from hydrographic data\index{hydrographic
% data!and geostrophic currents}. Note, constant-density surfaces cannot
% be inclined to constant-pressure surfaces for a fluid at rest.

В общем случае, зависимость потока от вертикальной координаты может
быть представлена в виде баротропной компоненты, постоянной по
вертикали, и бароклинной, меняющейся в этом направлении.
%
% In general, the variation of flow in the vertical can be decomposed
% into a barotropic component which is independent of depth, and a
% baroclinic component which varies with depth.
\end{paragraph}
\end{section}

\begin{section}{Пример использования гидрографических данных.}
% \section{An Example Using Hydrographic Data} 
Рассмотрим теперь конкретный пример численного расчета скоростей
геострофического течения с использованием методики Обработки данных
океанографических станций (JPOTSEditorialPanel, 1991). В этой книге
рассматриваются реальные примеры использования гидрографических
данных, собранных исследовательским судном Endeavorв Северной
Атлантике. Данные были собраны во время рейса \No~88 вдоль~\latlon{71}{W}
через Гольфсирим к югу от Кейп Код, Массачусетс, на станциях 61 и
64. Станция 61 расположена в Саргассово море в точке с глубиной 4260
метров, а станция 64~--- к северу от Гольфстрима в районе с глубиной
3892 метра. Измерения проводились глубинным терморезисторным
кислородным детектором MarkIIICDT/02 производства <<Нейл Браун
Инструмент Системс>>.
%
% Let's now consider a specific \index{geostrophic currents!from
% hydrographic data}numerical calculation of geostrophic velocity using
% generally accepted procedures from \textit{Processing of Oceanographic
% Station Data} (\textsc{jpots} Editorial Panel, 1991). The book has
% worked examples using hydrographic data\index{hydrographic data!from
% Endeavor} collected by the \textsc{r/v} \textit{Endeavor} in the north
% Atlantic. Data were collected on Cruise 88 along 71\degrees W across
% the Gulf Stream\index{Gulf Stream!south of Cape Cod} south of Cape
% Cod, Massachusetts at stations 61 and 64. Station 61 is on the
% Sargasso Sea side of the Gulf Stream in water 4260 m deep. Station 64
% is north of the Gulf Stream in water 3892 m deep. The measurements
% were made by a Conductivity-Temp\-erature-Depth-Oxygen Profiler, Mark
% III CTD/02\index{CTD}, made by Neil Brown Ins\-truments Systems.

Этот прибор регистрирует температуру, соленость и давление со
скоростью 22 отсчета в секунду, причем цифровые данные усредняются по
2-х децибарному интервалу при погружении вглубь. Данные выдаются через
2-х децибарные интервалы центрированные по нечетным значениям величины
давления, т.к. первый отсчет делается на поверхности, а первый
интервал осреднения простирается до давления 2 децибара с центром при
давлении 1 децибар. Затем данные сглаживаются биномиальным фильтром и
линейно интерполируются к стандартным уровням приведенным в первых
трех столбцах таблиц 10.2 и 10.3. Всяобработкаосуществляетсяавтоматически.
%
% The \textsc{ctd} sampled temperature, salinity, and pressure 22 times
% per second, and the digital data were averaged over 2 dbar intervals
% as the \textsc{ctd} was lowered in the water. Data were tabulated at 2
% dbar pressure intervals centered on odd values of pressure because the
% first observation is at the surface, and the first averaging interval
% extends to 2 dbar, and the center of the first interval is at 1
% dbar. Data were further smoothed with a binomial filter and linearly
% interpolated to standard levels reported in the first three columns of
% tables 10.2 and 10.3. All processing was done by computer.

$\delta (S, t, p)$~вычисляется по значениям $t$, $S$, $p$ в
соответствующем слое и приведены в пятой колонке таблиц 10.2 и
10.3. Среднее значение аномалии удельного объёма~$<\delta >$
приводится для данного слоя между отсчетными поверхностями давления
как среднее между значениями~$\delta (S, t, p)$ верхней и нижней
поверхности слоя. (см. теорему о среднем в курсе математического
анализа). В последней колонке $(10^{-5}\Delta\Phi)$ приводится
произведение средней аномалии удельного объёма на толщину слоя в
децибарах. Таким образом, последняя колонка дает значение перепада
геопотенциала~$\Delta \Phi$ между точками~$P_1$ и~$P_2$ в
соответствующем слое, полученная интегрированием (10.16) по толщине
слоя.
%
% $\delta (S, t, p)$ in the fifth column of tables 10.2 and 10.3 is
% calculated from the values of $t, S, p$ in the layer.  $<\delta >$ is
% the average value of specific volume anomaly for the layer between
% standard pressure levels. It is the average of the values of $\delta
% (S, t, p)$ at the top and bottom of the layer (\textit{cf.} the
% mean-value theorem of calculus). The last column $(10^{-5}
% \Delta\Phi)$ is the product of the average specific volume anomaly of
% the layer times the thickness of the layer in decibars. Therefore, the
% last column is the geopotential anomaly $\Delta \Phi$ calculated by
% integrating (10.16) between $P_1$ at the bottom of each layer and
% $P_2$ at the top of each layer.

Расстояние между станциями $L = 110\,935$, среднее значение параметра
Кориолиса~$f = 0.88104 \times 10^{-4}$ и знаменатель в (10.17)
равен~$0.10231\spm$. Эти данные используются для расчета
геострофических течений относительно уровня 2000 децибар, приведенных
в таблице 10.4 и изображенных на Рис. 10.8.
%
% The distance between the stations is $L = 110,935$ m; the average
% Coriolis parameter\index{Coriolis parameter} is $f = 0.88104 \times
% 10^{-4}$; and the denominator in (10.17) is 0.10231 s/m. This was used
% to calculate the geostrophic currents relative to 2000 decibars
% reported in table 10.4 and plotted in figure 10.8.

Обратите внимание, что на Рис. 10.8 нет Экмановских
течений. Экмановские течения не являются геострофическими, поэтому они
не вносят вклад в деформацию поверхности. Опосредованный вклад
проявляет себя через явление Экмановской накачки (см. Рис. 12.7).
%
% Notice that there are no Ekman\index{Ekman layer} currents in figure
% 10.8.  Ekman currents are not geostrophic, so they don't contribute
% directly to the topography. They contribute only indirectly through
% Ekman pumping (see figure 12.7).

\begin{table}[t!]
\caption{Расчет относительных (сдвиговых) геострофических течений} 
\begin{narrower}
по данным 88-го рейса «Эндевор» на станции 61.  
(\latlonmin{36}{40.03}{N}, \latlonmin{70}{59.59}{W} ; 23 августа 1982 г.; 1102Z)
\end{narrower}
\renewcommand{\baselinestretch}{0.0}
\begin{small}
% \centering
% \renewcommand{\baselinestretch}{0.0} \small
% \begin{tabular*}{108mm}{@{}rrrrrrl}
% \multicolumn{7}{@{}l@{}}{\bfseries Table 10.2 Computation of Relative Geostrophic Currents.} \\
% & \multicolumn{6}{@{}l@{}}{\bfseries \rule{0mm}{2.4ex}Data from Endeavor Cruise 88, Station 61} \\
% & \multicolumn{6}{@{}l@{}}{\bfseries (36\degrees 40.03'N, 70\degrees 59.59'W; 
%   \rule[-1ex]{0mm}{3.5ex}23 August 1982; 1102Z)} \\
\begin{center}
\begin{tabular}{rrrrrrl}
\hline
Давление&$t$ & $S$ &$\sigma (\theta)$&$\delta(S,t,p)$ &$<\delta >$&$10^{-5}\Delta\Phi$ \\ 
$\dBar$&$\degCent{}$ &  &$\kgpcm$&$10^{-8}\cubmpkg$&$10^{-8}\cubmpkg$&$\sqmpsqs$\\
\rule[-1ex]{0mm}{1ex}&  \\
\hline
\rule[-1ex]{0mm}{1ex}&  \\
0&      25.698& 35.221& 23.296& 457.24& \\
 &            &       &       &       & 457.26& 0.046\\
1&      25.698& 35.221& 23.296& 457.28& \\
 &            &       &       &       & 440.22& 0.396\\
10&     26.763& 36.106& 23.658& 423.15& \\
 &            &       &       &       & 423.41& 0.423\\
20&     26.678& 36.106& 23.658& 423.66& \\
 &            &       &       &       & 423.82& 0.424\\
30& 26.676& 36.107& 23.659& 423.98& \\
 &            &       &       &       & 376.23& 0.752\\
50& 24.528& 36.561& 24.670& 328.48& \\
 &            &       &       &       & 302.07& 0.755\\
75& 22.753& 36.614& 25.236& 275.66& \\
 &            &       &       &       & 257.41& 0.644\\
100&    21.427& 36.637& 25.630& 239.15& \\
 &            &       &       &       & 229.61& 0.574\\
125&    20.633& 36.627& 25.841& 220.06& \\
 &            &       &       &       & 208.84& 0.522\\
150&    19.522& 36.558& 26.086& 197.62& \\
 &            &       &       &       & 189.65& 0.948\\
200&    18.798& 36.555& 26.273& 181.67& \\
 &            &       &       &       & 178.72& 0.894\\
250&    18.431& 36.537& 26.354& 175.77& \\
 &            &       &       &       & 174.12& 0.871\\
300&    18.189& 36.526& 26.408& 172.46& \\
 &            &       &       &       & 170.38& 1.704\\
400&    17.726& 36.477& 26.489& 168.30& \\
 &            &       &       &       & 166.76& 1.668\\
500&    17.165& 36.381& 26.557& 165.22& \\
 &            &       &       &       & 158.78& 1.588\\
600&    15.952& 36.105& 26.714& 152.33& \\
 &            &       &       &       & 143.18& 1.432\\
700&    13.458& 35.776& 26.914& 134.03& \\
 &            &       &       &       & 124.20& 1.242\\
800&    11.109& 35.437& 27.115& 114.36& \\
 &            &       &       &       & 104.48& 1.045\\
900&    8.798&  35.178& 27.306& 94.60&  \\
 &            &       &       &       & 80.84&  0.808\\
1000&   6.292&  35.044& 27.562& 67.07&  \\
 &            &       &       &       & 61.89&  0.619\\
1100&   5.249&  35.004& 27.660& 56.70&  \\
 &            &       &       &       & 54.64&  0.546\\
1200&   4.813&  34.995& 27.705& 52.58&  \\
 &            &       &       &       & 51.74&  0.517\\
1300&   4.554&  34.986& 27.727& 50.90&  \\
 &            &       &       &       & 50.40&  0.504\\
1400&   4.357&  34.977& 27.743& 49.89&  \\
 &            &       &       &       & 49.73&  0.497\\
1500&   4.245&  34.975& 27.753& 49.56&  \\
 &            &       &       &       & 49.30&  1.232\\
1750&   4.028&  34.973& 27.777& 49.03&  \\
 &            &       &       &       & 48.83&  1.221\\
2000&   3.852&  34.975& 27.799& 48.62&  \\
 &            &       &       &       & 47.77&  2.389\\
2500&   3.424&  34.968& 27.839& 46.92&  \\
 &            &       &       &       & 45.94&  2.297\\
3000&   2.963&  34.946& 27.868& 44.96&  \\
 &            &       &       &       & 43.40&  2.170\\
3500&   2.462&  34.920& 27.894& 41.84&  \\
 &            &       &       &       & 41.93&  2.097\\
4000&   2.259&  34.904& 27.901& 42.02 \\
\rule[-1ex]{0mm}{1ex}&  \\
\hline
\end{tabular} \\
\end{center}
\end{small}
% \end{tabular*} \\[0.5ex]
% \vspace{-3.ex}
\end{table}

\begin{table}[t!]
\caption{Расчет относительных (сдвиговых) геострофических течений по
данным 88-го рейса <<Эндевор>> на станции 64.  
(\latlonmin{37}{39.93}{N}, \latlonmin{71}{00.00}{W}; 24 августа 1982 г.; 0203Z)} 
% \centering
\renewcommand{\baselinestretch}{0.0} 
\begin{small}
\begin{center}
% \begin{tabular*}{108mm}{@{}rrrrrrl}
% \multicolumn{7}{@{}l@{}}{\bfseries Table 10.3 Computation of Relative Geostrophic Currents.} \\
% & \multicolumn{6}{@{}l@{}}{\bfseries \rule{0mm}{2.4ex}Data from Endeavor Cruise 88, Station 64} \\
% & \multicolumn{6}{@{}l@{}}{\bfseries (37\degrees 39.93'N, 71\degrees 0.00'W; \rule[-1ex]{0mm}{3.5ex}24 August 1982;  0203Z)} \\
\begin{tabular}{rrrrrrl}
\hline
Давление&$t$ & $S$ &$\sigma (\theta)$&$\delta(S,t,p)$ &$<\delta >$&$10^{-5}\Delta\Phi$ \\ 
$\dBar$&$\degCent{}$ &  &$\kgpcm$&$10^{-8}\cubmpkg$&$10^{-8}\cubmpkg$&$\sqmpsqs$\\
\rule[-1ex]{0mm}{1ex}&  \\
\hline
\rule[-1ex]{0mm}{1ex}&  \\
0&  26.148& 34.646& 22.722& 512.09&\\
 &            &       &       &       &     512.15& 0.051\\
1&  26.148& 34.646& 22.722& 512.21&\\
 &            &       &       &       &     512.61& 0.461\\
10& 26.163& 34.645& 22.717& 513.01&\\
 &            &       &       &       &     512.89& 0.513\\
20& 26.167& 34.655& 22.724& 512.76&\\
 &            &       &       &       &     466.29& 0.466\\
30& 25.640& 35.733& 23.703& 419.82&\\
 &            &       &       &       &     322.38& 0.645\\
50& 18.967& 35.944& 25.755& 224.93&\\
 &            &       &       &       &     185.56& 0.464\\
75& 15.371& 35.904& 26.590& 146.19&\\
 &            &       &       &       &     136.18& 0.340\\
100&    14.356& 35.897& 26.809& 126.16&\\
 &            &       &       &       &     120.91& 0.302\\
125&    13.059& 35.696& 26.925& 115.66&\\
 &            &       &       &       &     111.93& 0.280\\
150&    12.134& 35.567& 27.008& 108.20&\\
 &            &       &       &       &     100.19& 0.501\\
200&    10.307& 35.360& 27.185& 92.17&\\
 &            &       &       &       &     87.41&  0.437\\
250&    8.783&  35.168& 27.290& 82.64&\\
 &            &       &       &       &     79.40&  0.397\\
300&    8.046&  35.117& 27.364& 76.16&\\
 &            &       &       &       &     66.68&  0.667\\
400&    6.235&  35.052& 27.568& 57.19&\\
 &            &       &       &       &     52.71&  0.527\\
500&    5.230&  35.018& 27.667& 48.23&\\
 &            &       &       &       &     46.76&  0.468\\
600&    5.005&  35.044& 27.710& 45.29&\\
 &            &       &       &       &     44.67&  0.447\\
700&    4.756&  35.027& 27.731& 44.04&\\
 &            &       &       &       &     43.69&  0.437\\
800&    4.399&  34.992& 27.744& 43.33&\\
 &            &       &       &       &     43.22&  0.432\\
900&    4.291&  34.991& 27.756& 43.11&\\
 &            &       &       &       &     43.12&  0.431\\
1000&   4.179&  34.986& 27.764& 43.12&\\
 &            &       &       &       &     43.10&  0.431\\
1100&   4.077&  34.982& 27.773& 43.07&\\
 &            &       &       &       &     43.12&  0.431\\
1200&   3.969&  34.975& 27.779& 43.17&\\
 &            &       &       &       &     43.28&  0.433\\
1300&   3.909&  34.974& 27.786& 43.39&\\
 &            &       &       &       &     43.38&  0.434\\
1400&   3.831&  34.973& 27.793& 43.36&\\
 &            &       &       &       &     43.31&  0.433\\
1500&   3.767&  34.975& 27.802& 43.26&\\
 &            &       &       &       &     43.20&  1.080\\
1750&   3.600&  34.975& 27.821& 43.13&\\
 &            &       &       &       &     43.00&  1.075\\
2000&   3.401&  34.968& 27.837& 42.86&\\
 &            &       &       &       &     42.13&  2.106\\
2500&   2.942&  34.948& 27.867& 41.39&\\
 &            &       &       &       &     40.33&  2.016\\
3000&   2.475&  34.923& 27.891& 39.26&\\
 &            &       &       &       &     39.22&  1.961\\
3500&   2.219&  34.904& 27.900& 39.17&\\
 &            &       &       &       &     40.08&  2.004\\
4000&   2.177&  34.896& 27.901& 40.98  \\
\rule[-1ex]{0mm}{1ex}&  \\
\hline
\end{tabular} \\
\end{center}
\end{small}
% \vspace{-3ex}
\end{table}

\begin{table}[t!]
\caption{Computation of Relative Geostrophic Currents.}
\begin{center}
Data from Endeavor Cruise 88, Station 61 and 64
\end{center}
\begin{center}
\renewcommand{\baselinestretch}{0.0}
\begin{small}
\begin{tabular}{rrrrrrl}
% \begin{tabular*}{97mm}{@{}rrrrrrl}
% \multicolumn{6}{@{}l@{}}{\bfseries Table 10.4 Computation of Relative Geostrophic Currents.} \\
% & \multicolumn{5}{@{}l@{}}{\bfseries Data from Endeavor Cruise 88, Station 61 and 64}\rule[-1ex]{0mm}{3.5ex}\\
\hline
\rule[-1ex]{0mm}{1ex}&  \\
Давление &$10^{-5}\Delta\Phi_{61}$ & $\Sigma\Delta\Phi $ &$10^{-5}\Delta\Phi_{64}$ &$\Sigma\Delta\Phi $ & $V$ \\
$\dBar$&$\sqmpsqs$ &ст.~61$^\ast$ &$\sqmpsqs$ &ст.~64$^\ast$&$\mps$ \\
\rule[-1ex]{0mm}{1ex}&  \\
\hline
\rule[-1ex]{0mm}{1ex}&  \\
0&              &2.1872 &        &1.2583 &0.95\rule{0mm}{2.5ex}\\
 &      0.046  &       & 0.051              \\
1&              &2.1826 &        &1.2532 &0.95\\
 &      0.396  &       & 0.461              \\
10&             &2.1430 &        &1.2070& 0.96\\
 &      0.423  &       & 0.513              \\
20&             &2.1006 &          &1.1557& 0.97\\
 &      0.424  &       & 0.466              \\
30&             &   2.0583&        &1.1091& 0.97\\
 &      0.752  &       & 0.645              \\
50&             &   1.9830&        &    1.0446 &0.96\\
 &      0.755  &       & 0.464              \\
75&               & 1.9075&        &    0.9982 &0.93\\
 &      0.644  &       & 0.340              \\
100&               &    1.8431&        &    0.9642& 0.90\\
 &      0.574  &       & 0.302              \\
125&            &   1.7857&        &    0.9340& 0.87\\
 &      0.522  &       & 0.280              \\
150&               &    1.7335&        &    0.9060& 0.85\\
 &      0.948  &       & 0.501              \\
200&               &    1.6387&        &    0.8559& 0.80\\
 &      0.894  &       & 0.437              \\
250&            &   1.5493&        &    0.8122& 0.75\\
 &      0.871  &       & 0.397              \\
300&               &    1.4623&        &    0.7725& 0.71\\
 &      1.704  &       & 0.667              \\
400&               &    1.2919&        &    0.7058& 0.60\\
 &      1.668  &       & 0.527              \\
500&               &    1.1252&        &    0.6531& 0.48\\
 &      1.588  &       & 0.468              \\
600&               &    0.9664&        &    0.6063& 0.37\\
 &      1.432  &       & 0.447              \\
700&               &    0.8232&        &    0.5617& 0.27\\
 &      1.242  &       & 0.437              \\
800&               &    0.6990&        &    0.5180& 0.19\\
 &      1.045  &       & 0.432              \\
900&               &    0.5945&        &    0.4748& 0.12\\
 &      0.808  &       & 0.431              \\
1000&             & 0.5137&        &    0.4317& 0.08\\
 &      0.619  &       & 0.431              \\
1100&             & 0.4518&        &    0.3886& 0.06\\
 &      0.546  &       & 0.431              \\
1200&             & 0.3972&        &    0.3454& 0.05\\
 &      0.517  &       & 0.433              \\
1300&             & 0.3454&        &    0.3022& 0.04\\
 &      0.504  &       & 0.434              \\
1400&             & 0.2950&        &    0.2588& 0.04\\
 &      0.497  &       & 0.433              \\
1500&             & 0.2453&        &    0.2155& 0.03\\
 &      1.232  &       & 1.080              \\
1750&             & 0.1221&        &    0.1075& 0.01\\
 &      1.221  &       & 1.075              \\
2000&             & 0.0000&        &    0.0000& 0.00\\
 &      2.389  &       & 2.106              \\
2500&             & -0.2389&       & -0.2106& -0.03\\
 &      2.297  &       & 2.016              \\
3000&             & -0.4686&       & -0.4123& -0.06\\
 &      2.170  &       & 1.961              \\
3500&             & -0.6856&       & -0.6083& -0.08\\
 &      2.097  &       & 2.004              \\
4000&             & -0.8952&       & -0.8087& -0.09\\
\rule[-1ex]{0mm}{1ex}&  \\
\hline
\rule[-1ex]{0mm}{1ex}&  \\
% \multicolumn{6}{@{}l@{}}{$\ast$ Geopotential anomaly integrated from 2000 dbar level.\rule{0mm}{2.5ex}}\\
% \multicolumn{6}{@{}l@{}}{\ \ \ Velocity \rule{0mm}{2.5ex}is calculated from (10.17)}\\
\end{tabular}
\end{small}

$\ast$ Geopotential anomaly integrated from 2000 dbar level.
Velocity is calculated from (10.17)
\end{center}
\end{table}

\begin{figure}[t!]
\makebox[120mm][c]{\includegraphics{pics/profileandsection}}
\caption{\textbf{Слева:} Сдвиговые течения как функция глубины,
рассчитанные по гидрографическим данным, собранным в ходе рейсов
\emph{Эндевора} к югу от Кайп Код в августе 1982 года. Гольфстрим~---
это быстрое течение с глубиной менее 1000 децибар. Глубина нулевой
поверхности принята равной 2000 децибар. Справа: Сечение потенциальной
плотности sq через Гольфстрим вдоль меридиана \latlon{63.66}{W}
рассчитанное по термоэлектрическим данным \emph{Эндевора} 25--28
апреля 1986 года. Гольфстрим сосредоточен на сильно наклоненных
контурах, расположенных выше глубины 1000 м между~$\degrees{40}$
и~$\degrees{41}$. Вертикальный масштаб растянут в 425~раз по сравнению
с горизонтальным. (Данные отображены Линн Талли, Океанографический
Институт Скрипса).}
\label{profileandsection}
\end{figure}
%
% \begin{figure}[t!]
% %\centering
% \makebox[120mm][c]{\includegraphics{profileandsection}}
% \footnotesize
% Figure 10.8 \textbf{Left} Relative current as a function of
% depth\rule{0mm}{4ex} calculated from hydrographic\index{hydrographic
% data!from Endeavor} data collected by the \textit{Endeavor} cruise
% south of Cape Cod in August 1982. The Gulf Stream\index{Gulf
% Stream!cross section of} is the fast current shallower than 1000
% decibars. The assumed depth of no motion is at 2000 decibars.
% \textbf{Right} Cross section of potential density $\sigma_{\theta}$
% across the Gulf Stream along 63.66\degrees W calculated from
% \textsc{ctd}\index{CTD} data collected from \textit{Endeavor} on
% 25--28 April 1986. The Gulf Stream is centered on the steeply sloping
% contours shallower than 1000m between 40\degrees\ and 41\degrees.
% Notice that the vertical scale is 425 times the horizontal
% scale. (Data contoured by Lynn Talley, Scripps Institution of
% Oceanography).
% \label{profileandsection}
% \vspace{-3ex}
% \end{figure}
\end{section}

\begin{section}{Пояснения к вопросу о геострофических течениях}
% \section{Comments on Geostrophic Currents}
Теперь, после того как мы познакомились с тем, как рассчитываются
геострофические течения по гидрографическим данным, рассмотрим
некоторые ограничения теоретических основ этих расчетов и
соответствующих алгоритмов.
%
% Now that we know how to calculate geostrophic
% currents\index{geostrophic currents!comments on} from hydrographic
% data\index{hydrographic data!and geostrophic currents}, let's consider
% some of the limitations of the theory and techniques.

\begin{paragraph}{Преобразование сдвиговой скорости в величину скорости. }
% \paragraph{Converting Relative Velocity to Velocity}
Из гидрографических данных получают геострофические течения
относительно геострофического потока на определенном отсчетном
уровне. Как отсюда получить скорость геострофического течения
относительно земной системы отсчета?
%
% \index{geostrophic currents!relative to the earth}Hydrographic data
% give geo\-stro\-phic currents relative to geostrophic
% currents\index{geostrophic currents!relative} at some reference
% level. How can we convert the relative geostrophic velocities to
% velocities relative to the earth?

\begin{enumerate}
\item
\emph{Выбор поверхности нулевой скорости}. Традиционно, океанографы
предполагают существование \emph{неподвижной} поверхности, иногда называемой
поверхностью отсчета,, приблизительно на глубине~$2000\m$. Это
предположение используется для получения скоростей течений в 
таблице~10.4. На этой глубине скорости полагаются нулевыми, и течения
относительно этой поверхности получаются интегрированием вверх до
поверхности и вниз~--- до дна, что дает зависимость скорости от
глубины.

Существуют экспериментальные данные, подтверждающие существование
такой нулевой поверхности для осредненных течений (см., например,
Defant, 1961: 492).
%
% \vitem \textit{Assume a Level of no Motion}: Traditionally,
% oceanographers assume there is a \textit{\textit{level of no motion}},
% \index{reference surface|textbf}sometimes called a
% \textit{\textit{reference surface}}, roughly 2,000 m below the
% surface. This is the assumption used to derive the currents in table
% 10.4. Currents are assumed to be zero at this depth, and relative
% currents are integrated up to the surface and down to the bottom to
% obtain current velocity as a function of depth. There is some
% experimental evidence that such a level exists on average for mean
% currents (see for example, Defant, 1961: 492).

Дефант рекомендует выбирать поверхность отсчета там, где сдвиг течения
в вертикальном направлении принимает наименьшее значение. Обычно, это
соответствует глубине около 2 км. На основе этого подхода составлены
удобные карты поверхностных течений, поскольку эти течения, в среднем,
имеют большие скорости, чем глубинные. На Рис. 10.9 представлены
геопотенциальная аномалия и поверхностные течения в Тихом океане,
полученные относительно уровня 1000 децибар по давлению.
%
% Defant recommends choosing a reference level where the current shear
% in the vertical is smallest. This is usually near 2 km. This leads to
% useful maps of surface currents because surface currents tend to be
% faster than deeper currents. Figure 10.9 shows the geopotential
% anomaly and surface currents in the Pacific relative to the 1,000 dbar
% pressure level.

\begin{figure}[t!]
\makebox[120mm][c]{\includegraphics{pics/wyrtkiplot}}
\caption{Средняя геопотенциальная аномалия относительно уровня 1000
децибар в Тихом океане, по данным 36356 наблюдений. Высоты аномалий
приведены в геопотенциальных сантиметрах. Если бы скорость циркуляции
на уровне 1000 децибар была равна нулю, то этот рисунок представлял бы
собой топографическую поверхность Тихого океана. По данным Виртки
(1979).}
\label{fig:wyrtkiplot}
\end{figure}
%
% \begin{figure}[t!]
% \makebox[120mm][c]{\includegraphics{wyrtkiplot}}
% \footnotesize
% Figure 10.9. Mean \rule{0mm}{4ex}geopotential anomaly relative to the
% 1,000 dbar surface in the Pacific based on 36,356 observations. Height
% of the anomaly is in geopotential centimeters. If the velocity at
% 1,000 dbar were zero, the map would be the surface topography of the
% Pacific. After Wyrtki (1979).
% \label{fig:wyrtkiplot}
% \vspace{-4ex}
% \end{figure}

\item
\emph{Использование известных течений.} Параметры таких течений можно
получить с помощью измерителей скоростей течений или по данным
спутниковой альтиметрии.

При этом могут возникнуть проблемы, связанные с тем, что
гидрографические данные, используемые для расчета течений, получены в
разное время. Например, массив гидрографических данных может
формироваться за время от месяцев до десятилетий, тогда как параметры
течений могут измеряться в течении всего нескольких месяцев. Поэтому,
гидрография может быть несовместимой с данными по измерениям
течений. Иногда, течения и гидрография измеряются практически
синхронно, как показано на Рис. 10.10. В этом примере течения
непрерывно измерялись заякоренным измерителем (точки) в глубинном
западном погранслое и рассчитывались по данным термоэлектрического
измерителя,, полученным сразу после установки измерителей и
непосредственно перед их поднятием (сглаженные кривые). Сплошная линия
отображает течение в предположении существования нулевой поверхности
на глубине 2000 м, штриховая~--- течение по данным измерителя,
сглаженным по различным интервалам до и после сеансов
термоэлектрических измерений.
%
% \vitem \textit{Use known currents:} The known currents could be
% measured by current meters or by satellite altimetry. Problems arise
% if the currents are not measured at the same time as the hydrographic
% data\index{hydrographic data!and geostrophic currents}. For example,
% the hydrographic data may have been collected over a period of months
% to decades, while the currents may have been measured over a period of
% only a few months.  Hence, the hydrography may not be consistent with
% the current measurements.  Sometimes currents and hydrographic data
% are measured at nearly the same time (figure 10.10). In this example,
% currents were measured continuously by moored current meters (points)
% in a deep western boundary current and calculated from
% \textsc{ctd}\index{CTD} data taken just after the current meters were
% deployed and just before they were recovered (smooth curves). The
% solid line is the current assuming a level of no motion at 2,000 m,
% the dotted line is the current adjusted using the current meter
% observations smoothed for various intervals before or after the
% \textsc{ctd} casts.

\begin{figure}[t!]
\makebox[120mm][c]{\includegraphics{pics/whitplot}}
\caption{Термоэлектрические измерения течений позволяют оценивать их
изменения с глубиной, не прибегая к предположению о поверхности
нулевой скорости. Сплошная линия: профиль скорости в предположении
наличия нулевой поверхности на глубине 2000 децибар. Пунктирная линия:
профиль согласованный с измерениями течений, сделанными за 1--7
суток до термоэлектрических гидрографических измерений. (Графики
предоставлены Томом Витворфом, A\&M Университет, Техас).}
\label{fig:whitplot}
\end{figure}
%
% \begin{figure}[t!]
% \makebox[120mm][c]{\includegraphics{whitplot}}
% \footnotesize
% Figure 10.10 Current \rule{0mm}{3ex }meter measurements can be used
% with \textsc{ctd}\index{CTD} measurements to determine current as a
% function of depth avoiding the need for assuming a depth of no
% motion. Solid line: profile assuming a depth of no motion at 2000
% decibars. Dashed line: profile adjusted to agree with currents
% measured by current meters 1--7 days before the \textsc{ctd}
% measurements.  (Plots from Tom Whitworth, Texas A\&M University)
% \label{fig:whitplot}
% \vspace{-3ex}
% \end{figure}

\item
\emph{Используем законы сохранения.} Данные цепочек гидрографических
станций пересекающих пролив или покрывающих открытый бассейн, могут
использоваться, вместе с уравнениями сохранения массы и солей, для
расчета течений. Это пример обратной задачи (Вюнш, 1996, описывает
применения методов обратных задач в океанографии). В работе Мерьсера
etal. (2003) подробно описывается методика расчета поверхностной
циркуляции в восточном бассейне южной Атлантики с использованием
гидрографических данных Глобального эксперимента по океанической
циркуляции и прямых измерений течений в рамках модели конечных
элементов теории обратных задач.
%
% \vitem \textit{Use Conservation Equations}: Lines of hydrographic
% stations\index{hydrographic stations} across a strait or an ocean
% basin may be used with conservation of mass and salt to calculate
% currents. This is an example of an inverse problem (Wunsch, 1996
% describes the application of inverse methods in oceanography). See
% Mercier et al. (2003) for a description of how they determined the
% circulation in the upper layers of the eastern basins of the south
% Atlantic using hydrographic data from the World Ocean Circulation
% Experiment and direct measurements of current in a box model\index{box
% model} constrained by inverse theory.
\end{enumerate}
\end{paragraph}

\begin{paragraph}{Недостатки методики расчета течений по гидрографическим данным.}
Карты океанских течений, рассчитанных по данным гидрографии,
используются с начала XX века. Тем не менее, важно проанализировать
ограничения данной методики.
%
% \paragraph{Disadvantage of Calculating Currents from Hydrographic Data}
% \index{hydrographic data!disadvantage of}Currents calculated from
% hydrographic data\index{hydrographic data!and geostrophic currents}
% have been used to make maps of ocean currents since the early 20th
% century. Nevertheless, it is important to review the limitations of
% the technique.

\begin{enumerate}
\item
Гидрографические данные позволяют рассчитать течения только по
отношению к течениям на уровне, принятом за базовый.
%
% \vitem Hydrographic data\index{hydrographic data!and geostrophic
% currents} can be used to calculate only the current relative to a
% current at another level.

\item 
Геострофическое приближение не применимо к полосе $\degrees{2}$ около экватора,
где параметр Кориолиса стремится к нулю, т.к.~$\sin \theta \to 0$.

\item
Геострофическое приближение не учитывает действие сил трения.

% \vitem The assumption of a level of no motion may be suitable in the
% deep ocean, but it is usually not a useful assumption when the water
% is shallow such as over the continental shelf.

% \vitem Geostrophic currents cannot be calculated from hydrographic
% stations\index{hydrographic data!and geostrophic currents} that are
% close together. Stations must be tens of kilometers apart.
\end{enumerate}
\end{paragraph}


% \begin{paragraph}{Limitations of the Geostrophic Equations}
% % \paragraph{Limitations of the Geostrophic Equations}
% \index{geostrophic equations!limitations of}I began this
% section\index{geostrophic balance!limitations of} by showing that the
% geostrophic balance applies with good
% accuracy\index{accuracy!equations!geostrophic} to flows that exceed a
% few tens of kilometers in extent and with periods greater than a few
% days. The balance cannot, however, be perfect.  If it were, the flow
% in the ocean would never change because the balance ignores any
% acceleration of the flow. The important limitations of the geostrophic
% assumption\index{geostrophic approximation} are:
% 
% % \index{geostrophic equations!limitations of}I began this
% % section\index{geostrophic balance!limitations of} by showing that the
% % geostrophic balance applies with good
% % accuracy\index{accuracy!equations!geostrophic} to flows that exceed a
% % few tens of kilometers in extent and with periods greater than a few
% % days. The balance cannot, however, be perfect.  If it were, the flow
% % in the ocean would never change because the balance ignores any
% % acceleration of the flow. The important limitations of the geostrophic
% % assumption\index{geostrophic approximation} are:
% \begin{enumerate}
% \item Geostrophic currents\index{geostrophic currents!cannot change}
% cannot evolve with time because the balance ignores acceleration of
% the flow. Acceleration dominates if the horizontal dimensions are less
% than roughly 50 km and times are less than a few days. Acceleration is
% negligible, but not zero, over longer times and distances.
% %
% % \vitem Geostrophic currents\index{geostrophic currents!cannot change}
% % cannot evolve with time because the balance ignores acceleration of
% % the flow. Acceleration dominates if the horizontal dimensions are less
% % than roughly 50 km and times are less than a few days. Acceleration is
% % negligible, but not zero, over longer times and distances.
%
% \item The geostrophic balance\index{geostrophic balance!not near
% equator} does not apply within about $\degrees{2}$\ of the equator where
% the Coriolis force goes to zero because $\sin \varphi \rightarrow 0$.
% %
% % \vitem The geostrophic balance\index{geostrophic balance!not near
% % equator} does not apply within about 2\degrees\ of the equator where
% % the Coriolis force goes to zero because $\sin \varphi \rightarrow 0$.
%
% \item The geostrophic balance\index{geostrophic balance!ignores
% friction} ignores the influence of friction.
% %
% % \vitem The geostrophic balance\index{geostrophic balance!ignores
% % friction} ignores the influence of friction.
% \end{enumerate}

\begin{paragraph}{Точность.}
Струб et. al. (1997) показал, что скорости течений, рассчитанных по
данным о наклонах морской поверхности, полученных из спутниковой
альтиметрии, имеет точность~$\acc{3}{5}{\cmps}$. Ушида, Имаваки и Ху
(1998) сравнили скорости течений в системе Куросио, измеренные по
данным дрейфующих буёв со скоростями, рассчитанными по спутниковым
данным в предположении геострофического баланса.

Используя данные о наклонах морской поверхности вдоль трассы в $12.5\km$, 
они оценили разность между двумя измерениями величиной~$\pm 16\cm$,
для скоростей течений до 150 см/сек, т.е. около $10\%$. Дхонс, Ваттс и
Россби (1989) проводили измерения скорости течения Гольфстрим к северу
от мыса Гаттерас и сравнивали свои результаты со скоростями,
рассчитанными по гидрографическим данным в предположении
геострофического баланса. Они установили, что скорости в центральных
областях течения на глубинах до $500\m$ были на 10~--25 см/сек
больше, чем скорости полученные из геострофических уравнений, с
использованием измерений на глубине $2000\m$. Максимальная скорость в
центре течения превышала 150 см/сек , поэтому погрешность составляла
около~$10\%$. После добавления поправки за кривизну траектории
Гольфстрима, вносящую дополнительный член в выражение для ускорения в
геострофических уравнениях, разница между рассчитанной и наблюденной
скоростями упала до величины не превосходящей~$5$--$10\cmps$ 
($\approx 5\%$).
%
% \paragraph{Accuracy} 
% Strub et al. (1997) showed that currents\index{geostrophic
% currents!measured by altimetry} calculated from satellite altimeter
% measurements of sea-surface slope have an
% accuracy\index{accuracy!altimeter} of $\pm$3--5 cm/s. Uchida, Imawaki,
% and Hu (1998) compared currents measured by drifters\index{drifters!in
% Kuroshio} in the Kuroshio\index{Kuroshio!geostrophic balance in} with
% currents calculated from satellite altimeter data assuming geostrophic
% balance. Using slopes over distances of 12.5 km, they found the
% difference between the two measurements was $\pm$16 cm/s for currents
% up to 150 cm/s, or about 10\%. Johns, Watts, and Rossby (1989)
% measured the velocity of the Gulf Stream\index{Gulf Stream!northeast
% of Cape Hatteras} northeast of Cape Hatteras and compared the
% measurements with velocity calculated from hydrographic
% data\index{hydrographic data!and geostrophic currents} assuming
% geostrophic balance. They found that the measured velocity in the core
% of the stream, at depths less than 500 m, was 10--25 cm/s faster than
% the velocity calculated from the geostrophic equations using measured
% velocities at a depth of 2000 m\index{geostrophic currents!and level
% of no motion}\index{geostrophic currents!in Gulf Stream}. The maximum
% velocity in the core was greater than 150 cm/s, so the error was
% $\approx 10$\%. When they added the influence of the curvature of the
% Gulf Stream, which adds an acceleration term to the geostrophic
% equations, the difference in the calculated and observed velocity
% dropped to less than 5--10 cm/s ($\approx 5$\%).
\end{paragraph}
\end{section}

\begin{section}{Течения по данным гидрографических разрезов}
% \section{Currents From Hydrographic Sections}
Гидрографические данные, собираемые на судах вдоль их курса, часто
используются для построения вертикального профиля плотности по
разрезам, проходящим через линию курса. Поперечные сечения линий тока
в течениях зачастую демонстрируют резкое погружение изопикнических
поверхностей с сильным контрастом плотности по обе стороны области
течения. Параметры бароклинных течений вдоль сечений могут быть
оценены с использованием методики, впервые предложенной Маргулесом
(1906) и описанной Дефантом (1961; 453). Эта методика позволяет
океанографам оценить скорость и направление течений пересекающих
плоскость сечения сравнительно простой процедурой.
%
% \index{hydrographic sections}Lines of hydrographic data along ship
% tracks are often used to produce contour plots of density in a
% vertical section along the track. Cross-sections of currents sometimes
% show sharply dipping density surfaces with a large contrast in density
% on either side of the current. The baroclinic currents in the section
% can be estimated using a technique first proposed by Margules (1906)
% and described by Defant (1961: 453). The technique allows
% oceanographers to estimate the speed and direction of currents
% perpendicular to the section by a quick look at the section.

Для вывода уравнения Маргулиса рассмотрим наклон 
$\partial z/\partial x$ стационарной границы между двумя водными
массами с плотностями~$\rho_1$ и~$\rho_2$ (см. Рис. 10.11). При
расчете изменения скорости поперек этой поверхности, предположим
однородность слоёв, соответствующих этим массам и примем, для
определенности, что $\rho_1 < \rho_2$, причем оба слоя положим
находящихся в геострофическом равновесии. Несмотря на все эти, на
первый взгляд, жесткие предположения (в реальном океане нет идеальных
поверхностей, нет однородных слоев с резкими границами и т.д.)
предлагаемая методика до сих пор оказывается практически полезной.
%
% To derive Margules' equation, consider the slope $\partial z/\partial
% x$ of a stationary interface between two water masses with densities
% $\rho_1$ and $\rho_2$ (see figure 10.11). To calculate the change in
% velocity across the interface we assume homogeneous layers of density
% $\rho_1 < \rho_2$ both of which are in geostrophic
% equilibrium\index{geostrophic currents!from slope of density
% surfaces}. Although the ocean does not have an idealized interface
% that we assumed, and the water masses do not have uniform density, and
% the interface between the water masses is not sharp, the concept is
% still useful in practice.

\begin{figure}[t!]
\makebox[120mm][c]{\includegraphics{pics/Fig10-10}}
\caption{Наклоны~$\beta$ морской поверхности и наклон~$\gamma$ ?
границы между двумя однородными, движущимися слоями с плотностями~$\rho_1$ 
и~$\rho_2$ в северном полушарии. Согласно Ньюману и Пирсону (1966; 166).}
\label{fig:Fig10-10}
\vspace{-3ex}
\end{figure}
%
% \begin{figure}[t!]
% %\vspace{-3ex}
% %\centering
% \makebox[120mm][c]{\includegraphics{Fig10-10}}
% \footnotesize
% Figure 10.11 Slopes $\beta$ of the \rule{0mm}{4ex}sea surface and the
% slope $\gamma$ of the interface between two homogeneous, moving
% layers, with density $\rho_1$ and $\rho_2$ in the northern
% hemisphere. After Neumann and Pierson (1966: 166)
%
% \label{fig:Fig10-10}
% \vspace{-3ex}
% \end{figure}

Изменение давления на границе слоев составляет: 
\begin{equation}
\delta p = \frac{\partial p}{\partial x}\,\delta x 
           + \frac{\partial p}{\partial z}\, \delta z ,
\end{equation}
а вертикальный и горизонтальный градиенты давления получаются 
из (10.6):
\begin{equation}
\frac{\partial p}{\partial z}= - \rho_1 g + \rho_1 f v_1
\end{equation}
Таким образом,
\begin{subequations}
\begin{align}
\delta p_1&=-\rho_1fv_1 \, \delta x + \rho_1 g \, \delta z \\
\delta p_2&=-\rho_2fv_2 \, \delta x + \rho_2 g \, \delta z \\ \notag
\end{align}
\end {subequations}
На неподвижной границе раздела должны выполняться 
условия~$\delta p_1 = \delta p_2$. Приравнивая (10.20а) к (10.20b), 
деля обе части на $\delta x$ и разрешая
относительно $\delta z/\delta x$, получаем:
\begin{displaymath}
\frac{\delta z}{\delta x}\equiv \tan \gamma 
  =\frac{f}{g}\left(\frac{\rho_2\,v_2 - \rho_1\,v_1}{\rho_2 -\rho_1}\right)
\end{displaymath}
Учитывая, что $\rho_1 \approx \rho_2$, для малых~$\beta$ и~$\gamma$ имеем 
\begin{subequations}
\begin{align}
 \tan \gamma &\approx \frac{f}{g}\left(\frac{\rho_1}{\rho_2 - \rho_1}\right)(v_2-v_1) \\
\tan \beta_1&=-\frac{f}{g}\, v_1 \\
\tan \beta_2&=-\frac{f}{g}\, v_2
\end{align}
\end {subequations}
где $\beta$~--- это наклон морской поверхности, а $\gamma$~--- наклон
границы раздела между двумя водными массами. Поскольку перепады
плотности внутри водной массы малы, этот наклон приблизительно в 1000
раз больше, чем наклон поверхностей постоянного давления.
%
% The change in pressure on the interface is:
% \begin{equation}
% \delta p = \frac{\partial p}{\partial x}\,\delta x + \frac{\partial p}{\partial
% z}\, \delta z ,
% \end{equation}
% and the vertical and horizontal pressure gradients are obtained 
% from (10.6):
% \begin{equation}
% \frac{\partial p}{\partial z}= - \rho_1 g + \rho_1 f v_1
% \end{equation}
% Therefore:
% \begin{subequations}
% \begin{align}
% \delta p_1&=-\rho_1fv_1 \, \delta x + \rho_1 g \, \delta z \\
% \delta p_2&=-\rho_2fv_2 \, \delta x + \rho_2 g \, \delta z \\ \notag
% \end{align}
% \end {subequations}
% The boundary conditions require $\delta p_1 = \delta p_2$ on the
% interface if the interface is not moving. Equating (10.20a) with (10.20b), 
% dividing by $\delta x$, and solving for $\delta z/\delta x$
% gives:
% \begin{displaymath}
% \frac{\delta z}{\delta x}\equiv \tan \gamma =\frac{f}{g}\left(\frac{\rho_2\,v_2
% - \rho_1\,v_1}{\rho_2 -\rho_1}\right)
% \end{displaymath}
% Because $\rho_1 \approx \rho_2$, and for small $\beta$ and $\gamma$,
% \begin{subequations}
% \begin{align}
% \tan \gamma &\approx \frac{f}{g}\left(\frac{\rho_1}{\rho_2 - \rho_1}\right)(v_2-v_1) \\
% \tan \beta_1&=-\frac{f}{g}\, v_1 \\
% \tan \beta_2&=-\frac{f}{g}\, v_2
% \end{align}
% \end {subequations}
% where $\beta$ is the slope of the sea surface, and $\gamma$ is the
% slope of the interface between the two water masses. Because the
% internal differences in density are small, the slope is approximately
% 1000 times larger than the slope of the constant pressure surfaces.

Рассмотрим применение этой методики к Гольфстриму (см. Рис. 10.8). Из
Рис. 10.8 имеем $\varphi = \degrees{36}$, $\rho_1 = 1026.7\kgpcm$, 
$\rho_2 = 1027.5\kgpcm$ на
глубине 500 децибар. Если использовать поверхность $\sigma_t = 27.1$ для
оценки наклона между двумя водными массами, то граница изменяется от
глубины 350 м до 650 м на протяжении 70 км, что 
дает $\tan \gamma = 4300 \times 10^{-6} = 0.0043$ 
и~$\Delta v = v_2 - v_1 = -0.38\mps$. Полагая $v_2 = 0$,
получаем $v_1 = 0.38\mps$. Эта грубая оценка скорости Гольфстрима
неплохо согласуется со скоростью на глубине 500 м, рассчитанной по
гидрографическим данным (Таблица 10.4) в предположении неподвижного
уровня, соответствующего глубине 2000 децибар.
%
% Consider the application of the technique to the Gulf
% Stream\index{Gulf Stream} (figure 10.8). From the figure: $\varphi =
% 36$\degrees, $\rho_1 = 1026.7$ kg/m$^3$, and $\rho_2 = 1027.5$
% kg/m$^3$ at a depth of 500 decibars. If we use the $\sigma_t = 27.1$
% surface to estimate the slope between the two water masses, we see
% that the surface changes from a depth of 350 m to a depth of 650 m
% over a distance of 70 km. Therefore, $\tan \gamma = 4300 \times
% 10^{-6} = 0.0043$, and $\Delta v = v_2 - v_1 = -0.38$ m/s. Assuming
% $v_2 = 0$, then $v_1 = 0.38$ m/s.  This rough estimate of the velocity
% of the Gulf Stream\index{Gulf Stream!velocity of} compares well with
% velocity at a depth of 500m calculated from hydrographic
% data\index{hydrographic data!across Gulf Stream} (table 10.4) assuming
% a level of no motion at 2,000 decibars.

Наклоны поверхностей постоянной плотности хорошо видны на Рис. 10.8. 
Графики этих поверхностей позволяют быстро оценить направления
течений и их приблизительные скорости. В то же время, при
использовании данных Таблицы 10.4, получим для наклона морской
поверхности величину $8.4 \times 10^{-6}$ или 0.84 м на 100 км.
%
% The slope of the constant-density surfaces are clearly seen in figure
% 10.8. And plots of constant-density surfaces can be used to quickly
% estimate current directions and a rough value for the speed. In
% contrast, the slope of the sea surface is $8.4 \times 10^{-6}$ or 0.84
% m in 100 km if we use data from table 10.4.

Заметим, что в системе Гольфстрима поверхности постоянной плотности
наклонены вниз в восточном направлении, тогда как наклоны топографии
морской поверхности направлены к востоку вверх, т.е. поверхности
постоянной плотности и постоянного давления имеют здесь
противоположные знаки.
%
% Note that constant-density surfaces in the Gulf Stream\index{Gulf
% Stream!density surfaces} slope downward to the east, and that
% sea-surface topography slopes upward to the east. Constant pressure
% and constant density surfaces have opposite slope.

В том случае, когда резкая граница между двумя водными массами
достигает поверхности, образуется океанический фронт, по своим
свойствам аналогичный атмосферным фронтам.
%
% If the sharp interface between two water masses reaches the surface,
% it is an oceanic front, which has properties that are very similar to
% atmospheric fronts.

Вихри в районе Гольфстрима имеют как теплые, так и холодные внутренние
области (Рис. 10.12). Применение метода Маргулиса к этим мезомаштабным
вихрям позволяет определить направление потока. Антициклонические
вихри (вращающиеся по часовой стрелке в северном полушарии) имеют
теплые центральные области ($\rho_1$ залегает глубже в центре) и поверхности
постоянного давления изгибаются вверх, а уровень морской поверхности
выше в центре вихря. Для циклонических вихрей картина обратная.
%
% Eddies in the vicinity of the Gulf Stream\index{Gulf Stream!eddies}
% can have warm or cold cores (figure 10.12). Application of Margules'
% method to these mesoscale eddies\index{mesoscale eddies} gives the
% direction of the flow. Anticyclonic eddies (clockwise rotation in the
% northern hemisphere) have warm cores ($\rho_1$ is deeper in the center
% of the eddy than elsewhere) and the constant-pressure surfaces bow
% upward. In particular, the sea surface is higher at the center of the
% ring. Cyclonic eddies are the reverse\index{geostrophic currents!from
% hydrographic data|)}.

\begin{figure}[h!]
\makebox[120mm][c]{\includegraphics{pics/rings}}
\caption{Формы поверхностей постоянного давления~$p_i$ и границы между
двумя водными массами $\rho_1$ и~$\rho_2$ в случае, когда верхний слой
вращается быстрее нижнего. \textbf{Слева:} антициклонический вихрь с теплым
ядром. \textbf{Справа:} циклонический, холодный вихрь. Морская поверхность~$p_0$
выпукла вверх в теплом вихре, а поверхности постоянной плотности~---
вниз. Кружок с точкой обозначает течение, направленное на читателя,
кружок с крестиком~--- от читателя. Согласно данным Дефанта (1961; 466).}
\label{fig:rings}
\end{figure}
%
% \begin{figure}[h!]
% \vspace{-1ex}
% \makebox[120mm][c]{\includegraphics{rings}}
% \footnotesize
% Figure 10.12 Shape of constant-pressure \rule{0mm}{4ex}surfaces $p_i$
% and the interface between two water masses of density $\rho_1, \rho_2$
% if the upper is rotating faster than the lower.  \textbf{Left:}
% Anticyclonic motion, warm-core eddy. \textbf{Right:} Cyclonic,
% cold-core eddy. Note that the sea surface $p_0$ slopes up toward the
% center of the warm-core ring, and the constant-density surfaces slope
% down toward the center. Circle with dot is current toward the reader,
% circle with cross is current away from the reader. After Defant (1961:
% 466).
% \label{fig:rings}
% \vspace{-4ex}
% \end{figure}
\end{section}

\begin{section}{Измерения параметров течений лагранжевым методом}
% \section{Lagrangian Measurements of Currents}
В океанологии и гидродинамике различают лагранжевый и эйлеровский
подходы. В первом случае, отслеживается траектория жидкой частицы, а
во втором~--- измеряется скорость потока в данной фиксированной
точке.
%
% \index{Lagrangian measurements}Oceanography and fluid mechanics
% distinguish between two techniques for measuring currents: Lagrangian
% and Eulerian. Lagrangian techniques follow a water particle. Eulerian
% techniques measure the velocity of water at a fixed position.

\begin{paragraph}{Основы метода}
% \paragraph{Basic Technique}
Метод Лагранжа основан на отслеживании положения поплавка (дрифтера)
связанного с определенным водным объёмом на поверхности или на
некоторой глубине. Средняя скорость за определенный период
определяется как отношение расстояния между двумя положениями дрифтера
в начальный и конечный моменты периода к его величине. Возникающие при
этом ошибки имеют следующую природу:
%
% Lagrangian techniques track the position of a drifter designed to
% follow a water parcel either on the surface or deeper within the water
% column. The mean velocity over some period is calculated from the
% distance between positions at the beginning and end of the period
% divided by the period. Errors are due to:
\begin{enumerate}
\item
Дрифтер не связан жестко с локальным объёмом воды, т.к. поверхностный
ветер постоянно сносит его относительно водной массы.
%
% \vitem The failure of the drifter to follow a parcel of water. We
% assume the drifter stays in a parcel of water, but wind blowing on the
% surface float of a surface drifter can cause the drifter to move
% relative to the water.

\item
Имеют место ошибки в определении положения дрифтера.
%
% \vitem Errors in determining the position of the drifter.

\item
Селекция в наблюдаемых данных, связанная с тем, что дрифтеры сносит в
области конвергенции, а зоны дивергенции, при этом, остаются не
покрытыми наблюдениями.
%
% \vitem Sampling errors. Drifters go only where
% drifters\index{drifters!accuracy of current measurements} want to
% go. And drifters want to go to convergent zones. Hence drifters tend
% to avoid areas of divergent flow.
\end{enumerate}

\begin{figure}[t!]
\makebox[120mm][c]{\includegraphics{pics/argos}}
\caption{Система <<Аргос>> использует сигналы, передаваемые с буёв для
определения их положения. Спутник принимает сигнал от буя В и измеряет
скорость изменения частоты сигнала~--- доплеровского сдвига~--- как
функцию от положения буя и расстояния от него до спутника. При этом,
буй ВВ даст такой же доплеровский сдвиг, как и буй В. Значения
доплеровского сдвига передаются наземной станции Е, которая направляет
информацию в центр обработки А через пункт управления К. Дитрих
et. al. (1980; 149).}
\label{fig:argos}
\end{figure}
%
% \begin{figure}[t!]
% %\vspace{-2ex}
% \makebox[120mm][c]{\includegraphics{argos}}
% \footnotesize
% Figure 10.13 System\rule{0mm}{4ex} Argos\index{Argos system} uses
% radio signals transmitted from surface buoys to determine the position
% of the buoy. A satellite receives the signal from the buoy B. The time
% rate of change of the signal, the Doppler shift $F$, is a function of
% buoy position and distance from the satellite's track. Note that a
% buoy at BB would produce the same Doppler shift as the buoy at B. The
% recorded Doppler signal is transmitted to ground stations E, which
% relays the information to processing centers A via control stations
% K. After Dietrich et al. (1980: 149).
% \label{fig:argos}
% \vspace{-4ex}
% \end{figure}
\end{paragraph}

\begin{paragraph}{Спутниковый мониторинг поплавковых измерителей течений}
% \paragraph{Satellite Tracked Surface Drifters}
Поплавковый измеритель поверхностных течений состоит из поплавка и
плавучего якоря. Его текущее положение определяется метеорологическими
спутниками с помощью системы <<Аргос>> (Свенсон и Шау, 1990) или
рассчитывается по GPS данным, непрерывно передаваемым буйковым
передатчиком и привязанным к береговой линии.
%
% \index{Lagrangian measurements!satellite tracked surface
% drifters}\index{satellite tracked surface
% drifters}\index{drifters}Surface drifters consist of a drogue plus a
% float. Its position is determined by the Argos\index{Argos system}
% system on meteorological satellites (Swenson and Shaw, 1990) or
% calculated from \textsc{gps} data recorded continuously by the buoy
% and relayed to shore.

Буи, входящие в систему <<Аргос>>, снабжены передатчиками, работающими
на одной строго фиксированной и стабилизированной
частоте~$F_0$. Спутниковая аппаратура принимает сигнал от буя и
определяет доплеровский частотный сдвиг как функцию времени~$t$
(Рис. 10.13). Частота, принимаемая спутником, дается выражением:
\begin{displaymath}
F=\frac{dR}{dt}\,\frac{F_0}{c} + F_0
\end{displaymath}
где $R$~---- расстояние между спутником и буём, $c$~--- скорость
света. Чем ближе буй к спутнику, тем быстрее меняется частота.
При~$F=F_0$ достигается стационарность изменения расстояния. В этот
момент $R$~достигает минимума, и скорость спутника перпендикулярна
линии, соединяющей спутник и буй. Время наибольшего сближения и
скорость изменения доплеровской частоты в этот момент позволяют
определить положение буя относительно орбиты спутника с точностью
до~$\degrees{180}$ (линии B и BB на Рис. 10.13). Этой неопределенности
удается избежать благодаря точному знанию орбиты спутника и
многократным наблюдениям одного и того же буя.
%
% Argos-tracked buoys carry a radio transmitter with a very stable
% frequency $F_0$. A receiver on the satellite receives the signal and
% determines the Doppler shift $F$ as a function of time $t$ (figure
% 10.13). The Doppler frequency is
% \begin{displaymath}
% F=\frac{dR}{dt}\,\frac{F_0}{c} + F_0
% \end{displaymath}
% where $R$ is the distance to the buoy, $c$ is the velocity of
% light. The closer the buoy to the satellite the more rapidly the
% frequency changes. When $F = F_0$ the range is a minimum. This is the
% time of closest approach, and the satellite's velocity vector is
% perpendicular to the line from the satellite to the buoy. The time of
% closest approach and the time rate of change of Doppler frequency at
% that time gives the buoy's position relative to the orbit with a
% 180\degrees\ ambiguity (B and BB in the figure). Because the orbit is
% accurately known, and because the buoy can be observed many times, its
% position can be determined without ambiguity.

Точность, определяемого таким образом положения буя, зависит от
стабильности передаваемой им частоты. Система <<Аргос>> обеспечивает
точность положения~$\acc{1}{2}{\km}$ при 1--8 наблюдениях за сутки в
зависимости от широты. Поскольку $1\cmps\approx 1\kmpdy$ и поскольку
характерные скорости океанских течений составляют $100$--$200\cmps$,
то такая точность представляется вполне приемлемой.
%
% The accuracy\index{accuracy!Argos} of the calculated position depends
% on the stability of the frequency transmitted by the buoy. The Argos
% system\index{Argos system} tracks buoys with an
% accuracy\index{drifters!accuracy of current measurements} of
% $\pm$(1--2) km, collecting 1--8 positions per day depending on
% latitude. Because 1 cm/s $\approx$ 1 km/day, and because typical
% values of currents in the ocean range from one to two hundred
% centimeters per second, this is an very useful accuracy.
\end{paragraph}

\begin{paragraph}{«Дырчатый» («дырявый носок») поплавковый измеритель течений}
% \paragraph{Holey-Sock Drifters}
Наиболее широкое распространение получили т.н. <<дырчатые>>
поплавковые измерители, движение которых отслеживается
спутниками. Типичная конструкция такого измерителя представляет собой
цилиндрический плавучий якорь из искусственной кожи диаметром 1 м и
длиной 15 м с 14-ю jбольшими отверстиями по краям. Вес якоря
компенсируется поплавком , расположенным в 3 мерах ниже
поверхности. Этот затопленный поплавок связан с полупогруженной
павучей платформой, на которой установлен передатчик системы
<<Аргос>>. Этот буй был разработан для программы исследований
поверхностных течений и прошел многократные испытания. Ниилер и
др. (1995) тщательно промерили скорости сноса буя поверхностными
ветрами и получили величину направленную~$\degrees{12 \pm 9}$ вправо
от направления ветра составляющую
\begin{equation}
U_s = \left( 4.32\pm 0.67 \times\right) 10^{-2} \frac{U_{10}}{DAR} 
      + \left( 11.04\pm 1.63 \right) \frac{D}{DAR}
\end{equation}
где DAR~--- коэффициент драгирования, определяемый как отношение
площади якоря поперечной потоку к сумме площадей соединения и плавучей
платформы, D~--- разность скоростей потока между верхней и нижней
гранями якоря. Типичное значение DAR для морских дрифтеров составляет
величину 40, и дрейф $U_s < 1\cmps$ для~$U_{10} < 10\mps$.
%
% \index{Lagrangian measurements!holey-sock
% drifters}\index{drifters!holey-sock}The most widely used,
% satellite-tracked drifter is the holey-sock drifter. It consists of a
% cylindrical drogue of cloth 1 m in diameter by 15 m long with 14 large
% holes cut in the sides. The weight of the drogue is supported by a
% float set 3 m below the surface. The submerged float is tethered to a
% partially submerged surface float carrying the Argos\index{Argos
% system} transmitter.
%
% The buoy was designed for the Surface Velocity Program and extensively
% tested. Niiler et al. (1995) carefully measured the rate at which wind
% blowing on the surface float pulls the drogue through the water, and
% they found that the buoy moves $12\pm9$\degrees\ to the right of the
% wind at a speed
% \begin{equation}
% U_s = \left( 4.32\pm 0.67 \times\right) 10^{-2} \frac{U_{10}}{DAR} +
% \left( 11.04\pm 1.63 \right) \frac{D}{DAR}
% \end{equation}
% where $DAR$ is the drag area ratio defined as the drogue's drag area
% divided by the sum of the tether's drag area and the surface float's
% drag area, and $D$ is the difference in velocity of the water between
% the top of the cylindrical drogue and the bottom. Drifters typically
% have a $DAR$ of 40, and the drift $U_s < 1$ cm/s for $U_{10} < 10$
% m/s.
\end{paragraph}

\begin{paragraph}{Плавучие измерители <<Арго>>}
% \paragraph{Argo Floats}
Среди подповерхностных (подводных) измерителей наиболее широкое
распространение получили измерители <<Арго>> (Рис. 10.14). Их
конструкция позволяет им дрейфовать между поверхностью и заданной
глубиной. Большинство измерителей дрейфуют в течение 10-и дней на
глубине 1 км, погружаясь до 2 км и затем поднимаясь на
поверхность. При подъеме они измеряют профиль температуры и солености
в функции давления (глубины). Измерители остаются на поверхности в
течение нескольких дней, передавая данные на береговые станции по
системе <<Аргос>>, а затем опять погружаются на глубину до 1 км. Каждый
измеритель снабжен источником питания, позволяющему ему
функционировать в таком циклическом режиме в течение нескольких
лет. Таким образом, измерители этого класса позволяют получать данные
о скоростях течений на глубине 1 км и распределении плотности в
верхнем слое океана. Три тысячи измерителей <<Арго>> были размещены во
всех частях Мирового океана в ходе Глобального эксперимента по
усвоению данных (GODAE).
%
% The most widely used subsurface floats are the Argo
% floats.\index{floats!Argo} The floats (figure 10.14) are designed to
% cycle between the surface and some predetermined depth. Most floats
% drift for 10 days at a depth of 1 km, sink to 2 km, then rise to the
% surface. While rising, they profile temperature and salinity as a
% function of pressure (depth). The floats remains on the surface for a
% few hours, relays data to shore via the Argos system\index{Argos
% system}, then sink again to 1 km. Each float carries enough power to
% repeat this cycle for several years. The float thus measures currents
% at 1 km depth and density distribution in the upper ocean. Three
% thousand Argo floats are being deployed in all parts of the ocean for
% the Global Ocean Data Assimilation Experiment
% \textsc{godae}\index{Global Ocean Data Assimilation
% Experiment!floats}.\index{floats}

\begin{figure}[h!]
\makebox[120mm][c]{\includegraphics{pics/alace}}
\caption{Автономный лагранжевый измеритель циркуляции.(ALACE) является
прототипом поплавкового измерителя течений системы <<Аргос>> и
предназначен для измерения течений на глубине 1 км. \textbf{Слева:} Схема
измерителя. Для всплытия, гидравлическая помпа перекачивает масло из
внутреннего резервуара во внешнюю камеру, уменьшая общую плотность
измерителя. Для погружения, защелкивающийся клапан открывается, и
масло перетекает обратно во внутренний резервуар. Антенна смонтирована
в верхней части измерителя. \textbf{Справа:} Подробная схема гидравлической
системы. Мотор вращает наклонную пластину, приводящую в движение
поршень, перекачивающий гидравлическое масло. По данным Девиса и
др. (1992).}
\label{fig:alace}
\end{figure}
%
% \begin{figure}[h!]
% \vspace{-2ex}
% \makebox[120mm][c]{\includegraphics{alace}}
% \footnotesize
% Figure 10.14 The Autonomous \rule{0mm}{4ex}Lagrangian Circulation
% Explorer (ALACE) floats\index{floats!ALACE} is the prototype for the
% Argos floats. It measures currents at a depth of 1 km. \textbf{Left:}
% Schematic of the drifter. To ascend, the hydraulic pump moves oil from
% an internal reservoir to an external bladder, reducing the drifter's
% density. To descend, the latching valve is opened to allow oil to flow
% back into the internal reservoir.  The antenna is mounted to the end
% cap. \textbf{Right:} Expanded schematic of the hydraulic system. The
% motor rotates the wobble plate actuating the piston which pumps
% hydraulic oil. After Davis et al. (1992).
% \label{fig:alace}
% \vspace{-2ex}
% \end{figure}
\end{paragraph}

\begin{paragraph}{Измерение течений с помощью трассеров.}
% \paragraph{Lagrangian Measurements Using Tracers}
Наиболее распространенным методом измерения течений в глубине океана
является отслеживание определенных объемов воды, содержащих
компоненты, не встречающиеся в естественных условиях. В ходе ядерных
испытаний в 50-е годы и благодаря недавнему экспоненциальному росту
фреонов в атмосфере, подобные трассеры в большом количестве были
выброшены в океан. В параграфе 13.4 приводится список трассеров,
используемых в океанографии. Распределение молекул трассеров
используется для оценок параметров движения водных масс. Эта методика
оказывается наиболее эффективной для оценки глубинных течений при
усреднении за несколько декад и измерениях турбулентного
перемешивания, обсуждаемого в \S8.4.
%
% \index{Lagrangian measurements!tracers}\index{tracers}The most common
% method for measuring the flow in the deep ocean is to track parcels of
% water containing molecules not normally found in the ocean.  Thanks to
% atomic bomb tests in the 1950s and the recent exponential increase of
% chlorofluorocarbons in the atmosphere, such tracers have been
% introduced into the ocean in large quantities. See \S 13.4 for a list
% of tracers used in oceanography. The distribution of trace molecules
% is used to infer the movement of the water. The technique is
% especially useful for calculating velocity of deep water masses
% averaged over decades and for measuring turbulent mixing discussed in
% \S 8.4.

\begin{figure}[t!]
\makebox[120mm][c]{\includegraphics{pics/tritium}}
\caption{Распределения трития вдоль сечения западного бассейна
северной Атлантики, измеренное в 1972 (Верх), и в 1982 (Низ). Данные
приведены в т.н. тритиевых единицах, определяемых как 
$10^{18}\times \text{(количество атомов трития)}/
\text{(количество атомов водорода)}$. Данные
скорректированы на уровень активности, который имел бы место на 1
января 1981ё года. Сравните эти данные профилем плотности на
Рис. 13.10. По данным Тоггвейлера (1994).}
\label{fig:tritium}
\end{figure}
%
% \begin{figure}[t!]
% \makebox[120mm][c]{\includegraphics{tritium}}
% \footnotesize
% Figure 10.15 Distribution of \rule{0pt}{3ex} tritium along a section
% through the western basins in the north Atlantic, measured in 1972
% (\textbf{Top}) and remeasured in 1981 (\textbf{Bottom}). Units are
% tritium units, where one tritium unit is $10^{18}$ (tritium
% atoms)/(hydrogen atoms) corrected to the activity levels that would
% have been observed on 1 January 1981. Compare this figure to the
% density in the ocean shown in figure 13.10. After Toggweiler (1994).
% \label{fig:tritium}
% \vspace{-5ex}
% \end{figure}

Распределение трассирующих молекул рассчитывается из данных об их
концентрации в пробах, собранных вдоль гидрологических разрезов и на
гидрологических станциях. Сбор и обработка проб являются
дорогостоящими и длительными процедурами, поэтому существует очень
немного повторных данных по одним и тем же сечениям. На Рис. 10.15
показаны две карты распределения трития в северной Атлантике
построенные по данным, собранным в 1972--1973 годах в рамках программы
Георазрезов и в 1981 году. На разрезе видно, что тритий, поступивший в
атмосферу во время ядерных испытаний в период с 50-х годов до 1972
года, проник до глубин ниже 4 км только к северу от~\latlon{40}{N}
к 1971 году и до \latlon{35}{N} к 1981 году. Это свидетельствует о
малости скоростей глубинных течений, порядка $1.6\mmps$ в данном
примере.
%
% The distribution of trace molecules is calculated from the
% concentration of the molecules in water samples collected on
% hydrographic sections\index{hydrographic sections} and
% surveys. Because the collection of data is expensive and slow, there
% are few repeated sections. Figure 10.15 shows two maps of the
% distribution of tritium in the north Atlantic collected in 1972--1973
% by the Geosecs Program and in 1981, a decade later. The sections show
% that tritium, introduced into the atmosphere during the atomic bomb
% tests in the atmosphere in the 1950s to 1972, penetrated to depths
% below 4 km only north of 40\degrees N by 1971 and to 35\degrees N by
% 1981. This shows that deep currents are very slow, about 1.6 mm/s in
% this example.

Ввиду малости скоростей глубинных течений, возникает вопрос о
механизме формирования наблюдаемого распределения трассеров. Как
турбулентная диффузия, так и адвекция, связанная с течениями, могут
объяснить наблюдаемую картину, поэтому правомерен вопрос, что
демонстрирует Рис. 10.15~--- среднюю глубинную циркуляцию в Атлантике
или распределение трития турбулентной диффузией?
%
% Because the deep currents are so small, we can question what process
% are responsible for the observed distribution of tracers. Both
% turbulent diffusion and advection by currents can fit the
% observations. Hence, does figure 10.15 give mean currents in the deep
% Atlantic, or the turbulent diffusion of tritium?

\begin{figure}[t!]
\makebox[120mm][c]{\includegraphics{pics/Fig10-16-bw}}
\caption{Температура и течения в океане по данным AVHRR
анализа. Поверхностные течения оценивались по перемещениям
температурных и осадочных деталей при сравнении двух
изображений. Применялся специальный пространственный фильтр для
усиления резкости границ водных масс. Теплые воды отмечены темным
оттенком. С разрешения «Ошен Имаджинг», Солана Бич, Калифорния.}
\label{Fig10.16.bw}
\end{figure}
%
% \begin{figure}[t!]
% \makebox[120mm][c]{\includegraphics{Fig10-16-bw}}
% \footnotesize
% Figure 10.16 Ocean \rule{0mm}{4ex}temperature and current patterns are
% combined in this \textsc{avhrr} \index{Advanced Very High Resolution
% Radiometer (AVHRR)}analysis. Surface currents were computed by
% tracking the displacement of small thermal or sediment features
% between a pair of images. A directional edge-enhancement filter was
% applied here to define better the different water masses. Warm water
% is shaded darker. From Ocean Imaging, Solana Beach, California, with
% permission.
% \label{Fig10.16.bw}
% \vspace{-4ex}
% \end{figure}

Другими информативными трассерами являются температура и соленость
воды. Эти наблюдения будут рассмотрены в \S13.4 , где описываются
основные методы исследования глубинной циркуляции. Здесь отметим, что
данные о поверхностной температуре океана, полученные в системе AVHRR,
являются дополнительным источником информации о течениях.
%
% Another useful tracer is the temperature and salinity of the water. I
% will consider these observations in \S 13.4 where I describe the core
% method for studying deep circulation. Here, I note that \textsc{avhrr}
% \index{Advanced Very High Resolution Radiometer (AVHRR)}observations
% of surface temperature of the ocean are an additional source of
% information about currents.

Последовательные инфракрасные изображения океанской поверхности
используются для расчета смещений температурных деталей
(Рис. 10.16). Методика особенно эффективна для исследований
изменчивости течений вблизи берегов, где по береговым ориентирам можно
точно определить смещения температурных аномалий. В некоторые сезоны,
таким образом были обнаружены большие температурные контрасты в ряде
регионов Мирового океана.
%
% Sequential infrared images of surface temperature are used to
% calculate the displacement of features in the images (figure
% 10.16). The technique is especially useful for surveying the
% variability of currents near shore. Land provides reference points
% from which displacement can be calculated accurately, and large
% temperature contrasts can be found in many regions in some seasons.

Однако у этой методики есть два существенных ограничения.
%
% There are two important limitations.
\begin{enumerate}
\item
Многие районы Мирового океана часто закрыты сплошной облачностью, что
не позволяет проводить наблюдения поверхности.
%
% \vitem Many regions have extensive cloud cover, and the ocean cannot
% be seen.

\item
Как правило, потоки параллельны температурным фронтам и сильные
течения могут существовать вдоль фронтов, даже если последние не
перемещаются. Поэтому существенным является отслеживание
мелкомасштабных вихрей в потоке вблизи фронта, а не положение самого
фронта.
%
% \vitem Flow is primarily parallel to temperature fronts, and strong
% currents can exist along fronts even though the front may not move. It
% is therefore essential to track the motion of small eddies embedded in
% the flow near the front and not the position of the front.
\end{enumerate}
\end{paragraph}

\begin{figure}[b!]
\makebox[120mm][c]{\includegraphics{pics/duckies}}
\caption{Траектории, по которым двигались бы резиновые утята, если бы
они были выброшены в море 10 января, но в различные годы. Пять
траекторий были выбраны из 48 модельных расчетов, охватывающих период
с 1946 и 1993 гг. Траектории начинаются 10 января и прослеживаются в
течение двух лет (черные квадраты). Серые квадраты указывают положение
игрушек на 16 ноября года смыва за борт. Серый кружок указывает место,
где игрушки были впервые выброшены на берег около Ситки в 1992
году. Табличка в левом нижнем углу показывает периоды соответствующие
приведенным траекториям. По данным Эббесмейра и Инграхама (1994).}
\label{fig:duckies}
\end{figure}
%
% \begin{figure}[b!]
% \vspace{-3ex}
% \makebox[120mm][c]{\includegraphics{duckies}}
% \footnotesize
% Figure 10.17 Trajectories that \rule{0mm}{5ex}spilled rubber duckies
% would have followed had they been spilled on January 10 of different
% years. Five trajectories were selected from a set of 48 simulations of
% the spill each year between 1946 and 1993. The trajectories begin on
% January 10 and end two years later (solid symbols). Grey symbols
% indicate positions on November 16 of the year of the spill. The grey
% circle gives the location where rubber ducks first came ashore near
% Sitka in 1992. The code at lower left gives the dates of the
% trajectories. After Ebbesmeyer and Ingraham (1994).
% \label{fig:duckies}
% %\vspace{-2ex}
% \end{figure}

\begin{paragraph}{Игрушки помогают океанографам.}
% \paragraph{The Rubber Duckie Spill}
10 января 1992 года 12.2 метровый контейнер с 29000-ми игрушек для
ванной, в том числе с резиновыми утятами, смыло за борт контейнеровоза
в точке с координатами \latlon{44.7}{N} и~\latlon{178.1}{E}
(Рис. 10.17).  Десять месяцев спустя, игрушки стало выбрасывать на
берег около Ситки на Аляске. При аналогичном происшествии 27-го мая
1990 года 80000 пар обуви было смыто за борт контейнеровоза <<Ганза
Кариер>> в точке \latlon{48}{N} и~\latlon{161}{W}.
%
% \index{Rubber Duckie Spill}On January 10, 1992 a 12.2-m container with
% 29,000 bathtub toys, including rubber ducks (called rubber duckies by
% children) washed overboard from a container ship at 44.7\degrees N,
% 178.1\degrees E (figure 10.17). Ten months later the toys began
% washing ashore near Sitka, Alaska. A similar accident on May 27, 1990
% released 80,000 Nike-brand shoes at 48\degrees N, 161\degrees W when
% waves washed containers from the \textit{Hansa Carrier}.

Эти события, а также находки, время от времени, игрушек и обуви,
оказались в хорошем соответствии с численными моделями расчета
траекторий утечек нефти, проделанными Эббесмейром и Инграхамом (1992,
1994). Они рассчитали возможные траектории выпавших за борт игрушек,
используя численную модель поверхностной циркуляции океана (OSCURS),
как ветровые течения, рассчитанные по ежедневным данным об атмосферном
давлении на уровне морской поверхности, предоставленных флотским
центром океанографических данных. После коррекции расчетов с учетом
увеличения парусности игрушек на 50\% и уменьшения угла отклонения
на~$\degrees{5}$, они точно предсказали появление выброшенных игрушек
около Ситки 16 ноября 1992 года, десять месяцев после их смыва за
борт.
%
% The spills and eventual recovery of the toys and shoes proved to be
% good tests of a numerical model for calculating the trajectories of
% oil spills developed by Ebbesmeyer and Ingraham (1992, 1994). They
% calculated the possible trajectories of the spilled toys using the
% Ocean Surface Current Simulations \textsc{oscurs} numerical model
% driven by winds calculated from the Fleet Numerical Oceanography
% Center's daily sea-level pressure data. After modifying their
% calculations by increasing the windage coefficient by 50\% for the
% toys and by decreasing their angle of deflection function by 5\degrees
% , their calculations accurately predicted the arrival of the
% toys\index{drifters!rubber duckie} near Sitka, Alaska on November 16,
% 1992, ten months after the spill.
\end{paragraph}
\end{section}


\begin{figure}[b!]
\makebox[120mm][c]{\includegraphics{pics/moorings}}
\caption{\textbf{Слева:} Пример размещения измерителя на поверхности
моря, установленного группой буйковых измерений Океанографического
Института Вудс Холл. \textbf{Справа:} подводныйизмеритель,
размещенныйэтойжегруппой. Бейкер (1981: 410--411).}
\label{fig:moorings}
\end{figure}
%
% \begin{figure}[b!]
% \vspace{-3ex}
% \makebox[120mm][c]{\includegraphics{moorings}}
% \footnotesize
% Figure 10.18 \textbf{Left:} An example \rule{0mm}{3ex}of a surface
% mooring of the type deployed by the Woods Hole Oceanographic
% Institution's Buoy Group.  \textbf{Right:} An example of a subsurface
% mooring deployed by the same group.  After Baker (1981: 410--411).
% \label{fig:moorings}
% %\vspace{-3ex}
% \end{figure}

\begin{section}{Измерители течений, основанные на эйлеровском подходе 
к гидродинамике.}
% \section{Eulerian Measurements}
Существует много различных типов измерителей, использующий
гидродинамический подход Эйлера и работающих как на судах, так и на
якорных станциях.
%
% \index{Eulerian measurements}Eulerian measurements are made by many
% different types of instruments on ships and moorings.

Якорные станции устанавливаются с судов на время от нескольких месяцев
до года и более. Установка и последующий демонтаж оборудования делают
эту методику дорогостоящей, поэтому в настоящее время развернуто всего
несколько измерителей подобного типа.

Подводные измерители, подобные тому, который показан на Рис. 10.18
справа, имеют ряд преимуществ по сравнению со снабженными
поверхностными поплавками, а именно: у них отсутствует поверхностный
поплавок, положение которого постоянно подвергается воздействию
сильных изменчивых поверхностных течений, они не заметны и не
привлекают лишнего внимания, расположены на достаточной глубине, чтобы
не попасть в рыболовные сети. Результаты измерений заякоренных
датчиков подвержены ошибкам, основными источниками которых являются:
%
% Moorings (figure 10.18) are placed on the sea floor by ships. The
% moorings may last for months to longer than a year. Because the
% mooring must be deployed and recovered by deep-sea research ships, the
% technique is expensive and few moorings are now being deployed. The
% subsurface mooring shown on the right in the figure is preferred for
% several reasons: it does not have a surface float that is forced by
% high frequency, strong, surface currents; the mooring is out of sight
% and it does not attract the attention of fishermen; and the floatation
% is usually deep enough to avoid being caught by fishing
% nets. Measurements made from moorings have errors due to:
\begin{enumerate}
\item
Перемещения датчиков, которые для подводных измерителей заметно
меньше, чем у датчиков с поверхностным поплавком, в силу чего
последние используются редко.
%
% \vitem Mooring motion. Subsurface moorings move least. Surface
% moorings in strong currents move most, and are seldom used.

\item
Рабочий период заякоренных измерителей недостаточно длителен, чтобы
корректно оценить среднюю скоростьилимежгодовуюизменчивостьскорости.
%
% \vitem Inadequate Sampling. Moorings tend not to last long enough to
% give accurate estimates of mean velocity or interannual variability of
% the velocity.

\item
Датчики довольно быстро облепляются морскими организмами, особенно у
измерителей, расположенных вблизи от поверхности в течение нескольких
недель и более.
%
% \vitem Fouling of the sensors by marine organisms, especially
% instruments deployed for more than a few weeks close to the surface.
\end{enumerate}

\begin{paragraph}{Доплеровские акустические измерители профилей течений.}
% \paragraph{Acoustic-Doppler Current Meters and Profilers}
Наиболее распространенным типом измерителей скоростей течений и
профилей их параметров, работающих на эйлеровом принципе, являются
акустические измерители. Обычно, такие измерители излучают звук в виде
трех или четырех узких пучков в различных направлениях и принимают
отраженный планктоном и мелкими пузырьками воздуха сигнал, частота
которого сдвинута относительно частоты излучения на величину
пропорциональную радиальной скорости отражателя. Комбинируя данные по
трем или четырем пучкам, оценивают горизонтальную скорость течения, в
предположении малости скорости отражателя относительно морской воды.
%
% \index{Eulerian measurements!acoustic-doppler current
% profiler}\index{acoustic-doppler current profiler}The most common
% Eulerian measurements of currents are made using sound. Typically, the
% current meter or profiler transmits sound in three or four narrow
% beams pointed in different directions. Plankton and tiny bubbles
% reflect the sound back to the instrument. The Doppler shift of the
% reflected sound is proportional to the radial component of the
% velocity of whatever reflects the sound. By combining data from three
% or four beams, the horizontal velocity of the current is calculated
% assuming the bubbles and plankton do not move very fast relative to
% the water.

Обычно применяются два типа акустических измерителей. В системе
доплеровского акустического измерителя профиля течений (ADCP)
измеряется доплеровское смещение сигнала, отраженного от водных масс,
подобно тому, как это делается в радиолокации при исследовании
рассеяния радиоимпульсов в зависимости от расстояния от
измерителя. Данные, поступающие от нескольких излучателей, работающих
в режиме узконаправленных пучков, комбинируются для оценки
горизонтальной скорости течения как функции от расстояния до
излучателя. При измерениях с судов, пучки направляются по диагонали
вниз, под 3--4-мя углами относительно курса судна. При килевой
установке измерителя пучок направляется по диагонали вверх.
%
% Two types of acoustic current meters are widely used. The
% Acoustic-Doppler Current Profiler, called the \textsc{adcp}, measures
% the Doppler shift of sound reflected from water at various distances
% from the instrument using sound beams projected into the water just as
% a radar measures radio scatter as a function of range using radio
% beams projected into the air. Data from the beams are combined to give
% profiles of current velocity as a function of distance from the
% instrument. On ships, the beams are pointed diagonally downward at
% 3--4 horizontal angles relative to the ship's bow. Bottom-mounted
% meters use beams pointed diagonally upward.

Судовые измерители широко используются для построения профилей
скоростей течений в диапазоне 200--300 м под поверхностью моря при
переходах между гидрографическими станциями. Поскольку судно движется
относительно океанского дна, его скорость, как по величине, так и по
направлению, должна быть точно известна. Начиная с девяностых годов,
эта задача решается с помощью GPS навигации.
%
% Ship-board instruments are widely used to profile currents within 200
% to 300 m of the sea surface while the ship steams between hydrographic
% stations\index{hydrographic stations!and acoustic Doppler current
% profiler}. Because a ship moves relative to the bottom, the ship's
% velocity and orientation must be accurately known. \textsc{gps} data
% have provided this information since the early 1990s.

Доплеровские акустические измерители течений значительно проще ADCP
систем. Они излучают непрерывный сигнал и измеряют локальную скорость
вблизи самого измерителя, а не профиль скорости на различных
расстояниях. Они устанавливаются на заякоренных платформах и, иногда,
совмещаются с электронными измерителями температуры, плотности и
солености (CTD) и передают данные о скорости как функции времени в
течение многих дней и месяцев. На Рис. 10.19 представлен подобный
измеритель, разработанный <<Аандера Инструментс>>. Измерители CTD типа
используются на гидрографических станциях для профилирования скоростей
течений.
%
% Acoustic-Doppler current meters are much simpler than the
% \textsc{adcp}. They transmit continuous beams of sound to measure
% current velocity close to the meter, not as a function of distance
% from the meter. They are placed on moorings and sometimes on a
% \textsc{ctd}.  Instruments on moorings record velocity as a function
% of time for many days or months. The Aanderaa current meter (figure
% 10.19) in the figure is an example of this type. Instruments on
% \textsc{ctd}s\index{CTD} profile currents from the surface to the
% bottom at hydrographic stations.

\begin{figure}[t!]
\makebox[120mm][c]{\includegraphics{pics/RCM9}}
\caption{Пример якорного акустического измерителя течений RCM-9
сконструированного <<Аандера Инструментс>>. Две компоненты
горизонтальной скорости измеряются акустической системой, а
направление относительно севера~--- инерционным компасом, работающим
на эффекте Холла. Источник питания, электроника, система записи
информации смонтированы в прочном корпусе. Точность определения
скорости течения составляет $\pm 0.15\cmps$ по величине и
$\pm\degrees{5}$ по направлению.}
\label{fig:RCM9}
\end{figure}
%
% \begin{figure}[t!]
% %\vspace{-3ex}
% \makebox[120mm][c]{\includegraphics{RCM9}}
% \footnotesize
% Figure 10.19 An example of a \rule{0mm}{5ex}moored acoustic current
% meter, the \textsc{rcm 9} produced by Aanderaa Instruments. Two
% components of horizontal velocity are measured by an acoustic system,
% and the directions are referenced to north using an internal
% Hall-effect compass. The electronics, data recorder, and battery are
% in the pressure-resistant housing. Accuracy\index{accuracy!current
% meter} is $\pm$0.15 cm/s and $\pm$5\degrees. (Courtesy Aanderaa
% Instruments)
% \label{fig:RCM9}
% \vspace{-2ex}
% \end{figure}
\end{paragraph}
\end{section}

\begin{section}{Важные выводы.}
% \section{Important Concepts}
\begin{enumerate}
\item
Распределение давления в океане практически точно совпадает с
соответствующим гидростатическому равновесию, поэтому давление с
высокой точностью вычисляется по данным измерения температуры и
проводимости воды по уравнению состояния. Гидрографические данные
позволяют получить поле давления в океане с точностью до постоянной.
%
% \item
% Pressure distribution is almost precisely the hydrostatic pressure
% obtained by assuming the ocean is at rest. Pressure is therefore
% calculated very accurately from measurements of temperature and
% conductivity as a function of pressure using the equation of state of
% seawater. Hydrographic data\index{hydrographic data!and altimetry}
% give the relative, internal pressure field of the ocean.

\item
Течения в океане с большой точностью определяются геострофическим
балансом, когда сила Кориолиса уравновешена горизонтальным градиентом
давления, за исключением поверхностного и придонного пограничных
слоев.
%
% \vitem Flow in the ocean is in almost exact geostrophic
% balance\index{geostrophic balance} except for flow in the upper and
% lower boundary layers. Coriolis force almost exactly balances the
% horizontal pressure gradient.

\item
Спутниковые альтиметрические наблюдения позволяют построить топографию
поверхности океана и измерить поверхностные геострофические течения,
для чего необходимо знать конфигурацию геоида. Если форма геоида не
известна, то альтиметрические данные используются для исследования
изменчивости топографии и геострофических течений.
%
% \vitem Satellite altimetric observations of the oceanic topography
% give the surface geostrophic current\index{geostrophic
% currents!measured by altimetry}. The calculation of topography
% requires an accurate geoid\index{geoid}. If the geoid\index{geoid} is
% not known, altimeters can measure the change in topography as a
% function of time, which gives the change in surface geostrophic
% currents.

\item
TOPEX/POSEIDON и Jason являются, в настоящее время, самыми
совершенными спутниковыми альтиметрическими системами, измеряющими
топографию океанской поверхности и её изменчивость с точностью $\pm 4\cm$.
%
% \vitem Topex/Poseidon\index{Topex/Poseidon} and Jason\index{Jason} are
% the most accurate altimeter systems, and they can measure the
% topography\index{topography!measured by altimetry} or changes in
% topography with an accuracy\index{accuracy!topography} of $\pm$4 cm.

\item
Гидрографические данные используются для расчета скоростей
геострофических течений в глубине океана относительного известного
потока на некотором горизонте, в качестве которого могут быть выбраны
или поверхность океана, где течения измеряются по спутниковым данным,
или некоторый горизонт с нулевой скоростью на глубине $1$--$2\km$.
%
% \vitem Hydrographic data\index{hydrographic data!and geostrophic
% currents} are used to calculate the internal geostrophic
% currents\index{geostrophic currents!relative to level of no motion} in
% the ocean relative to known currents at some level. The level can be
% surface currents measured by altimetry or an assumed level of no
% motion at depths below 1--2 km.

\item
Часть общего потока не зависящая от глубины, называется баротропным
потоком, изменяющаяся с глубиной~--- бароклинным. Гидрографические
данные позволяют оценивать только бароклинную составляющую потока.
%
% \vitem Flow in the ocean that is independent of depth is called
% barotropic flow, flow that depends on depth is called baroclinic
% flow. Hydrographic data\index{hydrographic data!and geostrophic
% currents} give only the baroclinic flow.

\item
Геосторофический поток стационарен, поэтому реальные течения в океане
не являются сторого геострофическими. Геострофическое приближение не
применимо в близэкваториальной зоне, где параметр Кориолиса обращается
в нуль.
%
% \vitem Geostrophic flow cannot change with time, so the flow in the
% ocean is not exactly geostrophic. The geostrophic method does not
% apply to flows at the equator where the Coriolis force vanishes.

\item
Наклоны поверхностей постоянной плотности или температуры, измеряемые
по гидрографическим сечениям, могут использоваться для определения
потока нормального к сечению.
%
% \vitem Slopes of constant density or temperature surfaces seen in a
% cross-section of the ocean can be used to estimate the speed of flow
% through the section.

\item
Лагранжевая методика оценки течений отслеживает движение конкретного
объема воды путем наблюдений за поверхностными дрейфующими
измерителями, получением данных о положении заглубленных буёв или
оценкой траекторий химических трассеров, таких, например, как тритий.
%
% \vitem Lagrangian techniques measure the position of a parcel of water
% in the ocean. The position can be determined using surface drifters or
% subsurface floats\index{drifters}, or chemical tracers such as
% tritium.

\item
Измерители течений, использующие принцип Эйлера, дают скорость потока
в данной фиксированной точке. Измерения проводятся с заякоренных
платформ или с использованием акустических измерителей профилей
течений с кораблей или гидрографических датчиков.
%
% \vitem Eulerian techniques measure the velocity of flow past a point
% in the ocean.  The velocity of the flow can be measured using moored
% current meters or acoustic velocity profilers on ships,
% \textsc{ctd}s\index{CTD} or moorings.
\end{enumerate}
\end{section}

\end{chapter}