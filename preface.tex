% -*- coding: utf-8 -*-

\begin{chapter}{Предисловие редактора перевода}

\textit{\dots{}будет завершено после окончания работы над проектом.}

\bigskip 

Перевод главы~\ref{chap:1}~--- Николай Колдунов, 
главы~\ref{chap:2}~--- Николай Колдунов,
главы~\ref{chap:3}~--- Николай Колдунов, 
главы~\ref{chap:4}~--- Галина Суркова, Дмитрий Чечин, 
главы~\ref{chap:5}~--- редактор перевода, 
главы~\ref{chap:6}~--- Николай Колдунов, 
главы~\ref{chap:7}~--- Николай Колдунов,
главы~\ref{chap:8}~--- Николай Колдунов,
главы~\ref{chap:9}~--- Настя К., редактор перевода,
главы~\ref{chap:10}~--- Тронь Александр Анатольевич,
главы~\ref{chap:11}~--- редактор перевода,
главы~\ref{chap:12}~--- Павел Файман,
главы~\ref{chap:13}~--- редактор перевода,
главы~\ref{chap:14}~--- Таня Алексеева, редактор перевода,
главы~\ref{chap:15}~--- Анна Акимова,
главы~\ref{chap:16}~--- редактор перевода,
главы~\ref{chap:17}~--- Виктор Колдунов, редактор перевода.

Переводчики искренне благодарны всем, кто внес свои предложения по
улучшению качества перевода и сообщал о найденных в нем ошибках. Особо
следует отметить решающий вклад участника форума oceanographers.ru под
псевдонимом Jo.

Текущая версия перевода и его английский оригинал доступны по адресу:
\begin{center}
\href{https://introocean.github.io/}%
{\url{https://introocean.github.io/}}.
\end{center}

Сообщить о найденных ошибках, а также внести любые другие предложения
по улучшению качества учебника или принять участие в работе над ним
можно в сообществе GitHub, посвященном данному
переводу:
\begin{center}
\href{https://github.com/introocean/introocean-ru/}%
{\url{https://github.com/introocean/introocean-ru/}},
\end{center}
либо его оригиналу:
\begin{center}
\href{https://github.com/introocean/introocean-en/}%
{\url{https://github.com/introocean/introocean-en/}}.
\end{center}

На данном этапе работы над переводом особо важна вычитка глав 14--17,
которые пока еще не проверялись никем из специалистов. Также команда
переводчиков будет рада любым предложениям по включению в текст
сведений о научных результатах, полученных после выхода в свет
последней версии оригинала.
\end{chapter}

\begin{chapter}{Предисловие}
Эта книга предлагается вниманию студентов старших курсов и выпускников
таких специальностей, как метеорология, океанотехника или
океанология, учебные программы которых довольно разнообразны.  С
учётом данного обстоятельства, изложение базовых идей и концепций будет
преобладать над математическими формулами.
%%
%% This book is written for upper-division undergraduates and new
%% graduate students in meteorology, ocean engineering, and
%% oceanography. Because these students have a diverse background, I have
%% emphasized ideas and concepts more than mathematical derivations.


В отличие от большинства других учебников, этот распространяется
бесплатно в цифровом виде через Интернет. Причины для такого решения
следующие:
%%
%% Unlike most books, I am distributing this book for free in digital
%% format via the world-wide web. I am doing this for two reasons:
\begin{enumerate}

\item
Подготовка учебника к печати после завершения работы над ним обычно
занимает один или два года. За это время он как правило успевает устареть.
Рэндол Ларсон пишет в своей статье для \textit{Syllabus}: <<По моему мнению,
учебники для технических специальностей суть ни что иное, как разбазаривание
природных ресурсов. В момент своей публикации они уже устарели.
Такой короткий ``срок годности'' ведёт к тому, что студенты вообще не хотят
с ними связываться.>>~\cite{Larson:2002}.
Благодаря публикации в цифровой форме, появляется возможность ежегодного
обновления книги, которая тем самым сохранит свою актуальность.
%%
%% \vitem Textbooks are usually out of date by the time they are
%% published, usually a year or two after the author finishes writing the
%% book. Randol Larson, writing in \textit{Syllabus}, states: ``In my
%% opinion, technology textbooks are a waste of natural
%% resources. They're out of date the moment they are published. Because
%% of their short shelf life, students don't even want to hold on to
%% them''---(Larson, 2002). By publishing in electronic form, I can make
%% revisions every year, keeping the book current.

\item
Стоимость книг, опубликованных в развитых странах, оказывается
непосильной для многих учащихся из стран с худшим уровнем жизни.
Национальное управление по аэронавтике и исследованию космического
пространства США (NASA) дарит этот учебник студентам всего
мира.
%%
%% \vitem Many students, especially in less-developed countries cannot
%% afford the high cost of textbooks from the developed world. This then
%% is a gift from the US National Aeronautics and Space Administration
%% \textsc{nasa} to the students of the world.
\end{enumerate}

\section*{Благодарности}
%% \section*{Acknowledgements}
Автор учебника использовал его в своей преподавательской деятельности в
течение нескольких лет и хочет поблагодарить многих студентов, как
тех, кто посещал его лекции, так и читателей со всего мира, которые
сообщали о плохо написанных разделах, неоднозначностях, конфликтующих
обозначениях и других ошибках.
Также автор благодарен профессору Фреду Шлеммеру (Техасский
университет A\&M, Галвестон) за обширные комментарии к материалу книги
на основе её применения в собственной педагогической практике.
%%
%% I have taught from the book for several years, and I thank the many
%% students in my classes and throughout the world who have pointed out
%% poorly written sections, ambiguous text, conflicting notation, and
%% other errors.  I also thank Professor Fred Schlemmer at Texas A\&M
%% Galveston who, after using the book for his classes, has provided
%% extensive comments about the material.

Автор выражает свою благодарность многочисленным коллегам, которые
внесли свой вклад в виде иллюстраций, комментариев и другой полезной
информации. Особого упоминания заслуживают Aanderaa Instruments, Bill
Allison, Kevin Bartlett, James Berger, Gerben de Boer, Daniel
Bourgault, Don Chambers, Greg Crawford, Thierry De Mees, Richard
Eanes, Peter Etnoyer, Tal Ezer, Gregg Foti, Nevin S. Fu\v{c}kar, Luiz
Alexandre de Araujo Guerra, Hazel Jenkins, Andrew Kiss, Jody Klymak,
Judith Lean, Christian LeProvost, Brooks Martner, Nikolai Maximenko,
Kevin McKone, Mike McPhaden, Thierry De Mees, Pim van Meurs, Gary
Mitchum, Joe Murtagh, Peter Niiler, Nuno Nunes, Ismael
N\'{u}\~{n}ez-Riboni, Alex Orsi, Kym Perkin, Mark Powell, Richard Ray,
Joachim Ribbe, Will Sager, David Sandwell, Sea-Bird Electronics, Achim
Stoessel, David Stooksbury, Tom Whitworth, Carl Wunsch и многие
другие.
%%
%% I also wish to thank many colleagues for providing figures, comments,
%% and helpful information. I especially wish to thank Aanderaa
%% Instruments, Bill Allison, Kevin Bartlett, James Berger, Gerben de
%% Boer, Daniel Bourgault, Don Chambers, Greg Crawford, Thierry De Mees,
%% Richard Eanes, Peter Etnoyer, Tal Ezer, Gregg Foti, Nevin
%% S. Fu\v{c}kar, Luiz Alexandre de Araujo Guerra, Hazel Jenkins, Andrew
%% Kiss, Jody Klymak, Judith Lean, Christian LeProvost, Brooks Martner,
%% Nikolai Maximenko, Kevin McKone, Mike McPhaden, Thierry De Mees, Pim
%% van Meurs, Gary Mitchum, Joe Murtagh, Peter Niiler, Nuno Nunes, Ismael
%% N\'{u}\~{n}ez-Riboni, Alex Orsi, Kym Perkin, Mark Powell, Richard Ray,
%% Joachim Ribbe, Will Sager, David Sandwell, Sea-Bird Electronics, Achim
%% Stoessel, David Stooksbury, Tom Whitworth, Carl Wunsch and many
%% others.

Все ошибки и опечатки, допущенные в книге, автор относит исключительно
на свой счёт и будет благодарен за любые комментарии и предложения по
её улучшению.
%%
%% Of course, I accept responsibility for all mistakes in the
%% book. Please send me your comments and suggestions for improvement.

Использованные графические материалы были получены из многих
источников. В частности, автор хочет поблагодарить Link Ji
за большое количество предоставленных карт,
а также коллег из Центра космических исследований
Техасского университета в Остине. Дон Джонсон не только перерисовал
многие готовые иллюстрации, но и создал новые на основании
набросков. Трей Моррис разметил предметный указатель.
%%
%% Figures in the book came from many sources. I particularly wish to
%% thank Link Ji for many global maps, and colleagues at the University
%% of Texas Center for Space Research. Don Johnson redrew many figures
%% and turned sketches into figures. Trey Morris tagged the words used in
%% the index.

Особой благодарности заслуживает поддержка, оказанная в ходе работы над книгой
Лабораторией реактивного движения НАСА, проектами Topex/Poseidon и Jason
(контракты 960887 и~1205046).
%%
%% I especially thank \textsc{nasa}'s Jet Propulsion Laboratory and the
%% Topex/Poseidon and Jason Projects for their support of the book
%% through contracts 960887 and 1205046.

Фотография на обложке изображает остров-курорт Курумба (атолл
Северный Мале, Мальдивские о-ва). Автор снимка~--- Jagdish Agara,
правообладатель~--- Corbis. Дизайн обложки разработан Доном Джонсоном.
%%
%% Cover photograph of the resort island of Kurumba in North Male Atoll
%% in the Maldives was taken by Jagdish Agara (copyright Corbis). Cover
%% design is by Don Johnson.

Подготовка книги к печати была выполнена в системе \LaTeXe\ при помощи
TeXShop~2.14 на компьютере Intel iMac под управлением OS-X~10.4.11.
Автор особенно благодарен Gerben Wierda за очень полезный пакет i-Installer,
который сделал эту работу возможной, Ричарду Коху, Dirk Olmes
и многим другим за разработку пакета программ TeXShop,
а также Эндрю Киссу (Университет Нового Южного Уэльса, Канберра, Австралия)
за помощь с hyperref.
Работа с \LaTeXe\ оказалась удовольствием.
Для оформления рисунков использовался Adobe Illustrator.
%%
%% The book was produced in \LaTeXe\ using TeXShop 2.14 on an Intel iMac
%% computer running OS-X 10.4.11. I especially wish to thank Gerben
%% Wierda for his very useful i-Installer package that made it all
%% possible, Richard Koch, Dirk Olmes and many others for writing the
%% TeXShop software package, and Andrew Kiss at the University of New
%% South Wales in Canberra Australia for help in using the hyperref
%% package. The \LaTeXe\ software is a pleasure to use. All figures were
%% drawn in Adobe Illustrator.
\end{chapter}
