% -*- coding: utf-8 -*-

\begin{chapter}{Исследовательское путешествие}
Влияние океана на погоду и климат часто обсуждается в новостях. Кто не слышал 
об Эль-Ниньо, изменении погоды, Атлантическом сезоне ураганов и штормовых 
%% изменении погоды --- weather patterns
нагонах? Однако, какую именно роль в этих процессах играет океан, и почему 
нас заботит подобный вопрос?

\begin{section}{Зачем изучать физику океана?}
Ответ зависит от наших интересов, которые, в свою очередь, определяются тем,
как мы используем океан. Следующие три аспекта имеют особую важность:

\begin{itemize}
\item 
Океаны~--- источники пищи. Поэтому мы интересуемся процессами, происходящими 
в море так же, как фермеры~--- погодой и климатом. Океану не просто присущи 
некоторые погодные условия, такие как изменения температуры и течения; 
важность этих характеристик в том, что они определяют биологическую 
продуктивность моря. С другой стороны, атмосферные условия редко влияют 
на плодородие почвы, за исключением разве что небольшого количества азота, 
фиксируемого молниями.

\item
Океаны используются человеком. Мы строим различные сооружения на побережье 
или просто в море, добываем нефть и газ, транспортируем грузы, а также 
отдыхаем: купаемся, катаемся на лодках, рыбачим, занимаемся серфингом 
и подводным плаванием. Таким образом, нам интересны процессы, которые 
влияют на эту деятельность, особенно волны, ветры, течения и температура.

\item
Океаны влияют на погоду и климат: распределение дождей, засух, 
наводнения, региональный климат и развитие штормов, ураганов и тайфунов; 
они играют ведущую роль в процессах глобального потепления. 
%% фраза "It dominates global warming processes" отсутствует в PDF-версии
Следовательно, нам интересно взаимодействие океана с атмосферой, 
особенно потоки тепла и воды проходящие через поверхность моря, 
транспорт тепла океанами, а также их влияние на климат и синоптическую 
ситуацию.
\end{itemize}

Эти темы влияют на выбор объектов изучения. Объекты определяют, что мы меряем, 
как производим измерения и где. Некоторые процессы локальны, такие как 
разрушение волн на пляже, некоторые~--- региональны, такие как влияние севера 
Тихого океана на погоду Аляски, а некоторые~--- глобальны, такие как влияние 
океанов на изменение климата и глобальное потепление. 

Если эти причины для изучения океана действительно важны, давайте начнём 
наше исследовательское путешествие. Любому путешествию необходим пункт 
назначения. Какой же следует избрать нам?
\end{section}

\begin{section}{Цели}
В целом, я надеюсь, что студенты почерпнут из этого учебника
представление о главных концептуальных схемах (или теориях) физической 
океанографии, лежащих в её основе, о том, какой путь в процессе их построения 
довелось пройти науке, а также о причинах их широкого признания. Помимо этого 
мы познакомимся с методами, которыми океанографы извлекают закономерности из
океана случайностей и рассмотрим роль эксперимента в океанографии
(перефразируя Shamos, 1995:p.89).

В частности, я ожидаю, что читательская аудитория будет в итоге способна 
описать физические процессы, происходящие в океанах и прибрежных зонах, 
взаимодействие океана и атмосферы, распределение океанских ветров, течений, 
потоков тепла и водных масс. В тексте общим идеям уделено большее внимание, 
чем математическим выкладкам. Мы постараемся ответить на следующие вопросы:

\begin{enumerate}
\item
Какова основа нашего понимания физики океана?

\begin{itemize}
  \item
  Что такое физические свойства морской воды? 

  \item
  Каковы важные термодинамические и динамические процессы влияющие на океан? 

  \item
  Какие уравнения описывают эти процессы, и как они выведены? 

  \item
  Какие допущения мы использовали для их вывода? 

  \item
  Имеют ли эти уравнения полезные решения? 

  \item
  Насколько хорошо эти решения описывают процесс? То есть, каковы 
  экспериментальные основания теорий? 

  \item
  Какие процессы плохо понятны? Какие~--- хорошо?
\end{itemize}

\item
Каковы источники информации о физических переменных?
\begin{itemize}
  \item
  Какие инструменты используются для измерения каждой переменной? 

  \item
  Каковы их точность и пределы измерения? 

  \item
  Какие данные существуют за длительный период времени? 

  \item
  Какая платформа используется: cпутники, корабли, буи, буйковые станции?
\end{itemize}

\item
Какие процессы важны? Некоторые важные процессы, которые мы будем изучать, 
включают:
\begin{itemize}
  \item
  накопление и транспорт тепла в океанах;

  \item
  обмен теплом с атмосферой и роль океана в климате; 

  \item
  ветровое и температурное воздействие на поверхностный слой перемешивания; 

  \item
  ветровая циркуляция (включая экмановский перенос, экмановскую подкачку
%% "Экмановская циркуляция" и "Экмановский насос" в русском переводе "Динамики
%% атмосферы и океана" А. Гилла называется "экмановский перенос" и "экмановская 
%% подкачка" соотв.?
  глубинной циркуляции, а также апвеллинг). 

  \item
  динамика океанических течений (в частности, геострофические течения и роль 
  завихренности); 

  \item
  формирование водных масс;
%% следующие три пункта отличаются в html-версии оригинала и pdf.
  \item
  термохалинная циркуляция океана; 

  \item
  экваториальная динамика и Эль-Ниньо; 

  \item
  общая циркуляция океанов, а также Мексиканского залива; 

  \item
  математические модели циркуляции; 

  \item
  волны в океане (в том числе поверхностные волны, внутренние колебания, 
  приливы и цунами); 

  \item
  волны в мелкой воде, прибрежные процессы и предсказание приливов.
\end{itemize}

\item
Каковы основные течения и водные массы в океане, что определяет их 
распределение, и как океан взаимодействует с атмосферой?
\end{enumerate}
\end{section}

\begin{section}{Организация (структура)}
Перед тем, как начать путешествие, мы обычно стараемся узнать о тех местах, 
которые собираемся посетить, для чего изучаем карты и путеводители. 
В нашей книге путеводителями будут статьи и книги, написанные океанографами. 
Мы начнем с краткого обзора того, что известно об океанах. 
Затем продолжим описанием океанских бассейнов, чтобы понять, как форма 
морей влияет на физические процессы в воде. Далее мы рассмотрим внешние силы, 
ветер и тепло, действующие на океан, и его отклик на них. Во время изучения 
будут изложены необходимые теоретические сведения и представлены натурные 
данные.

К тому времени, когда мы достигнем главы 7, нам необходимо будет понять 
уравнения, описывающие динамическую реакцию океана. Так, мы рассмотрим 
уравнения движения, влияние вращения Земли и вязкости. Это приведёт к 
изучению ветровых океанических течений, геострофического приближения и важности 
постоянства вихря.

В дальнейшем мы обсудим некоторые частные примеры: глубинную циркуляцию, 
экваториальный океан и Эль-Ниньо, а также циркуляцию отдельных частей океана. 
Затем рассмотрим роль математического моделирования в описании океана. 
В конце мы изучим прибрежные процессы, волны, приливы, предсказание волн 
и приливов, цунами и штормовые нагоны.
\end{section}

\begin{section}{Общая картина}
Океан представляет собой одну из частей географической оболочки. 
Он оказывает влияние на атмосферные процессы путем переноса массы, момента и
энергии через водную поверхность. Речной сток, вместе с растворёнными в нем
минеральными веществами, тоже в конечном итоге оказывается в океане. 
Накопленные осадочные материалы со временем становятся скальной
породой на суше. Следовательно, понимание океана важно для получения картины 
всей Земли как системы в целом, так и вопросов глобальной смены климата и
%% important problems such as global change or global warming
%% из контекста не вполне ясно, о каком "изменении" идет речь
глобального потепления в частности. На более низком уровне, физическая 
океанография и метеорология сближаются. Океан обеспечивает обратную связь,
%% отсутствует логическое согласование этих двух предложений
замедляющую изменения в состоянии атмосферы.

Я надеюсь, вы обратите внимание на то, что при описании динамики океана мы 
будем использовать теорию, натурные (эмпирические) данные и численные модели. 
Их необходимо рассматривать вместе, по отдельности они не самодостаточны.

\begin{enumerate}
\item
Процессы в океане нелинейны и турбулентны, а теория нелинейных, турбулентных 
потоков в сложных бассейнах не очень хорошо развита. Теории, используемые 
для описания океана~--- сильно упрощённые приближения реальности. 

\item
Натурные измерения разбросаны в пространстве и во времени. Они обеспечивают 
грубое описание усреднённого по времени потока, но большинство процессов 
во многих регионах ещё мало исследованы. 

\item
Численные модели включают наиболее реалистичные теоретические идеи, они могут 
помочь интерполировать натурные исследования во времени и пространстве 
и используются для предсказания климатических изменений, течений и волн. 
Однако, численные равенства являются приближениями непрерывных аналитических 
уравнений, описывающих жидкий поток; они не содержат никакой информации о 
потоке между узловыми точками, в силу чего пока не могут использоваться 
для полного описания турбулентного потока, наблюдающегося в океане. 
\end{enumerate}

Соединяя теорию и натурные измерения в численных моделях, мы избегаем 
сложностей, связанных с их использованием по отдельности. Способы комбинирования
этих подходов непрерывно совершенствуются, что ведёт к гораздо 
более точному описанию океана. Конечная цель~--- узнать океан так хорошо, 
чтобы можно было предсказывать будущие перемены в окружающей среде, 
включая изменения климата или реакцию рыбных ресурсов на перелов.

Объединение теории, натурных исследований и компьютерных моделей относительно 
молодо. Четыре десятилетия экспоненциального роста вычислительной мощности
привели к появлению массово доступных настольных компьютеров, способных 
моделировать важные физические процессы и динамику океана.

\begin{quotation}
Все мы, люди науки, знаем, что компьютер стал важнейшим исследовательским 
инструментом~\dots{} научные расчёты достигли того уровня, на котором 
они становятся инструментом научных и инженерных изысканий наравне
с лабораторным экспериментом и математической теорией (langer(1999)).
\end{quotation}


%% http://oceanworld.tamu.edu/resources/ocng_textbook/chapter01/Images/Fig1-1s.jpg
%% Рисунок 1.1 Данные, численные модели и теория необходимы для понимания океана. В конечном счёте понимание системы океан-атмосфера-суша позволит предсказывать будущие состояния системы.

Объединение теории, натурных исследований и компьютерных моделей предполагает 
новый путь развития океанологии. В прошлом океанограф должен был бы сформулировать 
теорию, собрать данные для её проверки, а затем опубликовать результаты. 
Теперь задачи стали настолько специализированными, что мало кто может всё 
это проделать в одиночку. Немногие преуспели одновременно в построении теорий, 
сборе данных и разработке численных моделей. Вместо этого всё больше и больше 
работы делается командами учёных и инженеров.
\end{section}

\begin{section}{Дополнительная литература}
Если вы знаете об океанах и океанографии не слишком много, я предлагаю 
вам начать с книги Маклениша, особенно её четвёртой главы, посвящённой 
<<чтению океана>>. 
%% Маклениша --- MacLeish ???
%% MacLeish's book, especially his Chapter 4 on "Reading the ocean
По моему мнению, в ней даётся наиболее полное нетехническое описание того, 
как океанографы идут к пониманию океана.

Вы также можете извлечь немало полезных сведений из соответствующих глав любой 
океанографической книги начального уровня. Особый интерес представляют работы 
таких авторов, как Gross, Pinet, или Thurman. Три публикации Открытого 
Университета, включенные в список литературы, ориентированы на более 
подготовленного читателя. 

Gross, M. Grant and Elizabeth Gross (1996) Oceanography, 
A View of Earth 7th Edition. Upper Saddle River, New Jersey: Prentice Hall. 

Mac Leish William (1989) The Gulf Stream: Encounters With the Blue God. 
Boston: Houghton Mifflin Company. 

Pinet, Paul R. (2000) Invitation to oceanography. 2nd Edition. Sudbury, 
Massachusetts: Jones and Bartlett Publishers.

Open University (1989) Ocean Circulation. Oxford: Pergamon Press. 

Open University (1989) Seawater: Its Composition, Properties and Behaviour. 
Oxford: Pergamon Press. 

Open University (1989) Waves, Tides and Shallow-Water Processes. 
Oxford: Pergamon Press. 

Thurman, Harold V. and Elizabeth A. Burton (2001) Introductory oceanography. 
9th Edition. Upper Saddle River, New Jersey: Prentice Hall.
\end{section}

\end{chapter}
