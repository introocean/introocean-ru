% -*- coding: utf-8 -*-

\begin{chapter}{Исследовательское путешествие}
Влияние океана на погоду и климат часто обсуждается в новостях. Кто не слышал 
об Эль-Ниньо об изменении погоды, Атлантическом сезоне ураганов и штормовых 
нагонах. Однако, какова именно роль океана? И почему нас это заботит?

\begin{section}{Зачем изучать физику океана?}
Ответ зависит от наших интересов, которые в свою очередь зависят от того 
как мы используем океан. Важными являются три широких темы:

\begin{itemize}
\item 
Океаны~--- источники пищи. Поэтому мы интересуемся процессами в море также как 
фермеры интересуются погодой и климатом. В океане не просто есть погодные 
условия, такие как изменения температуры и течения, но океанские погодные 
условия удобряют море. Атмосферная погода редко удобряет поля, за исключением 
небольшого количества азота фиксируемого молниями.

\item
Океаны используются человеком. Мы строим здания на побережье или просто в море, 
мы используем океаны для транспорта, и для отдыха~--- купаемся, катаемся на 
лодках, рыбачим, занимаемся серфингом и подводным плаванием. Поэтому нам 
интересны процессы которые влияют на эту деятельность, особенно волны, ветра, 
течения и температура.

\item
Океаны влияют на погоду и климат. Океаны влияют на распределение дождей, засух, 
наводнения, региональный климат и развитие штормов, ураганов и тайфунов. 
Поэтому нам интересно взаимодействие океана с атмосферой, особенно потоки тепла 
и воды проходящие через поверхность моря, транспорт тепла океанами, влияние 
океана на климат и синоптическую ситуацию.
\end{itemize}

Эти темы влияют на выбор объектов изучения. Объекты определяют что мы меряем, 
как производим измерения и где. Некоторые процессы локальны, такие как 
разрушение волн на пляже, некоторые региональны, такие как влияние севера 
Тихого океана на погоду Аляски, а некоторые глобальны, такие как влияние 
океанов на изменение климата и глобальное потепление. Если эти причины для
изучения океана действительно важны, давайте начнём наше исследовательское
путешествие. Любому путешествию необходим пункт назначения. Каков же он у нас?
\end{section}

\begin{section}{Цели}
На самом основном уровне, я надеюсь что студенты, прочитав этот текст, получат 
представление о главных концептуальных схемах (или теориях), которые формируют 
основу физической океанографии, как к ним пришли и почему они так широко 
принимаются, как океанографы достигают упорядочивания беспорядочного океана 
и роли эксперимента в океанографии (перефразируя Shamos, 1995:p.89).

При более обстоятельном изучении, я ожидаю что вы будете способны описать 
физические процессы происходящие в океанах и прибрежных зонах, взаимодействие 
океана и атмосферы, распределение океанских ветров, течений, потоков тепла 
и водных масс. В тексте идеям уделено большее внимание чем математическим 
выкладкам. Мы постараемся ответить на такие вопросы как:

Какова основа нашего понимания физики океана?

\begin{itemize}
\item
Что такое физические свойства морской воды? 

\item
Каковы важные термодинамические и динамические процессы влияющие на океан. 

\item
Какие уравнения описывают эти процессы и как они выведены? 

\item
Какие допущения мы использовали для их вывода? 

\item
Имеют ли эти уравнения полезные решения? 

\item
Насколько хорошо эти решения описывают процесс? То есть, каковы 
экспериментальные основания теорий? 

\item
Какие процессы плохо понятны? Какие хорошо?
\end{itemize}

Каковы источники информации о физических переменных?
\begin{itemize}
\item
Какие инструменты используются для измерения каждой переменной? 

\item
Каковы их точность и пределы измерения? 

\item
Какие данные существуют за длительный период времени? 

\item
Какая платформа используется? Спутники, корабли, буи, буйковые станции?
\end{itemize}

Какие процессы важны? Некоторые важные процессы мы будем изучать, как то:
\begin{itemize}
\item
Накопление и транспорт тепла в океанах 

\item
Обмен теплом с атмосферой и роль океана в климате. 

\item
Ветровое и температурное воздействие на поверхностный слой перемешивания. 

\item
Ветровая циркуляция включая Экмановскую циркуляцию, Экмановский насос 
глубинной циркуляции и апвеллинг. 

\item
Динамика океанических течений, включая геострофические течения и роль 
завихренности. 

\item
Формирование водных масс. 

\item
Термохалинная циркуляция океана. 

\item
Экваториальная динамика и Эль Ниньо. 

\item
Общая циркуляция океанов а также Мексиканского залива. 

\item
Математические модели циркуляции. 

\item
Волны в океане, включая поверхностные волны, внутренние колебания, приливы 
и цунами. 

\item
Волны в мелкой воде, прибрежные процессы и предсказание приливов
\end{itemize}

Каковы основные течения и водные массы в океане, что определяет их 
распределение, и как океан взаимодействует с атмосферой?
\end{section}

\begin{section}{Организация (структура)}

Перед тем как начать путешествие, мы обычно стараемся узнать о тех местах 
которые собираемся посетить. Мы изучаем карты и путеводители. В нашей книге 
путеводителями будут статьи и книги написанные океанографами. Мы начнем 
с краткого обзора того что известно об океанах. Затем продолжим описанием 
океанских бассейнов, чтобы понять как форма морей влияет на физические 
процессы в воде. Далее, мы изучим внешние силы, ветер и тепло, действующие 
на океан и его отклик на них. Во время изучения будет изложено необходимое 
количество теории и представлены натурные данные.

К тому времени когда мы достигнем главы 7, нам необходимо будет понять 
уравнения описывающие динамическую реакцию океана. Таким образом мы рассмотрим 
уравнения движения, влияние вращения Земли и завихренности. Это приведёт к 
изучению ветровых океанических течений, геострофического приближения и важности 
постоянства вихря.

Ближе к концу мы обсудим некоторые частные примеры: глубинную циркуляцию, 
экваториальный океан и Эль Ниньо, а также циркуляцию отдельных частей океана. 
Далее мы рассмотрим роль математического моделирования в описании океана. 
В конце мы изучим прибрежные процессы, волны, приливы, предсказание волн 
и приливов, цунами и штормовые нагоны.
\end{section}

\begin{section}{Общая картина}
Я надеюсь вы обратите внимание на то что при описании динамики океана мы 
будем использовать теорию, натурные (эмпирические) данные и численные модели. 
Их необходимо рассматривать вместе, по отдельности они не самодостаточны.

\begin{enumerate}
\item
Процессы в океане нелинейны и турбулентны, а теория нелинейных, турбулентных 
потоков в сложных бассейнах не очень хорошо развита. Теории, используемые 
для описания океана~--- сильно упрощённые приближения реальности. 

\item
Натурные измерения разбросаны в пространстве и во времени. Они обеспечивают 
грубое описание усреднённого по времени потока, но большинство процессов 
во многих регионах ещё мало исследованы. 

\item
Численные модели включают наиболее реалистичные теоретические идеи, они могут 
помочь интерполировать натурные исследования во времени и пространстве 
и используются для предсказания климатических изменений, течений и волн. 
Однако численные равенства являются приближениями непрерывных аналитических 
уравнений описывающих жидкий поток, они не содержат никакой информации о 
потоке между узловыми точками, и не могут пока использоваться для полного 
описания турбулентного потока наблюдающегося в океане. 
\end{enumerate}

Соединяя теорию и натурные измерения в численных моделях мы избегаем 
сложностей связанных с их использованием по отдельности. Продолжающееся 
усовершенствование способов комбинирования этих подходов ведёт к гораздо 
более точному описанию океана. Конечная цель этого узнать океан так хорошо 
чтобы можно было предсказывать будущие изменения в окружающей среде, 
включая изменения климата или реакцию рыбоохранных органов на перелов.

Объединение теории, натурных исследований и компьютерных моделей относительно 
молодо. Четыре десятилетия экспоненциального роста мощи компьютеров 
сделали доступными настольные компьютеры способные моделировать важные 
физические процессы и динамику океана.

\begin{quotation}
Все те кто вовлечён в науку знают что компьютер стал важнейшим инструментом 
для исследований... научные расчёты достигли той точки где они находятся 
наравне с лабораторным экспериментом и математической теорией как 
инструмент для научных и инженерных изысканий. (langer(1999) ).
\end{quotation}


%% http://oceanworld.tamu.edu/resources/ocng_textbook/chapter01/Images/Fig1-1s.jpg
%% Рисунок 1.1 Данные, численные модели и теория необходимы для понимания океана. В конечном счёте понимание системы океан-атмосфера-суша позволит предсказывать будущие состояния системы.

Объединение теории, натурных исследований и компьютерных моделей предполагает 
новый путь развития океанологии. В прошлом океанограф должен был бы выдумать 
теорию, собрать данные для того чтобы её проверить и опубликовать результаты. 
Теперь задачи стали настолько специализированными что мало кто может все 
это делать в одиночку. Немногие замечательно создают теории, собирают 
данные и разрабатывают численные модели. Вместо этого всё больше и больше 
работы делается командами учёных и инженеров.
\end{section}

\begin{section}{Дальнейшее чтение}
Если вы знаете немного об океанах и океанографии, я предлагаю вам начать 
с чтения книги Маклениша "Reading the Ocean", особенно её четвёртой главы. 
По моему мнению это наиболее полное не техническое описание того как 
океанографы понимают океан.

Вы также можете извлечь пользу из чтения отдельных глав любой 
океканографической книги начального уровня. Книги таких авторов как 
Gross, Pinet, или Thurman особенно интересны. Три текста выпущенные 
Открытым Университетом обеспечивают немного более продвинутое рассмотрение 
различных вопросов. 

Gross, M. Grant and Elizabeth Gross (1996) Oceanography, 
A View of Earth 7th Edition. Upper Saddle River, New Jersey: Prentice Hall. 

Mac Leish William (1989) The Gulf Stream: Encounters With the Blue God. 
Boston: Houghton Mifflin Company. 

Pinet, Paul R. (2000) Invitation to oceanography. 2nd Edition. Sudbury, 
Massachusetts: Jones and Bartlett Publishers.

Open University (1989) Ocean Circulation. Oxford: Pergamon Press. 

Open University (1989) Seawater: Its Composition, Properties and Behaviour. 
Oxford: Pergamon Press. 

Open University (1989) Waves, Tides and Shallow-Water Processes. 
Oxford: Pergamon Press. 

Thurman, Harold V. and Elizabeth A. Burton (2001) Introductory oceanography. 
9th Edition. Upper Saddle River, New Jersey: Prentice Hall.
\end{section}

\end{chapter}