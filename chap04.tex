% -*- coding: utf-8 -*-

\begin{chapter}{Влияние атмосферы на океан}
% \chapter{Atmospheric Influences}
Солнце и земная атмосфера прямо и косвенно оказывают определяющее
влияние на все динамические процессы в океане. Основные внешние по
отношению к океану источники и стоки энергии~--- свет, испарение,
тепловое длинноволновое излучение и потоки явного тепла с поверхности
океана. Влияние ветра на циркуляцию поверхностных вод океана проникает
до глубины около километра. Глубокое перемешивание до определенной
степени управляет глубинными океаническими течениями.
%
% The sun \index{sun}and the atmosphere drive directly or indirectly almost 
% all dynamical processes in the ocean. The dominant external sources and 
% sinks of energy are sunlight, evaporation, infrared emissions from
% the sea surface, and sensible heating of the sea by warm or cold winds.
% Winds drive the ocean's surface circulation down to depths of around
% a kilometer. Wind and tidal mixing\index{mixing!tidal}
% drive the deeper currents in the ocean.

Океан, в свою очередь, влияет на атмосферную циркуляцию. Неравномерное
пространственное распределение потоков тепла между океаном и
атмосферой приводит к возникновению ветров в атмосфере. Солнечное
излучение прогревает воды океана в тропиках. Испарение с прогретой
поверхности океана приводит к скрытому переносу тепла из океана в
атмосферу. Ветры и океанические течения переносят тепло от экватора к
полюсам, откуда оно передается в космос.
%
% The ocean, in turn, is the dominant source of heat that drives the 
% atmospheric circulation.\index{atmospheric circulation!driven by ocean}
% The uneven distribution of heat loss and gain by the ocean leads to winds
% in the atmosphere. Sunlight warms the tropical ocean, which evaporate,
% transferring heat in the form of water vapor to the atmosphere. The heat
% is released when the vapor condenses as rain. Winds and ocean currents
% carry heat poleward, where it is lost to space.

Поскольку атмосфера влияет на динамику океана, а океан, в свою
очередь, также влияет на атмосферную циркуляцию, мы должны
рассматривать океан и атмосферу как единую динамическую систему. В
этой главе мы рассмотрим обмен теплом и влагой между атмосферой и
океаном. Далее мы исследуем влияние атмосферных движений на океан и
обмен моментом движения, приводящий к формированию дрейфовых
течений. Затем мы рассмотрим, как океан и атмосфера взаимодействуют в
районе Тихого океана в процессе явления Эль-Ниньо.
%
% Because the atmosphere drives the ocean, and the ocean drives the
% atmosphere, we must consider the ocean and the atmosphere as a coupled
% dynamic system. In this chapter we will look mainly at the exchange of
% momentum between the atmosphere and the ocean. In the next chapter,
% we will look at heat exchanges. In chapter 14 we will look at how
% the ocean and the atmosphere interact in the Pacific to produce El Ni\~{n}o.

 
\begin{section}{Земля в космическом пространстве.}
% \section{The Earth in Space}
Орбита Земли вокруг Солнца по своей форме близка к окружности со
средним радиусом~$1.5\times 10^{-8}$~км. Эксцентриситет орбиты мал,
0,0168. Таким образом, Земля на 3,4\% дальше от Солнца в положении
афелия, чем в перигелии, положении наиболее близком к
Солнцу. Положение перигелия достигается в январе, при этом точное
время его наступления смещается ежегодно примерно на 20~мин. В
1995~г. Земля находилась в перигелии 3 января. Ось вращения Земли
наклонена к плоскости земной орбиты под углом~23.45°
(рис. 4.1). Положение Земли при этом таково, что солнечные лучи падают
на земной экватор под прямым углом в дни весеннего и осеннего
равноденствий, 21~марта и 21~сентября.
%
% Earth's \index{earth!in spaceabout the sun\index{sun} is nearly circular 
% at a mean distance of \(1.5 \times 10^8\) km. The eccentricity of the orbit
% is small, 0.0168. Thus earth is 3.4\% further from the Sun\index{sun}
% at aphelion than at perihelion, the time of closest approach to the sun.
% Perihelion occurs every year in January, and the exact time changes by
% about 20 minutes per year. In 1995, it occurred on 3 January. Earth's axis
% of rotation is inclined 23.45\degrees\ to the plane of earth's orbit around
% the sun\index{sun} (figure 4.1). The orientation is such that
% the sun\index{sun!equinox} is directly overhead at the Equator on the
% vernal and autumnal equinoxes, which occur on or about 21 March
% and 21 September each year.


%%Рис. 4.1. Взаимное положение Земли и Солнца. Эллиптичность земной
%%орбиты вокруг Солнца и наклон земной оси вращения по отношению к
%%плоскости орбиты приводит к неравномерному распределению тепла между
%%сезонами.

Широтные дуги 23,45 град. в северном и южном полушарии Земли
называются Тропиком Рака и Тропиком Козерога, соответственно. Область,
располагающаяся между этими широтными кругами, называется тропиками. В
результате эллиптичности земной орбиты максимум средней солнечной
инсоляции на земной поверхности в целом приходится на январь. В
результате наклона земной оси максимум солнечной инсоляции для
внетропических районов приходится на 21~июня в северном полушарии и на
21~июля в южном полушарии.
%
% The latitudes of 23.45\degrees\ North and South are the Tropics of Cancer
% and Capricorn respectively. The tropics lie equatorward of these latitudes.
% As a result of the eccentricity of earth's orbit, maximum solar
% insolation\index{insolation!maximum} averaged over the surface of the earth
% occurs in early January each year. As a result of the inclination of earth's
% axis of rotation, the maximum insolation at any location outside the tropics
% occurs around 21 June in the northern hemisphere, and around 21 December
% in the southern hemisphere.

Если бы приходящая солнечная радиация мгновенно распределялась по
земной поверхности, то максимальные температуры наблюдались бы в
январе (в положении перигелия~--- примеч.перев.). Напротив, при
медленном перераспределении получаемого от Солнца тепла северное
полушарие более всего должно прогреваться летом (при наибольшем угле
падения солнечных лучей~--- примеч.перев.). Из этого следует, что в
реальности перераспределение тепла воздушными и океанскими течениями
требует значительного времени.
%
% If solar heat was rapidly redistributed over earth, maximum
% temperature would occur in January. Conversely, if heat were poorly
% redistributed, maximum temperature in the northern hemisphere would occur
% in summer. So it is clear that heat is not rapidly redistributed by winds
% and currents.
\end{section}

\begin{section}{Атмосферная циркуляция.}
% \section{Atmospheric Wind Systems}
На рис.~4.2 показано среднее годовое распределение приземного ветра и
поля давления для 1989~г. На карте видны зона наиболее сильных
западных ветров характерных для широтного пояса 40--60?, который
называют поэтому «ревущие сороковые», самые слабые ветры~--- в
субтропическом поясе около 30~градусов широты, пассаты с восточной
составляющей в тропической зоне, и более слабые восточные ветры вдоль
экватора. Скорость и направление ветра в атмосфере зависят от
неравномерного пространственного распределения радиационного баланса,
площади континентов и вертикальной циркуляции в атмосфере.
%
% Figure 4.2 shows the distribution of sea-level winds and pressure averaged
% over the year 1989. The map shows strong winds from the west between 
% 40\degrees\ to 60\degrees\ latitude, the roaring forties, weak winds in
% the subtropics near 30\degrees\ latitude, trade winds from the east in
% the tropics, and weaker winds from the east along the Equator. The strength
% and direction of winds in the atmosphere is the result of uneven distribution
% of solar heating and continental land masses and the circulation of winds
% in a vertical plane in the atmosphere.

%%Рис. 4.2. Карта среднего годового приземного атмосферного давления по
%%данным 40-летнего Реанализа ECMWF (Kallberg et al., 2005).

Простейшая схема распределения атмосферных ветров (рис.~4.3)
показывает, что большое влияние оказывается экваториальной конвекцией
и другими процессами в верхней атмосфере. Средняя скорость ветра над
oкеанами~(Wentz et al., 1984)
\begin{equation}
U_{10} = 7.4\mps.
\end{equation}
%
% A cartoon of the distribution of winds in the atmosphere (figure 4.3) shows
% that the surface winds are influenced by equatorial convection and other 
% processes higher in the atmosphere. The mean value \index{wind!global mean}of
% winds over the ocean is (Wentz et al. 1984):
%\begin{equation}
%U_{10} = 7.4 \text{ m/s}
%\end{equation}



%%Рис. 4.3. Упрощенная схема атмосферной циркуляции, управляемой
%%нагреванием тропиков и выхолаживанием высоких широт. Сверху:
%%Меридиональные ячейки в атмосфере и влияние вращения Земли на
%%направление ветра. Снизу: Вертикальный меридиональный разрез
%%показывающий две основные ячейки меридиональной циркуляции (по данным
%%The Open University (1989a))

Карты приземного ветра демонстрируют сезонную изменчивость. Наибольшие
изменения наблюдаются в районах Индийского океана и западной части
Тихого океана (рис.~4.4). Оба района находятся под влиянием Азиатского
муссона. Зимой в нижней атмосфере в районе интенсивного выхолаживания
над Сибирью формируется область высокого давления, по ее периферии
холодный воздух перемещается с северо-запада на юго-восток над Японией
и далее, прогреваясь над теплым океанским течением Куросио. Летом
формирование термической депрессии в поле атмосферного давления над
Тибетом способствует притоку теплого влажного воздуха с Индийского
океана, с приходом которого начинается «сезон дождей» в Индии.
%
% Maps of surface winds change somewhat with the seasons. The largest
% changes are in the Indian Ocean and the western Pacific Ocean (figure 4.4).
% Both regions are strongly influenced by the Asian monsoon. In winter,
% the cold air mass over Siberia creates a region of high pressure at
% the surface, and cold air blows southeastward across Japan and on across
% the hot Kuroshio\index{Kuroshio}, extracting heat from the ocean.
% In summer, the thermal low over Tibet draws warm, moist air from
% the Indian Ocean leading to the rainy season over India.

%%Рис. 4.2. Карта средней годовой скорости ветра на высоте 10 м в июне,
%%июле и в августе, построенная по результатам 40-летнего Реанализа
%%ECMWF (Kallberg et al., 2005).

%%Рис. 4.2. Карта средней годовой скорости ветра на высоте 10 м в июне,
%%июле и в августе, построенная по результатам 40-летнего Реанализа
%%ECMWF (Kallberg et al., 2005).
\end{section}  

\begin{section}{Планетарный пограничный слой.}
% \section{The Planetary Boundary Layer}
% The atmosphere within 100 m of the sea surface is influenced by the turbulent
% drag of the wind on the sea and the fluxes of heat through the surface.
% This is the \textit{atmospheric boundary layer}. 
% \index{atmospheric boundary layer|textbf}It's thickness $Z_i$ varies from
% a few tens of meters for weak winds blowing over water colder than the air
% to around a  kilometer for stronger winds blowing over water
% warmer than the air.

% The lowest part of the atmospheric boundary layer is the surface layer. 
% Within this layer, which has thickness of $\approx 0.1 Z_i$, vertical fluxes 
% of heat and momentum are nearly constant.

% Wind speed varies as the logarithm of height within the surface layer for
% neutral stability. See ``The Turbulent Boundary Layer Over a Flat Plate'' in
% Chapter 8. Hence, the height of a wind measurement is important. Usually, 
% winds are reported as the value of wind at a height 10 m above the
% sea $U_{10}$.
\end{section}

\begin{section}{Наблюдения за ветром.}
% \section{Measurement of Wind}
Основной автор перевода этой части~--- Дмитрий Чечин, кафедра метеорологии МГУ

Измерение ветровых характеристик проводится уже не первое
столетие. Мори (1855) был первым, кто собрал и систематизировал данные
по ветру и сделал первые карты ветра. В настоящее время в Национальной
Администрации по Атмосфере и Океану США (NOAA USA) собраны,
отредактированы и переведены в цифровой формат миллионы данных
наблюдений за ветром за период более 100 лет. Результатом этой работы
стала база данных COADS (Comprehensive Ocean, Atmosphere Data Set),
которая подробно рассматривается в §5.5. Эта база данных широко
используется для исследования атмосферного влияния на океан.
Современные сведения о характеристиках ветра у земной поверхности
поступают из разных источников. Далее перечислены в порядке убывания
относительной важности наиболее значимые сведения о методах получения
характеристик ветра.
%
% \index{wind!measurement of|(}Wind at sea has been measured for centuries. 
% Maury (1855) was the first to systematically collect and map wind reports.
% Recently, the US National Atmospheric and Oceanic Administration
% \textsc{noaa} has collected, edited, and digitized millions of observations
% going back over a century. The resulting 
% \textit{International Comprehensive Ocean, Atmosphere Data Set} 
% \textsc{icoads} \index{ICOADS (international comprehensive ocean-atmosphere 
% data set)}discussed in \S 5.5 is widely used for studying atmospheric
% forcing of the ocean.
  

% Our knowledge of winds at the sea surface come from many sources. Here are 
% the more important, listed in a crude order of relative importance:


\begin{paragraph}{Шкала Бофорта.}
% \paragraph{Beaufort Scale}
Наиболее распространенным источником сведений о ветре являются данные
измерений скорости ветра в соответствии со шкалой Бофорта. Даже в 1990
году 60\%~данных наблюдений за ветром, поступивших из Северной
Атлантики, были представлены на основе шкалы Бофорта. Шкала основана
на наблюдаемых с борта судна характеристиках водной поверхности, в
частности, на площади покрытия пеной и форме волн (таблица 4.1).
%
% \index{wind!Beaufort scale}By far the most common source of wind data up 
% to 1991 have been reports of speed based on the Beaufort scale. The scale
% is based on features, such as foam coverage and wave shape, seen by
% an observer on a ship (table 4.1).

Шкала была предложена Адмиралом Сэром Ф.Бофортом в 1806~г. для
определения силы воздействия ветра на паруса судна. Она была одобрена
Британским Адмиралтейством в 1838 году и вскоре была принята в
повсеместное использование.
%
% The scale was originally proposed by Admiral Sir F. Beaufort in 1806 to 
% give the force of the wind on a ship's sails. It was adopted by the
% British Admiralty in 1838 and it soon came into general use.


В 1874~году Международный Метеорологический Комитет признал шкалу
Бофорта в качестве международного стандарта. В 1926~году была принята
модифицированная шкала, в баллах Бофорта, которой соответствовали
скорости ветра на высоте 6~метров над поверхностью океана. В 1946~году
шкала была вновь пересмотрена, она была расширена для учета более
сильных скоростей ветра, а высота, для которой определялась скорость
ветра, стала соответствовать 10~метрам. В основе шкалы 1946~года лежит
эмпирическое соотношение $U_{10} = 0.836 B^{3/2}$ де $B$~--- баллы по
шкале Бофорта а $U_{10}$~--- скорость ветра на высоте 10~метров,
выраженная в метрах в секунду (List, 1966). В последнее время
различные группы ученых пересматривали шкалу Бофорта, сравнивая
скорость ветра, рассчитанную по шкале, с измерениями, выполненными с
помощью анемометров, установленных на кораблях на известной
высоте. Рекомендуемые по результатам этих работ соотношения
представлены в таблице~4.1.
%
% The International Meteorological Committee adopted the force scale for
% international use in 1874. In 1926 they adopted a revised scale giving
% the wind speed at a height of 6 meters corresponding to the Beaufort Number.
% The scale was revised again in 1946 to extend the scale to higher wind speeds
% and to give the equivalent wind speed at a height of 10 meters. The 1946
% scale was based on the equation $U_{10} = 0.836 B^{3/2}$, where
% $B = $ Beaufort Number and $U_{10}$ is the wind speed in meters per second
% at a height of 10 meters (List, 1966). More recently, various groups have
% revised the Beaufort scale by comparing Beaufort force with ship measurements
% of winds. Kent and Taylor (1997) compared the various revisions of the scale
% with winds measured by ships having anemometers at known heights.
% Their recommended values are given in table 4.1.


Таблица 4.1 Шкала Бофорта и состояние моря.

\begin{tabular}{llll}
Балл Бофорта & Словесная характеристика ветра & м/с & Видимое состояние моря \\
0 & Штиль & 0 &
Зеркально гладкая морская поверхность \\

1 & Тихий Ветер & 1.2 &
Рябь в виде чешуи; пены на гребнях нет. \\

2 & Легкий ветер & 2.8 &
Небольшие волны; гребни волн гладкие, не опрокидываются \\

3 & Слабый Ветер & 4.9 & 
Различимые волны; гребни начинают опрокидываться; редкие барашки. \\

4 & Умеренный Ветер & 7.7 &
Волны становятся удлиненными; барашки во многих местах. \\

5 & Свежий ветер & 10.5 &
Средние волны; многочисленные барашки; появляются мелкие брызги. \\

6 & Сильный Ветер & 13.1 & 
Образуются крупные волны; барашки повсеместно; больше брызг. \\

7 & Крепкий Ветер & 15.8 & 
Волны громоздятся; гребни срываются; пена ложится полосами по склонам
волн. \\

8 & Очень крепкий Ветер & 18.8 &
Длинные, умеренно высокие волны; по краям гребней взлетают брызги;
отчетливые полосы пены, сносимой ветром. \\

9 & Шторм & 22.1 & 
Волны высокие; качка; пена широкими плотными полосами ложится по
ветру; брызги ухудшают видимость. \\

10 & Сильный Шторм & 25.9 & 
Очень высокие волны с нависающими гребнями; поверхность моря белая от
пены, которую ветер выдувает крупными хлопьями; сильная качка,
видимость ухудшена. \\

11 & Жестокий шторм & 30.2 & 
Исключительно высокие волны; море покрыто белыми хлопьями пены;
видимость еще более ухудшается. \\

12 & Ураган & 35.2 &
Воздух наполнен брызгами и пеной; море полностью белое, покрытое
пеной; видимость сильно ухудшена. \\
\end{tabular}

По Кенту и Тэйлору (Kent and Taylor, 1997)
%
% \begin{table}[t!] {\textbf{\footnotesize{Table 4.1 Beaufort Wind Scale and State of the Sea}}}
% \index{wind!Beaufort scale}
% \\[1ex]
% \begin{footnotesize}
% \begin{tabular}{@{}clrp{70mm}@{}} \hline
% Beaufort    & Descriptive & m/s & Appearance \rule{0ex}{2.5ex}of the Sea \\
% Number & term \  &  & \\[0.5ex]
% \hline  %\\
% 0 & Calm \rule{0ex}{2.5ex}  &   0 & Sea like a mirror. \\
% 1 & Light Air&  1.2 &   Ripples with appearance of scales; no foam crests.\\
% 2 & Light Breeze &  2.8 &   Small wavelets; crests of glassy appearance, \\
% \ &\ &\ & \hspace{1em}not breaking. \\
% 3 & Gentle breeze&  4.9 &   Large wavelets; crests begin to break; scattered\\
% \ &\ &\ & \hspace{1em}whitecaps.\\
% 4 & Moderate breeze & 7.7 & Small waves, becoming longer; numerous whitecaps. \\
% 5 & Fresh breeze &  10.5 &  Moderate waves, taking longer to form; many\\
% \ &\ &\ & \hspace{1em}whitecaps; some spray. \\
% 6 & Strong breeze & 13.1 &  Large waves forming; whitecaps everywhere; \\
% \ &\ &\ &\hspace{1em}more spray. \\
% 7 & Near gale & 15.8 &  Sea heaps up; white foam from breaking waves begins\\
% \ &\ &\ & \hspace{1em}to be blown into streaks.\\
% 8 & Gale &      18.8  & Moderately high waves of greater length; edges of \\
% \ &\ &\ & \hspace{1em}crests begin to break into spindrift; foam is blown \\
% \ &\ &\ & \hspace{1em}in well-marked streaks.\\
% 9 & Strong gale &   22.1  & High waves; sea begins to roll; dense streaks of foam;
% \\
% \ &\ &\ & \hspace{1em}spray may reduce visibility. \\
% 10 &    Storm &     25.9  & Very high waves with overhanging crests; sea takes \\
% \ &\ &\ & \hspace{1em}white appearance as foam is blown in very dense \\
% \ &\ &\ & \hspace{1em}streaks; rolling is heavy and visibility reduced. \\
% 11 &    Violent storm & 30.2  & Exceptionally high waves; sea covered with white \\
% \ &\ &\ & \hspace{1em}foam patches; visibility still more reduced.  \\
% 12 &    Hurricane & 35.2  & Air is filled with foam; sea completely white\\
% \ &\ &\ &  \hspace{1em}with driving  \rule[-2.5ex]{0ex}{0.5ex}spray;
% visibility greatly reduced.\\
% \hline
% \end{tabular} \\[0.5ex]
% From Kent and Taylor (1997)
% \end{footnotesize}
% \vspace{-4ex}
% \end{table}


Наблюдатели на борту обычно передают данные метеонаблюдений, в том
числе силу ветра по шкале Бофорта, четыре раза в день, в полночь, в
6~часов утра, в полдень и в 6~часов вечера по Гринвичу. Сообщения
кодируются и передаются по радио в национальные метеорологические
службы. При этом данные могут нести в себе погрешности следующего
характера.

\begin{enumerate}
\item 
Корабли неравномерно разбросаны по океану. Они стараются не заходить в
высокие широты зимой и избегать ураганов летом. В южном полушарии
кораблей гораздо меньше, чем в северном.

\item
Наблюдатели могут ошибочно характеризовать состояние поверхности
океана, а именно на этой характеристике основана шкала Бофорта.

\item
Ошибки могут возникнуть при кодировке данных, что может привести
кневерному определению месторасположения наблюдателя.

\item
В целом, точность возможно не выше 10\%.
\end{enumerate}
%
% Observers on ships everywhere in the world usually report weather
% observations, including Beaufort force, at the same four times every day.
% The times are at 0000Z, 0600Z, 1200Z and 1800Z, where Z indicates Greenwich
% Mean Time. The reports are coded and reported by radio to national
% meteorological agencies. The biggest error in the reports is the sampling
% error\index{sampling error}. Ships are unevenly distributed over the ocean.
% They tend to avoid high latitudes in winter and hurricanes in summer, and few
% ships cross the southern hemisphere (figure 4.5). Overall, the
% accuracy\index{accuracy!winds!Beaufort} is around 10\%.

%% Figure 4.5 Location of surface observations made from volunteer
%% observing ships and reported to national meteorological agencies. From
%% NOAA, National Ocean Service.
\end{paragraph}

\begin{paragraph}{Скаттерометры.}
% \paragraph{Scatterometers}
Наблюдения ветра над океаном все чаще осуществляются с помощью
оборудования, установленного на спутниках, и скаттерометры это самый
распространенный источник наблюдений. Скаттерометрый это прибор, по
принципу радара измеряющий рассеяние сантиметровых радиоволн от
коротких, сантиметровых волн на поверхности океана. Площадь моря
покрыта рябью, состоящей из маленьких волн, амплитуда которых зависит
от направления и скорости ветра. Скаттерометр измеряет рассеяние по
2--4~направлениям, далее по данным измерений высчитывается скорость
и направление ветра.
%
% \index{wind!from scatterometers} \index{scatterometers}Observations of winds
% at sea now come mostly from scatterometers on satellites (Liu, 2002).
% The scatterometer is a instrument very much like a radar that measures the
% scatter of centimeter-wavelength radio waves from small,
% centimeter-wavelength waves on the sea surface. The area of the sea covered
% by small waves, their amplitude, and their orientation, depend on wind speed
% and direction. The scatterometer measures scatter from 2--4 directions, from
% which wind speed and direction are calculated.

Скаттерометры на ERS-1 и ERS-2 осуществляют глобальное измерение
ветровых характеристик из космоса с 1991 года. Скаттерометр НАСА,
установленный на ADEOS, проводил измерения ветра в течение полугода,
начиная с ноября 1996 и вплоть до преждевременного падения
спутника. Его сменил Quicksat, запущенный в 1999. Прибор,
установленный на Quicksat, каждые 24 часа показывает данные, собранные
с 90\% поверхности океана.
%
% The scatterometers on \textsc{ers-1} and 2 have made global measurements of
% winds from space since 1991. The \textsc{nasa}
% scatterometer\index{scatterometers} on \textsc{adeos} measured winds for
% a six-month period beginning November 1996 and ending with the premature
% failure of the satellite. It was replaced by another scatterometer on
% QuikScat, launched on 19 June 1999. Quikscat\index{scatterometer!Quikscat}
% views 93\% of the ocean every 24 hr with a resolution of 25 km. 

Фрейлих и Данбар (Freilich and Dunbar, 1999) сообщают, что, в целом,
скаттрометр NASA на ADEOS измерял скорость ветра с точностью до 
$\pm 1,3$~м/с. Ошибка в измерении направления ветра была 
$\pm 17^\circ$. Пространственное разрешение было 25 км. Ошибки в вычислении
скорости были следствием недостатка знаний о зависимости рассеяния от
скорости ветра, неизвестного влияния поверхностных пленок, ошибок в
измерениях. Калиброванные данные Quicksat имеют точность~$\pm 1$~м/с.
%
% Freilich and Dunbar (1999) report that, overall, the \textsc{nasa}
% scatterometer\index{scatterometers!accuracy of} on \textsc{adeos} measured
% wind speed with an accuracy\index{accuracy!winds!scatterometer}
% of $\pm 1.3$ m/s. The error in wind direction was $\pm17$\degrees. Spatial
% resolution was 25 km. Data from QuikScat\index{QuikScat} has an accuracy
% of $\pm 1$ m/s.


Так как скаттерометры «видят» определенную площадь поверхности океана
один раз в день, или раз в два дня, то требуются дополнительные
численные процедуры, в частности математические метеорологические
модели, помогающие получить ежедневные карты ветра без разрывов в
пространственном покрытии.
%
% Because scatterometers\index{scatterometers} view a specific oceanic area
% only once a day, the data must be used with numerical weather models
% to obtain 6-hourly wind maps required for some studies.

% \begin{paragraph}{Windsat.}
% Windsat\index{Windsat} is an experimental, polarimetric, microwave radiometer
% developed by the US Navy that measures the amount and polarization
% of microwave radiation emitted from the sea at angles between 50\degrees\
% to 55\degrees\ relative to the vertical and at five radio frequencies.
% It was launched on 6 January 2003 on the Coriolis satellite. The received
% radio signal is a function of wind speed, sea-surface temperature, water
% vapor in the atmosphere, rain rate, and the amount of water in cloud drops.
% By observing several frequencies simultaneously, data from the instrument
% are used for calculating the surface wind speed and direction, sea-surface
% temperature, total precipitable water, integrated cloud liquid water,
% and rain rate over the ocean regardless of time of day or cloudiness. 

% Winds are calculated over most of the ocean on a 25-km grid once a day.
% Winds measured by Windsat have an accuracy of $\pm 2$ m/s in speed
% and $\pm 20$\degrees\ in direction over the range of 5--25 m/s.
\end{paragraph}


\begin{paragraph}{Cпециальный микроволновый радиометр SMM/I.}
% \paragraph{Special Sensor Microwave SSM/I}
Cпециальный микроволновый радиометр SMM/I~--- еще один прибор
спутникового базирования, широко используемый для измерения скорости
ветра, устанавливаемый с 1987 года на спутниках U.S. Defense
Meteorological Satellite Program, орбиты которых совпадают с полярными
орбитами метеоспутников NOAA. Прибор измеряет микроволновую радиацию,
испускаемую поверхностью океана под углом около 60° от
вертикали. Излучение является функцией скорости ветра, водяного пара в
атмосфере и водности облачных капель. Одновременные измерения на
нескольких частотах позволяют рассчитать скорость ветра у поверхности.
%
% Another satellite instrument that is used to measure wind speed is the
% Special-Sensor Microwave/Imager (\textsc{ssm/i}) carried since 1987 on the
% satellites of the U.S. Defense Meteorological Satellite Program in orbits
% similar to the \textsc{noaa} polar-orbiting meteorological satellites.
% The instrument measures the microwave radiation emitted from the sea at
% an angle near 60\degrees\ from the vertical. The radio signal is a function
% of wind speed, water vapor in the atmosphere, and the amount of water in
% cloud drops. By observing several frequencies simultaneously, data from
% the instrument are used for calculating the surface wind speed, water vapor,
% cloud water, and rain rate.

Измерения скорости ветра с помощью SSM/I имеют точность~$\pm 2$~м/с. При
совмещении данных этих измерений с результатами объективного анализа
фактического поля ветра, рассчитанного по численной модели
Европейского Центра Среднесрочного прогноза Погоды на изобарической
поверхности 1000 гПа, направление ветра может быть посчитано с
точностью $\pm 22$~° (Atlas, Hoffman, and Bloom, 1993). Глобальные данные
на регулярной сетке с пространственным разрешением 2.5° по долготе и
2.0° по широте поступают с 1987 года каждые 6 часов (Atlas et al.,
1996). Но необходимо помнить, что угол обзора прибора ограничен и он
дает данные по определенной видимой им зоне океана только один раз в
день, поэтому 6-часовые карты с данными по ветру в узлах регулярной
сетки имеют большие погрешности.
%
% Winds measured by \textsc{ssm/i} have an accuracy\index{accuracy!winds!SSM/I}
% of $\pm$ 2 m/s in speed. When combined with \textsc{ecmwf} 1000 mb wind
% analyses, wind direction can be calculated with an accuracy
% of $\pm 22$\degrees\ (Atlas, Hoffman, and Bloom, 1993). Global, gridded data
% are available since July 1987 on a 0.25\degrees\  grid every 6 hours.
% But remember, the instrument views a specific oceanic area only once a day,
% and the gridded, 6-hourly maps have big gaps.
\end{paragraph}

\begin{paragraph}{Анемометры на борту}
Следующий распространенный источник данных по характеристикам ветра,
поступающих в метеослужбы, это результаты измерений с использованием
анемометров, установленных на суднах. Измерения проводятся четыре раза
в сутки в установленные сроки по Гринвичу и передаются по радио в
метеослужбы.
%
% Satellite observations are supplemented by winds reported to meteorological
% agencies by observers reading ane\-mom\-eters on ships. The anemometer is
% read four times a day at the standard Greenwich times and reported via radio
% to meteorological agencies. 
%
Эти сообщения также могут содержать следующие погрешности:

1.Измерения имеют пространственный и временной разброс. Очень мало
  кораблей проводят измерения с помощью анемометров.

1.Может оказаться, что после установки поверка анемометра не
  проводилась ни разу.

1.Наблюдатель обычно производит снятие показаний анемометра в течение
  нескольких секунд, поэтому измерение отражает мгновенное значение
  скорости и направления ветра, в то время как получение среднего
  значения за час требует измерений в течение нескольких
  минут. Необходимо помнить, что ветер может быть порывистым, и
  измерения могут содержать ошибки порядка 10\%--30\%.

1.Данные наблюдений передаются по радио в виде закодированных
  сообщений, и при кодировке могут быть допущены ошибки. Такие ошибки
  могут привести к тому что данные с корабля могут иметь неверную
  привязку к координатам, как это показано на рис. 4.4.

%
% Again, the biggest error is the sampling error\index{sampling error}. Very
% few ships carry calibrated anemometers. Those that do tend to be commercial
% ships participating in the Volunteer Observing Ship program (figure 4.5).
% These ships are met in port by scientists who check the instruments and
% replace them if necessary, and who collect the data measured at sea.
% The accuracy\index{accuracy!winds!ship} of wind measurements from these ships
% is about $\pm 2$ m/s.
\end{paragraph}

\begin{paragraph}{Поверенные Анемометры на Судах.}
Малое количество судов имеют поверенные анемометры. Часто это
коммерческие суда, участвующие в the Volunteer Observing Ship
program. Ученые, собирающие данные наблюдений в море, встречают эти
суда в порту и поверяют или заменяют приборы при
необходимости. Наивысшая точность метода~$\pm 2$~м/с.  
\end{paragraph}

\begin{paragraph}{Поверенные анемометры на погодных буях.}
% \paragraph{Calibrated Anemometers on Weather Buoys}
Наиболее точные измерения параметров ветра производятся поверенными
анемометрами на заякоренных погодных буях. К сожалению, таких буев
мало, возможно, лишь около сотни их рассеяно по миру. Некоторые, такие
как сеть буев Tropical Atmosphere Ocean TAO в тропиках Тихого океана,
предоставляют данные из отдаленных областей, куда редко заходят суда,
однако большая часть буев расположена вблизи береговых линий. NOAA
курирует буи у побережья Соединенных Штатов и сеть TAO в Тихом
океане. Данные с прибрежных буев осредняются за восемь минут до
окончания часа и данные наблюдений отсылаются на берег через спутник.
Наивысшая точность анемометров на буях, курируемых National Data Buoy
Center характеризуется погрешностью~$\pm 1$~м/с или 10\% для скорости ветра
и~$\pm 10^\circ$ для направления ветра (Beardsley et al., 1997).
%
% The most accurate measurements of winds at sea are made by calibrated
% anemometers on moored weather buoys. Unfortunately there are few such buoys,
% perhaps only a hundred scattered around the world. Some, such as Tropical
% Atmosphere Ocean \textsc{tao} array in the tropical Pacific (figure 14.14)
% provide data from remote areas rarely visited by ships, but most tend to be
% located just offshore of coastal areas.
% \textsc{noaa} operates buoys offshore of the United States and the
% \textsc{tao} array in the Pacific. Data from the coastal buoys are averaged
% for eight minutes before the hour, and the observations are transmitted
% to shore via satellite links.
% 
% The best accuracy of anemometers on buoys operated by the \textsc{us}
% National Data Buoy Center is the greater of \(\pm\)1 m/s or 10\% for wind
% speed and $\pm 10$\degrees\ for wind direction (Beardsley et al. 1997).
% \index{wind!measurement of|)}  
\end{paragraph}
\end{section}

\begin{section}{Изучение состояния поверхности с помощью численных моделей погоды}
% \section{Calculations of Wind}
Измерения параметров ветра поступают в различное время из различных
точек земного шара со спутников, судов, буев. Для того, чтобы
использовать эти наблюдения для расчета среднемесячных значений
характеристик ветра над морем, необходимо выполнить осреднение и
интерполяцию данных в узлы регулярной пространственной сетки. Еще
менее полезными данные наблюдений окажутся, если их использовать в
численных моделях океанических течений. Вы сталкиваетесь с очень
распространенной проблемой: как учесть все наблюдения сделанные в один
день и определить ветровые характеристики над океаном, скажем, при
фиксированной сетке разбиения?
%
% \index{numerical models!numerical weather models}Satellites, ships, and buoys
% measure winds at various locations and times of the day. If you wish to use
% the observations to calculate monthly averaged winds over the sea, then the
% observations can be averaged and gridded. If you wish to use wind data in
% numerical models of the ocean's currents, then the data will be less useful.
% You are faced with a very common problem: How to take all observations made
% in a six-hour period and determine the winds over the ocean on a fixed grid?

Самым лучшим источником данных о ветре в узлах регулярной сетки
являются выходные данные численных погодных моделей. Используемый
метод получения таких данных называется методом последовательных
приближений или задачей усвоения данных. «Измерения используются для
составления начальных условий для модели, затем осуществляется
интегрирование по времени до определенного момента в будущем, когда
будут доступны данные новых наблюдений. Тогда в модель вводятся
обновленные начальные данные (Bennett, 1992: 67)».
%
% One source of gridded winds over the ocean is the
% \textit{surface analysis}\index{surface analysis|textbf} calculated by
% numerical weather models\index{wind!from numerical weather models}.
% The strategy used to produce the six-hourly gridded winds is called
% \textit{sequential estimation techniques}
% \index{sequential estimation techniques|textbf}or
% \textit{data assimilation}\index{data assimilation|textbf}. ``Measurements
% are used to prepare initial conditions for the model, which is then
% integrated forward in time until further measurements are available.
% The model is thereupon re-initialized'' (Bennett, 1992: 67). The initial
% condition is called the \textit{analysis}.

Обычно используются все доступные данные наблюдений, включая данные с
наземных метеостанций, данные о давлении и температуре, переданные с
судов и буев, данные о ветре со скаттерометров космического
базирования и данные с метеоспутников. Модель интерполирует данные
измерений, чтобы создать начальные условия, согласующиеся с
предыдущими и текущими наблюдениями. Дэлей (Daley, 1991) достаточно
подробно описывает данный метод.
%
% Usually, all available measurements are used in the analysis, including
% observations from weather stations on land, pressure and temperature
% reported by ships and buoys, winds from
% scatterometers\index{scatterometers}\index{wind!from scatterometers}
% in space, and data from meteorological satellites. The model interpolates
% the measurements to produce an analysis consistent with previous and present
% observations. Daley (1991) describes the techniques in considerable detail.
%\end{paragraph}
%
%\paragraph{Surface Analysis from Numerical Weather Models}
Возможно, наиболее широко распространена модель погоды, используемая
Европейским Центром Среднесрочных Прогнозов (ECMWF). Она рассчитывает
параметры ветра у поверхности и потоки тепла (см. часть 5) каждые
шесть часов на сетке 1° х 1° с использованием явной модели
пограничного слоя. Расчетные значения сохраняются затем в узлах сетки
с разрешением 2.5° .
%
% Perhaps the most widely used weather model is that run by the European Centre
% for Medium-range Weather Forecasts \textsc{ecmwf}. It calculates a surface
% analysis\index{surface analysis}, including surface winds and heat
% fluxes\index{heat flux} (see Chapter 5) every six hours on a
% 1\degrees\ $ \times $ 1\degrees\ grid from an explicit boundary-layer model.
% Calculated values are archived on a 2.5\degrees grid. Thus the wind maps from
% the numerical weather models lack the detail seen in maps from scatterometer
% data, which have a 1/4\degrees\  grid.

Параметры ветра, рассчитываемые в ECMWF, имеют относительно высокую
точность. По оценкам (Freilich and Dunbar, 1999) точность расчета
скорости ветра на высоте 10~метров составляет около~$\pm 1,5$~м/с, для
направления~$\pm 18^\circ$.
%
% \textsc{ecmwf} calculations of winds have relatively good
% accuracy\index{accuracy!winds!calculated}. Freilich and Dunbar (1999)
% estimated that the accuracy for wind speed at 10 meters is
% $\pm 1.5$ m/s, and $\pm 18$\degrees\ for direction. 

Точность для южного полушария возможно такая же как и для северного,
так как континенты южного полушария, в связи с меньшей площадью, не
так сильно искажают перенос воздуха, как в северном полушарии. Кроме
того, скаттерометры дают точные данные о расположении штормов и
фронтов над океаном.
%
% Accuracy in the southern hemisphere is probably as good as in the northern
% hemisphere because continents do not disrupt the flow as much as in the
% northern hemisphere, and because scatterometers\index{scatterometers} give
% accurate positions of storms and fronts over the ocean.
%
% The \textsc{noaa} National Centers for Environmental Prediction and
% the US Navy also produces global analyses and forecasts every six hours.

Other surface-analysis values used in oceanography include: 1) output
from the numerical weather model run by the NOAA National Centers for
Environmental Prediction, 2) the Planetary Boundary-Layer Data set
produced by the U. S. Navy's Fleet Numerical Oceanography Center FNOC
; and 3) surface wind maps for the tropics produced at Florida State
University (Goldenberg and O' Brien?, 1981).

Другие аналитически рассчитываемые данные, используемые в
океанографии, включают: 1)выходные данные численных погодных моделей,
используемых в NOAA National Centers for Environmental Prediction, 2)
база данных по планетарному пограничному слою, разработанная
американском Центре численной океанографии военно-морского флота
(U. S. Navy's Fleet Numerical Oceanography Center, FNOC); 3) карты
ветра у поверхности океана для тропических широт, составленные в
государственном университете штата Флорида, США (Goldenberg and O'
Brien?, 1981).
%\end{paragraph}

\begin{paragraph}{Реанализ выходных данных численных моделей погоды}
% \paragraph{Reanalyzed Data from Numerical Weather Models}
Выходные данные численных моделей атмосферной циркуляции доступны на
протяжении десятилетий. В течение этого времени модели постоянно
изменялись, так как метеорологи стремятся к увеличению точности
прогнозов. Таким образом, потоки, рассчитанные моделями, не постоянны
во времени. Их изменения могут оказаться больше чем межгодовая
изменчивость потоков (White, 1996). Чтобы свести к минимуму такие
ошибки, метеослужбы собрали все имеющиеся данные и провели их реанализ
с помощью лучших из имеющихся моделей с целью осуществления единого,
внутренне согласованного анализа состояния поверхности океана.
%
% \index{numerical models!numerical weather models!reanalysis from}
% \index{wind!from numerical weather models}Surface analyses of weather over
% some regions have been produced for more than a hundred years, and over
% the whole earth since about 1950. Surface analyses calculated by numerical
% models of the atmospheric circulation have been available for decades.
% Throughout this period, the methods for calculating surface analyses have
% constantly changed as meteorologists worked to make ever more accurate
% forecasts. Fluxes calculated from the analyses are therefore not consistent
% in time. The changes can be larger than the interannual variability of the
% fluxes (White, 1996). To minimize this problem, meteorological agencies have
% taken all archived weather data and reanalyzed them using the best numerical
% models to produce a uniform, internally-consistent, surface
% analysis\index{surface analysis}.

Базы данных реанализа используются теперь для изучения динамики океана
и атмосферы. Данные анализа поверхности используются для решения
задач, требующих современной, текущей информации. Например, если вы
проектируете какое-либо сооружение на шельфе, то вам скорее всего
понадобятся данные реанализа за десятилетия; если вы управляете
работой этого сооружения, то вы будете наблюдать за данными анализа
поверхности океана и прогнозами, даваемыми метеослужбами каждые шесть
часов.  
%
% The reanalyzed data are used to study oceanic and atmospheric processes
% in the past. Surface analyses\index{surface analysis} issued every six hours
% from weather agencies are used only for problems that require up-to-date
% information. For example, if you are designing an offshore structure, you
% will probably use decades of reanalyzed data. If you are operating an
% offshore structure, you will watch the surface analysis and forecasts put out
% every six hours by meteorological agencies.
\end{paragraph}

\begin{paragraph}{Источники данных реанализа.}
% \paragraph{Sources of Reanalyzed Data}
Данные реанализа потоков у поверхности предоставляются
метеорологическими центрами, занимающимися численными моделями
прогноза погоды.
%
% \index{numerical models!numerical weather models!sources of reanalyzed data}
% Reanalyzed surface flux data are available from national meteorological
% centers operating numerical weather prediction models.

\begin{enumerate}
\item
The Центр исследования окружающей среды США (NCEP), работающие
совместно с Национальным центром атмосферных исследований (NCAR),
создали базу данных (NCEP/NCAR реанализ), на основе перерасчета данных
по погоде за 40 лет с 1958 по 1998 год, используя свою прогностическую
численную модель версии от 25 января 1995 года. Период, подверженный
реанализу, охватитывает также время с 1948 по 1957 года; все текущие
данные также подвергаются реанализу с шестимесячной задержкой выхода
набора данных. При реанализе используются данные наблюдений с суши и с
моря, а также данные космического зондирования. Данные реанализа
предоставляются каждые 6 часов на сетке T62 в узлах глобальной сетки
размером 194х94 узла с пространственным разрешением 209 км и 24
уровнями по вертикали. Важные подразделы данных реанализа, в
частности, поверхностные потоки, доступны на компакт-дисках (Kalnay et
al., 1996; Kistler et al., 1999). Европейский центр среднесрочных
прогнозов погоды провел реанализ погоды за 17 лет, для периода
1979–1993 гг. используя практически те же данные о подстилающей
поверхности, судовые и спутниковые данные, что использовались и при
создании базы реанализа NCEP/NCAR.
%
% The U.S. National Centers for Environmental Predictions, working with the
% National Center for Atmospheric Research have produced the \textsc{ncep/ ncar}
% reanalysis based on 51 years of weather data from 1948 to 2005 using the
% 25 January 1995 version of their forecast model. The reanalysis period is
% being extended forward to include all date up to the present with about
% a three-day delay in producing data sets. The reanalysis uses surface
% and ship observations plus sounder data from satellites. Reanalysis products
% are available every six hours on a T62 grid having $192 \times 94$ grid
% points with a spatial resolution of 209 km and with 28 vertical levels.
% Important subsets of the reanalysis, including surface fluxes, are available
% on \textsc{cd--rom} (Kalnay et al. 1996; Kistler et al. 2000).

\item
Европейский центр проводит реанализ данных с целью покрыть 40-летний
период с 1957 по 1997 год. Пространственное разрешение будет 83 км;
временное разрешение будет 6 часов. При реанализе будет использована
большая часть доступной спутниковой информации, включая данные со
спутников ERS-1 и ERS-2, а также SSM/I. Анализ будет включать модель
океанских волн, которая будет рассчитывать высоту волн.
%
% The European Centre for Medium-range Weather Forecasts \textsc{ecmwf} has
% reanalyzed 45 years of weather data from September 1957 to August 2002
% (\textsc{era}-40) using their forecast model of 2001 (Uppala et al. 2005).
% The reanalysis uses mostly the same surface and ship data used by the
% \textsc{ncep/ncar} reanalysis plus data from the \textsc{ers}-1 and
% \textsc{ers}-2\index{ERS satellites} satellites and \textsc{ssm/i}.
% The \textsc{era}-40 full-resolution products are available every six hours
% on a N80 grid having $160 \times 320$ grid points with a spatial resolution
% of 1.125\degrees\ and with 60 vertical levels. The  \textsc{era}-40
% basic-resolution products are available every six hours with a spatial
% resolution of 2.5\degrees\ and with 23 vertical levels. The reanalysis
% includes an ocean-wave model that calculates ocean wave heights and wave
% spectra every six hours on a 1.5\degrees\ grid.


\item
Отдел усвоения данных Центра управления полетов НАСА (The Data
Assimilation Office at NASA's Goddard Space Flight Center) завершил
реанализ за период с 1 марта 1980 по 13 декабря 1993, позднее
продленный до февраля 1995. Анализируемые данные имеют временное
разрешение 6 ч., пространственное 2° х 2.5° (91 на 144 узла сетки), 20
уровней по вертикали. При анализе использовались данные, поступавшие в
режиме реального времени из Центра исследований окружающей среды США
(NCEP), а такжы данные TOVS от спутников Национальной администрации по
атмосфере и океану США (NOAA) и данные о ветре из наблюдений за
дрейфом облаков (Schubert, Rood, and Pfaendtner, 1993). Особое
внимание при анализе уделялось усвоению спутниковых данных с
использованием численной модели погоды Годдаровской системы наблюдений
(the Goddard Earth Observing System).
\end{enumerate}
\end{paragraph}
\end{section}

\begin{section}{Напряжение трения.}
% \section{Wind Stress}
% \index{wind stress|textbf}The wind, by itself, is usually not very
% interesting. Often we are much more interested in the force of the wind,
% or the work done by the wind. The horizontal force of the wind on the sea
% surface is called the \textit{wind stress}. Put another way, it is the
% vertical transfer of horizontal momentum. Thus momentum is transferred from
% the atmosphere to the ocean by wind stress.

% Wind stress $T$ is calculated from:
%
% \begin{equation}
% T = \rho_a \,C_D U_{10}^2
% \end{equation}
% where $\rho_a = 1.3$ kg/m$^3$ is the density of air, $U_{10}$ is wind speed
% at 10 meters, and $C_D$ is the
% \textit{drag coefficient}\index{drag!coefficient|textbf}.
% $C_D$ is measured using the techniques described in \S5.6. Fast response
% instruments measure wind fluctuations within 10--20 m of the sea surface,
% from which $T$ is directly calculated. The correlation of $T$ with $U_{10}^2$
% gives $C_D$ (figure 4.6).

% Various measurements of $C_D$ have been published based on careful
% measurements of turbulence\index{turbulence!measurement of} in the marine
% boundary layer. Trenberth et al. (1989) and Harrison (1989) discuss the
% accuracy\index{accuracy!drag coefficient} of an effective drag
% coefficient\index{drag!coefficient} relating wind
% stress\index{wind stress!and drag coefficient} to wind velocity on a global
% scale. Perhaps the best of the recently published values are those of
% Yelland and Taylor (1996) and Yelland et al. (1998) who give:

% \begin{subequations}
% \begin {align}
% 1000 \, C_D = & \,0.29 + \frac{3.1}{U_{10}} + \frac{7.7}{U_{10}^2}
%   & \left( 3 \le U_{10} \le 6 \text{ m/s}\right) \\
% 1000 \, C_D = & \,0.60 + 0.071 \, U_{10}
%   & \left( 6 \le U_{10} \le 26 \text{ m/s}\right)
% \end{align}
% \end{subequations}
% for neutrally stable boundary layer. Other values are listed in their 
% table 1 and in figure 4.6.
\end{section}

\begin{section}{Современные концепции.}
%\section{Important Concepts}
%
% \begin{enumerate}
% \item
% Sunlight is the primary energy source driving the atmosphere and ocean.
% \vitem
% There is a boundary layer at the bottom of the atmosphere where wind speed
% decreases with as the boundary is approached, and in which fluxes of heat and
% momentum are constant in the lower 10--20 meters.
% \vitem
% Wind is measured many different ways. The most common until 1995 was from
% observations made at sea of the Beaufort force\index{wind!Beaufort scale}
% of the wind.
%\vitem
% Since 1995, the most important source of wind measurements is from
% scatterometers\index{scatterometers}\index{wind!from scatterometers} on
% satellites. They produce daily global maps with 25 km resolution. 
% \vitem
% The surface analysis from numerical models of the
% atmosphere\index{wind!from numerical weather models} is the most useful
% source of global, gridded maps of wind velocity for dates before 1995. It
% also is a useful source for 6-hourly maps. Resolution is 100-250 km.
% \vitem
% The flux of momentum from the atmosphere to the ocean, the wind
% stress\index{wind stress}, is calculated from wind speed using a drag
% coefficient\index{drag!coefficient}.
%\end{enumerate}
\end{section}
\end{chapter}
