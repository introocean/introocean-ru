% -*- coding: utf-8 -*-

\begin{chapter}{Немного истории}
Наши знания о существовании океанских течений, ветров и приливов
насчитывают тысячи лет. Полинезийские мореплаватели совершали торговые
путешествия на большие расстояния в Тихом океане уже в 4000 до нашей
эры (Service, 1996). Пифейцы в 325 году до н. э. исследовали Атлантику
от Италии до Норвегии. Арабские торговцы в Средние века использовали
свои знания о ветрах и течениях в Индийском океане для того чтобы
установить торговые связи с Китаем, а позже с Занзибаром на
Африканском побережье. Связь между приливами и луной с солнцем была
описана в Самоведе, документе индийской Ведической религии, между 2000
и 1400 годами д.н.э. (Pugh, 1987). Современные океанографы, которые
доверяют только тому что измерено инструментами, могли бы многому
поучиться у тех кто зарабатывал себе на жизнь в океане.

Современные европейские знания об океане начинаются с
исследовательских экспедиций Бартоломео Диаша (1487--1488), Христофора
Колумба (1492--1494), Васко де Гама (1497--1499), Фернанда
(Фердинанда) Магелана (1519--1522), и многих других. Они заложили
основы для появления в начале 16 века глобальных торговых маршрутов,
протянувшихся от Испании до Филиппин. Эти маршруты были основаны на
хороших знаниях о пассатах, западных ветрах и западных прибрежных
течениях в Атлантическом и Тихом океанах (Couper, 1983: 192--193).

За первыми европейскими исследовательским экспедициями вскоре
последовали научные экспедиции которыми руководили (среди многих
других) Джеймс Кук на кораблях Endeavour, Resolution, и Adventure,
Чарльз Дарвин (1809--1882) на Бигле, сэр Джеймс Кларк Росс и сер
Джон Росс которые проводили исследования в Арктике и Антрктике с
кораблей Викторя, Изабелла и Эребус, а также Эрдвард Форбс
(1815--1854) который изучал вертикальное распределение жизни в
океанах. Другие собирали вместе наблюдения и строили различные карты,
так Эдмонд Халлей картировал пассаты и муссоны, а Бенджамин Франклин
нанёс на карту Гольфстрим.

Медленные корабли 19 и 20 веков уступили в конце 20 века дорогу
спутникам. Сейчас спутники исследуют океаны, атмосферу и сушу. Их
данные применяемые в численных моделях, позволяют изучать Землю как
единую систему. Впервые мы можем изучать как биологические, химические
и физические системы взаимодействуют между собой и влияют на
окружающую среду.

%% http://oceanworld.tamu.edu/resources/ocng_textbook/chapter02/Images/Fig2-1s.jpg
%% Рисунок 2.1 Пример из периода глубоководных исследований: Путь судна
%% Челленджер во время British Challenger Expedition 1872- 1876 ( взято
%% из Wurst, 1964)

\begin{section}{Определения}
Долгая история изучения океана привела к появлению различных
специализированных дисциплин, каждая из которых обладает своими
собственными интересами и набором терминов. Наиболее важными
дисциплинами являются:

\begin{description}
\item[Океанография] занимается изучением океана как среды. Целью этой
науки является получение количественного описания океана достаточного
для того чтобы с некоторой достоверностью предсказывать будущее его
состояние.

\item[Геофизика] изучает физику Земли.

\item[Физическая океанография] изучает физические характеристики и
динамику океанов. Основными интересами этой науки являются
взаимодействие океана с атмосферой, тепловой баланс океана,
формирование водных масс, течения и процессы в прибрежных
областях. Многими Физическая Океанография рассматривается как раздел
геофизики.

\item[Геофизическая гидромеханика] изучает динамику движений жидкости
в масштабах в которых ощущается влияние вращения Земли. Метеорология и
океанография используют геофизическую гидромеханику для расчёта полей
планетарных течений.

\item[Гидрография] занимается составлением морских карт, таких как
карты глубины океана, течений, полей плотности в океане и приливов.
\end{description}
\end{section}

\begin{section}{Периоды исследований океана}
Исследование океана можно условно разделить на различные периоды 
(Wust 1964). Я расширил это разделение до конца 20 века.

\begin{enumerate}
\item  
Период поверхностной океанографии: с древнейших времён до 1873. Этот
период характеризуется систематизацией наблюдений за ветрами,
течениями, волнами, температурой и другими явлениями поддающимися
наблюдению с палубы корабля. Известными примерами являются карты
пассатов составленные Халлеем, карта Гольфстрима составленная
Франклином и книга Матью Фонтайна Маури «Физическая география моря»
(Matthew Fontaine Maury’s Physical Geography for the Sea.)

\item
Период глубоководных исследований: 1873--1914. Характеризуется
различными по значимости океанографическими экспедициями направленными
на выяснение поверхностных и глубинных характеристик океана возле
колониальных земель. Основным примером здесь является экспедиция
Челленджера (рисунок 2.1), но можно назвать также экспедиции Газели и
Фрама.

\item
Период национальных систематических исследований: 1925--1940.
Характеризуется детальным исследованием колониальных областей. Как
примеры можно привести изучение Атлантики Метеором (рисунок 2.2) и
экспедицию Дискавери.

\item
Период новых методов: 1947--1956. Характеризуется длительными
исследованиями с использованием новых инструментов (рисунок 2.3). Как
пример можно привести сейсмическое изучение Атлантики с судна Вема, в
результате которого Хеезеном (Heezen) были составлены карты морского
дна.

\item
Период международного взаимодействия: 1957--1978. Характеризуется
многонациональными исследованиями океанов и процессов в них
происходящих. Примеры: Программа Атлантический Полярный Фронт
(Atlantic Polar Front Program), рейсы NORPAC, рейсы в ходе
Международного Геофизического Года и Международной Декады Изучения
Океана (рисунок 2.4), исследования с одновременным участием нескольких
десятков кораблей --- эксперименты MODE, POLYMODE, NORPAX и JASIN.

\item
Эра спутников: 1978--1995. Характеризуется глобальными
исследованиями процессов в океане из космоса. Примеры: Seasat, 
NOAA 6--10, NIMBUS-7, Geosat, Topex/Poseidon и ERS-1, ERS-2.

\item
Эра изучения Земли как системы: 1995--. Характеризуется глобальным
изучением взаимодействия биологических, химических и физических
процессов в океане, атмосфере и на суше с использованием данных
полученных in situ (на месте) и из космоса в численных моделях. Для
океана примерами являются World Ocean Circulation Experiment (WOCE)
(Рисунок 2.5) и Topex/Poseidon (рисунок 2.6), Sea WiFS и Join Global
Ocean Flux Study (JGOFS).
\end{enumerate}

%% Рисунок 2.2 Пример из эры национальных систематических
%% исследований. Путь НИС Метеор во время German Meteor
%% Expedition. (Взято из Wust, 1964).
\end{section}

\begin{section}{Вехи в понимании океана}
Что все эти программы и экспедиции говорят нам об океане? Давайте
взглянем на развитие наших знаний начиная с 17 века. В начале прогресс
был очень медленным. Первые наблюдения были сделаны учёными которые
думать не думали что они океанографы, если такой термин в те времена
вообще существовал. Позже появились более детальные описания
океанографических экспериментов сделанные учёными специализирующимися
именно на изучении океана. 

%% Рисунок 2.3 Пример из периода новых методов. Экспедиция судна Атлантис
%% принадлежащего Океанографическому Институту в Вудсхоле (взято из
%% Wurst,1964).

%% Рисунок 2.4 Пример из периода международного взаимодействия. Измерения
%% проведённые в ходе Атлантической программы Международного
%% геофизического года 1957--1959 (взято из Wurst, 1969).

%% Рисунок 2.5 Эксперимент по исследованию циркуляции мирового океана
%% (WOCE). Профили проделанные одновременно в ходе этого глобального
%% исследования океана многими научно исследовательскими судами.

%% Рисунок 2.6 Пример из периода спутников. Треки спутника Topex /
%% Poseidon над Тихим океаном за 10 дней.

\begin{description}
\item[1685] Эдмонд Халлей (Edmond Halley) в результате изучения океанской
системы ветров и течений опубликовал в журнале Философские труды
(Philosophical Transactions) за номером 16 страницы 153--169 работу
«Историческая оценка пассатов и муссонов наблюдаемых в морях между и
вблизи тропиков и попытка установить физическую причину возникновения
названных ветров» («An Historical Account of the Trade Winds, and
Monsoons, observable in the Seas between and near the Tropicks, with
an attempt to assign the Physical cause of the said Winds»)

\item[1735] Джорж Хадли (George Hadley) опубликовал свою теорию возникновения
пассатов основанную на сохранении углового момента в тех же
Философских трудах (Philosophical Transactions, 39: 58–62.), статья
«Обсуждение причин возникновения пассатов» («Concerning the Cause of
the General Trade-Winds»).

%% Рисунок 2.7 Вариант карты Гольфстрима Франклина и Фолгера.

\item[1751] Генри Эллис (Henry Ellis) провёл в районе тропиков первое
измерение температуры на глубине и обнаружил под тёплым поверхностным
слоем холодные воды что указывало на то что поступили они из полярных
районов.

\item[1769] Бенджамин Франклин во время работы почтмейстером,
используя информацию о маршрутах кораблей курсирующих между Англией и
Новой Англией, собранную его кузеном Тимоти Фолгером (Timothy Folger),
создал первую карту Гольфстрима.

\item[1775] Лаплас публикует свою теорию приливов.

\item[1800] Коунт Румфорд (Count Rumford) предлагает вариант
меридиональной циркуляции океана в которой вода опускается на глубину
возле полюсов и поднимается на поверхность возле Экватора.

\item[1847] Matthew Fontain Maury публикует первую карту ветров и
течений основанную на судовых записях. Маури предложил практику
международного обмена данными об окружающей среде взятыми из судовых
журналов, для составления на их основе различных карт.

\item[1872--1876] Экспедиция Челленджера, которая ознаменовала начало
систематического изучения биологии, химии и физики океанов.

\item[1885] Pillsbury произвёл прямые измерения Флоридского течения с
заякоренного корабля.

\item[1910--1913] Вильгельм Бъеркнес (Vilhelm Bjerknes) опубликовал
книгу «Динамическая Метеорология и Гидрография» (Dynamic Meteorology
and Hydrography) заложившую основы геофизической гидродинамики. В ней
он развивает идеи фронтов, dynamic meter, геострофических течений,
взаимодействия океана и атмосферы, циклонов.

\item[1912] Основание Морской Биологической лаборатории в
Калифорнийском университете. Позднее она стала Институтом океанографии
имени Скрипса.

\item[1930] Основание Океанографического Института в Вудсхоле.

%% Рисунок 2.8 Осреднённая по времени поверхностная циркуляция океана
%% построенная на основе данных полученных приблизительно за столетие
%% океанографических экспедиций (взято из Tolmazin, 1985).

\item[1942] Публикация Свердрупом, Джонсоном и Флемингом труда
«Океаны» («The Oceans»), первого всеобъемлющего обзора
океанографических знаний.

\item[После 2 МВ] Создание кафедр океанографии в различных
университетах, включая Орегонский Университет, Техасский Унверситет,
Университет Майами, Университет Род Айланда, а также создание в разных
странах Океанографических институтов и лабораторий.

\item[1947--1950] Свердруп, Стоммел и Манк публикуют свои теории
ветровой циркуляции океана. Вместе эти три работы заложили основы
нашего понимания океанской циркуляции.

\item[1949] Начало изучения Калифорнийского течения в рамках программы
California Cooperative Fisheries Investigation of the California
Current. Самое детальное из когда либо проводившихся исследование
прибрежного течения.

\item[1952] Кромвел и Монтгомери открывают экваториальное
противотечение в Тихом океане.

\item[1955] Брюс Хамон и Нейл Браун разрабатывают CTD зонд для
измерений электропроводности и температуры как функции глубины.

\item[1958] Стоммел публикует свою теорию глубинной циркуляции океана.

\item[1963] Корпорация Сиппикан (Sippican Corporation (Tim Francis,
William Van Allen Clark, Graham Campbell, and Sam Francis)) изобретает
невозвратный батитермограф XBT который в настоящее время является
наверное самым широко используемым океанографическим прибором в мире.

\item[1969] Кирк Брайан и Михаэль Кокс разрабатывают первую численную
модель океанской циркуляции

\item[1978] NASA запускает первый океанографический спутник Seasat. В
ходе этого проекта были разработаны приёмы которые после
использовались поколениями спутников дистанционного зондирования.

\item[1979--1981] Terry Joyce, Rob Pinkel, Lloyd Regier, F. Rowe and
J. W. Young занимаются разработками приведшими в итоге к созданию
доплеровского измерителя течений для измерения поверхностных течений с
движущихся судов, инструмента который широко используется в
океанографии.

\item[1988] NASA Earth System Science Committee возглавляемый
Францисом Бретертоном показал в общих чертах как все системы Земли
связаны между собой, таким образом были разрушены барьеры разделяющие
традиционную астрофизику, экологию, геологию, метеорологию и
океанографию.

\item[1992] Рус Дэвис и Даг Веб изобретают автономный погружающийся
буй который способен постоянно измерять течения на глубине до 2х
километров.

\item[1992] Nasa и CNES разрабатывают и запускают спутник Topex /
Poseidon который картирует океанские поверхностные течения, волны и
приливы каждые 10 дней.

\item[1997] Уолли Брокер предполагает что изменения в глубинной
циркуляции океанов регулируют наступление ледниковых периодов и что
глубинная циркуляция в Атлантике может быть нарушена в результате чего
северное полушарие погрузится в новый ледниковый период.
\end{description}

Более полную информацию об истории физической океанографии вы можете
найти в Приложении А книги фон Аркса (W.S. von Arx) «An Introduction
to Physical Oceanography.»

Данные накопленные в течении веков океанских экспедиций были
использованы для того чтобы описать океан. Большинство работ было
посвящено описанию устойчивого состояния океана, его течениям как
поверхностным так и глубинным, и его взаимодействию с
атмосферой. Основное описание было закончено к началу 70х. Рисунок 2.8
демонстрирует пример из того времени, он изображает поверхностную
циркуляцию океана. Более поздние работы пытались описать динамические
процессы в океане для того чтобы научиться предсказывать его годовую и
межгодовую изменчивость, а также понять роль океана в глобальных
процессах.
\end{section}

\begin{section}{Эволюция некоторых теоретических представлений.}
Теоретическое понимание океанических процессов основано на
классической физике, соединённой с всё более развивающимися
представлениями о хаотических системах в математике и применениями их
к теории турбулентности. Даты приведенные ниже являются
приблизительными.

\begin{description}
\item[19й век] Разработка аналитической гидродинамики. Кульминацией
этой работы была книга Ламба (lamb) Hydrodynamics. Бъеркнес
разрабатывает геострофический метод, широко используемый в
метеорологии и океанографии.

\item[1925--40] Разработка теорий турбулентности основанных на идеях
аэродинамики и длинны перемешивания. Работы Прандтля и фон Кармана.

\item[1940--1970] Усовершенствование теорий турбулентности на базе
статистических корреляций и идее изотропического постоянства
турбулентности. Книги Бачелора (Bachelor(1967)), Хинза (Hinz (1975)) и
других.

\item[1970--] Численные исследования турбулентной геофизической
гидродинамики на высокоскоростных компьютерах.

\item[1985--] Механика хаотических процессов. Применение к
гидродинамике это только начало. Большинство движение в атмосфере и в
океане по существу могут быть непредсказуемыми.
\end{description}
\end{section}

\begin{section}{Роль наблюдений в океанографии.}
На основе этого небольшого обзора теоретических идей в океанологии
можно предположить что наблюдения очень важны для понимания
океанов. The theory describing a convecting, wind-forced, turbulent
.uid in a rotating coordinate system has never been su.ciently well
known that important features of the oceanic circulation could be
predicted before they were observed. Почти всегда для понимания
океанических процессов учёные обращаются к наблюдениям.

Мы можем подумать что многочисленные экспедиции проведённые с 1873
года должны дать хорошее описание океанов. Их результаты действительно
впечатляющи. Сотни экспедиций были проведены во всех океанах. Несмотря
на это большая часть океана мало исследована.

К 2000 году большинство районов океана исследовалось от поверхности до
дна только один раз. Некоторые районы, такие как Атлантика,
исследовались таким образом трижды: во время международного
геофизического года (1959), во время Geochemical Sections cruises в
начале 70х, и во время World Ocean Circulation Experiment с 1991 по
1996 годы. All areas will be under sampled. Это проблема выборки
(смотри ящик). Наших измерений океана недостаточно для того чтобы
предсказывать его изменчивость и реакцию на различные внешние
силы. Отсутствие репрезентативных наблюдений~--- наибольший источник
ошибок в нашем понимании океана.

Выбор массива океанологических данных. Большинство существующих
океанологических данных организованы в большие массивы
данных. Например спутниковые данные обрабатываются и распространяются
группами работающими вместе с NASA. Данные с судов и собираются и
классифицируются другими группами. Теперь океанографы в своей работе
всё больше и больше полагаются на данные собранные другими.

Использование данных полученных другими вызывает следующие вопросы:
\begin{enumerate}
   \item Насколько точны эти данные?
   \item Каковы ограничения этого набора данных?
   \item Возможно ли сравнить этот набор данных с другими?
\end{enumerate}

Человеку работающему с публичными или закрытыми наборами данных
необходимо знать ответы на эти вопросы.

Если вы собираетесь использовать данные полученные другими, то вот
несколько основополагающих принципов которыми вы должны
руководствоваться:

\begin{enumerate}
\item
Используйте хорошо документированные наборы данных. Полностью ли
документация описывает источники измерений, шаги которым следовали при
обработке данных и критерии используемые для отбрасывания неверных
значений? Включает ли набор данных номер версии для того чтобы
прослеживать изменения в этом наборе.

\item
Пользуйтесь проверенными (валидированными) данными. Хорошо ли
задокументирована точность данных? Определялась ли точность исходя из
сравнения с другими измерениями той же переменной? Была валидация
глобальной или региональной?

\item
Используйте данные которые уже были использованы другими и на которые
ссылаются в научных статьях. Некоторые наборы данных широко
используются. Те кто получил данные используют их в своих публикациях
и другие учёные им доверяют.

\item 
И наоборот не используйте данные только потому что они легко
доступны. Знаете ли вы источник данных? Например сейчас доступно много
версий электронных карт морского дна на 5 мильной сетке. Некоторые из
них основаны на первых данных полученных U.S Defense Mapping Agency, а
другие на данных полученных со спутника ETOPO-5. Не полагайтесь на
мнение коллег об источнике данных. Найдите документацию. Если
документации нет, ищите другие данные.
\end{enumerate}
\end{section}

\begin{section}{Ошибки измерений}
Ошибки измерений вызываются наборами данных не репрезентативных по
отношению к генеральной совокупности измеряемой
переменной. Генеральная совокупность это набор всех возможных
измерений, а наши измерения это выборка из генеральной
совокупности. Мы предполагаем что каждое измерение сделано с
абсолютной точностью.

Чтобы определить что в ваших измерениях присутствует ошибка, вы должны
полностью определить проблему которую хотите исследовать. Это
определяет генеральную совокупность. Затем вы должны определить
представляют ли измерения генеральную совокупность. Все эти шаги
необходимы.

Допустим вы хотите измерить среднегодовую температуру поверхности
океана для того чтобы посмотреть идёт ли глобальное потепление. Для
этой проблемы генеральной совокупностью являются все возможные
измерения поверхностной температуры во всех регионах и во все
месяцы. Для того чтобы среднее измерений и реальное среднее совпадали,
измерения должны быть однородно распределены на протяжении года и над
всеми районами океана, также они должны быть достаточно плотными для
того чтобы включать в себя все важные процессы изменчивости в
пространстве и во времени. Это невозможно. Корабли обходят районы
штормов, такие как высокие широты зимой, таким образом корабельные
измерения не могут представлять генеральную совокупность поверхностных
температур. Спутники не могут однородно измерять поверхностную
температуру на протяжении дневного цикла, а спутниковым наблюдениям за
температурой в высоких широтах зимой мешают постоянные облака, тем не
менее в большинстве регионов они обеспечивают измерения однородные по
пространству на протяжении года. Если дневная изменчивость мала,
спутниковые данные будут более репрезентативными чем данные с судов.

Исходя из вышесказанного что океанологические наблюдения редко
представляют собой генеральную совокупность переменной которую мы
хотим изучать. У нас всегда есть ошибка измерений.

Определяя ошибку измерений мы должны чётко для себя разделять ошибку
измерений и инструментальную ошибку. Инструментальная ошибка
происходит из за неточности инструмента. Ошибка измерений происходит
из за неспособности сделать измерения. Рассмотрим пример приведённый
выше: определение средней температуры на поверхности. Если измерения
производятся с судов с помощью термометров, каждое измерение обладает
небольшой ошибкой, поскольку термометры не идеальны. Это
инструментальная ошибка. А если судно зимой не заходит в высокие
широты, то отсутствие измерений в высоких широтах зимой~--- ошибка
измерений.

Метеорологи разрабатывавшие Tropical Rainfall Mapping Mission нашли
ошибку измерений в измерениях дождя. Их результаты являются общими и
могут быть применены и к другим переменным. Интересующимся этой
проблемой можно посоветовать обратиться к North \& Nakamoto (1989).

Планирование эксперимента Наблюдения очень важны для океанографии, но
они дороги, так как корабельное время дорого и спутники тоже
удовольствие не из дешёвых. Поэтому океанографический эксперимент
должен быть хорошо спланирован. Рассказ о планировании эксперимента не
совсем уместен в главе об истории, но возможно эта тема заслуживает
нескольких коротких замечаний так как она нечасто упоминается в книгах
по океанографии, но ей уделяется много внимания в текстах посвящённым
другим наукам. Планирование эксперимента чрезвычайно важно, поскольку
неправильно спланированный эксперимент приводит к сомнительным
результатом, в ходе него могут измеряться не те переменные или вообще
получаться бесполезные данные.

Первый и наиболее важный аспект в планировании любого эксперимента это
перед тем как вы решите что и как вы будете измерять понять зачем вы
хотите проводить данные измерения.

\begin{enumerate}
\item 
Какова цель наблюдений? Вы хотите проверить гипотезу или описать
процесс?

\item 
Какая точность должна быть у измерений?

\item
Какое пространственное и временное разрешение необходимо? Какова
продолжительность измерений?
\end{enumerate}

Рассмотрим, например, как цель измерений будет изменять способ которым
вы должны проводить измерения температуры и солёности.
\begin{enumerate}
\item
Если например в нашу задачу входит описание водных масс в каком ни
будь бассейне, тогда раз в 20--50 лет необходимо проводить измерения
с вертикальным разрешением 20--50 метров и горизонтальным
разрешением 50--300 км.

\item
Если целью является описание вертикального перемешивания в океане,
тогда необходимо проводить измерения с вертикальным разрешением
0,5--1,0 мм и расстоянием между станциями наблюдений 50--1000 км,
каждый час в течении многих дней.
\end{enumerate}

Точность, определённость и линейность Если уж мы говорим об
экспериментах, то сейчас самое время представить три концепции которые
понадобятся нам на протяжении всей книги когда мы будем говорить об
экспериментах: определённость, точность и линейность измерений.

Точность это разница между измеренным и истинным значением.

Определённость это разница между повторяющимися измерениями.

Разницу между точностью и определённостью обычно иллюстрируют на
простом примере стрельбы из винтовки по мишени. Точность это среднее
расстояние между центром мишени и местом попадания. Определённость это
среднее расстояние между попаданиями. Таким образом десять попаданий
могут быть сгруппированы внутри круга с диаметром 10 см с центром
отстоящим от центра мишени на 20 см. Тогда точность будет равняться 20
см, а определённость 5 см.

Линейность требует того чтобы то что выдаёт инструмент было линейной
функцией того что он измеряет. Нелинейные инструменты подстраивают
изменчивость к постоянному значению. Значит нелинейная реакция
приводит к неверным средним значениям. Нелинейность может быть также
важна как и точность. Например пусть
\begin{eqnarray}
\mbox{Выход} & = & \mbox{вход} +0.1 (\mbox{Вход})^2 \\
\mbox{Вход}  & = & a \sin \omega t
\end{eqnarray}
Тогда
\begin{eqnarray}
\mbox{Выход} & = & a \sin \omega t + 0.1 (a \sin \omega t)^2 \\
\mbox{Выход} & = & a \sin \omega t + 1/2 a^2 - 1/2 a^2 \cos 2\omega t
\end{eqnarray}
Обратите внимание на то что среднее значение входа~--- ноль, в то
время как выход этого нелинейного инструмента имеет среднее значение
$0,05a2$ плюс такой же член умноженный на косинус с удвоенной
частотой. В основном если вход обладает частотами $w1$ и $w2$, то выход
нелинейного инструмента имеет частоты $w1\pm w2$. Линейность особенно важна
когда инструмент должен измерять среднее значение турбулентной
переменной. Например когда мы измеряем течения на небольшой глубине, у
поверхности, где ветра и волны вызывают большую изменчивость течений,
нам необходимы «линейные» измерители течения.

Чувствительность к другим переменным Ошибки могут быть связаны с
влиянием других переменных. Например измерения электропроводности
чувствительны к температуре. Таким образом ошибки при измерении
температуры в солемере приводят к ошибкам в измеренных значениях
электропроводности и солёности.
\end{section}

\begin{section}{Важные концепции.}
Я надеюсь из сказанного выше вы поняли что:
\begin{enumerate}
\item
Океаны не очень хорошо изучены. Всё что мы знаем основано на
информации собранной за период океанографических экспедиций
насчитывающий чуть больше века с 1978 года дополненной данными
спутников.

\item
К настоящему моменту наших знаний об океане достаточно для того чтобы
описать его циркуляцию осреднённую по времени, современные работы уже
начинают описывать изменчивость.

\item
Наблюдения важны для понимания океана. Некоторые процессы были
предсказаны теоретически до того как наблюдались.

\item
Океанографы всё больше и больше полагаются на наборы данных полученных
другими. Эти данные обладают ошибками и ограничениями которые вы
должны знать и понимать перед их использованием.

\item
Планирование эксперимента по меньшей мере так же важно как его
проведение.

\item
Ошибки измерений появляются тогда когда наблюдения не отображают
изучаемый процесс. Эти ошибки наибольший источник ошибок в
океанографии.
\end{enumerate}
\end{section}

\end{chapter}
