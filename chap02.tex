% -*- coding: utf-8 -*-

\begin{chapter}{Немного истории}
Наши знания о существовании океанских течений, ветров и приливов
насчитывают тысячи лет. Полинезийские мореплаватели совершали торговые
путешествия на большие расстояния в Тихом океане уже 
в 4000~до~н.~э.\ (Service, 1996). Пифей в 325~до~н.~э.\ исследовал Атлантику
от Италии до Норвегии. Арабские торговцы в Средние века использовали
свои знания о ветрах и течениях в Индийском океане для того, чтобы
установить торговые отношения с Китаем, а позже~--- с Занзибаром на
побережье Африки. Связь между Солнцем, Луной и приливами была
описана в Сама-веде, одном из священных писаний индуизма ведического 
периода (2000--1400~до~н.~э.) (Pugh, 1987). Некоторые океанографы, 
доверяющие только инструментальным измерениям, могли бы многому
поучиться у тех, кто зарабатывал себе на жизнь в океане.

Современные европейские знания об океане начинаются с
исследовательских экспедиций Бартоломеу Диаша (1487--1488), Христофора
Колумба (1492--1494), Васко да Гама (1497--1499), Фернана
Магеллана (1519--1522) и многих других. Они заложили
основы для появления в начале XVI~века глобальных торговых маршрутов,
протянувшихся от Испании до Филиппин. Эти маршруты были основаны на
хороших знаниях о пассатах, западных ветрах и западных прибрежных
течениях в Атлантическом и Тихом океанах (Couper, 1983: 192--193).

За первыми европейскими исследовательским экспедициями вскоре
последовали научные, которыми руководили, в частности,
Джеймс Кук (1728--1779) на кораблях <<Индевор>>, <<Резольюшен>> 
и~<<Эдвенчур>>,
%% англ. имена кораблей --- в предм. указатель 
%% Endeavour, Resolution, Adventure
%% википедия:Резолюшн, БСЭ: Резольюшен
%% http://slovari.yandex.ru/dict/bse/article/00039/73900.htm
%% http://ru.wikipedia.org/wiki/%D0%9A%D1%83%D0%BA_%D0%94%D0%B6%D0%B5%D0%B9%D0%BC%D1%81
%% имена людей --- тоже. Даты жизни --- тоже в указатель, чтобы не 
%% загромождать текст? И нужны ли они вообще (кому надо --- найдут)?
%% Нестыковка: в предыдущем абзаце даны даты путешествий, а в этом --- жизни?
Чарльз Дарвин (1809--1882) на <<Бигле>>, сэр Джеймс Кларк Росс (1800--1862)
%% Beagle
и сэр Джон Росс (1777--1856), которые проводили исследования в Арктике 
и Антарктике с кораблей <<Виктори>>, <<Изабелла>> и~<<Эребус>>, а также 
%% Victory, Isabella, Erebus
%% транслитерация "Виктори" не подтверждена источниками
Эдвард Форбс (1815--1854), изучавший вертикальное распределение жизни 
в~океанах. Другие обобщали наблюдения и строили на их основе различные карты,
так, Эдмунд Галлей (1656--1742) картировал пассаты и~муссоны, 
а Бенджамин Франклин (1706--1790) нанёс на карту Гольфстрим.

Медленные корабли XIX и~XX~веков уступили в конце XX~века дорогу
спутникам, дрейфующим буям и другим автоматическим приборам. 
Сейчас спутники исследуют океаны, атмосферу и сушу. Тысячи дрейфующих буев
ведут наблюдения на глубинах до двух километров. Полученные с их помощью
данные обрабатываются при помощи численных моделей и позволяют изучать Землю как
единую систему. Впервые в истории науки мы получили возможность узнать, 
как биологические, химические и физические системы взаимодействуют между 
собой и влияют на окружающую среду.

%% http://oceanworld.tamu.edu/resources/ocng_textbook/chapter02/Images/Fig2-1s.jpg
%% Рисунок 2.1 Пример из периода глубоководных исследований: Путь судна
%% Челленджер во время British Challenger Expedition 1872- 1876 ( взято
%% из Wurst, 1964)

\begin{section}{Определения}
Долгая история изучения океана привела к появлению различных
специализированных дисциплин, каждая из которых обладает своими
собственными интересами и терминологией. Среди этих дисциплин наиболее важны
следующие:

\begin{description}
\item[Океанография] занимается изучением океана как среды. Целью этой
науки является получение количественного описания океана, достаточного
для того, чтобы с некоторой достоверностью предсказывать его будущее
состояние.

\item[Геофизика] изучает физику Земли.

\item[Физическая океанография] изучает физические характеристики и
динамику океанов. Основными интересами этой науки являются
взаимодействие океана с атмосферой, тепловой баланс океана,
формирование водных масс, течения и процессы в прибрежных
областях. Многими физическая океанография рассматривается как раздел
геофизики.

\item[Геофизическая гидродинамика] изучает динамику движения жидкости
%% не гидромеханика, а гидродинамика?
%% http://www.multitran.ru/c/m.exe?CL=1&l1=1&s=fluid+dynamics
%% http://ru.wikipedia.org/wiki/%D0%93%D0%B8%D0%B4%D1%80%D0%BE%D0%BC%D0%B5%D1%85%D0%B0%D0%BD%D0%B8%D0%BA%D0%B0
%% т.к. речь идет о динамике --- то "гидродинамика"
в масштабах, в которых ощущается влияние вращения Земли. Метеорология и
океанография используют геофизическую гидродинамику для расчёта полей
планетарных течений.

\item[Гидрография] занимается составлением морских карт, таких как
карты глубины океана, течений, полей плотности в океане и приливов.

\item[Earth-system Science] изучает Землю как единую систему, в состав 
%% http://en.wikipedia.org/wiki/Earth_System_Science
%% http://ru.wikipedia.org/wiki/%D0%9D%D0%B0%D1%83%D0%BA%D0%B8_%D0%BE_%D0%97%D0%B5%D0%BC%D0%BB%D0%B5
%% согласно этим статьям, нет конкретно такой науки, это общий термин для
%% всех наук о Земле?
которой входит множество взаимодействующих подсистем, таких как океан, 
атмосфера, криосфера и биосфера. Отдельным важным объектом исследований
служат изменения, происходящие в этих подсистемах под влиянием деятельности
человечества.
\end{description}
\end{section}

\begin{section}{Периоды исследований океана}
Исследование океана можно условно разделить на различные периоды 
(Wust 1964). Рассмотрим эту классификацию, расширив её до конца XX~века:

\begin{enumerate}
\item  
Период поверхностной океанографии: с древнейших времён до~1873. 
Систематизация наблюдений за ветрами,
течениями, волнами, температурой и другими явлениями, поддающимися
наблюдению с палубы корабля. Известными примерами достижений той эпохи 
служат карты пассатов, составленные Галлеем, карта Гольфстрима Франклина 
и книга Мэтью Фонтейна Мори (1806--1873) <<Физическая география моря>>.
%% (Physical Geography for the Sea.) --- англ. книгу в индекс???
%% Полное название другое? The physical geography of the sea and its meteorology
%% http://bse.sci-lib.com/article078157.html

\item
Период глубоководных исследований: 1873--1914.
Различные по значимости океанографические экспедиции, цель 
которых~--- выяснение поверхностных и глубинных характеристик океана возле
колониальных земель. Важнейший пример~--- экспедиция
<<Челленджера>> (рис~2.1), но можно назвать также экспедиции <<Газели>> 
и~<<Фрама>>.

%% Figure 2.1 Example from the era of deep-sea exploration: 
%% Track of the H.M.S. Challenger during the British Challenger 
%% Expedition 1872-1876. From Wust (1964).

\item
Период национальных систематических исследований: 1925--1940.
Детальное изучение колониальных областей. Как
примеры можно привести изучение Атлантики <<Метеором>> (рис.~2.2) и
экспедицию <<Дискавери>>.

%% Figure 2.2 Example of a survey from the era of national systematic surveys. 
%% Track of the R/V Meteor during the German Meteor Expedition. From Wust (1964).

\item
Период новых методов: 1947--1956. Долговременные
исследования с использованием новых инструментов (рис.~2.3). Как
пример можно привести сейсмическое изучение Атлантики с судна <<Вема>>, в
результате которого Б.~Хейзеном (1924--1977) были составлены карты морского дна.
%% Heezen -- Хейзен/Хизен: http://slovari.yandex.ru/dict/bse/article/00086/39500.htm

%% Figure 2.3 Example from the era of new methods. The cruises of the 
%% R/V Atlantis out of Woods Hole Oceanographic Institution. From Wust (1964).

\item
Период международной кооперации: 1957--1978.
Многонациональные исследования океанов и происходящих в них процессов.
Примеры: Программа Атлантический Полярный Фронт
(Atlantic Polar Front Program), рейсы NORPAC, рейсы в ходе
Международного геофизического года и Международной декады изучения
океана (рис.~2.4), исследования с одновременным участием нескольких
десятков кораблей~--- эксперименты MODE, POLYMODE, NORPAX и JASIN.

%% Figure 2.4 Example from the era of international cooperation. 
%% The sections in the International Geophysical Year Atlantic 
%% Program 1957-1959. From Wust (1964).

\item
Эра спутников: 1978--1995. Глобальное
изучение океанических процессов из космоса. Примеры: Seasat, 
NOAA 6--10, NIMBUS-7, Geosat, Topex/Poseidon, ERS-1 и~ERS-2.

\item
Эра изучения Земли как системы: 1995--. Изучение в глобальных масштабах
взаимодействия биологических, химических и физических
процессов в океане, атмосфере и на суше с использованием численных моделей
и входных данных для них, полученных как in situ (то есть, непосредственно 
в океане), так и из космоса. В случае океана это
World Ocean Circulation Experiment (WOCE) (рис.~2.5) 
и Topex/Poseidon (рис.~2.6), Join Global
Ocean Flux Study (JGOFS), Global Ocean Data Assimilation Experiment (GODAE),
а также спутники SeaWiFS, Jason, QuikScat, Aqua и~Terra.
%% проверить, что в самом деле спутники, а что --- названия экспериментов
\end{enumerate}

%% Figure 2.5 World Ocean Circulation Experiment: Tracks of research ships 
%% making a one-time global survey of the oceans of the world. 
%% From World Ocean Circulation Experiment.

%% Figure 2.6 Example from the era of satellites. Topex/Poseidon tracks 
%% in the Pacific Ocean during a 10-day repeat of the orbit. 
%% From Topex/Poseidon Project.

%% Рисунок 2.2 Пример из эры национальных систематических
%% исследований. Путь НИС Метеор во время German Meteor
%% Expedition. (Взято из Wust, 1964).
\end{section}

\begin{section}{Вехи в понимании океана}
Что же удалось узнать об океане в ходе исследовательских программ и
экспедиций, упомянутых в предыдущем разделе? Перечислим некоторые ключевые
достижения, начиная с XVII~в. Сначала прогресс
был очень медленным. Первые простые, но очень важные в перспективе
наблюдения были сделаны учёными, которые не считали себя океанографами, 
если такой термин в те времена вообще существовал. 
В дальнейшем пришла пора более детальных описаний и океанографических 
экспериментов, проделанных учёными, специализирующимися
именно на изучении океана. 

%% Рисунок 2.3 Пример из периода новых методов. Экспедиция судна Атлантис
%% принадлежащего Океанографическому Институту в Вудсхоле (взято из
%% Wurst,1964).

%% Рисунок 2.4 Пример из периода международного взаимодействия. Измерения
%% проведённые в ходе Атлантической программы Международного
%% геофизического года 1957--1959 (взято из Wurst, 1969).

%% Рисунок 2.5 Эксперимент по исследованию циркуляции мирового океана
%% (WOCE). Профили проделанные одновременно в ходе этого глобального
%% исследования океана многими научно исследовательскими судами.

%% Рисунок 2.6 Пример из периода спутников. Треки спутника Topex /
%% Poseidon над Тихим океаном за 10 дней.

\begin{description}
\item[1685] Эдмунд Галлей опубликовал результаты проведенного
изучения океанской системы ветров и течений в работе <<Историческая оценка 
пассатов и муссонов, наблюдаемых в морях между и вблизи тропиков, 
и попытка установить физическую причину возникновения
названных ветров>> (<<An Historical Account of the Trade Winds, and
Monsoons, observable in the Seas between and near the Tropicks, with
an attempt to assign the Physical cause of the said Winds>>, 
\textsl{Philosophical Transactions,} 16: 153--168).

\item[1735] Джордж Гадлей изложил свою теорию возникновения
пассатов, основанную на сохранении углового момента, в статье
<<О причинах возникновения пассатов>> (<<Concerning the Cause of
the General Trade-Winds>>, \textsl{Philosophical Transactions,} 39: 58--62).

%% Рисунок 2.7 Вариант карты Гольфстрима Франклина и Фолгера.

\item[1751] Генри Эллис провёл в районе тропиков первое
измерение температуры на глубине и обнаружил под тёплым поверхностным
слоем холодные воды, что указывало на их полярное происхождение.

\item[1769] Бенджамин Франклин во время работы почтмейстером 
создал первую карту Гольфстрима на основе информации о маршрутах кораблей, 
курсирующих между Англией и Новой Англией, собранной его кузеном 
Тимоти Фолгером.

\item[1775] Лаплас публикует свою теорию приливов.

\item[1800] Граф Румфорд предлагает вариант
меридиональной циркуляции океана, в которой вода опускается на глубину
возле полюсов и поднимается на поверхность возле экватора.

\item[1847] Мэтью Фонтейн Мори публикует первую карту ветров и
течений, основанную на судовых записях. Мори стал первопроходцем практики
международного обмена данными об окружающей среде; он предлагал за сведения из 
судовых журналов карты и таблицы, составленные на их основе.

\item[1872--1876] Экспедиция <<Челленджера>>, которая ознаменовала начало
систематического изучения биологии, химии и физики океанов.

\item[1885] Пильсбери произвёл прямые измерения Флоридского течения с
заякоренного корабля.

\item[1903] Основание Морской биологической ассоциации Сан-Диего.
Позднее она стала Институтом океанографии имени Скриппса в составе
Калифорнийского университета.
%% Перевод Marine Biological Association
%% по аналогии с "Морская биологическая ассоциация Великобритании"
%% (http://science.viniti.ru/index.php?option=com_content&task=view&id=917&Itemid=358)

\item[1910--1913] Вильгельм Бьеркнес опубликовал
книгу <<Динамическая метеорология и гидрография>> (\textsl{Dynamic Meteorology
and Hydrography}), заложившую основы геофизической гидродинамики. В ней
он развивает понятия фронтов, динамического метра, геострофических течений,
взаимодействия океана и атмосферы, циклонов.

\item[1930] Основание Океанографического Института в Вудсхоле.
%% ??? Вудс-Хол http://slovari.yandex.ru/dict/bse/article/00008/56900.htm

%% Рисунок 2.8 Осреднённая по времени поверхностная циркуляция океана
%% построенная на основе данных полученных приблизительно за столетие
%% океанографических экспедиций (взято из Tolmazin, 1985).

\item[1942] Публикация Свердрупом, Джонсоном и Флемингом труда
<<Океаны>> (<<The Oceans>>), первого всеобъемлющего обзора
океанографических знаний.
%% ??? на самом деле, название полнее: 
%% The Oceans: Their Physics, Chemistry and General Biology
%% http://en.wikipedia.org/wiki/Harald_Sverdrup

\item[После 2-й Мировой Войны] Потребность в средствах обнаружения 
подводных лодок привела к тому, что военно-морские силы многих государств 
существенно расширили свои программы по изучению моря. В связи с этим
были открыты кафедры океанографии в различных
университетах, включая Орегонский и Техасский университеты,
университет Майами, университет Род-Айленда, а также созданы 
океанографические институты и лаборатории в других странах.

\item[1947--1950] Свердруп, Стоммел и Манк публикуют свои теории
ветровой циркуляции океана. Вместе эти три работы заложили основы
нашего понимания океанской циркуляции.

\item[1949] Начало изучения Калифорнийского течения в рамках программы
California Cooperative Fisheries Investigation of the California
Current, которая стала самым детальным исследованием прибрежного течения
из когда-либо проводившихся.

\item[1952] Кромвелл и Монтгомери открывают экваториальное
противотечение в Тихом океане.

\item[1955] Брюс Хамон и Нейл Браун разрабатывают зонд~CTD, предназначенный
%% Хамон -- Хеймон???
для измерения электропроводности и температуры как функции глубины.

\item[1958] Стоммел публикует свою теорию глубинной циркуляции океана.

\item[1963] Корпорация <<Сиппикан>> (Тим Фрэнсис,
Вильям Ван Аллен Кларк, Грэхем Кемпбелл и Сэм Фрэнсис) изобретает
отрывной батитермограф XBT (Expendable BathyThermograph), который 
в настоящее время является, наверное, самым широко используемым 
океанографическим прибором в мире.

\item[1969] Кирк Брайан и Майкл Кокс разрабатывают первую численную
модель океанской циркуляции.

\item[1978] NASA запускает первый океанографический спутник Seasat. 
Технологии, разработанные в ходе этого проекта, использовались последующими
поколениями спутников дистанционного зондирования.

\item[1979--1981] Терри Джойс, Роб Пинкель, Ллойд Ригер, F. Rowe и
J. W. Young занимаются разработками, которые в итоге привели к созданию
акустического доплеровского профилографа течений~--- популярного среди
океанографов инструмента, предназначенного для измерения скорости 
поверхностных течений с движущихся судов.

\item[1988] NASA Earth System Science Committee, возглавляемый
Фрэнсисом Брезертоном, показал в общих чертах взаимосвязь всех систем Земли.
Тем самым были разрушены барьеры, разделяющие традиционную астрофизику, 
экологию, геологию, метеорологию и океанографию.

\item[1990] Уолли Брокер предполагает, что изменения в глубинной
циркуляции океанов регулируют наступление ледниковых периодов, и что
глубинная циркуляция в Атлантике может быть нарушена, в результате чего
северное полушарие погрузится в новый ледниковый период.
%% добавить ссылку на статью
%% Broecker, W. S., and Denton, G. H., 1990, What drives glacial cycles?:
%% Scientific American, January, v. 262, no. 1, p. 48-56.

\item[1992] Рас Дэвис и Даг Вебб изобретают автономный погружающийся
буй, способный постоянно измерять течения на глубине до 2~км.

\item[1992] NASA и CNES разрабатывают и запускают спутник Topex/Poseidon,
который картирует океанские поверхностные течения, волны и приливы каждые 
10~дней.

\item[1993] Команда учёных проекта Topex/Poseidon публикует первые точные
глобальные карты приливов.
\end{description}

Более полная информация об истории физической океанографии доступна
в Приложении~А работы фон Аркса (W.S. von Arx) <<An Introduction
to Physical Oceanography>>.

Данные, накопленные в течении веков океанских экспедиций, были
использованы для составления исчерпывающего описания океана.
В большинстве работ рассматривалось его устойчивое состояние, 
течения, как поверхностные так и глубинные, а также его взаимодействие с
атмосферой. Система научных знаний на данном уровне сложилась в целом   
к началу 1970-х. Рис.~2.8 демонстрирует пример достижений того времени; 
он изображает поверхностную циркуляцию океана. 
В более поздних работах делается попытка описать динамические
процессы в океане для того, чтобы научиться предсказывать его годовую и
межгодовую изменчивость, а также понять роль океана в глобальных
процессах.
\end{section}

\begin{section}{Эволюция некоторых теоретических представлений}
Теоретическое понимание океанических процессов основано на
классической физике, объединённой со всё более развивающимися
представлениями о хаотических системах в математике и их применением
к теории турбулентности. Даты, приведенные ниже, приблизительны.

\begin{description}
\item[XIX~век] Становление аналитической гидродинамики. Кульминацией
этого процесса считается труд Ламба <<Гидродинамика>>. Бьеркнес
предлагает геострофический метод, широко используемый в
метеорологии и океанографии.

\item[1925--40] Разработка теорий турбулентности на основе
аэродинамики и понятия длины смешения турбулентного потока. 
Работы Прандтля и фон Кармана.
%% http://www.multitran.ru/c/m.exe?l1=1&l2=2&s=mixing-length
%% http://dic.academic.ru/dic.nsf/eng_rus_technic/200264/%D0%B4%D0%BB%D0%B8%D0%BD%D0%B0

\item[1940--1970] Развитие теорий турбулентности на базе
статистических корреляций и понятия однородной изотропной
турбулентности. Книги Бэтчелора (Bachelor(1967)), Хинце (Hinze (1975)) и
других.
%% Bachelor: Бэтчелор, Hinze: Хинце --- упомянуты здесь в такой транслитерации
%% http://slovari.yandex.ru/dict/bse/article/00081/14900.htm
%%
%% isotropic homogeneous turbulence.
%% Однородная и изотропная 
%% http://ru.wikipedia.org/wiki/%D0%A2%D1%83%D1%80%D0%B1%D1%83%D0%BB%D0%B5%D0%BD%D1%82%D0%BD%D0%BE%D1%81%D1%82%D1%8C

\item[1970--] Численные исследования турбулентной геофизической
гидродинамики при помощи выскопроизводительных компьютеров.

\item[1985--] Механика хаотических процессов. Её применение к
гидродинамике лишь начинается. Большинство процессов движения в атмосфере 
и~океане могут быть непредсказуемыми по своей природе.
\end{description}
\end{section}

\begin{section}{Роль наблюдений в океанографии}
На основе приведенного выше небольшого обзора теоретических основ океанологии
можно предположить, что наблюдения очень важны для понимания океана. В самом
деле, теория поведения жидкости во вращающейся системе координат с учётом 
конвекции, ветрового воздействия и турбулентности никогда не была развитой
настолько, чтобы предсказать важные свойства процессов циркуляции в океане 
до их обнаружения на практике. Почти всегда для понимания океанических 
процессов учёные обращаются к наблюдениям.

Может создаться впечатление, что многочисленные экспедиции, проведённые 
с 1873~г., должны дать хорошее описание мирового океана. Их результаты 
действительно впечатляют: cотни экспедиций были проведены во всех океанах. 
Но, несмотря на это, большая часть океана исследована слабо.

К 2000~г.\ большинство районов океана исследовалось, грубо говоря, от поверхности до
%% crudely --- возм., это не "грубо говоря", а "вчерне", "без особой тщательности"
дна только один раз. Некоторые районы, такие как Атлантика,
исследовались выборочно четырежды: в ходе экспедиции <<Метеора>> (1925), 
в рамках Международного геофизического года (1959), 
в течение Geochemical Sections cruises в начале 70-х
и во время World Ocean Circulation Experiment с~1991 по~1996~гг. 
К сожалению, выборки по всем районам не являются репрезентативными 
(подробнее об ошибках выборочного обследования см. врезку).
Наших измерений океана недостаточно для того, чтобы предсказывать его 
изменчивость и реакцию на различные внешние воздействия.
\emph{Отсутствие репрезентативных наблюдений~--- наибольший источник
ошибок в нашем понимании океана.}

Нехватка эмпирических данных служит весьма частой причиной существенных
концептуальных ошибок: 
\begin{quote}
\emph{Отсутствие фактического подтверждения трактовалось как подтверждение 
отсутствия.} Высокая сложность наблюдений за происходящими в океане явлениями
вела к тому, что феномен, который не удалось наблюдать, считался несуществующим
вообще. По мере увеличения возможностей, нашему взгляду всё отчетливее 
открывается сложность и тонкость происходящего. Wunsch (2002a).
\end{quote}
Как следствие, наше понимание океанических процессов зачастую слишком упрощено,
чтобы быть верным.

%\hrule
\fbox{%
%\parbox{\linewidth}{%
\vbox{
\begin{center}
Ошибка выборочного обследования
\end{center}
Ошибки выборочного обследования считаются в геонауках самым большим 
источником проблем. Причиной их служит использование наборов данных, 
не репрезентативных по отношению к генеральной совокупности измеряемой
переменной. Генеральная совокупность~--- это набор всех возможных
измерений, а наши измерения~--- выборка из генеральной
совокупности соответственно. Мы предполагаем, что каждое измерение сделано с
абсолютной точностью.

Чтобы понять, допущена ли ошибка выборочного обследования, требуется
прежде всего точно сформулировать проблему, которую предполагается исследовать. 
Тем самым задаётся генеральная совокупность. Затем следует выяснить,
представляют ли измерения данную совокупность. Оба эти шага
необходимы.

Допустим, нам требуется измерить среднегодовую температуру поверхности
океана, чтобы определить, идёт ли глобальное потепление. Для
этой проблемы генеральной совокупностью являются всевозможные
измерения поверхностной температуры во всех регионах и во все
месяцы. Для того, чтобы выборочное и реальное среднее совпадали,
измерения должны быть однородно распределены на протяжении года и по
всей площади океана; также они должны быть достаточно плотными для
того, чтобы включать в себя все важные процессы изменчивости в
пространстве и во времени. Это невозможно. Корабли обходят районы
%% В оригинале: If the sample mean is to equal the true mean, the samples must be uniformly distributed 
%% Почему "если"?
штормов, такие как высокие широты зимой, в силу чего корабельные
измерения не могут представлять генеральную совокупность поверхностных
температур. Спутники не в состоянии однородно измерять поверхностную
температуру на протяжении дневного цикла, а спутниковым наблюдениям за
температурой в высоких широтах зимой мешают постоянные облака; тем не
менее в большинстве регионов они обеспечивают измерения, однородные по
пространству на протяжении года. Если дневная изменчивость мала,
спутниковые данные будут более репрезентативными, чем данные с судов.

Исходя из вышесказанного ясно, что океанологические наблюдения редко
представляют собой генеральную совокупность переменной, которую мы
хотим изучать, и ошибка выборочного обследования неминуема.

Определяя ошибку выборочного обследования, мы должны чётко для себя разделять 
ошибку выборочного обследования и инструментальную. В самом деле,
инструментальная ошибка происходит вследствие неточности инструмента,
а ошибка выборочного обследования обусловлена невозможностью провести 
измерения. Рассмотрим пример, приведённый выше: определение средней температуры 
на поверхности. Если измерения производятся с судов при помощи термометров, 
каждое измерение обладает небольшой ошибкой, поскольку термометры не идеальны. 
Это инструментальная ошибка. С другой стороны, если судно зимой не заходит 
в высокие широты, то отсутствие  измерений в высоких широтах зимой~--- ошибка 
выборочного обследования.

Участники метеорологического проекта Tropical Rainfall Mapping Mission 
исследовали ошибку выборочного обследования на примере измерений дождя. 
Их результаты являются общими и могут быть применены к другим переменным. 
Интересующимся этой проблемой можно посоветовать обратиться к North \& Nakamoto (1989).
}%
}
%\hrule

\begin{paragraph}{Выбор массива океанологических данных.}
Большинство существующих
океанологических данных организовано в большие массивы.
Например, спутниковые данные обрабатываются и распространяются
группами учёных, сотрудничающими с NASA. Данные с судов и собираются, и
классифицируются другими коллективами. В настоящее время океанографы в своей 
деятельности всё больше и больше полагаются на данные, собранные другими.

Каждый, кто собирается работать как с публичными, так и с закрытыми 
наборами данных, полученными другими исследователями, должен предварительно
выяснить следующее:
\begin{enumerate}
   \item Насколько точны эти данные?
   \item Каковы ограничения этого набора данных?
   \item Как он согласуется с другими?
\end{enumerate}

Далее будут изложены несколько основополагающих принципов, которыми следует
руководствоваться при работе с такими данными.

\begin{enumerate}
\item
\emph{Используйте хорошо документированные наборы данных.} Полностью ли
документация описывает источники измерений, шаги, проведенные при
обработке данных, и критерии, согласно которым отбрасывались неверные
значения? Включает ли набор данных номер версии, позволяющий 
прослеживать изменения?

\item
\emph{Пользуйтесь проверенными (валидированными) данными.} Хорошо ли
задокументирована точность данных? Определялась ли точность, исходя из
сравнения с другими измерениями той же переменной? Была валидация
глобальной или региональной?

\item
\emph{Используйте данные, которые уже применялись другими, и на которые
ссылаются в научных статьях.} Широкая популярность некоторых наборов данных 
вполне обоснованна. Те, кто получил эти данные, использовали их в 
своих публикациях, и другие учёные им доверяют.

\item 
\emph{И наоборот, не следует пользоваться данными только потому, что они легко
доступны.} Известен ли источник данных? Например, сейчас доступно много
версий электронных карт морского дна на 5-мильной сетке. Некоторые из
них основаны на первых данных, полученных U.S Defense Mapping Agency, а
другие~--- на данных со спутника ETOPO-5. Не полагайтесь на
мнение коллег об источнике данных. Найдите документацию. Если
документации нет, ищите другие данные.
\end{enumerate}
\end{paragraph}


\begin{paragraph}{Планирование эксперимента.}
Наблюдения очень важны для океанографии, но
они дороги, так как корабельное время дорого и спутники тоже
удовольствие не из дешёвых. Поэтому океанографический эксперимент
должен быть хорошо спланирован. Рассказ о планировании эксперимента не
совсем уместен в главе об истории, но, возможно, эта тема заслуживает
нескольких коротких замечаний, так как она нечасто упоминается в книгах
по океанографии, но ей уделяется много внимания в текстах, посвящённым
другим наукам. Планирование эксперимента чрезвычайно важно, поскольку
неправильно спланированный эксперимент приводит к сомнительным
результатам, в ходе него могут измеряться не те переменные или вообще
получаться бесполезные данные.

Первый и наиболее важный аспект в планировании любого эксперимента:
перед тем, как будет принято решение, что и как будет измеряться, 
следует понять, зачем требуется проводить данные измерения.

\begin{enumerate}
\item 
Какова цель наблюдений: проверка гипотезы или описание процесса?

\item 
С какой точностью следует проводить измерения?

\item
Какое пространственное и временное разрешение необходимо? Какова
продолжительность измерений?
\end{enumerate}

Рассмотрим, например, как цель измерений будет определять способ, которым
следует проводить измерения температуры и солёности.
\begin{enumerate}
\item
Если, например, в нашу задачу входит описание водных масс в каком-нибудь 
бассейне, тогда раз в 20--50~лет требуется проводить измерения
с вертикальным разрешением 20--50~м и горизонтальным~--- 50--300~км.

\item
Если же целью является описание вертикального перемешивания 
в open equatorial Pacific,
тогда необходимо проводить измерения с вертикальным разрешением
0.5--1.0~мм и расстоянием между станциями наблюдений 50--1000~км
каждый час в течение многих дней.
\end{enumerate}
\end{paragraph}

\begin{paragraph}{Точность, определённость и линейность.}
Поскольку зашла речь об экспериментах, будет уместным представить 
три концепции, которые понадобятся нам на протяжении всей книги, 
когда мы будем касаться экспериментирования: определённость, точность 
и линейность измерений.

\emph{Точность}~--- это разница между измеренным и истинным значением.

\emph{Определённость}~--- это разница между повторяющимися измерениями.
%% ??? термин

Разницу между точностью и определённостью обычно иллюстрируют на
простом примере стрельбы из винтовки по мишени. Точностью в данном случае 
будет среднее расстояние между центром мишени и местом попадания, 
а определённостью~--- среднее расстояние между попаданиями. 
Таким образом, десять попаданий
могут быть сгруппированы внутри круга с диаметром 10~см с центром,
отстоящим от центра мишени на 20~см. Тогда точность будет равняться 20~см, 
а определённость~--- 5~см.

\emph{Линейность}~--- линейная зависимость выхода инструмента от измеряемой
величины. Нелинейные инструменты подстраивают
изменчивость к постоянному значению. ??? Следовательно, нелинейная реакция
приводит к неверным средним значениям. Нелинейность может быть так же
важна, как и точность. Например пусть
\begin{eqnarray}
\mbox{Выход} & = & \mbox{Вход} + 0.1 (\mbox{Вход})^2 \\
\mbox{Вход}  & = & a \sin \omega t
\end{eqnarray}
Тогда
\begin{eqnarray}
\mbox{Выход} & = & a \sin \omega t + 0.1 (a \sin \omega t)^2 \\
\mbox{Выход} & = & a \sin \omega t + \frac{0.1}{2} a^2 - \frac{0.1}{2}2 a^2 \cos 2\omega t
\end{eqnarray}
Обратите внимание на то что среднее значение входа~--- нуль, в то
время как выход этого нелинейного инструмента имеет среднее значение~$0.05a^2$ 
плюс такой же член, умноженный на косинус с удвоенной
частотой. В целом, если вход обладает частотами $\omega_1$ и $\omega_2$, 
то выход нелинейного инструмента имеет частоты $\omega_1\pm \omega_2$. 
Линейность особенно важна в случае, когда инструмент должен измерять 
среднее значение турбулентной переменной. Например, когда мы измеряем течения 
на небольшой глубине у поверхности, где ветры и волны вызывают большую 
изменчивость течений, нам необходимы <<линейные>> измерители течения.
\end{paragraph}

\begin{paragraph}{Чувствительность к другим переменным.}
Ошибки могут быть связаны с влиянием других переменных. 
Например, результаты измерения электропроводности чувствительны к температуре. 
Таким образом, ошибки при измерении температуры в солемере приводят к 
ошибкам в измеренных значениях электропроводности и солёности.
\end{paragraph}
\end{section}

\begin{section}{Важные концепции}
Автор надеется, что из сказанного выше читатели сделали следующие выводы:
\begin{enumerate}
\item
Океан изучен не очень хорошо. Всё, что мы о нём знаем, основано на
информации, собранной за период океанографических экспедиций,
насчитывающий чуть больше века и дополненной данными
спутников, накопленными с 1978~г.

\item
К настоящему моменту наших знаний об океане достаточно для того, чтобы
описать его циркуляцию, осреднённую по времени, в то время как более 
современные работы уже начинают затрагивать также его изменчивость.

\item
Наблюдения важны для понимания океана. Немногие процессы были
предсказаны теоретически до того, как наблюдались.

\item
Нехватка эмпирических данных ведёт к представлениям об океанических
процессах, которые зачастую слишком упрощены и даже неверны.

\item
Океанографы всё больше и больше полагаются на наборы данных, полученные
другими. Эти данные обладают ошибками и ограничениями, которые требуется
знать и понимать перед их использованием.

\item
Планирование эксперимента по меньшей мере так же важно, как его
проведение.

\item
Ошибки выборочного обследования появляются тогда, когда наблюдения не 
отображают изучаемый процесс. Эти ошибки~--- наибольший источник проблем в
океанографии.

\item
На данном этапе почти все наблюдения производятся при помощи спутников,
дрейфующих буев и других автоматических инструментов. Роль судовых наблюдений
неуклонно снижается.
\end{enumerate}
\end{section}

\end{chapter}
